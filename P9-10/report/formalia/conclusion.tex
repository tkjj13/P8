\chapter{Conclusion}
The aim of this project has been to design a controller which would enable a segway to be balancing in an upright position, as the segway can be seen as an inverted pendulum - a classical control problem that is inherently unstable. In the project, a remote controller has also been made, to be able to drive the segway wirelessly. 

To design the cascade controller, a model of the system has been derived. This model includes a model of the motors and wheels, as well as a model of the inverted pendulum. The models have been combined and linearised, after which transfer functions for the system are derived. The system model has been verified through tests. From these it is seen that the fit of the motors and wheels model is 84 \%, while the fit of the inverted pendulum model is 70\%. From these models, two controllers are designed, with the purpose of controlling the motors and stabilising the inverted pendulum. The controller for the motors and wheels is implemented as a P-controller, while the inverted pendulum controller is a PID-controller. The cascade controller, has through simulation, been proven to stabilise the model of the system, with an overshoot of 23.5 \%, whereas the requirement is 10\%. The requirements for the settling time and rise time are fulfilled. 
In the same simulation, a steady-state error of 12.3 \% is seen, which does not comply with the requirement of 0\% steady-state error.

To obtain data from the segway regarding the velocity of the wheels, encoders are used to measure the speed of the motors. To obtain the angle and angular velocity of the inverted pendulum, a gyroscope and an accelerometer are also used. The data from these three sensors are filtered using digital filters, to remove high-frequency noise. The filters are designed in such a way that the phase shift at low frequencies is small, in order to reduce the delay on the measurements at low frequencies. The filters are designed as low-pass Butterworth filters, which are then transformed to the z-domain using bilinear transformation. The filters work as designed for, but the computation time needed to run the filter code decreases the computation time needed for the wireless communication, and thus the filters are not implemented.

A remote controller is also included in the project, which makes it possible to drive the segway forwards and backwards, by sending commands from a PC. The remote controller also allows the user to receive data from the segway, regarding the tilting angle, the angular velocity and the speed of the segway. To facilitate this, a communication protocol has been designed, in which the data package structure and package header are defined.

In the acceptance test, it is seen that the designed controller after some parameter tuning makes the segway stable i.e. makes the inverted pendulum stand upright. The system is also able to re-stabilise after a light push. The segway cannot turn, but through the remote controller, the segway can be put to drive back and forth. Furthermore data can be sent from the segway to the remote controller.