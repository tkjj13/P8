%pagestyle{fancy} %enable headers and footers again

\begin{comment}
\pdfbookmark[0]{Danish title page}{label:titlepage_en}
\aautitlepage{%
  \englishprojectinfo{
    Project Title %title
  }{%
    Analoge kredsløb og systemer %theme
  }{%
    P3: 2. September 2014 - 17. December 2014 %project period
  }{%
    14gr313 % project group
  }{%
    %list of group members
    Amalie Vistoft Petersen\\
    Mikkel Krogh Simonsen\\
    Rasmus Gundorff Sæderup\\
    Simon Bjerre Krogh\\
    Thomas Kær Juel Jørgensen\\
    Thomas 'Godlike' Rasmussen
  }{%
    %list of supervisors
    Tom S. Pedersen

  }{%
    9 % number of printed copies
  }{%
    \today % date of completion
  }%
}{%department and address
  \textbf{Institut for Elektroniske Systemer}\\
  Fredrik Bajers Vej 7\\
  DK-9220 Aalborg Ø\\
  }{% the abstract
  Here is the abstract
}

\cleardoublepage

\end{comment}

\selectlanguage{english}
\pdfbookmark[0]{Title page}{label:titlepage_en}
\aautitlepage{%
  \englishprojectinfo{
    Experimental Investigation of Battery Lifetime and Massive Access in NB-IoT %title
  }{%
    MSc Project (Wireless Communication Systems)  %theme
  }{%
    9-10. Semester %project period
  }{%
    17gr950 % project group
  }{%
    %list of group members
    Mads Røgeskov Gotthardsen\\
    Thomas Kær Juel Jørgensen
  }{%
    %list of supervisors
    Petar Popovski\\
    Dong Min Kim
    }{
    Germán Corrales Madueño
    }{%
    2 % number of printed copies
  }
  {%
    \today % date of completion
  }%
}{%department and address
  \textrm{\textbf{Connectivity \\Department of Electronic Systems  }\\
  Fredrik Bajers Vej 7\\
  DK-9220 Aalborg Ø\\}
 }{
This project investigates the NB-IoT protocol. The focus is on two domains massiveness and energy consumption. The report completely separates the investigations of the domains. For the massiveness domain, an MDE is designed, leveraging on a previous project. The focus in this part is to add the functionality of emulating multiple devices in a single radio. The outcome is the ability to emulate 12 devices simultaneously. Each device is limited to receiving msg2 in the NRAP, as the baseline used is not fully developed. The baseline emulator used is the SRS NB-IoT device emulator. For the energy consumption domain, four parameters are investigated. These are the CP-format, deployment mode, frequency, and transmit power. The estimation performed, relies on actual measurements of these parameters, which is performed on a Quectel BC95 modem. The model used to estimate the battery lifetime is very simple. The results of this estimation show, that a device may only transmit for 0.468 s each day, to achieve a battery lifetime of 10 years. It is also highlighted, how the different parameters for the idle modes affect the battery lifetime.
}