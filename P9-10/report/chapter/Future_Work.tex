\chapter{Future work}
\label{ch:Future}
The focus of this chapter is to show the ideas and concepts, that was plan to be implemented or tested and is the direction future work on the project will have gone. This exclude the fixes and ideas mention in the \autoref{ch:dics} and \autoref{ch:con}.

There will be looked into these five ideas and concepts:

\begin{itemize}
\item Active and none-active devices
\item Channel models
\item ??new model??
\item ??time parameter??
\item Full emulator setup
\end{itemize}

%Active and unactive
%Channel
%New model
%Parameter for time in measurement
%Full out blown system.

\section{Active and None-active Devices}
A problem in the current version of the MDE, is that all devices is active at the start, which is only useful to test worst case scenarios, where a high number of devices wants to access the network at the same time. In most use cases, are there already devices connected to the network, some is trying to attach and other are in idle. With the concept use for the MDE, where the Co Phy is keeping the whole MDE synchronized, a timer system could be implemented into the device, most likely in the RRC layer, which starts the device up, according to a traffic plan for the individual device, and begins its attach procedure. It do not have to synchronize, as this is taken care of in the Co Phy, and when it has attached, transmitted its data and released again, it can go back to sleep, until next time it shall transmit. This is idea requires that the MDE to not crash at any point through the process, which the errors found right now does. It do not need to be able to go through the full attach, transmit, release procedure, but at least needs to be able to transmit msg3, which is the first message in the NRAP procedure, which affects other devices, besides from signal interference. If it only gets to the msg2, as the current MDE does, the devices will not interfere with each other and the massiveness problem will not be felt on the individual device.

If the turn device on and off feature gets added, the devices, that are turned off will still fill in the memory for a whole device. As the requirement for 60000 devices per km$^2$, known from \autoref{ch:NB-IoT}, for NB-IoT, it is not all device active at once, as the system will not be able to handle it. From different traffic plans, there will be a different number of active devices at once. A wanted feature is therefore, that the number of devices, that the MDE can emulated at once, is the number of possible active devices and they will be use by the devices, that at that moment have to transmit or otherwise communicate with the eNB. The other devices, which are in idle modes, will set to only keep the minimum amount of data it needs to save in a database and when the device needs to be on again, do it get a active device spot. This will require a bigger overhaul of the system, as some management is needed to put devices into active device spots and remove devices, when they are done. But if this feature is implemented, will the theory logical number of devices produce by the MDE be higher than the current version, and nearer the number defined in the requirement for NB-IoT.

\section{Channel models}
An aspect that not been looked into is the channel between the MDE and eNB. As there is use a wired as channel, the channel model only consist of a unchanged path loss. 

The implement channel models into the MDE, there is two solution for the transmission part. The first is to apply the channel model after the data from the different devices have been combined. This will only results in a single calculation, while the second solution is to do it before the data is combined, which require a calculation per device, that is transmitting. There are pros and cons for both solution, but with the second on being the optimal solution. This comes from, that the calculations can be made when the data is put into the buffer and not at a critical point in time, where the MDE needs to transmit the data and first there do the combining and then the channel calculation. Another advantage for the second solution, is that each device can have its own channel model, so it will be seen from the eNB, that the device is not placed in a big cluster at one point. The disadvantage for this solution is that the more calculations produce a bigger workload, which can affect the number of devices, the MDE can emulate.

For the receiving part, are there the same two solution, where the first is to apply the channel model, when the data is received and spread all this data out to the different devices or the second solution, where the data is spread out to the different device and then an individual channel model is applied. The advantages and disadvantages are the same, as for the transmission part, with the second solution being the most optimal.

The first solution in both case could be moved down to the USRP B210 and therefore not put a bigger workload onto the MDE with this feature. This will not work for the second solution, as the USRP B210 do not have the data from the individual devices and therefore can not apply the different channel models onto the data individual.

%\section{??New model??}
%\section{??Time parameter??}

\section{Full Emulator Setup}
Through this project, have the different domains, mention in \autoref{ch:Introduction}, been tested in different setups. A concept proposal, seen on \autoref{fig:Fsetup}, is a setup, where all three domains should be possible to emulate and measure. The concept is build upon, the other features in this chapter have been implemented. 

%%Beautiful picture

The setup contains the MDE for the massiveness domain, which will produce interference, both at a signal level and sharing resources at different layers of the protocol. The PSU will provide power and analyze the power consumption for the DUT. As the only domain not tested in this project, the reliability could be emulated by having an emulator at the input from the DUT, which emulates the wanted channel for the DUT. As reliability is mostly effected by the channel, this is thought to be enough to measure on this domain. Other parameter that effect the reliability are already built in the system, like the different repetition parameters from the NB-IoT protocol.

