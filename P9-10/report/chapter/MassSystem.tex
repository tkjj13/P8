\chapter{Massive IoT system overview}
\label{ch:MassOver}
 

The focus of this chapter is to provide an overview of the concepts and design of the \gls{MDE}. From \autoref{ch:NB-IoT} it is known that massiveness has an impact on the performance of the NB-IoT protocol. The MDE should be able to show and measure this.

%It has a high cost to test massiveness scenarios with a massive amount of hardware devices and will require a system to control them individually, especially if the desire is to test the impact of massiveness on the individual devices. A more optimal solution is instead of buying a massive amount of devices, to emulate them through software.

There are two ways to emulate massiveness, using a large number of actual devices that requires a system to control them individually or using software to achieve a large number of devices. Given the desire to emulate several thousand devices the best option is the latter.

The challenges when emulating a massive amount of devices is that every single device is complex, as seen throughout \autoref{ch:NB-IoT}. The complexity comes from the number of functionalities that are necessary. To emulate a massive number of devices takes a lot of CPU power. As each device needs to process the data. Another issue is having enough memory for the data they receive. Another challenge is the combination of the signals for the different devices before transmitting and the dividing the received signals to each device. This is is an issue as the system operates in real-time and the operations time for these tasks are very short. 


As seen in \autoref{tab:protocol_decision} from \autoref{ch:Introduction} there are many emulators available for a single NB-IoT device. As the protocol is still in development, these emulators are also still in the development stage. None of the found emulators can emulate more than one device. To build a \gls{MDE} the SRS NB-IoT emulator will act as a baseline. Its capabilities are expanded to include emulation of multiple devices. The previous project used the LTE version of the baseline\citep{thesis_report}. As mentioned in \autoref{ch:MassOver}, the design of the emulator will leverage on this project.

The SRS NB-IoT emulator is a bigger c++ code project structured up into classes. The code was never intended to emulate multiple devices, so several changes are needed, to achieve this.


%The concept to produce this type of emulator, with the focus on having as high a number of devices active, while they still keep all functionalities for the individual device, will be to split up the code up into common parts, which can be used for all devices, and individual parts, which are only used for the individual device. To make this split between parts, the baseline code will be investigated.

 

The key idea for the emulator is to split between active and inactive devices. This is done to achieve as many devices as possible. Each device needs to keep all functionalities of a device, however, some functionalities should be common for all devices. These functionalities refer to the radio layer functionalities, as only one radio is used for all devices it makes sense to only handle the radio functionalities once and share the data later to each device. To do this the code is split up into a common part, which can be used for all devices, and individual parts, which are only used for the individual device. To make this split between parts, the baseline code will is investigated.

To test the emulator, it needs to communicate with a BSE. A wire is used to connect the two, to limit the interference both to and from any potential real BS. AS BSE the Amarisoft LTE 100 code is used as it can handle 1000 devices \citep{Amarisoft_solutions} and is tested to synchronize with the baseline emulator devices that will be used. 

To convert the signals, that the emulators produce, from digital to analog and vice versa is there use a USRP B210 for each emulator, as it has been tested to work with both emulators. A 30 dB attenuator is added to compensate for the lack of path loss, as well as to protect the USRPs. This gives the whole setup seen in \autoref{fig:MassSetup}. 




%For the emulation of the eNB the Amarisoft LTE 100 code will be used as it can handle 1000 \gls{UE}s \citep{Amarisoft_solutions} and is tested to synchronize with the software based NB-IoT devices that will be used. As it is a commercial BSE, it is not open source and therefore harder to add new features and change all parameters, but as the focus is the devices, the software will still be used. A longer list of parameters for the Amarisoft LTE 100 code can be found in \appref{app:Amarisoft}.

%For the \gls{MDE} the code from SRS which have been tested to work with the software eNB from Amarisoft is chosen. This is chosen primarily because access to the source code is available through Keysight Technologies. As the code is designed for a single device some change have to be implemented, which requires knowledge of the original code. 

%The code will produce the infrastructure of the \gls{MDE} combining and transmitting the signals produced to the \gls{eNB}. The eNB will be emulated on secondary PC, as this frees up the PC running the \gls{MDE}. The NB-IoT spectrum overlaps with the LTE or GSM spectrum, meaning the spectrum is licensed, therefore a wire is used to connect \gls{MDE} and eNB. As the path loss in the wire is quite low, compared to transmission through the air, a 30 dB attenuator is added in between to increase the dynamic spectrum of the \gls{MDE} and eNB. To convert from digital signals to RF signals, a USRP B210 is connected to each the two emulators as these \gls{SDR}s are comparable with both softwares. The setup of the emulators can be seen on \autoref{fig:MassSetup}.

\begin{figure}[H]
\tikzsetnextfilename{MassSetup}
\centering
\resizebox{0.9\textwidth}{!}{
\input{figures/Mass_setup.tex}}
\caption{Overview of the emulator setup.}
\label{fig:MassSetup}
\end{figure}




%The emulation of the massiveness scenarios uses the Massive IoT and eNB (Amarisoft) components. It is possible to connect and external NB-IoT device be combining its signal with the signals from the other two components. But as the focus for this part of the emulator is to stress test the protocol this feature will not be used. The Orchestrator is not used either in this part, as it has not been prioritized to set the two system together and have the same control unit. Both components is started up from a config file. While the Amarisoft software already fits to the purpose as an eNB, the SRS software code need to be change to emulate multiple UE. To understand this changes, there will be looked into the original code

%\section{eNB emulator}
%For the emulation of the eNB the Amarisoft LTE 100 code will be used as it can handle 1000 \gls{UE} \todo{Insert ref to Amarisoft side} and is tested to synchronize with the software based NB-IoT devices that will be used. As it is a commercial BSE, it is not open source and therefore harder to add new features and change all parameters, but as the focus is the devices, the software will still be used. A longer list of parameters for the Amarisoft LTE 100 code can be found \todo{Insert appendix for this}

%\section{Massive device emulator}
%For the emulation of the massive amount of NB-IoT devices the code provide by SRS which have been tested to work with the software eNB from Amarisoft. As the code is design for a single device some change have to be implemented, which requires knowledge of the original code.



 

\section{Baseline emulator}

As stated before, the baseline emulator from SRS is still in development. The structure of the code has been copied from the LTE version and it follows the communication layer structure known from \autoref{ch:NB-IoT}. The emulator is only able to get to msg2 in the NRAP, as SRS has not developed the functionalities to transmit msg3 correctly. This blocks the possibilities to make a full attach and transmit data and all other functionalities after completing an attach. It can still be used as a proof of concept model that showcases the possibility of emulating a massive number of devices. 

%The code provide by SRS is still in development and is based on the code SRS developed to LTE, which was used in \citep{thesis_report}. The structure of the code is very similar and with that, some of the structure change and solutions can be copied over from the mentioned project.

%Unfortunately as the code is still in development, the msg3 transmission does not work, as it seems that it gets transmitted in a wrong time slot. \todo{Check up on if this is what we want to say} This block the possibility to get a complete attach request, so the project will only work up to the reception of msg2. The plans for the changes and solution for the parts including the full attach procedure and data transmission can be found in \autoref{ch:Future}.

\subsection{Structure}

\label{sub:MassStruct}

To structure the parts of the code that should be common for all devices and which will be kept as individual parts. The existing structure of the code and how it performs the different parts of the NB-IoT protocol should be understood. As stated before the code is split up into multiple classes and its structure is very similar to the LTE version of the emulator. There are two parts to the code, one is the srsLTE, which handles all lower layer functionalities and the API to the USRP B210. The other is the srsUE, where the main classes and control of the system is placed. The structure of the srsUE part can be seen in \autoref{fig:MassClass}. 

%The code is written in C++, with a little C in some of the lower level files. The code is divided up into two sections srsLTE, which contains all the lower layer functions and the radio layer, and srsUE, which emulate a device with all the different communication layers that handle the protocol. The code contains a number of classes for the different layers known from the OSI model, as seen in \autoref{fig:MassClass}, with the PHY layer being split up into even more classes. All these classes are underclasses for an overall class called ue, which therefore contains all the parts needed for running a device. The structure of the PHY layer can be seen in \autoref{fig:PhyClass}. As multiple tasks are needed to be handled at the same time in the different layers, threading is used in multiple classes (PHY, PHCH recv, PHCH workers, MAC, RRC, GW).

\begin{figure}[H]

\tikzsetnextfilename{MassClass}

\centering

\resizebox{0.5\textwidth}{!}{

\usetikzlibrary{arrows}
\begin{tikzpicture}

\draw  (-3,1) rectangle (2,0);
\draw  (-3,1.5) rectangle (2,2.5);
\draw  (-3,3) rectangle (2,4);
\draw  (-3,4.5) rectangle (2,5.5);
\node at (-0.5,0.5) {Radio};
\node at (-0.5,2) {PHY};
\node at (-0.5,3.5) {MAC};
\node at (-0.5,5) {RLC};
\draw  (-3,6) rectangle (2,7);
\draw  (-5,7.5) rectangle (-1,8.5);
\node at (-0.5,6.5) {PDCP};
\node at (-3,8) {RRC};
\draw  (0,7.5) rectangle (2,8.5);
\node at (1,8) {GW};
\draw  (-5,10.5) rectangle (-1,11.5);
\draw  (-3,10) rectangle (2,9);
\node at (-3,11) {USIM};
\node at (-0.5,9.5) {NAS};

\draw [-triangle 60](-1,1) -- (-1,1.5);
\draw [-triangle 60](0,1.5) -- (0,1);
\draw [-triangle 60](-1,2.5) -- (-1,3);
\draw [-triangle 60](0,3) -- (0,2.5);
\draw [-triangle 60](-1,4) -- (-1,4.5);
\draw [-triangle 60](0,4.5) -- (0,4);
\draw [-triangle 60](-1,5.5) -- (-1,6);
\draw [-triangle 60](0,6) -- (0,5.5);
\draw [-triangle 60](-2.5,7) -- (-2.5,7.5);
\draw [-triangle 60](-1.5,7.5) -- (-1.5,7);
\draw [-triangle 60](0.5,7) -- (0.5,7.5);
\draw [-triangle 60](1.5,7.5) -- (1.5,7);
\draw [-triangle 60](0.5,8.5) -- (0.5,9);
\draw [-triangle 60](1.5,9) -- (1.5,8.5);
\draw [-triangle 60](-3,0.25) -- (-4.5,0.25) -- (-4.5,7.5);
\draw (-3,1.75) -- (-4.5,1.75);
\draw (-3,3.25) -- (-4.5,3.25);
\draw (-3,4.75) -- (-4.5,4.75);
\draw [-triangle 60](-3.5,7.5) -- (-3.5,5.25)  -- (-3,5.25);
\draw [-triangle 60](-3.5,7.5) -- (-3.5,3.75)  -- (-3,3.75);
\draw [-triangle 60](-3.5,7.5) -- (-3.5,2.25)  -- (-3,2.25);
\draw [-triangle 60](-3.5,7.5) -- (-3.5,0.75) -- (-3,0.75);
\draw [-triangle 60](-4.5,8.5) -- (-4.5,10.5);
\draw [-triangle 60](-3.5,10.5) -- (-3.5,8.5);
\draw [-triangle 60](-2.5,8.5) -- (-2.5,9);
\draw [-triangle 60](-1.5,9) -- (-1.5,8.5);
\draw [-triangle 60](-2.5,10) -- (-2.5,10.5);
\draw [-triangle 60](-1.5,10.5) -- (-1.5,10);
\end{tikzpicture}}

\caption{Class overview of SRS emulation code.}

\label{fig:MassClass}

\end{figure}

The structure is designed like the OSI model, but do not follow the model completely, which can be seen in the functionalities seen below.

\begin{tabular}{l p{14cm}}
Radio & The radio controls and sets all the parameters needed to transmit and receive. It is the only class connected to the SDR API and sends the data to it for transmission and receives data from it too. \\
PHY & The PHY class handles synchronization, receiving and decoding of MIB and SIBs messages, and decoding and encoding of the data. \\
MAC & The MAC class keep track of TTI and give grants to the PHY class to decode and encode the DL and UL. It also takes control of the NRAP from the NPRACH class after msg1. \\
RRC & The RRC class is the overall control class for the whole device. It manages how far through the process the device is and if it got all the needed parameters. It is therefore connected to all lower classes, besides the radio class. \\
\end{tabular}

Even if the class layout look like the OSI model it does not follow blindly especially in the lower layers (PHY and MAC). This is mainly because many of the MAC layers tasks are moved to the PHY class. Normally the PHY layer is made up of physical components and do the task, that the SDR performs. The RLC, PDCP, GW, NAS, and USIM classes are not used much, as their functionality is for the process after the NRAP or at least after msg4 have been received, and have therefore not been through much development in the SRS code.

Between these classes are some interfaces, that handles the data, grants, parameters, and other information. These interfaces are individual classes. An overall interface class for functionalities used in multiple places is present in each of the interface classes. 

The PHY class handles many tasks and contains several smaller classes which take care of these tasks. The structure and description of these classes can be seen below.

\begin{figure}[H]

\centering

\tikzsetnextfilename{PhyClass}

\resizebox{0.5\textwidth}{!}{

\usetikzlibrary{arrows}
\begin{tikzpicture}

\draw  (-2.5,-1) rectangle (1.5,-3);
\draw  (-3,0) rectangle (2,-3.5);
\draw  (-7.5,-1) rectangle (-3.5,-3);
\draw  (-7.5,2.5) rectangle (2,0.5);
\draw  (-2.5,5) rectangle (1.5,3);
\node at (-1.75,-0.30) {Thread pool};
\node at (-0.5,-2) {PHCH workers};
\node at (-5.5,-2) {PHCH common};
\node at (-3.25,1.5) {PHCH recv};
\node at (-0.5,4) {NPRACH};
\draw [-triangle 60](-2.5,-2) -- (-3.5,-2);
\draw [-triangle 60](-7.5,1) -- (-8.5,1) -- (-8.5,-4.5);
\draw (-7.5,-2.5) -- (-8.5,-2.5);
\node at (-8.5,-4.75) {Radio};
\draw [-triangle 60](-0.5,0.5) -- (-0.5,0);
\draw [-triangle 60](-7.5,-1.5) -- (-8,-1.5)  -- (-8,5.5);
\draw (-7.5,2) -- (-8,2);
\node at (-8,5.75) {MAC};
\draw (-0.5,-3) -- (-0.5,-4) -- (-8,-4) -- (-8,-0);
\draw [-triangle 60](-0.5,2.5) -- (-0.5,3);
\draw [-triangle 60](-5.5,0.5) -- (-5.5,-1);
\draw [-triangle 60](-5.5,2.5) -- (-5.5,5.5);
\node at (-5.5,5.75) {RRC};
\draw (-2.5,4) -- (-8.5,4) -- (-8.5,0);
\end{tikzpicture}}

\caption{Class overview of the PHY layer from the SRS emulation code.}

\label{fig:PhyClass}

\end{figure}

\begin{tabular}{lp{12cm}}

PHCH recv & Controls the whole PHY class system. It keeps track of which state the system is in and activates the different underclasses in the PHY class. \\

PHCH common & Contains most parameters for the PHY underclasses, here under the RNTI's and metrics for UL and DL, and communication up to the MAC class. It also sends the transmission data to the radio class. \\

PHCH workers & Handles all processing of data after decoding of MIB-NB and NB-SIBs messages, including encoding and decoding of the data. There can be multiple workers for one device, but in this feature is not used. \\

Thread pool & Contains and controls the array of PHCH workers. \\    

NPRACH & Transmit msg1 in the NRAP and keep track of the parameters set for this transmission e.g. preamble id. \\

\end{tabular}

There are many other classes in the srsUE part, but they are a bi-product of the copied structure from the LTE version and are not executed, as either the feature have not been added yet or that its functionalities are first meant to be executed after msg2.

With the knowledge of the structure of the baseline emulator, the flow of the code will be looked into now.

\subsection{Execution of the baseline emulator}

 The emulator is build up into multiple sections which run at different times, as shown in \autoref{fig:MassStep}. This flow allows the emulator to synchronize and attach to a network. %The different threads go to different states, depending on which step in the process the emulator is in. The different step is shown on \autoref{fig:MassStep}.

\begin{figure}[H]

\centering

\tikzsetnextfilename{MassStep}

\resizebox{0.5\textwidth}{!}{

\usetikzlibrary{arrows}
\begin{tikzpicture}

\draw  (-5,5) rectangle (-2,3);
\draw  (-5,2.5) rectangle (-2,0.5);
\draw  (-1,2.5) rectangle (2,0.5);
\draw  (-1,0) rectangle (2,-2);
\draw  (-5,0) rectangle (-2,-2);
\draw  (-5,-2.5) rectangle (-2,-4.5);
\draw  (-1,-2.5) rectangle (2,-4.5);
\node at (-3.5,4) {Initialization};
\node at (-3.5,1.5) {Synchronization};
\node at (0.5,1.5) {MIB};
\node at (-3.5,-1) {SIB2};
\node at (0.5,-1) {SIB1};
\node at (-3.5,-3.5) {NPRACH};
\node at (0.5,-3.5) {RAP};
\draw [-triangle 60](-3.5,3) -- (-3.5,2.5);
\draw [-triangle 60](-2,1.5) -- (-1,1.5);
\draw [-triangle 60](0.5,0.5) -- (0.5,0);
\draw [-triangle 60](-1,-1) -- (-2,-1);
\draw [-triangle 60](-3.5,-2) -- (-3.5,-2.5);
\draw [-triangle 60](-2,-3.5) -- (-1,-3.5);
\end{tikzpicture}}

\caption{A simplification of the steps in the emulator process.}

\label{fig:MassStep}

\end{figure}

\textbf{Initialization} \\

The initialization step reads the config file, which includes a long list of parameters, that can be found in \appref{app:SRSconfig}. It afterward starts up all the different classes and sets pointers to the different interfaces between them. The PHY has a number of underclasses, as shown in \autoref{fig:PhyClass} which are initialized in the PHY thread and are closed down afterward. The main thread will wait on the PHY to finish the initialization, before sending an attach request, through the NAS class. This process is shown on \autoref{fig:InitStep}.

\begin{figure}[H]

\centering

\tikzsetnextfilename{InitStep}

\resizebox{0.5\textwidth}{!}{

\usetikzlibrary{arrows}
\begin{tikzpicture}


\draw  (-4,3) rectangle (-1,1.5);
\draw  (-8,3) rectangle (-5,1.5);
\draw  (-8,1) rectangle (-5,-0.5);
\draw  (-8,-1) rectangle (-5,-2.5);
\draw  (-8,-3) rectangle (-5,-4.5);
\draw  (-3.5,-1) rectangle (-0.5,-2.5);
\draw  (-3.5,-3) rectangle (-0.5,-4.5);
\draw  (-8,-5) rectangle (-5,-6.5);
\draw  (-4,-0.5) rectangle (0,-5);

\draw [-triangle 90](-4,2.25) -- (-5,2.25);
\draw [-triangle 90](-6.5,1.5) -- (-6.5,1);
\draw [-triangle 90](-6.5,-0.5) -- (-6.5,-1);
\draw [-triangle 90](-6.5,-2.5) -- (-6.5,-3);
\draw [-triangle 90](-6.5,-4.5) -- (-6.5,-5);
\draw [dash pattern=on 2pt off 3pt on 4pt off 4pt][-triangle 90](-5,0.25) -- (-2,0.25) -- (-2,-1);
\draw [dash pattern=on 2pt off 3pt on 4pt off 4pt][-triangle 90](-2,-4.5) -- (-2,-5.75) -- (-5,-5.75);
\draw [-triangle 90](-2,-2.5) -- (-2,-3);

\node at (-2.5,2.25) {Start};
\node at (-6.5,2.25) {Read config file};
\node at (-6.5,0.25) {Start PHY init};
\node at (-6.5,-1.75) {Radio init};
\node at (-6.5,-3.5) {Upper layers};
\node at (-6.5,-5.75) {Wait for PHY};
\node at (-2,-1.5) {Underclasses};
\node at (-2,-3.75) {PHCH recv init};
\draw [-triangle 60](-6.5,-6.5) -- (-6.5,-7.5);
\node at (-6.5,-7.75) {Attach request};
\node at (-0.50,-0.25) {PHY};
\node at (-6.5,-4) {init};
\node at (-2,-2) {init};
\end{tikzpicture}}

\caption{The process for the initialization. The dashed line indicates an event trigger to another thread.}

\label{fig:InitStep}

\end{figure}

\textbf{Synchronization and MIB-NB} \\

The synchronization is where the device tries to find the NPSS and NSSS in the right frequency. This is done in the PHCH recv class, which handles both the synchronization and decoding of MIB-NB. This functionality of the PHCH recv is triggered dependably on which state the class is in, as shown on \autoref{fig:RecvState}.

\begin{figure}[H]

\centering

\tikzsetnextfilename{RecvState}

\resizebox{0.5\textwidth}{!}{

\usetikzlibrary{arrows}
\begin{tikzpicture}

\draw  (-4.5,1.5) ellipse (2 and 1);
\draw  (-4.5,4.5) ellipse (2 and 1);
\draw  (-4.5,7.5) ellipse (2 and 1);
\draw  (-4.5,10.5) ellipse (2 and 1);
\draw  (-11,6) ellipse (2 and 1);
\draw [-triangle 90](-4.5,9.5) -- (-4.5,8.5);
\draw [-triangle 90](-4.5,6.5) -- (-4.5,5.5);
\draw [-triangle 90](-4.5,3.5) -- (-4.5,2.5);
\draw [-triangle 90](-11,7) -- (-11,12.5) -- (-4.5,12.5) -- (-4.5,11.5);
\draw [-triangle 90](-2.75,8) -- (-1.5,8) -- (-1.5,10.5) -- (-2.5,10.5);
\draw [-triangle 90](-6.5,10.5) -- (-7.5,10.5) -- (-7.5,6) -- (-9,6);
\draw [-triangle 90](-2.75,7) -- (-1.5,7) -- (-1.5,1.5) -- (-2.5,1.5);
\draw (-6.5,7.5) -- (-7.5,7.5);
\draw (-6.5,1.5) -- (-7.5,1.5) -- (-7.5,8.5);
\node at (-11,6) {Ilde};
\node at (-4.5,10.5) {Cell search};
\node at (-4.5,7.5) {Cell select};
\node at (-4.5,4.5) {Cell measure};
\node at (-4.5,1.5) {Cell camp};
\draw (-6.5,4.5) -- (-7.5,4.5);
\end{tikzpicture}}

\caption{State machine for PHCH recv. The whole process start in idle.}

\label{fig:RecvState}

\end{figure}

The PHCH recv is in idle until it gets triggered by the attach request, which goes through the NAS and RRC to the PHY class. The PHY class then call \textit{sync start}, so the state becomes \textit{cell search}.

In \textit{cell search}, the PHCH recv starts up the radio class and begins to receive signals from it. It looks for the NPSS and NSSS through all the different frequencies and chooses the strongest NPSS. If it does not find a frequency, where there is supposed to be a cell, it tries again. If it finds a cell, it will then try to locate the MIB-NB from the information obtained from the NPSS and NSSS. If it is found, the state changes to \textit{cell select}.

In \textit{cell select} the MIB-NB is received, decoded, and send to the MAC. In the case that it can not be decoded in the first try, it will try to do a soft combination with the next received data. This is possible due to the design of the MIB-NB  as seen in \autoref{sec:cellsync}. If the MIB-NB gets decode it will change state to \textit{cell measure} if there are no \gls{RSRP} measurements or change directly to  \textit{cell camp}. 

In \textit{cell measure} it measures the RSRP to correct the receiving parameters and before changing state to \textit{cell camp}.

In \textit{cell camp} the PHCH recv stage, where all further process will be handled, starts out by locating the NB-SIB messages.

From all stages, the process is triggered to go into idle state, if an error occurred or the system tries to reset. Here it will try to start up into cell search again unless it receives a close down trigger.

\textbf{PHCH recv cell camp state} \\

In \textit{cell camp} the PHCH recv is ready to activate the PHCH workers. A control signal informs the PHCH recv to transmit msg1 through the NPRACH class. This process can be seen on \autoref{fig:Cellcamp}.

\begin{figure}[H]

\centering

\tikzsetnextfilename{CellCamp}

\resizebox{0.7\textwidth}{!}{

\input{figures/Cellcamp.tex}}

\caption{The process for the PHCH recv, when it is in cell camp state.}

\label{fig:Cellcamp}

\end{figure}

At first, it needs to find an idle worker and as long as no worker is idle, it will be blocking the main thread for continue. After finding and starting a worker, it requests data from the radio, which is received through API for the USRP. If the data is not collected in time or if the emulator is out of sync with the API, it will return an out of sync. If this is returned, the PHCH recv will release the worker and send a signal to the RRC, that it is out of sync. This causes the state of the PHCH recv to jump back to idle. If the radio returns in sync, the data is received with synchronization parameters, such as \gls{CFO}.

In the case that the device has received both NB-SIB1 and NB-SIB2 at this point, a flag in MAC class is set indicating that  msg1 in the \gls{NRAP} needs to be sent. If it is set, the PHCH recv calls the NPRACH function that chooses a  preamble id and prepares the msg1. When it is done, the NPRACH class sends the msg1 data to the radio, which transmit it and close down the transmission afterward. This is the only place in the code, where it is not the PHCH worker, which handles uplink transmission.

When the process reaches this point, after transmitting the msg1, the PHCH worker thread is activated and it begins to handle the data. Meanwhile, the PHCH recv thread goes back to the start of the process, if it is still in cell camp state. If there are more PHCH workers, it can begin to handle the next data, before the previous PHCH worker is done, if not it waits for the PHCH worker to go into idle state.

\textbf{Downlink decoding} \\

The first part of the process in the PHCH workers thread is decoding of the DL data, received in the PHCH recv class. The process can be seen on \autoref{fig:DLwork}.

\begin{figure}[H]

\centering

\tikzsetnextfilename{DLwork}

\resizebox{0.8\textwidth}{!}{

\usetikzlibrary{arrows}
\begin{tikzpicture}

\draw (0,1) -- (-2,0) node (v2) {} -- (0,-1) -- (2,0) node (v7) {} -- (0,1) node (v15) {};
\draw (-6,-2)  -- (-8,-3) -- (-6,-4) node (v3) {} -- (-4,-3) node (v6) {} -- (-6,-2) node (v1) {};
\draw (-6,-5)  -- (-8,-6) -- (-6,-7) node (v5) {} -- (-4,-6) node (v8) {} -- (-6,-5) node (v4) {};
\draw  (4,-2) rectangle (8,-4);
\draw  (10,-2) rectangle (14,-4);
\draw  (7,-8) rectangle (11,-10);
\draw  (-2,-2) rectangle (2,-4);
\draw  (-8,-8) rectangle (-4,-10);
\draw [-triangle 60](-2,0) -- (-6,0) -- (-6,-2);
\draw [-triangle 60](-6,-4) -- (-6,-5);
\draw [-triangle 60](-6,-7) -- (-6,-8);
\draw [-triangle 60](-4,-3) -- (-2,-3);
\draw [-triangle 60](2,0) -- (6,0) -- (6,-2);
\draw [-triangle 60](8,-3) -- (10,-3);

\draw (0,-4) -- (3,-10.5) node (v9) {};
\node at (0,0) {Have DL grant};
\node at (-6,-3) {New DL grant};
\node at (-6,-6) {New UL grant};
\node at (0,-2.75) {Send new DL grant};
\node at (-6,-8.75) {Send new UL grant};
\node at (6,-3) {Decoding NPDSCH};
\node at (12,-2.75) {Send decode data to};
\node at (9,-8.6) {Generate};
\node at (3,-11.25) {Contiune with UL};
\node at (-6,-9.25) {to MAC};
\node at (9,-9.4) {and UL grant};
\node at (12,-3.25) {MAC};
\node at (0,-3.25) {to MAC};
\node at (-2.5,0.5) {No};
\node at (2.5,0.5) {Yes};
\node at (-3.5,-2.5) {Yes};
\node at (-5.5,-4.5) {No};
\node at (-3.5,-6) {No};
\node at (-5.5,-7.5) {Yes};
\node at (6,-4.85) {Error};
\node at (9,-2.5) {Succes};

\draw (-4,-6) -- (3,-10.5);
\draw (-4,-9) -- (3,-10.5);
\draw  -- (3,-10.5);
\draw (7,-9) -- (3,-10.5);
\draw [-triangle 60](3,-10.5) node (v12) {} -- (3,-11);
\draw (9,-5) node (v10) {} -- (7,-6) node (v11) {} -- (9,-7) node (v14) {} -- (11,-6) --  (9,-5) node (v13) {};
\draw (7,-6) -- (3,-10.5);
\draw [-triangle 60](9,-7) -- (9,-8);
\node at (9,-5.8) {Need};
\node at (9.5,-7.5) {Yes};
\node at (6,-6) {No};
\draw [-triangle 60](0,2) -- (0,1);
\draw [-triangle 60](6,-4) -- (6,-4.5) -- (9,-4.5) -- (9,-5);
\draw (12,-4) -- (12,-4.5) -- (8.5,-4.5);
\node at (9,-9) {ACK/NACK};
\node at (9,-6.3) {ACK/NACK};
\end{tikzpicture}}

\caption{The process for the downlink decoding in PHCH worker.}

\label{fig:DLwork}

\end{figure}

Firstly the PHCH worker asks the MAC if it has a DL grant from earlier. If not, it will assume that the received data is from NPDCCH, where it first searches for a new DL grant and if that is not in the data, it searches for a new UL grant. After the search for new grants, it continues with the UL.

If the MAC has a DL grant from earlier, it will try to decode the data, which is sent through the NPDSCH. If this is successful, it will send the decoded data up to MAC to handle it. After sending the data to the MAC for decoding, it checks if ACK/NACK is enabled. If this flag is set, the ACK/NACK is generated. Following this, the UL process begins.

\textbf{Uplink encoding and transmission} \\

After the DL, the system checks if it needs to send data through UL and have a UL grant. It then encodes the data and sends it along with the UL parameters to the PHCH common. The PHCH common calls the transmission function in the radio class. Next time the worker is finished with a run through the thread and it has nothing to transmit, the PHCH common tell the radio class to end transmission. This whole process is seen on \autoref{fig:ULwork}.

\begin{figure}[H]

\centering

\tikzsetnextfilename{ULwork}

\resizebox{0.8\textwidth}{!}{

\usetikzlibrary{arrows}
\begin{tikzpicture}


\draw (0,2.5) node (v1) {} -- (-2,1.5) -- (0,0.5) node (v5) {} -- (2,1.5) node (v8) {} -- (0,2.5);
\draw (0,-0.5) node (v2) {} -- (-2,-1.5) -- (0,-2.5) node (v6) {} -- (2,-1.5) node (v11) {} -- (0,-0.5);
\draw (0,-3.5) node (v3) {} -- (-2,-4.5) -- (0,-5.5) node (v7) {} -- (2,-4.5) node (v12) {} -- (0,-3.5);
\draw  (-2,-6.5) rectangle (2,-8.5);
\draw  (-2,-9.5) rectangle (2,-11.5);
\draw  (4,-9.5) rectangle (8,-11.5);
\draw (9,-10.5) node (v4) {} -- (11,-11.5) -- (13,-10.5) node (v10) {} -- (11,-9.5) node (v9) {} -- (9,-10.5);
\draw  (15,-9.5) rectangle (19,-11.5);
\draw  (15,-8.5) rectangle (19,-6.5);
\draw [-triangle 60](0,3.5) -- (0,2.5);
\draw [-triangle 60](0,0.5) -- (0,-0.5);
\draw [-triangle 60](0,-2.5) -- (0,-3.5);
\draw [-triangle 60](0,-5.5) -- (0,-6.5);
\draw [-triangle 60](0,-8.5) -- (0,-9.5);
\draw [-triangle 60](2,-10.5) -- (4,-10.5);
\draw [-triangle 60](2,1.5) -- (6,1.5) -- (6,-9.5);
\draw [-triangle 60](8,-10.5) -- (9,-10.5);
\draw [-triangle 60](11,-9.5) -- (11,-7.5) -- (15,-7.5);
\draw [-triangle 60](13,-10.5) -- (15,-10.5);
\draw (v11) -- (6,-1.5);
\draw (v12) -- (6,-4.5);
\node at (0,1.5) {TX enable};
\node at (0,-1.5) {RNTI};
\node at (0,-4.5) {UL grant};
\node at (0,-7.5) {Encode NPUSCH};
\node at (0,-10.5) {Set UL parameters};
\node at (6,-10.5) {Release worker};
\node at (11,-10.5) {Data to transmit};
\node at (17,-7.5) {End transmission};
\node at (17,-10.25) {Start/Continue};
\node at (17,-10.75) {transmission};
\node at (0,4) {Done with DL};
\node at (2.5,2) {No};
\node at (2.5,-1) {0};
\node at (2.5,-4) {No};
\node at (0.5,-6) {Yes};
\node at (0.5,-3) {!= 0};
\node at (0.5,0) {Yes};
\node at (13.5,-10) {Yes};
\node at (11.5,-9) {No};
\draw  (3,-8.5) rectangle (14,-12.5);
\draw  (14,-5.5) rectangle (20,-12.5);
\node at (4.5,-9) {PHCH common};
\node at (14.75,-6) {Radio};
\end{tikzpicture}}

\caption{The process for the uplink decoding in PHCH worker.}

\label{fig:ULwork}

\end{figure}

\textbf{NB-SIB1, NB-SIB2 and msg2 in NRAP} \\

These three messages are all received in the DL after the MIB-NB and in the msg2 case, after transmitted msg1. When the NPDCCH is decoded, there can be a grant to one of these messages and it will begin to look for it in the NPDSCH and decode it when it is received. After decoding it will go to the next step in the code.

To have the emulator emulate multiple devices, some parts of the original device is needed to be duplicated. The goal of these changes is to have a more than 15 individual and independent devices emulated.

\section{Development of MDE}

\label{sec:Changes}

This section describes the changes and implementations made to the baseline emulator and the reasoning behind these.

The concept mentioned in the start of this chapter is to focus on making some functionalities common for all devices. This is to minimize the workload, but still have the devices split enough to see them as independent devices.

\textbf{Radio}\\

Through the previous part about the structure of the code, it is seen that the radio class is at the bottom of the class structure. This class is the bottleneck for the simple strategy of copying the whole structure for each device, as there should only be one connection to the API of the USRP B210. Thus only one instance of the radio class should be generated.

The next point is to secure that all communication to the radio class is controlled. As seen in the previous structure section and in \autoref{fig:PhyClass}, the only classes that communicate with the radio class are PHCH recv (for synchronization and decoding of MIB), NPRACH (to transmit msg1 in the NRAP) and PHCH common (with the data from the PHCH worker).

In the transmission part, there is a need of combining all the signal correctly, compared to time slots and frequencies, while still being in sync with the BS. This will be fixed with a buffer system, that will receive the data and hold on to it until it is transmitted. A problem comes when the data have to be combined, as it has to be transmitted simultaneously. As different devices have the possibility for different power level, a power combiner is implemented, that takes this parameter into account.

In the receiving part, the data is sent only to the PHCH recv. Here there is also a need for a buffer so that all devices can get the data before the next data is received. 

With these two buffer system, the problem with combining and distributing the data is fixed, while the layers above can be copied and have all their functionalities intact. But with this setup, the bottleneck in the system will quickly be the workload on the CPU kernels, which is why it is looked into how to lower the overall workload. As different functionalities are tied to different classes, it will be best to look at which classes can be made common.

\textbf{PHCH recv}\\

The control class among the PHY underclasses, PHCH recv, can be a common class instead of an individual class. One of the pros of having it as a common class is that the buffer for receiving data can be moved up here, where it can be called from. Another pro is that the synchronization and the decoding of the MIB can be made common tasks, as these two steps in the process will not influence other devices. The PHCH recv will also keep all the devices synchronized by measuring the DL channel. The con will be that devices will not be able to get off sync individually if that is wanted to be tested.

The other underclasses in the PHY class all contain parts which should be kept individual, as they include some parameters meant for the individual device (NPRACH have preamble id and PHCH common have RNTI's). A solution could be to make a large PHCH worker array, where all devices get workers from. But to ensure that all devices always have a worker available, the PHCH worker array will also be copied. The higher layers classes will also be copied, as these all include the state the different devices are in and this is also needed to achieve independence between the devices.

With all classes, besides radio and PHCH recv being copied for each device emulated, some changes have to be made in the initialization.

\textbf{Initialization}\\

As seen in \autoref{fig:MassClass}, are both the MAC and RRC connected to the PHY class. There is also interactions between the PHCH recv and the other PHY underclasses, as seen on \autoref{fig:PhyClass}. So when the initialization step begins, the PHCH recv needs pointers to all the different MAC, RRC and other PHY classes to move data around and for its other tasks. A problem when implemented this, is that the MAC class needs the PHY class to be initialized before it can initialize itself and the PHCH recv is in need of all the MAC classes to be initialized before it can be initialized itself. This is done by splitting the PHY initialize up into two parts, where the PHY underclasses, except the PHCH recv, is initialized first. The higher layer classes (besides MAC) and radio are initialized next, then MAC and finally the PHCH recv is initialized. This process can be seen on \autoref{fig:InitNew}.

\begin{figure}[H]

\tikzsetnextfilename{InitNew}

\centering

\resizebox{0.5\textwidth}{!}{

\usetikzlibrary{arrows}
\begin{tikzpicture}


\draw  (-4,3) rectangle (-1,1.5);
\draw  (-8,3) rectangle (-5,1.5);
\draw  (-8,1) rectangle (-5,-0.5);
\draw  (-8,-1) rectangle (-5,-2.5);
\draw  (-8,-3) rectangle (-5,-4.5);
\draw  (-3,-1) rectangle (0,-2.5);
\draw  (-3,-9) rectangle (0,-10.5);
\draw  (-8,-5) rectangle (-5,-6.5);
\draw  (-3.5,-0.5) rectangle (0.5,-11);

\draw [-triangle 90](-4,2.25) -- (-5,2.25);
\draw [-triangle 90](-6.5,1.5) -- (-6.5,1);
\draw [-triangle 90](-6.5,-0.5) -- (-6.5,-1);
\draw [-triangle 90](-6.5,-2.5) -- (-6.5,-3);
\draw [-triangle 90](-6.5,-4.5) -- (-6.5,-5);
\draw [dash pattern=on 2pt off 3pt on 4pt off 4pt][-triangle 90](-5,0.25) -- (-1.5,0.25) -- (-1.5,-1);


\node at (-2.5,2.25) {Start};
\node at (-6.5,2.25) {Read config file};
\node at (-6.5,0.25) {Start PHY init};
\node at (-6.5,-1.75) {Radio init};
\node at (-6.5,-3.5) {Upper layers init};
\node at (-6.5,-5.75) {Wait for PHY};
\node at (-1.5,-1.75) {Underclasses init};
\node at (-1.5,-9.75) {PHCH recv init};


\node at (0,-0.25) {PHY};
\node at (-6.5,-4) {(besides MAC)};

\draw [-triangle 60](-6.5,-6.5) -- (-6.5,-7);
\draw  (-8,-7) rectangle (-5,-8.5);
\draw  (-8,-9) rectangle (-5,-10.5);
\draw [-triangle 60](-6.5,-8.5) -- (-6.5,-9);
\node at (-6.5,-7.75) {MAC init};
\node at (-6.5,-9.75) {Wait for PHY};
\draw [-triangle 60](-6.5,-10.5) -- (-6.5,-11.5);
\node at (-6.5,-12) {Attach request};
\draw [-triangle 60] (-3,-7) rectangle (0,-8.5);
\draw [-triangle 60](-1.5,-2.5) -- (-1.5,-7);
\draw [-triangle 60](-1.5,-8.5) -- (-1.5,-9);
\node at (-1.5,-7.75) {Wait for main};
\draw [dash pattern=on 2pt off 3pt on 4pt off 4pt][-triangle 60](-3,-1.75) -- (-4,-1.75) -- (-4,-5.75) -- (-5,-5.75);
\draw [dash pattern=on 2pt off 3pt on 4pt off 4pt][-triangle 60](-5,-7.75) -- (-3,-7.75);
\draw [dash pattern=on 2pt off 3pt on 4pt off 4pt][-triangle 60](-3,-9.75) -- (-5,-9.75);
\end{tikzpicture}}

\caption{Overview of the emulator setup.}

\label{fig:InitNew}

\end{figure}

Another change for the initialization step is that instead of calling the attach request in the NAS class. It is called directly in the PHCH recv, as else the receiver is started multiple times, instead of only once.

This way of splitting the classes up into common and individual for each device was also done in the previous project \citep{thesis_report}. This comes from the similarities of the two baselines used. The way that the common classes structured are different from each other though. The old version is seen on \autoref{fig:OldStruct}, while the version used in this project is seen on \autoref{fig:NewStruct}.

\begin{figure}[H]

\tikzsetnextfilename{OldStruct}

\centering

\resizebox{0.5\textwidth}{!}{

\input{figures/OldStruct.tex}}

\caption{Low layer design for the old project. Black boxes are not used. Blue lines are for DL and orange is for UL.}

\label{fig:OldStruct}

\end{figure}

\begin{figure}[H]

\tikzsetnextfilename{NewStruct}

\centering

\resizebox{0.5\textwidth}{!}{

\usetikzlibrary{arrows}
\definecolor{mycolor1}{rgb}{0.00000,0.44700,0.74100}%
\definecolor{mycolor2}{rgb}{0.85000,0.32500,0.09800}%
\begin{tikzpicture}

\draw  (-5,2.5) rectangle (-1,-1.5);
\draw  (-5,3) rectangle (-1,5);
\fill [black](-3,-1.5) rectangle (-1,0.5);
\draw  (1,2.5) rectangle (5,-1.5);
\draw  (1,3) rectangle (5,5);
\fill [black](3,-1.5) rectangle (5,0.5);
\draw  (-3,6) ellipse (0.1 and 0.1);
\draw  (-3,5.5) ellipse (0.1 and 0.1);
\draw  (-3,6.5) ellipse (0.1 and 0.1);
\draw  (3,6) ellipse (0.1 and 0.1);
\draw  (3,5.5) ellipse (0.1 and 0.1);
\draw  (3,6.5) ellipse (0.1 and 0.1);
\draw  (5.5,0.5) ellipse (0.1 and 0.1);
\draw  (6,0.5) ellipse (0.1 and 0.1);
\draw  (6.5,0.5) ellipse (0.1 and 0.1);
\draw  (7,2.5) rectangle (11,-1.5);
\fill [black]  (9,-1.5) rectangle (11,0.5);
\draw  (7,3) rectangle (11,5);
\draw  (9,6) ellipse (0.1 and 0.1);
\draw  (9,5.5) ellipse (0.1 and 0.1);
\draw  (9,6.5) ellipse (0.1 and 0.1);
\node at (-3,7.5) {Device 1};
\node at (3,7.5) {Device 2};
\node at (9,7.5) {Device n};
\node at (-3,4) {MAC};
\node at (3,4) {MAC};
\node at (9,4) {MAC};
\node at (-3,1.5) {PHY};
\node at (3,1.5) {PHY};
\node at (9,1.5) {PHY};
\node [white]at (-2,-0.5) {Recv};
\node [white]at (4,-0.5) {Recv};
\node [white]at (10,-0.5) {Recv};
\draw  (-5,-8.5) rectangle (11,-10.5);
\node at (3,-9.5) {Radio};
\draw  (8,-8.5) rectangle (11,-9.5);
\node at (9.5,-9) {Buffer};
\draw [-triangle 90](-4,2.5) -- (-4,3);
\draw [-triangle 90](-2,3) -- (-2,2.5);
\draw [-triangle 90](2,2.5) -- (2,3);
\draw [-triangle 90](4,3) -- (4,2.5);
\draw [-triangle 90](8,2.5) -- (8,3);
\draw [-triangle 90](10,3) -- (10,2.5);

\fill [black]  (-5,-3.5) rectangle (-1,-7.5);
\fill [white](-3,-7.5) rectangle (-1,-5.5);

\node [white] at (-3,-4.5) {PHY};
\node at (-2,-6.5) {Recv};
\draw [-triangle 60][mycolor1](-2,-8.5) -- (-2,-7.5);
\draw [-triangle 60][mycolor1](-1,-6.5) -- (0,-6.5) -- (0,-2.5) -- (-2,-2.5) -- (-4.5,-2.5) -- (-4.5,-1.5);
\draw [-triangle 60][mycolor1](-0.5,-2.5) -- (1.5,-2.5) -- (1.5,-1.5);
\draw [-triangle 60][mycolor2](-3.5,-1.5) -- (-3.5,-3) -- (9.5,-3) -- (9.5,-8.5);
\draw [mycolor2](2.5,-1.5) -- (2.5,-3);
\draw [mycolor2](8.5,-1.5) -- (8.5,-3);
\end{tikzpicture}}

\caption{Low layer design for this project. Black boxes are not used. Blue lines are for DL and orange is for UL.}

\label{fig:NewStruct}

\end{figure}

As seen on the two figures above, the big difference is that the PHCH recv class, used as the common class for all devices. In the old version it is placed in device 1, while in the new version, it is a full PHY class, which is named Co-Phy for future reference. By moving the common PHCH recv out of device 1 and into it owns, gives new possibilities to use other PHY class functions and underclasses e.g. having a PHCH worker in the Co-Phy to decode the NB-SIB1 and SIB2 steps, so not all device has to decode these messages. Other possibilities are listed in \autoref{ch:Future}. 

Another pro is that device 1 does not contain the PHCH recv class anymore and is not required to be active. The con is that the setup for the interfaces between the Co-Phy and the devices becomes more complicated, as seen earlier in this section. The choice of structure is mostly influenced by the possibilities, which can be found in \autoref{ch:Future}, as the structure, is freer from device 1.

\textbf{PHCH workers}\\

With the changes to having the PHCH recv being a common class, it is first when reaching the NB-SIB1 step, that the individual devices need to activate their PHCH worker. The \textit{cell camp} state, shown on \autoref{fig:Cellcamp} starts the workers in the baseline, which is still case for the MDE. This is done through a loop, where each device PHCH worker is started up after each other. As that the PHCH workers begin to work in their own thread at the end of the loop in the \textit{cell camp} state, the loop quickly loop around and the workers are started quickly after each other. The receiver buffer is placed outside this loop, so there is only received once per rotation of all devices. This requires that all PHCH workers are done before the next data needs to received or else will the PHCH recv get an out of sync flag and go back to idle state. This problem can be solved, by having a timer, that checks when the next data need to be received and then the late devices miss some data, but this has not been implemented, due to the lack of time resources in the project.

\subsection{Summary}

A baseline emulator capable of emulating one device has been changed to an MDE, which is cable of emulating multiple devices. The design of the MDE follows the concept, which was used in the previous project. Some changes were made to the lower structure to achieve multiple emulated devices. The baseline emulator is still in development, which has affected the development and result of the MDE. 

The synchronization and decoding of MIB-NB were made a common task for the Co-Phy. The Co-Phy is an extra class designed to handle most of the common task for all the devices. With this addition and the other changes, the MDE should be able to produce a massive number of devices, while still performing like the baseline, which only emulates a single device. This will be tested in the next chapter.

\chapter{Testing of Massive IoT system} \label{ch:mass_test}
The focus of this chapter is to showcase the emulator describe in \autoref{ch:MassOver}. This is done in a series of test, where it is compared to the original code and to see if the changed made for fill the goal set in \autoref{ch:MassOver}:

\textit{The goal of these changes will be to have a massive amount of individual devices emulated, without them effecting each other through their processing time and combining their signals and transmit it as one to the eNB.}

This will be done by comparing the changed code to the baseline code as well as testing the performance of the code at higher number of devices emulated.
The performance criteria that will be look into is:

\begin{itemize}
\item Error rate
\item Execution time
\item CPU usage
\item Memory usage
\end{itemize}

The error rate will showcase the stability of the code. This is done by running the code multiple times and to analyze the errors occurring. The errors can be analyzed from the log files, produce by the code, when executed.

The execution time will be look into to see if any processes in code is taking longer, after the changes. This will be measured by inserting prints of time stamps into the code, when the code goes from one step to another. 

CPU and memory usage is used to test where possible bottle necks can be compared to overloading the used computer. CPU usage will be measure with the CPU stat tool, which can measure the CPU usage on the individual processes at a sample rate of 3 Hz. The memory usage is measured using the system monitor tool, as all bigger buffers is allocated in the initialization and the used memory therefore is nearly static. 

The parameters used for the eNB and SRS code is the default settings shown in \todo{Insert app ref to the two apps}. The emulator have executed the code 20 times per different amount of devices until the emulator hit 100\% in error rate multiple times in a row. Beside testing the emulator with the changed code, the baseline code will also be tested, but as it can only emulate one devices, this will be what the code will be matched against.

\section{Error rate}
\label{sec:MTerror}
The error rate is analyze from the log messages coming from the emulator on its process through the code. These result can be seen in \autoref{fig:MT_error}.

\begin{figure}[H]
\tikzsetnextfilename{MT_error}
\centering
\resizebox{0.5\textwidth}{!}{
% This file was created by matlab2tikz.
%
%The latest updates can be retrieved from
%  http://www.mathworks.com/matlabcentral/fileexchange/22022-matlab2tikz-matlab2tikz
%where you can also make suggestions and rate matlab2tikz.
%
\definecolor{mycolor1}{rgb}{0.00000,0.44700,0.74100}%
\definecolor{mycolor2}{rgb}{0.85000,0.32500,0.09800}%
%
\begin{tikzpicture}

\begin{axis}[%
width=4.521in,
height=3.566in,
at={(0.758in,0.481in)},
scale only axis,
xmin=1,
xmax=15,
xlabel={Number of UEs},
ymin=0,
ymax=100,
ylabel={Error rate (\%)},
axis background/.style={fill=white},
title style={font=\bfseries},
title={Error rate},
legend style={at={(0.03,0.97)},anchor=north west,legend cell align=left,align=left,draw=white!15!black}
]
\addplot [color=mycolor1,solid]
  table[row sep=crcr]{%
1	25\\
2	40\\
3	25\\
4	15\\
5	40\\
6	30\\
7	30\\
8	20\\
9	15\\
10	50\\
11	30\\
12	20\\
13	100\\
14	100\\
15	100\\
};
\addlegendentry{Change code};

\addplot [color=mycolor2,solid]
  table[row sep=crcr]
\caption{Error rate for different amount of devices.}
\label{fig:MT_error}
\end{figure}

As it can be seen in \autoref{fig:MT_error}, the error rate is higher for the changed code, than the baseline which still is not perfect. Another important aspect is that when the number of devices hit 13, the error rate goes to 100\%, which shows the limit of the emulator. To get a better understanding of these error, a diagram over the different error is shown in \autoref{fig:MT_error_dist}.

\begin{figure}[H]
\tikzsetnextfilename{MT_error_dist}
\centering
\resizebox{0.8\textwidth}{!}{
% This file was created by matlab2tikz.
%
%The latest updates can be retrieved from
%  http://www.mathworks.com/matlabcentral/fileexchange/22022-matlab2tikz-matlab2tikz
%where you can also make suggestions and rate matlab2tikz.
%
\definecolor{mycolor1}{rgb}{0.20810,0.16630,0.52920}%
\definecolor{mycolor2}{rgb}{0.01651,0.42660,0.87863}%
\definecolor{mycolor3}{rgb}{0.02650,0.61370,0.81350}%
\definecolor{mycolor4}{rgb}{0.21783,0.72504,0.61926}%
\definecolor{mycolor5}{rgb}{0.64730,0.74560,0.41880}%
\definecolor{mycolor6}{rgb}{0.99377,0.74546,0.24035}%
\definecolor{mycolor7}{rgb}{0.97630,0.98310,0.05380}%
%
\begin{tikzpicture}

\begin{axis}[%
width=.8\textwidth,
height=.533\textwidth,
at={(0.533in,0.481in)},
scale only axis,
bar width=0.5,
xmin=-0.5,
xmax=15.5,
xtick={0,1,2,3,4,5,6,7,8,9,10,11,12,13,14,15},
xticklabels={{Base},{1},{2},{3},{4},{5},{6},{7},{8},{9},{10},{11},{12},{13},{14},{15}},
xlabel={Number of devices},
ymin=0,
ymax=100,
ylabel={Trials ($\%$)},
axis background/.style={fill=white},
title style={font=\bfseries},
title={Error distrubution},
legend style={at={(1.03,1)},anchor=north west,legend cell align=left,align=left,draw=white!15!black}
]
\addplot[ybar stacked,draw=black,fill=mycolor1,area legend] plot table[row sep=crcr] {%
0	90\\
1	75\\
2	60\\
3	75\\
4	85\\
5	60\\
6	70\\
7	70\\
8	80\\
9	85\\
10	50\\
11	70\\
12	80\\
13	0\\
14	0\\
15	0\\
};
\addlegendentry{No errors};

\addplot[ybar stacked,draw=black,fill=mycolor2,area legend] plot table[row sep=crcr] {%
0	0\\
1	0\\
2	0\\
3	0\\
4	0\\
5	0\\
6	0\\
7	0\\
8	5\\
9	0\\
10	0\\
11	0\\
12	0\\
13	5\\
14	0\\
15	0\\
};
\addlegendentry{Cell sync error};

\addplot[ybar stacked,draw=black,fill=mycolor3,area legend] plot table[row sep=crcr] {%
0	10\\
1	10\\
2	10\\
3	10\\
4	0\\
5	15\\
6	15\\
7	5\\
8	10\\
9	5\\
10	15\\
11	15\\
12	10\\
13	0\\
14	15\\
15	5\\
};
\addlegendentry{Radio error};

\addplot[ybar stacked,draw=black,fill=mycolor4,area legend] plot table[row sep=crcr] {%
0	0\\
1	10\\
2	30\\
3	15\\
4	15\\
5	25\\
6	10\\
7	25\\
8	5\\
9	5\\
10	30\\
11	5\\
12	5\\
13	80\\
14	65\\
15	60\\
};
\addlegendentry{Idle after MIB-NB};

\addplot[ybar stacked,draw=black,fill=mycolor5,area legend] plot table[row sep=crcr] {%
0	0\\
1	0\\
2	0\\
3	0\\
4	0\\
5	0\\
6	0\\
7	0\\
8	0\\
9	5\\
10	5\\
11	10\\
12	5\\
13	5\\
14	10\\
15	20\\
};
\addlegendentry{Transmission after NB-SIB1};

\addplot[ybar stacked,draw=black,fill=mycolor6,area legend] plot table[row sep=crcr] {%
0	0\\
1	0\\
2	0\\
3	0\\
4	0\\
5	0\\
6	5\\
7	0\\
8	0\\
9	0\\
10	0\\
11	0\\
12	0\\
13	10\\
14	10\\
15	10\\
};
\addlegendentry{NPRACH error};

\addplot[ybar stacked,draw=black,fill=mycolor7,area legend] plot table[row sep=crcr] 
\caption{The distribution of different errors for different amount of devices.}
\label{fig:MT_error_dist}
\end{figure}

As seen in \autoref{fig:MT_error_dist} is there six different types of errors that have occurred in the testing.

Radio error is the only error type that the baseline also have. It comes from miscommunication between the radio class and the API for the USRP B210. It occur when the process begins the search after SIB messages, where some radio parameters is changed. As the radio error is the only error occurring for the baseline, this should be the only error which are not produced by the changed made to the baseline.

The ilde after MIB error occur in the same part of the process as the radio error, where the emulator gets stuck and runs without retrying or closing down. This error type is the most occurring type of the errors produced by the changes, which will be the most optimal place to improve the error rate, especially for higher amount of devices.

The msg2 not received error is when all devices have gone through the NPRACH step and waiting on the msg2, which is never received or not registered if received. This error is very rare and have no tendencies.

Cell sync error is when the emulator shuts down before a synchronization is finalized and it goes to the MIB step. This errors is very rare, but as the error occurs so early in the process, these test runs will not give any data for any other step, beside initialization.

Transmission after SIB1 is an error that occurs sometimes at the SIB1 step and the radio class transmit a signal, which course the emulator to shut down. This error type have a tendency to occur with higher amount of devices and can indicate a bottleneck in the process to get a higher amount of devices.

NPRACH error is an error that occurs when some devices completes the NPRACH step, but other do not, as the system shuts down before hand. This error type have the same tendency as the transmission after SIB1 error and indicates the same thing.

To further test the emulator, the test will be split up into the different steps in the code process discussed in \autoref{sub:MassStruct}.

\section{Initialization}
The execution time for the initialization step is measured dependably on the number of devices emulated and the baseline is measured as well, which gives the results seen in \autoref{fig:MT_Init_Time}.

\begin{figure}[H]
\tikzsetnextfilename{MT_Init_Time}
\centering
\resizebox{0.5\textwidth}{!}{
% This file was created by matlab2tikz.
%
%The latest updates can be retrieved from
%  http://www.mathworks.com/matlabcentral/fileexchange/22022-matlab2tikz-matlab2tikz
%where you can also make suggestions and rate matlab2tikz.
%
\definecolor{mycolor1}{rgb}{0.00000,0.44700,0.74100}%
\definecolor{mycolor2}{rgb}{0.85000,0.32500,0.09800}%
%
\begin{tikzpicture}

\begin{axis}[%
width=0.9\textwidth,
height=0.6\textwidth,
at={(0.758in,0.481in)},
scale only axis,
xmin=-0.5,
xmax=15.5,
xtick={0,1,2,3,4,5,6,7,8,9,10,11,12,13,14,15},
xticklabels={{Baseline},{1},{2},{3},{4},{5},{6},{7},{8},{9},{10},{11},{12},{13},{14},{15}},
xlabel={Number of devices},
ymin=1.4,
ymax=3.2,
ylabel={Time (s)},
axis background/.style={fill=white},
title style={font=\bfseries},
title={Initialization time},
legend style={at={(0.03,0.97)},anchor=north west,legend cell align=left,align=left,draw=white!15!black}
]



\addplot [color=mycolor1,only marks,mark=*,mark options={solid}]
  table[row sep=crcr]{%
1	1.60657000541687\\
1	1.60206508636475\\
1	1.60402417182922\\
1	1.60998201370239\\
1	1.59828901290894\\
1	1.59890818595886\\
1	1.60678696632385\\
1	1.70481204986572\\
1	1.60345387458801\\
1	1.59799599647522\\
1	1.60137987136841\\
1	1.60408782958984\\
1	1.60380721092224\\
1	1.61764979362488\\
1	1.61315512657166\\
1	1.60408186912537\\
1	1.60214781761169\\
1	1.59820103645325\\
1	1.70323586463928\\
1	1.60239100456238\\
};
\addlegendentry{Change code};

\addplot [color=mycolor2,only marks,mark=*,mark options={solid}]
  table[row sep=crcr]{%
0	1.51574110984802\\
0	1.60954785346985\\
0	1.61224007606506\\
0	1.51666712760925\\
0	1.51392078399658\\
0	1.51035499572754\\
0	1.50713300704956\\
0	1.51292109489441\\
0	1.51016092300415\\
0	1.5099790096283\\
0	1.51415991783142\\
0	1.51640319824219\\
0	1.51323890686035\\
0	1.51066589355469\\
0	1.5165901184082\\
0	1.50853300094604\\
0	1.50311899185181\\
0	1.60737919807434\\
0	1.51729917526245\\
0	1.50886988639832\\
};
\addlegendentry{Baseline};

\addplot [color=mycolor1,only marks,mark=*,mark options={solid},forget plot]
  table[row sep=crcr]{%
2	1.70376014709473\\
2	1.69752597808838\\
2	1.69991588592529\\
2	1.70224094390869\\
2	1.70335793495178\\
2	1.70019602775574\\
2	1.81223583221436\\
2	1.70259284973145\\
2	1.70058512687683\\
2	1.90431618690491\\
2	1.69746613502502\\
2	1.70056891441345\\
2	1.7057900428772\\
2	1.6926589012146\\
2	1.69660997390747\\
2	1.69284415245056\\
2	1.70735216140747\\
2	1.69585204124451\\
2	1.69405889511108\\
2	1.70744895935059\\
};


\addplot [color=mycolor1,only marks,mark=*,mark options={solid},forget plot]
  table[row sep=crcr]{%
3	1.79403805732727\\
3	1.80174088478088\\
3	1.80632400512695\\
3	1.79517197608948\\
3	1.7999119758606\\
3	1.80256199836731\\
3	1.79768395423889\\
3	1.79976677894592\\
3	1.79294800758362\\
3	1.80160784721375\\
3	1.81014108657837\\
3	1.8038489818573\\
3	1.79714298248291\\
3	1.78692388534546\\
3	1.80100011825562\\
3	1.79488515853882\\
3	1.81216788291931\\
3	1.79761385917664\\
3	1.80774402618408\\
3	1.79226994514465\\
};

\addplot [color=mycolor1,only marks,mark=*,mark options={solid},forget plot]
  table[row sep=crcr]{%
4	1.98840284347534\\
4	1.98891401290894\\
4	1.90781497955322\\
4	1.90410208702087\\
4	1.89322519302368\\
4	1.89481401443481\\
4	1.89260292053223\\
4	1.89337205886841\\
4	1.89051008224487\\
4	1.99312591552734\\
4	1.89608120918274\\
4	1.98998594284058\\
4	1.9188380241394\\
4	1.88647603988647\\
4	1.89477205276489\\
4	1.89825892448425\\
4	1.90163683891296\\
4	1.89038610458374\\
4	1.91325998306274\\
4	1.88523697853088\\
};
\addplot [color=mycolor1,only marks,mark=*,mark options={solid},forget plot]
  table[row sep=crcr]{%
5	1.99140405654907\\
5	1.98323607444763\\
5	2.10928916931152\\
5	1.97859597206116\\
5	2.08593010902405\\
5	1.97863793373108\\
5	1.99062418937683\\
5	2.07845187187195\\
5	1.99165511131287\\
5	1.97789597511292\\
5	1.98802804946899\\
5	1.9954240322113\\
5	2.00401997566223\\
5	1.98706889152527\\
5	1.98496317863464\\
5	1.97689509391785\\
5	1.98235487937927\\
5	1.98183512687683\\
5	2.10645198822021\\
5	1.97618389129639\\
};
\addplot [color=mycolor1,only marks,mark=*,mark options={solid},forget plot]
  table[row sep=crcr]{%
6	2.07043600082397\\
6	2.08795213699341\\
6	2.08343291282654\\
6	2.09298491477966\\
6	2.07702803611755\\
6	2.07426118850708\\
6	2.1002459526062\\
6	2.09029412269592\\
6	2.08507609367371\\
6	2.07489895820618\\
6	2.08115887641907\\
6	2.09077501296997\\
6	2.08771395683289\\
6	2.07395005226135\\
6	2.07281303405762\\
6	2.1818699836731\\
6	2.08554911613464\\
6	2.0755889415741\\
6	2.09108996391296\\
6	2.08529710769653\\
};
\addplot [color=mycolor1,only marks,mark=*,mark options={solid},forget plot]
  table[row sep=crcr]{%
7	2.16757988929749\\
7	2.16262793540955\\
7	2.17564296722412\\
7	2.1759250164032\\
7	2.19155502319336\\
7	2.27151203155518\\
7	2.1676127910614\\
7	2.17567276954651\\
7	2.17186117172241\\
7	2.29317903518677\\
7	2.17238211631775\\
7	2.17540311813354\\
7	2.17659687995911\\
7	2.16815400123596\\
7	2.17439103126526\\
7	2.17733693122864\\
7	2.17026114463806\\
7	2.17389607429504\\
7	2.19961309432983\\
7	2.1946439743042\\
};
\addplot [color=mycolor1,only marks,mark=*,mark options={solid},forget plot]
  table[row sep=crcr]{%
8	2.27100706100464\\
8	2.26513481140137\\
8	2.26285195350647\\
8	2.2616548538208\\
8	2.27730703353882\\
8	2.27119493484497\\
8	2.26165103912354\\
8	2.25236415863037\\
8	2.26101899147034\\
8	2.36273288726807\\
8	2.25905799865723\\
8	2.27876305580139\\
8	2.29212021827698\\
8	2.26190590858459\\
8	2.27887606620789\\
8	2.25859999656677\\
8	2.26044893264771\\
8	2.26256895065308\\
8	2.26363897323608\\
8	2.26199913024902\\
};
\addplot [color=mycolor1,only marks,mark=*,mark options={solid},forget plot]
  table[row sep=crcr]{%
9	2.35985803604126\\
9	2.35758590698242\\
9	2.35490608215332\\
9	2.35805201530457\\
9	2.35813903808594\\
9	2.36619591712952\\
9	2.36291694641113\\
9	2.3615140914917\\
9	2.36494493484497\\
9	2.35645508766174\\
9	2.36352205276489\\
9	2.36592721939087\\
9	2.46607184410095\\
9	2.46248698234558\\
9	2.35874915122986\\
9	2.35979104042053\\
9	2.36403608322144\\
9	2.35437083244324\\
9	2.45976400375366\\
9	2.35649704933167\\
};
\addplot [color=mycolor1,only marks,mark=*,mark options={solid},forget plot]
  table[row sep=crcr]{%
10	2.4522500038147\\
10	2.45602107048035\\
10	2.45017790794373\\
10	2.46005892753601\\
10	2.4437301158905\\
10	2.45332884788513\\
10	2.45075607299805\\
10	2.44673299789429\\
10	2.44251298904419\\
10	2.45762705802917\\
10	2.45294904708862\\
10	2.48201680183411\\
10	2.45067620277405\\
10	2.57340097427368\\
10	2.44935178756714\\
10	2.45021200180054\\
10	2.55899405479431\\
10	2.55124402046204\\
10	2.45972394943237\\
10	2.4552149772644\\
};
\addplot [color=mycolor1,only marks,mark=*,mark options={solid},forget plot]
  table[row sep=crcr]{%
11	2.54409885406494\\
11	2.5580689907074\\
11	2.55131316184998\\
11	2.54538011550903\\
11	2.54991102218628\\
11	2.54586219787598\\
11	2.54519009590149\\
11	2.54685091972351\\
11	2.54958415031433\\
11	2.55115580558777\\
11	2.55340099334717\\
11	2.54899501800537\\
11	2.54859089851379\\
11	2.55042195320129\\
11	2.54939794540405\\
11	2.57132411003113\\
11	2.54307293891907\\
11	2.54036903381348\\
11	2.54952502250671\\
11	2.56598615646362\\
};
\addplot [color=mycolor1,only marks,mark=*,mark options={solid},forget plot]
  table[row sep=crcr]{%
12	2.63736891746521\\
12	2.6386501789093\\
12	2.63579487800598\\
12	2.76004385948181\\
12	2.64884686470032\\
12	2.63978505134583\\
12	2.65288496017456\\
12	2.6491858959198\\
12	2.64470791816711\\
12	2.65107917785645\\
12	2.83676886558533\\
12	2.63889098167419\\
12	2.65938687324524\\
12	2.6382257938385\\
12	2.63785290718079\\
12	2.65852999687195\\
12	2.74441695213318\\
12	2.64707398414612\\
12	2.63858604431152\\
12	2.65901207923889\\
};
\addplot [color=mycolor1,only marks,mark=*,mark options={solid},forget plot]
  table[row sep=crcr]{%
13	2.73925495147705\\
13	2.76844787597656\\
13	2.74468803405762\\
13	2.72794198989868\\
13	2.73288083076477\\
13	2.72759079933167\\
13	2.73981499671936\\
13	2.73808693885803\\
13	2.73037505149841\\
13	2.73060989379883\\
13	2.75299692153931\\
13	2.73150610923767\\
13	2.73392391204834\\
13	2.83988189697266\\
13	2.73668503761292\\
13	2.73319101333618\\
13	2.7269880771637\\
13	2.73366594314575\\
13	2.73546814918518\\
13	2.73145604133606\\
};
\addplot [color=mycolor1,only marks,mark=*,mark options={solid},forget plot]
  table[row sep=crcr]{%
14	2.93708896636963\\
14	2.8230938911438\\
14	2.82491183280945\\
14	2.81724190711975\\
14	2.83096098899841\\
14	2.82523608207703\\
14	2.92713284492493\\
14	2.83528399467468\\
14	2.81740093231201\\
14	2.94761514663696\\
14	2.84746098518372\\
14	2.92244791984558\\
14	2.82092189788818\\
14	2.82772517204285\\
14	2.82116103172302\\
14	2.8086531162262\\
14	2.82523202896118\\
14	2.82433700561523\\
14	2.85196495056152\\
14	2.82061409950256\\
};
\addplot [color=mycolor1,only marks,mark=*,mark options={solid},forget plot]
  table[row sep=crcr]{%
15	2.91355895996094\\
15	2.92956900596619\\
15	2.92799687385559\\
15	3.0291600227356\\
15	2.92018890380859\\
15	2.92904806137085\\
15	2.91579508781433\\
15	2.91569089889526\\
15	2.92359805107117\\
15	2.91793894767761\\
15	2.9355640411377\\
15	3.12653303146362\\
15	2.91362810134888\\
15	2.94397902488709\\
15	2.9135410785675\\
15	2.91744613647461\\
15	2.92062401771545\\
15	2.91158103942871\\
15	3.02637410163879\\
15	2.90584707260132\\
};

\addplot [color=red,solid,forget plot]
  table[row sep=crcr]{%
0	1.52591009015129\\
0.015	1.52732309482467\\
0.03	1.52873609949805\\
0.045	1.53014910417142\\
0.06	1.5315621088448\\
0.075	1.53297511351818\\
0.09	1.53438811819156\\
0.105	1.53580112286494\\
0.12	1.53721412753832\\
0.135	1.5386271322117\\
0.15	1.54004013688508\\
0.165	1.54145314155845\\
0.18	1.54286614623183\\
0.195	1.54427915090521\\
0.21	1.54569215557859\\
0.225	1.54710516025197\\
0.24	1.54851816492535\\
0.255	1.54993116959873\\
0.27	1.55134417427211\\
0.285	1.55275717894549\\
0.3	1.55417018361886\\
0.315	1.55558318829224\\
0.33	1.55699619296562\\
0.345	1.558409197639\\
0.36	1.55982220231238\\
0.375	1.56123520698576\\
0.39	1.56264821165914\\
0.405	1.56406121633252\\
0.42	1.56547422100589\\
0.435	1.56688722567927\\
0.45	1.56830023035265\\
0.465	1.56971323502603\\
0.48	1.57112623969941\\
0.495	1.57253924437279\\
0.51	1.57395224904617\\
0.525	1.57536525371955\\
0.54	1.57677825839293\\
0.555	1.5781912630663\\
0.57	1.57960426773968\\
0.585	1.58101727241306\\
0.6	1.58243027708644\\
0.615	1.58384328175982\\
0.63	1.5852562864332\\
0.645	1.58666929110658\\
0.66	1.58808229577996\\
0.675	1.58949530045333\\
0.69	1.59090830512671\\
0.705	1.59232130980009\\
0.72	1.59373431447347\\
0.735	1.59514731914685\\
0.75	1.59656032382023\\
0.765	1.59797332849361\\
0.78	1.59938633316699\\
0.795	1.60079933784036\\
0.81	1.60221234251374\\
0.825	1.60362534718712\\
0.84	1.6050383518605\\
0.855	1.60645135653388\\
0.87	1.60786436120726\\
0.885	1.60927736588064\\
0.9	1.61069037055402\\
0.915	1.6121033752274\\
0.93	1.61351637990077\\
0.945	1.61492938457415\\
0.96	1.61634238924753\\
0.975	1.61775539392091\\
0.99	1.61916839859429\\
1.005	1.62058140326767\\
1.02	1.62199440794105\\
1.035	1.62340741261443\\
1.05	1.6248204172878\\
1.065	1.62623342196118\\
1.08	1.62764642663456\\
1.095	1.62905943130794\\
1.11	1.63047243598132\\
1.125	1.6318854406547\\
1.14	1.63329844532808\\
1.155	1.63471145000146\\
1.17	1.63612445467484\\
1.185	1.63753745934821\\
1.2	1.63895046402159\\
1.215	1.64036346869497\\
1.23	1.64177647336835\\
1.245	1.64318947804173\\
1.26	1.64460248271511\\
1.275	1.64601548738849\\
1.29	1.64742849206187\\
1.305	1.64884149673524\\
1.32	1.65025450140862\\
1.335	1.651667506082\\
1.35	1.65308051075538\\
1.365	1.65449351542876\\
1.38	1.65590652010214\\
1.395	1.65731952477552\\
1.41	1.6587325294489\\
1.425	1.66014553412227\\
1.44	1.66155853879565\\
1.455	1.66297154346903\\
1.47	1.66438454814241\\
1.485	1.66579755281579\\
1.5	1.66721055748917\\
1.515	1.66862356216255\\
1.53	1.67003656683593\\
1.545	1.67144957150931\\
1.56	1.67286257618268\\
1.575	1.67427558085606\\
1.59	1.67568858552944\\
1.605	1.67710159020282\\
1.62	1.6785145948762\\
1.635	1.67992759954958\\
1.65	1.68134060422296\\
1.665	1.68275360889634\\
1.68	1.68416661356971\\
1.695	1.68557961824309\\
1.71	1.68699262291647\\
1.725	1.68840562758985\\
1.74	1.68981863226323\\
1.755	1.69123163693661\\
1.77	1.69264464160999\\
1.785	1.69405764628337\\
1.8	1.69547065095674\\
1.815	1.69688365563012\\
1.83	1.6982966603035\\
1.845	1.69970966497688\\
1.86	1.70112266965026\\
1.875	1.70253567432364\\
1.89	1.70394867899702\\
1.905	1.7053616836704\\
1.92	1.70677468834378\\
1.935	1.70818769301715\\
1.95	1.70960069769053\\
1.965	1.71101370236391\\
1.98	1.71242670703729\\
1.995	1.71383971171067\\
2.01	1.71525271638405\\
2.025	1.71666572105743\\
2.04	1.71807872573081\\
2.055	1.71949173040418\\
2.07	1.72090473507756\\
2.085	1.72231773975094\\
2.1	1.72373074442432\\
2.115	1.7251437490977\\
2.13	1.72655675377108\\
2.145	1.72796975844446\\
2.16	1.72938276311784\\
2.175	1.73079576779121\\
2.19	1.73220877246459\\
2.205	1.73362177713797\\
2.22	1.73503478181135\\
2.235	1.73644778648473\\
2.25	1.73786079115811\\
2.265	1.73927379583149\\
2.28	1.74068680050487\\
2.295	1.74209980517825\\
2.31	1.74351280985162\\
2.325	1.744925814525\\
2.34	1.74633881919838\\
2.355	1.74775182387176\\
2.37	1.74916482854514\\
2.385	1.75057783321852\\
2.4	1.7519908378919\\
2.415	1.75340384256528\\
2.43	1.75481684723865\\
2.445	1.75622985191203\\
2.46	1.75764285658541\\
2.475	1.75905586125879\\
2.49	1.76046886593217\\
2.505	1.76188187060555\\
2.52	1.76329487527893\\
2.535	1.76470787995231\\
2.55	1.76612088462569\\
2.565	1.76753388929906\\
2.58	1.76894689397244\\
2.595	1.77035989864582\\
2.61	1.7717729033192\\
2.625	1.77318590799258\\
2.64	1.77459891266596\\
2.655	1.77601191733934\\
2.67	1.77742492201272\\
2.685	1.77883792668609\\
2.7	1.78025093135947\\
2.715	1.78166393603285\\
2.73	1.78307694070623\\
2.745	1.78448994537961\\
2.76	1.78590295005299\\
2.775	1.78731595472637\\
2.79	1.78872895939975\\
2.805	1.79014196407312\\
2.82	1.7915549687465\\
2.835	1.79296797341988\\
2.85	1.79438097809326\\
2.865	1.79579398276664\\
2.88	1.79720698744002\\
2.895	1.7986199921134\\
2.91	1.80003299678678\\
2.925	1.80144600146016\\
2.94	1.80285900613353\\
2.955	1.80427201080691\\
2.97	1.80568501548029\\
2.985	1.80709802015367\\
3	1.80851102482705\\
3.015	1.80992402950043\\
3.03	1.81133703417381\\
3.045	1.81275003884719\\
3.06	1.81416304352056\\
3.075	1.81557604819394\\
3.09	1.81698905286732\\
3.105	1.8184020575407\\
3.12	1.81981506221408\\
3.135	1.82122806688746\\
3.15	1.82264107156084\\
3.165	1.82405407623422\\
3.18	1.8254670809076\\
3.195	1.82688008558097\\
3.21	1.82829309025435\\
3.225	1.82970609492773\\
3.24	1.83111909960111\\
3.255	1.83253210427449\\
3.27	1.83394510894787\\
3.285	1.83535811362125\\
3.3	1.83677111829463\\
3.315	1.838184122968\\
3.33	1.83959712764138\\
3.345	1.84101013231476\\
3.36	1.84242313698814\\
3.375	1.84383614166152\\
3.39	1.8452491463349\\
3.405	1.84666215100828\\
3.42	1.84807515568166\\
3.435	1.84948816035503\\
3.45	1.85090116502841\\
3.465	1.85231416970179\\
3.48	1.85372717437517\\
3.495	1.85514017904855\\
3.51	1.85655318372193\\
3.525	1.85796618839531\\
3.54	1.85937919306869\\
3.555	1.86079219774207\\
3.57	1.86220520241544\\
3.585	1.86361820708882\\
3.6	1.8650312117622\\
3.615	1.86644421643558\\
3.63	1.86785722110896\\
3.645	1.86927022578234\\
3.66	1.87068323045572\\
3.675	1.8720962351291\\
3.69	1.87350923980247\\
3.705	1.87492224447585\\
3.72	1.87633524914923\\
3.735	1.87774825382261\\
3.75	1.87916125849599\\
3.765	1.88057426316937\\
3.78	1.88198726784275\\
3.795	1.88340027251613\\
3.81	1.88481327718951\\
3.825	1.88622628186288\\
3.84	1.88763928653626\\
3.855	1.88905229120964\\
3.87	1.89046529588302\\
3.885	1.8918783005564\\
3.9	1.89329130522978\\
3.915	1.89470430990316\\
3.93	1.89611731457654\\
3.945	1.89753031924991\\
3.96	1.89894332392329\\
3.975	1.90035632859667\\
3.99	1.90176933327005\\
4.005	1.90318233794343\\
4.02	1.90459534261681\\
4.035	1.90600834729019\\
4.05	1.90742135196357\\
4.065	1.90883435663694\\
4.08	1.91024736131032\\
4.095	1.9116603659837\\
4.11	1.91307337065708\\
4.125	1.91448637533046\\
4.14	1.91589938000384\\
4.155	1.91731238467722\\
4.17	1.9187253893506\\
4.185	1.92013839402397\\
4.2	1.92155139869735\\
4.215	1.92296440337073\\
4.23	1.92437740804411\\
4.245	1.92579041271749\\
4.26	1.92720341739087\\
4.275	1.92861642206425\\
4.29	1.93002942673763\\
4.305	1.93144243141101\\
4.32	1.93285543608438\\
4.335	1.93426844075776\\
4.35	1.93568144543114\\
4.365	1.93709445010452\\
4.38	1.9385074547779\\
4.395	1.93992045945128\\
4.41	1.94133346412466\\
4.425	1.94274646879804\\
4.44	1.94415947347141\\
4.455	1.94557247814479\\
4.47	1.94698548281817\\
4.485	1.94839848749155\\
4.5	1.94981149216493\\
4.515	1.95122449683831\\
4.53	1.95263750151169\\
4.545	1.95405050618507\\
4.56	1.95546351085845\\
4.575	1.95687651553182\\
4.59	1.9582895202052\\
4.605	1.95970252487858\\
4.62	1.96111552955196\\
4.635	1.96252853422534\\
4.65	1.96394153889872\\
4.665	1.9653545435721\\
4.68	1.96676754824548\\
4.695	1.96818055291885\\
4.71	1.96959355759223\\
4.725	1.97100656226561\\
4.74	1.97241956693899\\
4.755	1.97383257161237\\
4.77	1.97524557628575\\
4.785	1.97665858095913\\
4.8	1.97807158563251\\
4.815	1.97948459030588\\
4.83	1.98089759497926\\
4.845	1.98231059965264\\
4.86	1.98372360432602\\
4.875	1.9851366089994\\
4.89	1.98654961367278\\
4.905	1.98796261834616\\
4.92	1.98937562301954\\
4.935	1.99078862769292\\
4.95	1.99220163236629\\
4.965	1.99361463703967\\
4.98	1.99502764171305\\
4.995	1.99644064638643\\
5.01	1.99785365105981\\
5.025	1.99926665573319\\
5.04	2.00067966040657\\
5.055	2.00209266507995\\
5.07	2.00350566975332\\
5.085	2.0049186744267\\
5.1	2.00633167910008\\
5.115	2.00774468377346\\
5.13	2.00915768844684\\
5.145	2.01057069312022\\
5.16	2.0119836977936\\
5.175	2.01339670246698\\
5.19	2.01480970714035\\
5.205	2.01622271181373\\
5.22	2.01763571648711\\
5.235	2.01904872116049\\
5.25	2.02046172583387\\
5.265	2.02187473050725\\
5.28	2.02328773518063\\
5.295	2.02470073985401\\
5.31	2.02611374452739\\
5.325	2.02752674920076\\
5.34	2.02893975387414\\
5.355	2.03035275854752\\
5.37	2.0317657632209\\
5.385	2.03317876789428\\
5.4	2.03459177256766\\
5.415	2.03600477724104\\
5.43	2.03741778191442\\
5.445	2.03883078658779\\
5.46	2.04024379126117\\
5.475	2.04165679593455\\
5.49	2.04306980060793\\
5.505	2.04448280528131\\
5.52	2.04589580995469\\
5.535	2.04730881462807\\
5.55	2.04872181930145\\
5.565	2.05013482397483\\
5.58	2.0515478286482\\
5.595	2.05296083332158\\
5.61	2.05437383799496\\
5.625	2.05578684266834\\
5.64	2.05719984734172\\
5.655	2.0586128520151\\
5.67	2.06002585668848\\
5.685	2.06143886136186\\
5.7	2.06285186603523\\
5.715	2.06426487070861\\
5.73	2.06567787538199\\
5.745	2.06709088005537\\
5.76	2.06850388472875\\
5.775	2.06991688940213\\
5.79	2.07132989407551\\
5.805	2.07274289874889\\
5.82	2.07415590342226\\
5.835	2.07556890809564\\
5.85	2.07698191276902\\
5.865	2.0783949174424\\
5.88	2.07980792211578\\
5.895	2.08122092678916\\
5.91	2.08263393146254\\
5.925	2.08404693613592\\
5.94	2.0854599408093\\
5.955	2.08687294548267\\
5.97	2.08828595015605\\
5.985	2.08969895482943\\
6	2.09111195950281\\
6.015	2.09252496417619\\
6.03	2.09393796884957\\
6.045	2.09535097352295\\
6.06	2.09676397819633\\
6.075	2.0981769828697\\
6.09	2.09958998754308\\
6.105	2.10100299221646\\
6.12	2.10241599688984\\
6.135	2.10382900156322\\
6.15	2.1052420062366\\
6.165	2.10665501090998\\
6.18	2.10806801558336\\
6.195	2.10948102025674\\
6.21	2.11089402493011\\
6.225	2.11230702960349\\
6.24	2.11372003427687\\
6.255	2.11513303895025\\
6.27	2.11654604362363\\
6.285	2.11795904829701\\
6.3	2.11937205297039\\
6.315	2.12078505764377\\
6.33	2.12219806231714\\
6.345	2.12361106699052\\
6.36	2.1250240716639\\
6.375	2.12643707633728\\
6.39	2.12785008101066\\
6.405	2.12926308568404\\
6.42	2.13067609035742\\
6.435	2.1320890950308\\
6.45	2.13350209970417\\
6.465	2.13491510437755\\
6.48	2.13632810905093\\
6.495	2.13774111372431\\
6.51	2.13915411839769\\
6.525	2.14056712307107\\
6.54	2.14198012774445\\
6.555	2.14339313241783\\
6.57	2.14480613709121\\
6.585	2.14621914176458\\
6.6	2.14763214643796\\
6.615	2.14904515111134\\
6.63	2.15045815578472\\
6.645	2.1518711604581\\
6.66	2.15328416513148\\
6.675	2.15469716980486\\
6.69	2.15611017447824\\
6.705	2.15752317915161\\
6.72	2.15893618382499\\
6.735	2.16034918849837\\
6.75	2.16176219317175\\
6.765	2.16317519784513\\
6.78	2.16458820251851\\
6.795	2.16600120719189\\
6.81	2.16741421186527\\
6.825	2.16882721653865\\
6.84	2.17024022121202\\
6.855	2.1716532258854\\
6.87	2.17306623055878\\
6.885	2.17447923523216\\
6.9	2.17589223990554\\
6.915	2.17730524457892\\
6.93	2.1787182492523\\
6.945	2.18013125392568\\
6.96	2.18154425859905\\
6.975	2.18295726327243\\
6.99	2.18437026794581\\
7.005	2.18578327261919\\
7.02	2.18719627729257\\
7.035	2.18860928196595\\
7.05	2.19002228663933\\
7.065	2.19143529131271\\
7.08	2.19284829598608\\
7.095	2.19426130065946\\
7.11	2.19567430533284\\
7.125	2.19708731000622\\
7.14	2.1985003146796\\
7.155	2.19991331935298\\
7.17	2.20132632402636\\
7.185	2.20273932869974\\
7.2	2.20415233337312\\
7.215	2.20556533804649\\
7.23	2.20697834271987\\
7.245	2.20839134739325\\
7.26	2.20980435206663\\
7.275	2.21121735674001\\
7.29	2.21263036141339\\
7.305	2.21404336608677\\
7.32	2.21545637076015\\
7.335	2.21686937543352\\
7.35	2.2182823801069\\
7.365	2.21969538478028\\
7.38	2.22110838945366\\
7.395	2.22252139412704\\
7.41	2.22393439880042\\
7.425	2.2253474034738\\
7.44	2.22676040814718\\
7.455	2.22817341282055\\
7.47	2.22958641749393\\
7.485	2.23099942216731\\
7.5	2.23241242684069\\
7.515	2.23382543151407\\
7.53	2.23523843618745\\
7.545	2.23665144086083\\
7.56	2.23806444553421\\
7.575	2.23947745020759\\
7.59	2.24089045488096\\
7.605	2.24230345955434\\
7.62	2.24371646422772\\
7.635	2.2451294689011\\
7.65	2.24654247357448\\
7.665	2.24795547824786\\
7.68	2.24936848292124\\
7.695	2.25078148759462\\
7.71	2.25219449226799\\
7.725	2.25360749694137\\
7.74	2.25502050161475\\
7.755	2.25643350628813\\
7.77	2.25784651096151\\
7.785	2.25925951563489\\
7.8	2.26067252030827\\
7.815	2.26208552498165\\
7.83	2.26349852965503\\
7.845	2.2649115343284\\
7.86	2.26632453900178\\
7.875	2.26773754367516\\
7.89	2.26915054834854\\
7.905	2.27056355302192\\
7.92	2.2719765576953\\
7.935	2.27338956236868\\
7.95	2.27480256704206\\
7.965	2.27621557171543\\
7.98	2.27762857638881\\
7.995	2.27904158106219\\
8.01	2.28045458573557\\
8.025	2.28186759040895\\
8.04	2.28328059508233\\
8.055	2.28469359975571\\
8.07	2.28610660442909\\
8.085	2.28751960910246\\
8.1	2.28893261377584\\
8.115	2.29034561844922\\
8.13	2.2917586231226\\
8.145	2.29317162779598\\
8.16	2.29458463246936\\
8.175	2.29599763714274\\
8.19	2.29741064181612\\
8.205	2.2988236464895\\
8.22	2.30023665116287\\
8.235	2.30164965583625\\
8.25	2.30306266050963\\
8.265	2.30447566518301\\
8.28	2.30588866985639\\
8.295	2.30730167452977\\
8.31	2.30871467920315\\
8.325	2.31012768387653\\
8.34	2.3115406885499\\
8.355	2.31295369322328\\
8.37	2.31436669789666\\
8.385	2.31577970257004\\
8.4	2.31719270724342\\
8.415	2.3186057119168\\
8.43	2.32001871659018\\
8.445	2.32143172126356\\
8.46	2.32284472593693\\
8.475	2.32425773061031\\
8.49	2.32567073528369\\
8.505	2.32708373995707\\
8.52	2.32849674463045\\
8.535	2.32990974930383\\
8.55	2.33132275397721\\
8.565	2.33273575865059\\
8.58	2.33414876332397\\
8.595	2.33556176799734\\
8.61	2.33697477267072\\
8.625	2.3383877773441\\
8.64	2.33980078201748\\
8.655	2.34121378669086\\
8.67	2.34262679136424\\
8.685	2.34403979603762\\
8.7	2.345452800711\\
8.715	2.34686580538437\\
8.73	2.34827881005775\\
8.745	2.34969181473113\\
8.76	2.35110481940451\\
8.775	2.35251782407789\\
8.79	2.35393082875127\\
8.805	2.35534383342465\\
8.82	2.35675683809803\\
8.835	2.3581698427714\\
8.85	2.35958284744478\\
8.865	2.36099585211816\\
8.88	2.36240885679154\\
8.895	2.36382186146492\\
8.91	2.3652348661383\\
8.925	2.36664787081168\\
8.94	2.36806087548506\\
8.955	2.36947388015844\\
8.97	2.37088688483181\\
8.985	2.37229988950519\\
9	2.37371289417857\\
9.015	2.37512589885195\\
9.03	2.37653890352533\\
9.045	2.37795190819871\\
9.06	2.37936491287209\\
9.075	2.38077791754547\\
9.09	2.38219092221884\\
9.105	2.38360392689222\\
9.12	2.3850169315656\\
9.135	2.38642993623898\\
9.15	2.38784294091236\\
9.165	2.38925594558574\\
9.18	2.39066895025912\\
9.195	2.3920819549325\\
9.21	2.39349495960588\\
9.225	2.39490796427925\\
9.24	2.39632096895263\\
9.255	2.39773397362601\\
9.27	2.39914697829939\\
9.285	2.40055998297277\\
9.3	2.40197298764615\\
9.315	2.40338599231953\\
9.33	2.40479899699291\\
9.345	2.40621200166628\\
9.36	2.40762500633966\\
9.375	2.40903801101304\\
9.39	2.41045101568642\\
9.405	2.4118640203598\\
9.42	2.41327702503318\\
9.435	2.41469002970656\\
9.45	2.41610303437994\\
9.465	2.41751603905331\\
9.48	2.41892904372669\\
9.495	2.42034204840007\\
9.51	2.42175505307345\\
9.525	2.42316805774683\\
9.54	2.42458106242021\\
9.555	2.42599406709359\\
9.57	2.42740707176697\\
9.585	2.42882007644035\\
9.6	2.43023308111372\\
9.615	2.4316460857871\\
9.63	2.43305909046048\\
9.645	2.43447209513386\\
9.66	2.43588509980724\\
9.675	2.43729810448062\\
9.69	2.438711109154\\
9.705	2.44012411382738\\
9.72	2.44153711850075\\
9.735	2.44295012317413\\
9.75	2.44436312784751\\
9.765	2.44577613252089\\
9.78	2.44718913719427\\
9.795	2.44860214186765\\
9.81	2.45001514654103\\
9.825	2.45142815121441\\
9.84	2.45284115588779\\
9.855	2.45425416056116\\
9.87	2.45566716523454\\
9.885	2.45708016990792\\
9.9	2.4584931745813\\
9.915	2.45990617925468\\
9.93	2.46131918392806\\
9.945	2.46273218860144\\
9.96	2.46414519327482\\
9.975	2.46555819794819\\
9.99	2.46697120262157\\
10.005	2.46838420729495\\
10.02	2.46979721196833\\
10.035	2.47121021664171\\
10.05	2.47262322131509\\
10.065	2.47403622598847\\
10.08	2.47544923066185\\
10.095	2.47686223533522\\
10.11	2.4782752400086\\
10.125	2.47968824468198\\
10.14	2.48110124935536\\
10.155	2.48251425402874\\
10.17	2.48392725870212\\
10.185	2.4853402633755\\
10.2	2.48675326804888\\
10.215	2.48816627272226\\
10.23	2.48957927739563\\
10.245	2.49099228206901\\
10.26	2.49240528674239\\
10.275	2.49381829141577\\
10.29	2.49523129608915\\
10.305	2.49664430076253\\
10.32	2.49805730543591\\
10.335	2.49947031010929\\
10.35	2.50088331478266\\
10.365	2.50229631945604\\
10.38	2.50370932412942\\
10.395	2.5051223288028\\
10.41	2.50653533347618\\
10.425	2.50794833814956\\
10.44	2.50936134282294\\
10.455	2.51077434749632\\
10.47	2.51218735216969\\
10.485	2.51360035684307\\
10.5	2.51501336151645\\
10.515	2.51642636618983\\
10.53	2.51783937086321\\
10.545	2.51925237553659\\
10.56	2.52066538020997\\
10.575	2.52207838488335\\
10.59	2.52349138955673\\
10.605	2.5249043942301\\
10.62	2.52631739890348\\
10.635	2.52773040357686\\
10.65	2.52914340825024\\
10.665	2.53055641292362\\
10.68	2.531969417597\\
10.695	2.53338242227038\\
10.71	2.53479542694376\\
10.725	2.53620843161713\\
10.74	2.53762143629051\\
10.755	2.53903444096389\\
10.77	2.54044744563727\\
10.785	2.54186045031065\\
10.8	2.54327345498403\\
10.815	2.54468645965741\\
10.83	2.54609946433079\\
10.845	2.54751246900417\\
10.86	2.54892547367754\\
10.875	2.55033847835092\\
10.89	2.5517514830243\\
10.905	2.55316448769768\\
10.92	2.55457749237106\\
10.935	2.55599049704444\\
10.95	2.55740350171782\\
10.965	2.5588165063912\\
10.98	2.56022951106457\\
10.995	2.56164251573795\\
11.01	2.56305552041133\\
11.025	2.56446852508471\\
11.04	2.56588152975809\\
11.055	2.56729453443147\\
11.07	2.56870753910485\\
11.085	2.57012054377823\\
11.1	2.5715335484516\\
11.115	2.57294655312498\\
11.13	2.57435955779836\\
11.145	2.57577256247174\\
11.16	2.57718556714512\\
11.175	2.5785985718185\\
11.19	2.58001157649188\\
11.205	2.58142458116526\\
11.22	2.58283758583864\\
11.235	2.58425059051201\\
11.25	2.58566359518539\\
11.265	2.58707659985877\\
11.28	2.58848960453215\\
11.295	2.58990260920553\\
11.31	2.59131561387891\\
11.325	2.59272861855229\\
11.34	2.59414162322567\\
11.355	2.59555462789904\\
11.37	2.59696763257242\\
11.385	2.5983806372458\\
11.4	2.59979364191918\\
11.415	2.60120664659256\\
11.43	2.60261965126594\\
11.445	2.60403265593932\\
11.46	2.6054456606127\\
11.475	2.60685866528608\\
11.49	2.60827166995945\\
11.505	2.60968467463283\\
11.52	2.61109767930621\\
11.535	2.61251068397959\\
11.55	2.61392368865297\\
11.565	2.61533669332635\\
11.58	2.61674969799973\\
11.595	2.61816270267311\\
11.61	2.61957570734648\\
11.625	2.62098871201986\\
11.64	2.62240171669324\\
11.655	2.62381472136662\\
11.67	2.62522772604\\
11.685	2.62664073071338\\
11.7	2.62805373538676\\
11.715	2.62946674006014\\
11.73	2.63087974473351\\
11.745	2.63229274940689\\
11.76	2.63370575408027\\
11.775	2.63511875875365\\
11.79	2.63653176342703\\
11.805	2.63794476810041\\
11.82	2.63935777277379\\
11.835	2.64077077744717\\
11.85	2.64218378212054\\
11.865	2.64359678679392\\
11.88	2.6450097914673\\
11.895	2.64642279614068\\
11.91	2.64783580081406\\
11.925	2.64924880548744\\
11.94	2.65066181016082\\
11.955	2.6520748148342\\
11.97	2.65348781950758\\
11.985	2.65490082418095\\
12	2.65631382885433\\
12.015	2.65772683352771\\
12.03	2.65913983820109\\
12.045	2.66055284287447\\
12.06	2.66196584754785\\
12.075	2.66337885222123\\
12.09	2.66479185689461\\
12.105	2.66620486156798\\
12.12	2.66761786624136\\
12.135	2.66903087091474\\
12.15	2.67044387558812\\
12.165	2.6718568802615\\
12.18	2.67326988493488\\
12.195	2.67468288960826\\
12.21	2.67609589428164\\
12.225	2.67750889895502\\
12.24	2.67892190362839\\
12.255	2.68033490830177\\
12.27	2.68174791297515\\
12.285	2.68316091764853\\
12.3	2.68457392232191\\
12.315	2.68598692699529\\
12.33	2.68739993166867\\
12.345	2.68881293634205\\
12.36	2.69022594101542\\
12.375	2.6916389456888\\
12.39	2.69305195036218\\
12.405	2.69446495503556\\
12.42	2.69587795970894\\
12.435	2.69729096438232\\
12.45	2.6987039690557\\
12.465	2.70011697372908\\
12.48	2.70152997840245\\
12.495	2.70294298307583\\
12.51	2.70435598774921\\
12.525	2.70576899242259\\
12.54	2.70718199709597\\
12.555	2.70859500176935\\
12.57	2.71000800644273\\
12.585	2.71142101111611\\
12.6	2.71283401578949\\
12.615	2.71424702046286\\
12.63	2.71566002513624\\
12.645	2.71707302980962\\
12.66	2.718486034483\\
12.675	2.71989903915638\\
12.69	2.72131204382976\\
12.705	2.72272504850314\\
12.72	2.72413805317652\\
12.735	2.72555105784989\\
12.75	2.72696406252327\\
12.765	2.72837706719665\\
12.78	2.72979007187003\\
12.795	2.73120307654341\\
12.81	2.73261608121679\\
12.825	2.73402908589017\\
12.84	2.73544209056355\\
12.855	2.73685509523693\\
12.87	2.7382680999103\\
12.885	2.73968110458368\\
12.9	2.74109410925706\\
12.915	2.74250711393044\\
12.93	2.74392011860382\\
12.945	2.7453331232772\\
12.96	2.74674612795058\\
12.975	2.74815913262396\\
12.99	2.74957213729733\\
13.005	2.75098514197071\\
13.02	2.75239814664409\\
13.035	2.75381115131747\\
13.05	2.75522415599085\\
13.065	2.75663716066423\\
13.08	2.75805016533761\\
13.095	2.75946317001099\\
13.11	2.76087617468436\\
13.125	2.76228917935774\\
13.14	2.76370218403112\\
13.155	2.7651151887045\\
13.17	2.76652819337788\\
13.185	2.76794119805126\\
13.2	2.76935420272464\\
13.215	2.77076720739802\\
13.23	2.7721802120714\\
13.245	2.77359321674477\\
13.26	2.77500622141815\\
13.275	2.77641922609153\\
13.29	2.77783223076491\\
13.305	2.77924523543829\\
13.32	2.78065824011167\\
13.335	2.78207124478505\\
13.35	2.78348424945843\\
13.365	2.7848972541318\\
13.38	2.78631025880518\\
13.395	2.78772326347856\\
13.41	2.78913626815194\\
13.425	2.79054927282532\\
13.44	2.7919622774987\\
13.455	2.79337528217208\\
13.47	2.79478828684546\\
13.485	2.79620129151883\\
13.5	2.79761429619221\\
13.515	2.79902730086559\\
13.53	2.80044030553897\\
13.545	2.80185331021235\\
13.56	2.80326631488573\\
13.575	2.80467931955911\\
13.59	2.80609232423249\\
13.605	2.80750532890587\\
13.62	2.80891833357924\\
13.635	2.81033133825262\\
13.65	2.811744342926\\
13.665	2.81315734759938\\
13.68	2.81457035227276\\
13.695	2.81598335694614\\
13.71	2.81739636161952\\
13.725	2.8188093662929\\
13.74	2.82022237096627\\
13.755	2.82163537563965\\
13.77	2.82304838031303\\
13.785	2.82446138498641\\
13.8	2.82587438965979\\
13.815	2.82728739433317\\
13.83	2.82870039900655\\
13.845	2.83011340367993\\
13.86	2.83152640835331\\
13.875	2.83293941302668\\
13.89	2.83435241770006\\
13.905	2.83576542237344\\
13.92	2.83717842704682\\
13.935	2.8385914317202\\
13.95	2.84000443639358\\
13.965	2.84141744106696\\
13.98	2.84283044574034\\
13.995	2.84424345041371\\
14.01	2.84565645508709\\
14.025	2.84706945976047\\
14.04	2.84848246443385\\
14.055	2.84989546910723\\
14.07	2.85130847378061\\
14.085	2.85272147845399\\
14.1	2.85413448312737\\
14.115	2.85554748780075\\
14.13	2.85696049247412\\
14.145	2.8583734971475\\
14.16	2.85978650182088\\
14.175	2.86119950649426\\
14.19	2.86261251116764\\
14.205	2.86402551584102\\
14.22	2.8654385205144\\
14.235	2.86685152518778\\
14.25	2.86826452986115\\
14.265	2.86967753453453\\
14.28	2.87109053920791\\
14.295	2.87250354388129\\
14.31	2.87391654855467\\
14.325	2.87532955322805\\
14.34	2.87674255790143\\
14.355	2.87815556257481\\
14.37	2.87956856724818\\
14.385	2.88098157192156\\
14.4	2.88239457659494\\
14.415	2.88380758126832\\
14.43	2.8852205859417\\
14.445	2.88663359061508\\
14.46	2.88804659528846\\
14.475	2.88945959996184\\
14.49	2.89087260463522\\
14.505	2.89228560930859\\
14.52	2.89369861398197\\
14.535	2.89511161865535\\
14.55	2.89652462332873\\
14.565	2.89793762800211\\
14.58	2.89935063267549\\
14.595	2.90076363734887\\
14.61	2.90217664202225\\
14.625	2.90358964669562\\
14.64	2.905002651369\\
14.655	2.90641565604238\\
14.67	2.90782866071576\\
14.685	2.90924166538914\\
14.7	2.91065467006252\\
14.715	2.9120676747359\\
14.73	2.91348067940928\\
14.745	2.91489368408265\\
14.76	2.91630668875603\\
14.775	2.91771969342941\\
14.79	2.91913269810279\\
14.805	2.92054570277617\\
14.82	2.92195870744955\\
14.835	2.92337171212293\\
14.85	2.92478471679631\\
14.865	2.92619772146969\\
14.88	2.92761072614306\\
14.895	2.92902373081644\\
14.91	2.93043673548982\\
14.925	2.9318497401632\\
14.94	2.93326274483658\\
14.955	2.93467574950996\\
14.97	2.93608875418334\\
14.985	2.93750175885672\\
15	2.93891476353009\\
};
\addlegendentry{Fitted line};
\end{axis}
\end{tikzpicture}%}
\caption{Execution time for the initialization for different amount of devices and the baseline. The fitted line is a linear approximation.}
\label{fig:MT_Init_Time}
\end{figure}

From \autoref{fig:MT_Init_Time} it can be seen that there is a linear tendency scaling with the number of devices. It is also seen that even if the baseline and the changed code emulates one device, the baseline have a lower execution time, with a estimated difference to be the same as a single step between different number of devices. The fitted line is estimated to be:

\begin{align}
&T_{init} (\text{NoD}) = 0.0942 \cdot \text{NoD} + 1.526 [s]
\end{align}

\section{Synchronization}
The execution time for the synchronization step is measured dependably on the number of devices emulated and the baseline is measured as well, which gives the results seen in \autoref{fig:MT_Sync_Time}. As the error type cell sync, mention in \autoref{sec:MTerror}, occurs in this step of the process, some measurements will be equal zero, as the execution time can not be calculated.

\begin{minipage}{0.48\textwidth}
\begin{figure}[H]
\tikzsetnextfilename{MT_Sync_Time}
\centering
\resizebox{0.9\textwidth}{!}{
% This file was created by matlab2tikz.
%
%The latest updates can be retrieved from
%  http://www.mathworks.com/matlabcentral/fileexchange/22022-matlab2tikz-matlab2tikz
%where you can also make suggestions and rate matlab2tikz.
%
\definecolor{mycolor1}{rgb}{0.00000,0.44700,0.74100}%
\definecolor{mycolor2}{rgb}{0.85000,0.32500,0.09800}%
%
\begin{tikzpicture}

\begin{axis}[%
width=\textwidth,
height=.66\textwidth,
at={(0.758in,0.481in)},
scale only axis,
xmin=-0.5,
xmax=15.5,
xtick={0,1,2,3,4,5,6,7,8,9,10,11,12,13,14,15},
xticklabels={{Base},{1},{2},{3},{4},{5},{6},{7},{8},{9},{10},{11},{12},{13},{14},{15}},
xlabel={Number of devices},
ymin=0,
ymax=1.8,
ylabel={Time [s]},
axis background/.style={fill=white},
title style={font=\bfseries},
title={Syncronation time},
legend style={at={(0.03,0.97)},anchor=north west,legend cell align=left,align=left,draw=white!15!black}
]
\addplot [color=mycolor1,only marks,mark=*,mark options={solid}]
  table[row sep=crcr]{%
1	0.650609970092773\\
1	0.653857946395874\\
1	0.640816926956177\\
1	0.642185926437378\\
1	0.645545959472656\\
1	1.06731581687927\\
1	0.649732112884521\\
1	0.653727054595947\\
1	1.06988906860352\\
1	0.648482799530029\\
1	0.650296926498413\\
1	0.653731107711792\\
1	0.650707960128784\\
1	0.656582117080688\\
1	0.651366949081421\\
1	0.653976202011108\\
1	0.653213024139404\\
1	0.650224924087524\\
1	0.648483037948608\\
1	0.654691934585571\\
};
\addlegendentry{Change code};

\addplot [color=mycolor1,only marks,mark=*,mark options={solid},forget plot]
  table[row sep=crcr]{%
2	0.646949052810669\\
2	0.649175882339478\\
2	0.650078058242798\\
2	0.658921957015991\\
2	0.644191026687622\\
2	1.06752800941467\\
2	0.652630090713501\\
2	0.649014949798584\\
2	0.650974988937378\\
2	0.660398006439209\\
2	0.641355037689209\\
2	0.652486085891724\\
2	0.646153926849365\\
2	0.645971059799194\\
2	0.647706985473633\\
2	0.652987003326416\\
2	0.651640892028809\\
2	0.64258599281311\\
2	0.644962072372437\\
2	0.649473905563354\\
};


\addplot [color=mycolor1,only marks,mark=*,mark options={solid},forget plot]
  table[row sep=crcr]{%
3	0.643748044967651\\
3	0.644491195678711\\
3	0.642645835876465\\
3	0.656013965606689\\
3	0.647945880889893\\
3	0.646684885025024\\
3	0.641174077987671\\
3	0.647907018661499\\
3	0.644546985626221\\
3	0.645722150802612\\
3	0.64906907081604\\
3	0.648266077041626\\
3	0.642682075500488\\
3	0.652300119400024\\
3	0.641947031021118\\
3	0.676903009414673\\
3	0.648000955581665\\
3	0.655172109603882\\
3	0.648041009902954\\
3	0.656927108764648\\
};
\addplot [color=mycolor1,only marks,mark=*,mark options={solid},forget plot]
  table[row sep=crcr]{%
4	0.648493051528931\\
4	0.64559006690979\\
4	0.654642105102539\\
4	0.6754150390625\\
4	0.646654844284058\\
4	0.642754077911377\\
4	0.639609098434448\\
4	0.653938055038452\\
4	0.650071144104004\\
4	0.656721115112305\\
4	0.643237829208374\\
4	0.651222229003906\\
4	0.652673006057739\\
4	0.65081000328064\\
4	0.672778844833374\\
4	0.651262998580933\\
4	0.649074077606201\\
4	0.647204875946045\\
4	0.646042108535767\\
4	0.64453911781311\\
};
\addplot [color=mycolor1,only marks,mark=*,mark options={solid},forget plot]
  table[row sep=crcr]{%
5	0.650202989578247\\
5	0.648781776428223\\
5	0.640866994857788\\
5	0.649247169494629\\
5	0.65114688873291\\
5	0.643033027648926\\
5	0.644155025482178\\
5	0.657757997512817\\
5	0.650070905685425\\
5	0.65401291847229\\
5	0.646816968917847\\
5	0.64539098739624\\
5	0.654246091842651\\
5	0.655587911605835\\
5	0.645053863525391\\
5	0.64438796043396\\
5	0.648882150650024\\
5	0.652736902236938\\
5	0.643465995788574\\
5	0.653812170028687\\
};
\addplot [color=mycolor1,only marks,mark=*,mark options={solid},forget plot]
  table[row sep=crcr]{%
6	0.646232128143311\\
6	0.648225069046021\\
6	1.40721011161804\\
6	0.651366949081421\\
6	0.652253866195679\\
6	0.650762796401978\\
6	0.648574829101563\\
6	0.640298843383789\\
6	0.653538942337036\\
6	0.645024061203003\\
6	0.661158084869385\\
6	0.647090911865234\\
6	0.642549991607666\\
6	0.642514944076538\\
6	0.643571853637695\\
6	0.65564489364624\\
6	1.07363796234131\\
6	0.647010087966919\\
6	0.646505117416382\\
6	0.6438148021698\\
};
\addplot [color=mycolor1,only marks,mark=*,mark options={solid},forget plot]
  table[row sep=crcr]{%
7	1.07072401046753\\
7	0.652054071426392\\
7	1.07058906555176\\
7	0.648339986801147\\
7	0.652436017990112\\
7	0.642619132995605\\
7	0.643392086029053\\
7	0.648621082305908\\
7	0.645454883575439\\
7	0.652534961700439\\
7	0.64702582359314\\
7	0.644365072250366\\
7	0.659003973007202\\
7	0.652891874313354\\
7	0.646991968154907\\
7	0.655962944030762\\
7	0.651911020278931\\
7	0.649963140487671\\
7	0.650946855545044\\
7	0.650607109069824\\
};
\addplot [color=mycolor1,only marks,mark=*,mark options={solid},forget plot]
  table[row sep=crcr]{%
8	0.649052143096924\\
8	0.655632972717285\\
8	0.653681039810181\\
8	0.651021003723145\\
8	0.645121097564697\\
8	0.651643991470337\\
8	0.644710063934326\\
8	0.643614053726196\\
8	0.641947984695435\\
8	0.648487091064453\\
8	0.652446985244751\\
8	0.648170948028564\\
8	0.646777153015137\\
8	0.644640922546387\\
8	0.652256011962891\\
8	0.646914958953857\\
8	0.647845983505249\\
8	0.647209167480469\\
8	0.65062403678894\\
8	0\\
};
\addplot [color=mycolor1,only marks,mark=*,mark options={solid},forget plot]
  table[row sep=crcr]{%
9	0.650902032852173\\
9	0.653230905532837\\
9	0.651026964187622\\
9	1.37894892692566\\
9	0.658339023590088\\
9	0.6616530418396\\
9	0.650697946548462\\
9	0.65686297416687\\
9	0.646430015563965\\
9	0.652285814285278\\
9	0.657356977462769\\
9	0.648330926895142\\
9	0.652124166488647\\
9	0.650056838989258\\
9	0.649123907089233\\
9	1.69176697731018\\
9	0.648163080215454\\
9	0.653515100479126\\
9	0.648416042327881\\
9	0.652014017105103\\
};
\addplot [color=mycolor1,only marks,mark=*,mark options={solid},forget plot]
  table[row sep=crcr]{%
10	0.652431964874268\\
10	0.653368949890137\\
10	0.653976917266846\\
10	0.647805213928223\\
10	0.651004791259766\\
10	0.657984972000122\\
10	0.652050971984863\\
10	0.653388023376465\\
10	0.645644187927246\\
10	0.653051853179932\\
10	0.655381917953491\\
10	0.653103113174438\\
10	0.649111986160278\\
10	0.650975942611694\\
10	0.646430015563965\\
10	0.657355070114136\\
10	0.655450105667114\\
10	0.647177934646606\\
10	0.656944990158081\\
10	0.651263952255249\\
};
\addplot [color=mycolor1,only marks,mark=*,mark options={solid},forget plot]
  table[row sep=crcr]{%
11	0.651514053344727\\
11	0.65205192565918\\
11	0.658270835876465\\
11	0.649635076522827\\
11	1.07549595832825\\
11	0.654492855072021\\
11	0.653318881988525\\
11	0.65693998336792\\
11	0.65533185005188\\
11	1.07469415664673\\
11	0.647542953491211\\
11	0.678914070129395\\
11	0.654986143112183\\
11	0.646422863006592\\
11	0.650253057479858\\
11	0.652107954025269\\
11	0.655264139175415\\
11	0.651924133300781\\
11	0.645050048828125\\
11	0.648700952529907\\
};
\addplot [color=mycolor1,only marks,mark=*,mark options={solid},forget plot]
  table[row sep=crcr]{%
12	0.65892505645752\\
12	0.650825977325439\\
12	0.649593114852905\\
12	0.650028944015503\\
12	0.647899150848389\\
12	0.658643007278442\\
12	0.649430990219116\\
12	0.666985988616943\\
12	0.659577131271362\\
12	0.653954982757568\\
12	0.65661096572876\\
12	0.647140026092529\\
12	1.07442212104797\\
12	0.653070211410522\\
12	0.676941156387329\\
12	0.655467987060547\\
12	0.646998167037964\\
12	0.653201103210449\\
12	0.649198055267334\\
12	0.652733087539673\\
};
\addplot [color=mycolor1,only marks,mark=*,mark options={solid},forget plot]
  table[row sep=crcr]{%
13	0.647382974624634\\
13	0.653558015823364\\
13	0.647633075714111\\
13	0.650227069854736\\
13	0.655040979385376\\
13	0.644393920898438\\
13	0.656842947006226\\
13	0.650371074676514\\
13	1.69415903091431\\
13	0.651180028915405\\
13	0.649358034133911\\
13	0.650844097137451\\
13	0.659465074539185\\
13	0.648830890655518\\
13	0.65994119644165\\
13	0.655570030212402\\
13	0.661641120910645\\
13	0.654593944549561\\
13	0.656475782394409\\
13	0\\
};
\addplot [color=mycolor1,only marks,mark=*,mark options={solid},forget plot]
  table[row sep=crcr]{%
14	0.643074035644531\\
14	0.657819986343384\\
14	0.644295215606689\\
14	0.660433053970337\\
14	0.653297185897827\\
14	0.658216953277588\\
14	0.653433084487915\\
14	0.649707078933716\\
14	0.653749942779541\\
14	0.649873971939087\\
14	0.647374868392944\\
14	0.680761098861694\\
14	0.653578996658325\\
14	0.646941900253296\\
14	0.657058954238892\\
14	0.654543876647949\\
14	0.644098997116089\\
14	0.649263143539429\\
14	0.653286933898926\\
14	1.39285397529602\\
};
\addplot [color=mycolor1,only marks,mark=*,mark options={solid},forget plot]
  table[row sep=crcr]{%
15	0.654828071594238\\
15	0.648122787475586\\
15	0.656240224838257\\
15	0.646528005599976\\
15	0.648042917251587\\
15	0.657634973526001\\
15	0.656691074371338\\
15	0.654543161392212\\
15	0.665755033493042\\
15	0.655173063278198\\
15	0.675770044326782\\
15	0.679379940032959\\
15	0.654716014862061\\
15	0.655852794647217\\
15	0.657128095626831\\
15	0.646656036376953\\
15	0.647743940353394\\
15	0.651691913604736\\
15	0.657409906387329\\
15	0.656056880950928\\
};
\addplot [color=mycolor2,only marks,mark=*,mark options={solid}]
  table[row sep=crcr]
\caption{Execution time for the synchronization for different amount of devices and the baseline}
\label{fig:MT_Sync_Time}
\end{figure}
\end{minipage}%
\hfill
\begin{minipage}{0.48\textwidth}
\begin{figure}[H]
\tikzsetnextfilename{MT_Sync_His}
\centering
\resizebox{0.9\textwidth}{!}{
% This file was created by matlab2tikz.
%
%The latest updates can be retrieved from
%  http://www.mathworks.com/matlabcentral/fileexchange/22022-matlab2tikz-matlab2tikz
%where you can also make suggestions and rate matlab2tikz.
%
\definecolor{mycolor1}{rgb}{0.00000,0.44700,0.74100}%
%
\begin{tikzpicture}

\begin{axis}[%
width=\textwidth,
height=.66\textwidth,
at={(0.758in,0.481in)},
scale only axis,
xmin=0.6,
xmax=1.8,
xlabel={Number of devices},
ymin=0,
ymax=250,
ylabel={Trials},
axis background/.style={fill=white},
title style={font=\bfseries},
title={Histogram for Sync}
]
\addplot[fill=mycolor1,fill opacity=0.6,draw=black,ybar interval,area legend] plot table[row sep=crcr] 
\caption{The distribution for the execution time for synchronization for all different amount of devices}
\label{fig:MT_Sync_His}
\end{figure}
\end{minipage}

As seen in \autoref{fig:MT_Sync_Time} is the execution time for different amount of devices behaving equal to each other. Another aspect seen on the figure is that some measurements have taken some extra time to execute, but is align at the same time values, which also indicated on the histogram in \autoref{fig:MT_Sync_His}. Here it is seen that most measurements is placed at 0.6 s to 0.8 s and the amount at the other points is much lower.



\section{MIB decoding}
The execution time for the MIB decoding step is measured dependably on the number of devices emulated and the baseline is measured as well, which gives the results seen in \autoref{fig:MT_MIB_Time}. All these measurements is for a full decoding of MIB and all retries have been removed.
As all earlier occurred errors will infect that there is no measurement point for further steps in the process, the cell sync errors from the synchronization step will still infect the given measurements here and further on.

\begin{minipage}{0.48\textwidth}
\begin{figure}[H]
\tikzsetnextfilename{MT_MIB_Time}
\centering
\resizebox{0.9\textwidth}{!}{
% This file was created by matlab2tikz.
%
%The latest updates can be retrieved from
%  http://www.mathworks.com/matlabcentral/fileexchange/22022-matlab2tikz-matlab2tikz
%where you can also make suggestions and rate matlab2tikz.
%
\definecolor{mycolor1}{rgb}{0.00000,0.44700,0.74100}%
\definecolor{mycolor2}{rgb}{0.85000,0.32500,0.09800}%
%
\begin{tikzpicture}

\begin{axis}[%
width=\textwidth,
height=.66\textwidth,
at={(0.758in,0.481in)},
scale only axis,
xmin=-0.5,
xmax=15.5,
xtick={0,1,2,3,4,5,6,7,8,9,10,11,12,13,14,15},
xticklabels={{Baseline},{1},{2},{3},{4},{5},{6},{7},{8},{9},{10},{11},{12},{13},{14},{15}},
xlabel={Number of devices},
ymin=0,
ymax=2,
ylabel={Time (s)},
axis background/.style={fill=white},
title style={font=\bfseries},
title={MIB decoding time},
legend style={at={(0.03,0.97)},anchor=north west,legend cell align=left,align=left,draw=white!15!black}
]
\addplot [color=mycolor1,only marks,mark=*,mark options={solid}]
  table[row sep=crcr]{%
1	1.18452787399292\\
1	1.02464199066162\\
1	1.16689705848694\\
1	0.904258966445923\\
1	0.843533992767334\\
1	1.13420486450195\\
1	0.824519872665405\\
1	0.961822032928467\\
1	0.891458988189697\\
1	1.3114321231842\\
1	0.72556209564209\\
1	1.26286196708679\\
1	1.25451898574829\\
1	1.12048602104187\\
1	1.02167701721191\\
1	1.09917998313904\\
1	0.873126983642578\\
1	1.14201807975769\\
1	1.21537899971008\\
1	1.05097317695618\\
};
\addlegendentry{Change code};

\addplot [color=mycolor1,only marks,mark=*,mark options={solid},forget plot]
  table[row sep=crcr]{%
2	0.742633819580078\\
2	1.12349700927734\\
2	0.862571001052856\\
2	1.09928011894226\\
2	1.06075310707092\\
2	0.841415166854858\\
2	0.824921131134033\\
2	1.05261611938477\\
2	1.0511691570282\\
2	1.34838080406189\\
2	1.16666889190674\\
2	0.892491817474365\\
2	0.863969087600708\\
2	0.991780996322632\\
2	1.16264510154724\\
2	1.15249800682068\\
2	1.24425196647644\\
2	0.853885173797607\\
2	1.44308090209961\\
2	0.932136058807373\\
};


\addplot [color=mycolor1,only marks,mark=*,mark options={solid},forget plot]
  table[row sep=crcr]{%
3	1.35272288322449\\
3	0.896404027938843\\
3	1.19529414176941\\
3	0.721312999725342\\
3	1.00501799583435\\
3	0.974997997283936\\
3	1.27371096611023\\
3	1.09565401077271\\
3	1.19446611404419\\
3	0.736949920654297\\
3	0.784852981567383\\
3	0.874388933181763\\
3	0.932243824005127\\
3	0.821744918823242\\
3	1.09404897689819\\
3	1.14502000808716\\
3	0.893469095230103\\
3	1.19056606292725\\
3	1.06648683547974\\
3	1.04886889457703\\
};
\addplot [color=mycolor1,only marks,mark=*,mark options={solid},forget plot]
  table[row sep=crcr]{%
4	1.21173596382141\\
4	0.894090890884399\\
4	0.950580835342407\\
4	1.31642508506775\\
4	0.83543586730957\\
4	0.925927877426147\\
4	0.976491928100586\\
4	1.28235483169556\\
4	0.803546905517578\\
4	0.790771007537842\\
4	1.27620601654053\\
4	1.01348495483398\\
4	1.07200884819031\\
4	1.01012992858887\\
4	0.839915037155151\\
4	0.962180137634277\\
4	1.07453393936157\\
4	1.29357409477234\\
4	1.40396094322205\\
4	1.24524092674255\\
};
\addplot [color=mycolor1,only marks,mark=*,mark options={solid},forget plot]
  table[row sep=crcr]{%
5	1.29511213302612\\
5	1.01446914672852\\
5	1.20752096176147\\
5	1.20313382148743\\
5	0.805004119873047\\
5	1.30468487739563\\
5	0.815110921859741\\
5	0.758973121643066\\
5	1.11016798019409\\
5	0.761851072311401\\
5	0.955629825592041\\
5	1.00391411781311\\
5	1.32148098945618\\
5	1.17293000221252\\
5	0.923401832580566\\
5	0.895086050033569\\
5	1.20391082763672\\
5	0.76114296913147\\
5	0.784970045089722\\
5	1.32375192642212\\
};
\addplot [color=mycolor1,only marks,mark=*,mark options={solid},forget plot]
  table[row sep=crcr]{%
6	1.12626886367798\\
6	0.784417867660522\\
6	0.973766088485718\\
6	1.34400105476379\\
6	1.03182911872864\\
6	1.00979208946228\\
6	1.18366503715515\\
6	0.977426052093506\\
6	0.831204891204834\\
6	1.34424090385437\\
6	1.26865386962891\\
6	0.815386056900024\\
6	0.72773003578186\\
6	1.09482884407043\\
6	0.776089191436768\\
6	1.34079313278198\\
6	0.740194082260132\\
6	0.825405120849609\\
6	1.03018093109131\\
6	1.05475306510925\\
};
\addplot [color=mycolor1,only marks,mark=*,mark options={solid},forget plot]
  table[row sep=crcr]{%
7	1.14058113098145\\
7	0.969480037689209\\
7	1.2895450592041\\
7	1.25373196601868\\
7	0.869355916976929\\
7	0.836460828781128\\
7	0.883985042572021\\
7	1.33135199546814\\
7	1.10523915290833\\
7	1.06331515312195\\
7	0.834944009780884\\
7	0.796628952026367\\
7	1.09889316558838\\
7	1.26145696640015\\
7	1.31127190589905\\
7	0.929186105728149\\
7	1.09361791610718\\
7	0.924177885055542\\
7	1.28277397155762\\
7	1.24324703216553\\
};
\addplot [color=mycolor1,only marks,mark=*,mark options={solid},forget plot]
  table[row sep=crcr]{%
8	1.21215677261353\\
8	0.811891078948975\\
8	1.35208201408386\\
8	1.25351810455322\\
8	1.24444103240967\\
8	0.96881103515625\\
8	1.32427191734314\\
8	1.24557089805603\\
8	0.985246181488037\\
8	1.10344481468201\\
8	0.947085857391357\\
8	1.03846478462219\\
8	0.940639019012451\\
8	1.27187204360962\\
8	0.932521820068359\\
8	1.20112204551697\\
8	0.952279090881348\\
8	0.981828927993774\\
8	1.12882900238037\\
8	0\\
};
\addplot [color=mycolor1,only marks,mark=*,mark options={solid},forget plot]
  table[row sep=crcr]{%
9	1.07991504669189\\
9	1.19017219543457\\
9	1.26013112068176\\
9	1.21215891838074\\
9	0.786477088928223\\
9	1.14683389663696\\
9	0.953114032745361\\
9	0.919471025466919\\
9	1.3333580493927\\
9	1.07340693473816\\
9	0.95987606048584\\
9	0.84372091293335\\
9	0.995597839355469\\
9	1.08392119407654\\
9	0.971073150634766\\
9	1.1183009147644\\
9	1.08447599411011\\
9	0.850917100906372\\
9	0.973115921020508\\
9	1.12152695655823\\
};
\addplot [color=mycolor1,only marks,mark=*,mark options={solid},forget plot]
  table[row sep=crcr]{%
10	1.25671696662903\\
10	1.05311989784241\\
10	0.929455041885376\\
10	1.19342279434204\\
10	1.05944609642029\\
10	1.01689720153809\\
10	0.967733860015869\\
10	1.21085119247437\\
10	1.22501087188721\\
10	1.08909916877747\\
10	0.941591024398804\\
10	1.1600980758667\\
10	1.24331402778625\\
10	0.889430046081543\\
10	1.32221412658691\\
10	1.01067590713501\\
10	0.999495983123779\\
10	1.3286120891571\\
10	1.03497505187988\\
10	0.809139966964722\\
};
\addplot [color=mycolor1,only marks,mark=*,mark options={solid},forget plot]
  table[row sep=crcr]{%
11	0.941027879714966\\
11	1.35822701454163\\
11	0.977467060089111\\
11	1.29018783569336\\
11	1.10762286186218\\
11	1.12313294410706\\
11	1.08049416542053\\
11	1.08900213241577\\
11	1.15175604820251\\
11	1.06713485717773\\
11	1.1618390083313\\
11	1.21526479721069\\
11	1.33760190010071\\
11	0.884803056716919\\
11	1.23175096511841\\
11	0.910167932510376\\
11	1.27922487258911\\
11	1.03915095329285\\
11	1.28432893753052\\
11	1.26249885559082\\
};
\addplot [color=mycolor1,only marks,mark=*,mark options={solid},forget plot]
  table[row sep=crcr]{%
12	0.858820915222168\\
12	1.21957588195801\\
12	0.880445003509521\\
12	1.05029010772705\\
12	1.23261284828186\\
12	1.30926299095154\\
12	0.879901170730591\\
12	1.27121210098267\\
12	1.13944888114929\\
12	0.958610057830811\\
12	1.08663201332092\\
12	1.01077389717102\\
12	1.08854699134827\\
12	1.2690908908844\\
12	1.22531986236572\\
12	1.09908199310303\\
12	1.27269697189331\\
12	1.25594997406006\\
12	1.10074591636658\\
12	0.738483905792236\\
};
\addplot [color=mycolor1,only marks,mark=*,mark options={solid},forget plot]
  table[row sep=crcr]{%
13	0.83938193321228\\
13	0.998347997665405\\
13	1.32503414154053\\
13	0.880541086196899\\
13	1.16189002990723\\
13	0.742758989334106\\
13	0.765993118286133\\
13	1.25955986976624\\
13	1.0792031288147\\
13	0.740605115890503\\
13	1.33235692977905\\
13	1.11989402770996\\
13	1.13771796226501\\
13	1.11200213432312\\
13	1.1292519569397\\
13	0.961405992507935\\
13	1.27758502960205\\
13	1.24095487594604\\
13	1.33164381980896\\
13	0\\
};
\addplot [color=mycolor1,only marks,mark=*,mark options={solid},forget plot]
  table[row sep=crcr]{%
14	0.835340976715088\\
14	1.21743702888489\\
14	1.1116509437561\\
14	0.764328956604004\\
14	1.3510730266571\\
14	1.34011006355286\\
14	0.861910104751587\\
14	1.09282898902893\\
14	1.14283895492554\\
14	1.34170484542847\\
14	0.792457103729248\\
14	0.864109992980957\\
14	0.750169038772583\\
14	1.22165703773499\\
14	0.819968938827515\\
14	0.823435068130493\\
14	0.993892908096313\\
14	0.913672924041748\\
14	0.91166090965271\\
14	0.859905958175659\\
};
\addplot [color=mycolor1,only marks,mark=*,mark options={solid},forget plot]
  table[row sep=crcr]{%
15	0.958617925643921\\
15	1.01214718818665\\
15	1.23916387557983\\
15	0.954987049102783\\
15	1.02038717269897\\
15	0.949368000030518\\
15	1.08639097213745\\
15	1.20632886886597\\
15	1.28288388252258\\
15	1.04977297782898\\
15	0.788140058517456\\
15	0.933758020401001\\
15	1.16064596176147\\
15	0.740518093109131\\
15	1.11970591545105\\
15	1.19137382507324\\
15	0.931004047393799\\
15	0.758590936660767\\
15	1.19861912727356\\
15	1.03083515167236\\
};
\addplot [color=mycolor2,only marks,mark=*,mark options={solid}]
  table[row sep=crcr]
\caption{Execution time for the decoding the MIB for different amount of devices and the baseline. A single measurement for the base line is placed at 5.0834 s, which is not shown on this figure.}
\label{fig:MT_MIB_Time}
\end{figure}
\end{minipage}%
\hfill
\begin{minipage}{0.48\textwidth}
\begin{figure}[H]
\tikzsetnextfilename{MT_MIB_His}
\centering
\resizebox{0.9\textwidth}{!}{
% This file was created by matlab2tikz.
%
%The latest updates can be retrieved from
%  http://www.mathworks.com/matlabcentral/fileexchange/22022-matlab2tikz-matlab2tikz
%where you can also make suggestions and rate matlab2tikz.
%
\definecolor{mycolor1}{rgb}{0.00000,0.44700,0.74100}%
%
\begin{tikzpicture}

\begin{axis}[%
width=\textwidth,
height=.66\textwidth,
at={(0.758in,0.481in)},
scale only axis,
xmin=0.7,
xmax=1.5,
xlabel={Time [s]},
ymin=0,
ymax=7,
ylabel={Counts},
axis background/.style={fill=white},
title style={font=\bfseries},
title={Histogram for Mib}
]
\addplot[fill=mycolor1,fill opacity=0.6,draw=black,ybar interval,area legend] plot table[row sep=crcr] {%
x	y\\
0.721	3\\
0.72823	0\\
0.73546	6\\
0.74269	1\\
0.74992	1\\
0.75715	5\\
0.76438	1\\
0.77161	1\\
0.77884	3\\
0.78607	4\\
0.7933	1\\
0.80053	2\\
0.80776	2\\
0.81499	4\\
0.82222	4\\
0.82945	5\\
0.83668	5\\
0.84391	1\\
0.85114	1\\
0.85837	6\\
0.8656	1\\
0.87283	3\\
0.88006	4\\
0.88729	5\\
0.89452	2\\
0.90175	1\\
0.90898	3\\
0.91621	2\\
0.92344	4\\
0.93067	5\\
0.9379	3\\
0.94513	4\\
0.95236	5\\
0.95959	4\\
0.96682	6\\
0.97405	4\\
0.98128	2\\
0.98851	3\\
0.99574	2\\
1.00297	4\\
1.0102	6\\
1.01743	3\\
1.02466	3\\
1.03189	2\\
1.03912	1\\
1.04635	7\\
1.05358	3\\
1.06081	3\\
1.06804	3\\
1.07527	3\\
1.0825	7\\
1.08973	5\\
1.09696	6\\
1.10419	3\\
1.11142	3\\
1.11865	6\\
1.12588	3\\
1.13311	3\\
1.14034	5\\
1.14757	2\\
1.1548	4\\
1.16203	3\\
1.16926	1\\
1.17649	1\\
1.18372	3\\
1.19095	4\\
1.19818	4\\
1.20541	6\\
1.21264	4\\
1.21987	3\\
1.2271	2\\
1.23433	2\\
1.24156	6\\
1.24879	4\\
1.25602	6\\
1.26325	2\\
1.27048	6\\
1.27771	5\\
1.28494	2\\
1.29217	2\\
1.2994	1\\
1.30663	3\\
1.31386	1\\
1.32109	5\\
1.32832	5\\
1.33555	4\\
1.34278	3\\
1.35001	3\\
1.35724	1\\
1.36447	0\\
1.3717	0\\
1.37893	0\\
1.38616	0\\
1.39339	0\\
1.40062	1\\
1.40785	0\\
1.41508	0\\
1.42231	0\\
1.42954	0\\
1.43677	1\\
1.444	1\\
};
\end{axis}
\end{tikzpicture}%}
\caption{The distribution for the execution time for decoding the MIB for all different amount of devices.}
\label{fig:MT_MIB_His}
\end{figure}
\end{minipage}

As seen in \autoref{fig:MT_MIB_Time} do the execution time for the MIB decoding step have the same tendency across different amount of devices. The spread is bigger compared to the synchronazitation step, which also can be seen when comparing the histogram for the two steps, \autoref{fig:MT_MIB_His} and \autoref{fig:MT_Sync_His}. The baseline have the same tendency as the changed code. Mention before are these execution times only for a full decoding. In \autoref{fig:MT_MIB_Tries} is it shown how many tries the different measurements needed before completing the MIB decoding step.

\begin{figure}[H]
\tikzsetnextfilename{MT_MIB_Tries}
\centering
\resizebox{0.5\textwidth}{!}{
% This file was created by matlab2tikz.
%
%The latest updates can be retrieved from
%  http://www.mathworks.com/matlabcentral/fileexchange/22022-matlab2tikz-matlab2tikz
%where you can also make suggestions and rate matlab2tikz.
%
\definecolor{mycolor1}{rgb}{0.20810,0.16630,0.52920}%
\definecolor{mycolor2}{rgb}{0.02650,0.61370,0.81350}%
\definecolor{mycolor3}{rgb}{0.64730,0.74560,0.41880}%
\definecolor{mycolor4}{rgb}{0.97630,0.98310,0.05380}%
%
\begin{tikzpicture}

\begin{axis}[%
width=\textwidth,
height=.66\textwidth,
at={(0.607in,0.481in)},
scale only axis,
bar width=0.5,
xmin=-0.5,
xmax=15.5,
xtick={0,1,2,3,4,5,6,7,8,9,10,11,12,13,14,15},
xticklabels={{Baseline},{1},{2},{3},{4},{5},{6},{7},{8},{9},{10},{11},{12},{13},{14},{15}},
xlabel={Number of devices},
ymin=0,
ymax=1,
ylabel={Trials},
axis background/.style={fill=white},
title style={font=\bfseries},
title={Number of tries for decoding MIB},
legend style={at={(1.03,1)},anchor=north west,legend cell align=left,align=left,draw=white!15!black}
]
\addplot[ybar stacked,draw=black,fill=mycolor1,area legend] plot table[row sep=crcr] {%
0	0.65\\
1	0.8\\
2	0.85\\
3	0.75\\
4	0.9\\
5	0.65\\
6	0.85\\
7	0.85\\
8	0.736842105263158\\
9	0.9\\
10	0.95\\
11	0.95\\
12	1\\
13	0.842105263157895\\
14	0.95\\
15	0.95\\
};
\addlegendentry{1 Attempts};

\addplot[ybar stacked,draw=black,fill=mycolor2,area legend] plot table[row sep=crcr] {%
0	0.25\\
1	0.15\\
2	0.05\\
3	0.2\\
4	0.1\\
5	0.25\\
6	0\\
7	0.15\\
8	0.263157894736842\\
9	0.1\\
10	0.05\\
11	0.05\\
12	0\\
13	0.157894736842105\\
14	0.05\\
15	0.05\\
};
\addlegendentry{2 Attempts};

\addplot[ybar stacked,draw=black,fill=mycolor3,area legend] plot table[row sep=crcr] {%
0	0.1\\
1	0.05\\
2	0.05\\
3	0.05\\
4	0\\
5	0.05\\
6	0.15\\
7	0\\
8	0\\
9	0\\
10	0\\
11	0\\
12	0\\
13	0\\
14	0\\
15	0\\
};
\addlegendentry{3 Attempts};

\addplot[ybar stacked,draw=black,fill=mycolor4,area legend] plot table[row sep=crcr] 
\caption{The distribution for number of tries for decoding the MIB for different amount of devices.}
\label{fig:MT_MIB_Tries}
\end{figure}

It is seen in \autoref{fig:MT_MIB_Tries} that baseline code is not different from the changed code at a lower amount of devices. At a higher amount of devices does it seems that the changed code is more efficient, as these levels is the amount of tries lower.

\section{SIB1}
The execution time for the SIB1 decoding step is measured dependably on the number of devices emulated and the baseline is measured as well, which gives the results seen in \autoref{fig:MT_SIB1_Time}. The test is executed with the different amount of devices, but the results shown in \autoref{fig:MT_SIB1_Time} is only for the first device which also will be the procedure for the following steps. As both the radio error and ilde after MIB error occurs in this step of the process, the amount of measurement points are lowered. \todo{shown time difference from first to last SIB mes}

\begin{figure}[H]
\tikzsetnextfilename{MT_SIB1_Time}
\centering
\resizebox{0.5\textwidth}{!}{
% This file was created by matlab2tikz.
%
%The latest updates can be retrieved from
%  http://www.mathworks.com/matlabcentral/fileexchange/22022-matlab2tikz-matlab2tikz
%where you can also make suggestions and rate matlab2tikz.
%
\definecolor{mycolor1}{rgb}{0.00000,0.44700,0.74100}%
\definecolor{mycolor2}{rgb}{0.85000,0.32500,0.09800}%
%
\begin{tikzpicture}

\begin{axis}[%
width=\textwidth,
height=.66\textwidth,
at={(0.758in,0.481in)},
scale only axis,
xmin=-0.5,
xmax=15.5,
xtick={0,1,2,3,4,5,6,7,8,9,10,11,12,13,14,15},
xticklabels={{Base},{1},{2},{3},{4},{5},{6},{7},{8},{9},{10},{11},{12},{13},{14},{15}},
xlabel={Number of devices},
ymin=0,
ymax=14,
ylabel={Time [s]},
axis background/.style={fill=white},
title style={font=\bfseries},
title={SIB1 execution time},
legend style={at={(0.03,0.97)},anchor=north west,legend cell align=left,align=left,draw=white!15!black}
]
\addplot [color=mycolor1,only marks,mark=*,mark options={solid}]
  table[row sep=crcr]{%
1	2.06326198577881\\
1	2.06440305709839\\
1	3.98324704170227\\
1	3.34370994567871\\
1	2.064288854599\\
1	2.06284713745117\\
1	3.34417414665222\\
1	0\\
1	3.98412108421326\\
1	0\\
1	0\\
1	3.34290599822998\\
1	3.98381996154785\\
1	2.70346093177795\\
1	2.7038140296936\\
1	3.34348487854004\\
1	2.70364594459534\\
1	3.34404802322388\\
1	2.70409893989563\\
1	0\\
};
\addlegendentry{MDE};

\addplot [color=mycolor1,only marks,mark=*,mark options={solid},forget plot]
  table[row sep=crcr]{%
2	0\\
2	2.06386804580688\\
2	3.9830470085144\\
2	2.70327496528625\\
2	3.98435401916504\\
2	3.34425902366638\\
2	0\\
2	0\\
2	3.34448194503784\\
2	3.98420310020447\\
2	3.98417806625366\\
2	0\\
2	0\\
2	0\\
2	2.06371593475342\\
2	0\\
2	3.98412013053894\\
2	6.54380798339844\\
2	3.34428119659424\\
2	0\\
};

\addplot [color=mycolor1,only marks,mark=*,mark options={solid}]
  table[row sep=crcr]{%
3	0\\
3	3.34311699867249\\
3	2.06403684616089\\
3	2.06435298919678\\
3	2.06442618370056\\
3	3.98329901695251\\
3	0\\
3	2.06374597549438\\
3	3.98314690589905\\
3	2.70358204841614\\
3	3.34318780899048\\
3	3.98370504379272\\
3	0\\
3	2.70299601554871\\
3	0\\
3	0\\
3	3.34412288665771\\
3	2.70408582687378\\
3	2.06382894515991\\
3	2.06484198570251\\
};
\addplot [color=mycolor1,only marks,mark=*,mark options={solid},forget plot]
  table[row sep=crcr]{%
4	3.34450602531433\\
4	2.70465612411499\\
4	3.9842541217804\\
4	0\\
4	0\\
4	2.70361804962158\\
4	3.34443497657776\\
4	3.3442051410675\\
4	3.98412203788757\\
4	0\\
4	2.70440602302551\\
4	2.70354795455933\\
4	3.98445606231689\\
4	3.34337210655212\\
4	2.7031729221344\\
4	2.06350088119507\\
4	2.7041220664978\\
4	3.98393082618713\\
4	2.7035870552063\\
4	2.06354808807373\\
};
\addplot [color=mycolor1,only marks,mark=*,mark options={solid},forget plot]
  table[row sep=crcr]{%
5	2.70302295684814\\
5	2.0638120174408\\
5	2.70368385314941\\
5	3.34371113777161\\
5	3.34336185455322\\
5	0\\
5	0\\
5	2.06377196311951\\
5	3.34426617622375\\
5	0\\
5	0\\
5	2.06288194656372\\
5	2.70433521270752\\
5	2.70352816581726\\
5	0\\
5	0\\
5	0\\
5	0\\
5	3.34436511993408\\
5	3.34376907348633\\
};
\addplot [color=mycolor1,only marks,mark=*,mark options={solid},forget plot]
  table[row sep=crcr]{%
6	2.70398306846619\\
6	3.98413896560669\\
6	2.70379996299744\\
6	2.06457901000977\\
6	3.34415984153748\\
6	2.06386399269104\\
6	0\\
6	3.98335313796997\\
6	2.06371712684631\\
6	3.34421420097351\\
6	3.98331713676453\\
6	0\\
6	3.98413610458374\\
6	2.06382918357849\\
6	0\\
6	3.34377503395081\\
6	0\\
6	0\\
6	3.34429097175598\\
6	2.70383501052856\\
};
\addplot [color=mycolor1,only marks,mark=*,mark options={solid},forget plot]
  table[row sep=crcr]{%
7	2.70354390144348\\
7	0\\
7	2.70460295677185\\
7	3.34423208236694\\
7	0\\
7	2.70423698425293\\
7	2.06377696990967\\
7	0\\
7	0\\
7	0\\
7	3.98403000831604\\
7	3.34456205368042\\
7	3.98315095901489\\
7	3.34323811531067\\
7	2.70326495170593\\
7	3.34363293647766\\
7	0\\
7	2.70313405990601\\
7	3.98417210578918\\
7	2.70344686508179\\
};
\addplot [color=mycolor1,only marks,mark=*,mark options={solid},forget plot]
  table[row sep=crcr]{%
8	3.34394001960754\\
8	3.9837920665741\\
8	3.98305892944336\\
8	2.06428194046021\\
8	0\\
8	2.06423997879028\\
8	3.9834680557251\\
8	0\\
8	2.70423197746277\\
8	3.34431982040405\\
8	0\\
8	2.7035391330719\\
8	0\\
8	3.34361791610718\\
8	2.06392502784729\\
8	3.98410606384277\\
8	2.70362114906311\\
8	3.34315776824951\\
8	3.98356890678406\\
8	3.98443794250488\\
};
\addplot [color=mycolor1,only marks,mark=*,mark options={solid},forget plot]
  table[row sep=crcr]{%
9	2.06394577026367\\
9	3.34412384033203\\
9	3.98399186134338\\
9	3.9841251373291\\
9	2.06348896026611\\
9	0\\
9	3.34316611289978\\
9	2.06286597251892\\
9	3.98410415649414\\
9	2.06404900550842\\
9	2.06381487846375\\
9	2.06389117240906\\
9	0\\
9	2.06410598754883\\
9	2.70342183113098\\
9	3.34323501586914\\
9	3.34321093559265\\
9	3.98393988609314\\
9	2.06367611885071\\
9	3.34430599212646\\
};
\addplot [color=mycolor1,only marks,mark=*,mark options={solid},forget plot]
  table[row sep=crcr]{%
10	0\\
10	3.98293900489807\\
10	3.34414911270142\\
10	2.06450819969177\\
10	0\\
10	0\\
10	0\\
10	3.34374499320984\\
10	0\\
10	0\\
10	3.98308205604553\\
10	2.06371092796326\\
10	2.06403684616089\\
10	2.06367516517639\\
10	0\\
10	2.7036759853363\\
10	2.06395602226257\\
10	2.70334911346436\\
10	0\\
10	0\\
};
\addplot [color=mycolor1,only marks,mark=*,mark options={solid},forget plot]
  table[row sep=crcr]{%
11	3.34377312660217\\
11	2.70394802093506\\
11	2.7042019367218\\
11	3.98395895957947\\
11	0\\
11	3.34369802474976\\
11	2.70436382293701\\
11	2.70322489738464\\
11	3.34422206878662\\
11	2.70361518859863\\
11	3.98409199714661\\
11	3.34430408477783\\
11	0\\
11	3.98311901092529\\
11	3.98332715034485\\
11	2.06374907493591\\
11	0\\
11	0\\
11	3.98366022109985\\
11	2.06314301490784\\
};
\addplot [color=mycolor1,only marks,mark=*,mark options={solid},forget plot]
  table[row sep=crcr]{%
12	2.7039110660553\\
12	2.06496000289917\\
12	2.70287799835205\\
12	2.7045431137085\\
12	3.98310804367065\\
12	2.70452785491943\\
12	3.34418082237244\\
12	0\\
12	3.98356199264526\\
12	0\\
12	3.34408807754517\\
12	3.34393095970154\\
12	3.98390102386475\\
12	0\\
12	3.98423719406128\\
12	3.34313797950745\\
12	2.70388698577881\\
12	3.34319090843201\\
12	2.06460905075073\\
12	2.06348609924316\\
};
\addplot [color=mycolor1,only marks,mark=*,mark options={solid},forget plot]
  table[row sep=crcr]{%
13	0\\
13	0\\
13	0\\
13	0\\
13	3.98369693756104\\
13	0\\
13	0\\
13	0\\
13	0\\
13	0\\
13	0\\
13	0\\
13	0\\
13	0\\
13	3.98429584503174\\
13	0\\
13	0\\
13	0\\
13	0\\
13	2.06700801849365\\
};
\addplot [color=mycolor1,only marks,mark=*,mark options={solid},forget plot]
  table[row sep=crcr]{%
14	0\\
14	2.06355404853821\\
14	0\\
14	0\\
14	0\\
14	0\\
14	0\\
14	3.34336185455322\\
14	0\\
14	0\\
14	3.34429407119751\\
14	0\\
14	0\\
14	0\\
14	0\\
14	0\\
14	12.9436860084534\\
14	0\\
14	0\\
14	0\\
};
\addplot [color=mycolor1,only marks,mark=*,mark options={solid},forget plot]
  table[row sep=crcr]{%
15	3.37162494659424\\
15	0\\
15	2.06366395950317\\
15	2.70392298698425\\
15	0\\
15	3.98395490646362\\
15	0\\
15	12.9445569515228\\
15	0\\
15	0\\
15	0\\
15	0\\
15	3.34374690055847\\
15	0\\
15	3.98381996154785\\
15	4.64827418327332\\
15	0\\
15	2.06406998634338\\
15	0\\
15	0\\
};
\addplot [color=mycolor2,only marks,mark=*,mark options={solid}]
  table[row sep=crcr]
\caption{Execution time for the decoding the SIB1 step for different amount of devices and the baseline.}
\label{fig:MT_SIB1_Time}
\end{figure}

As seen in \autoref{fig:MT_SIB1_Time} do the baseline and changed code have the same tendency around four different time values, with some measurements way off. Compared to the previous steps, this step have long execution time and the steps between the time values is around 600 ms.
\todo{Maybe make a histogram here to}

\section{SIB2}
The execution time for the SIB2 decoding step is measured dependably on the number of devices emulated and the baseline is measured as well, which gives the results seen in \autoref{fig:MT_SIB2_Time}. The transmission after SIB1 error narrows the number of measurement points down even further for this step in the process.

\begin{figure}[H]
\tikzsetnextfilename{MT_SIB2_Time}
\centering
\resizebox{0.5\textwidth}{!}{
% This file was created by matlab2tikz.
%
%The latest updates can be retrieved from
%  http://www.mathworks.com/matlabcentral/fileexchange/22022-matlab2tikz-matlab2tikz
%where you can also make suggestions and rate matlab2tikz.
%
\definecolor{mycolor1}{rgb}{0.00000,0.44700,0.74100}%
\definecolor{mycolor2}{rgb}{0.85000,0.32500,0.09800}%
%
\begin{tikzpicture}

\begin{axis}[%
width=4.521in,
height=3.566in,
at={(0.758in,0.481in)},
scale only axis,
xmin=-0.5,
xmax=15.5,
xtick={0,1,2,3,4,5,6,7,8,9,10,11,12,13,14,15},
xticklabels={{Baseline},{1},{2},{3},{4},{5},{6},{7},{8},{9},{10},{11},{12},{13},{14},{15}},
xlabel={Number of UEs},
ymin=1.12,
ymax=1.18,
ylabel={Time (s)},
axis background/.style={fill=white},
title style={font=\bfseries},
title={SIB2 execution time},
legend style={at={(0.03,0.97)},anchor=north west,legend cell align=left,align=left,draw=white!15!black}
]
\addplot [color=mycolor1,only marks,mark=*,mark options={solid}]
  table[row sep=crcr]{%
1	1.14861416816711\\
1	1.14764595031738\\
1	1.14869999885559\\
1	1.14867520332336\\
1	1.14768314361572\\
1	1.14871907234192\\
1	1.14760398864746\\
1	0\\
1	1.14761400222778\\
1	0\\
1	0\\
1	1.14873504638672\\
1	1.14866995811462\\
1	1.14762902259827\\
1	1.14844989776611\\
1	1.14772701263428\\
1	1.14866614341736\\
1	1.14762902259827\\
1	1.14766502380371\\
1	0\\
};
\addlegendentry{Change code};

\addplot [color=mycolor1,only marks,mark=*,mark options={solid}]
  table[row sep=crcr]{%
2	0\\
2	1.14869999885559\\
2	1.14872193336487\\
2	1.14743685722351\\
2	1.14759993553162\\
2	1.1476469039917\\
2	0\\
2	0\\
2	1.14758086204529\\
2	1.14762592315674\\
2	1.14773392677307\\
2	0\\
2	0\\
2	0\\
2	1.14763808250427\\
2	0\\
2	1.14763689041138\\
2	1.14846801757813\\
2	1.14778089523315\\
2	0\\
};
\addlegendentry{Baseline};

\addplot [color=mycolor1,only marks,mark=*,mark options={solid},forget plot]
  table[row sep=crcr]{%
3	0\\
3	1.1484739780426\\
3	1.14770102500916\\
3	1.14761209487915\\
3	1.14775395393372\\
3	1.14868903160095\\
3	0\\
3	1.14869093894959\\
3	1.14865207672119\\
3	1.14784407615662\\
3	1.14785313606262\\
3	1.14755582809448\\
3	0\\
3	1.14864993095398\\
3	0\\
3	0\\
3	1.14760112762451\\
3	1.14760613441467\\
3	1.14757800102234\\
3	1.14759707450867\\
};
\addplot [color=mycolor1,only marks,mark=*,mark options={solid},forget plot]
  table[row sep=crcr]{%
4	1.14768004417419\\
4	1.14766192436218\\
4	1.1485378742218\\
4	0\\
4	0\\
4	1.14876699447632\\
4	1.14763498306274\\
4	1.14748001098633\\
4	1.14754796028137\\
4	0\\
4	1.14770913124084\\
4	1.14860796928406\\
4	1.14766693115234\\
4	1.14847683906555\\
4	1.14886403083801\\
4	1.14751315116882\\
4	1.14781403541565\\
4	1.14758896827698\\
4	1.14863586425781\\
4	1.14876008033752\\
};
\addplot [color=mycolor1,only marks,mark=*,mark options={solid},forget plot]
  table[row sep=crcr]{%
5	1.14869904518127\\
5	1.14763689041138\\
5	1.14762902259827\\
5	1.14758992195129\\
5	1.14868807792664\\
5	0\\
5	0\\
5	1.14759087562561\\
5	1.14765095710754\\
5	0\\
5	0\\
5	1.14874196052551\\
5	1.14760994911194\\
5	1.14874196052551\\
5	0\\
5	0\\
5	0\\
5	0\\
5	1.1476788520813\\
5	1.14865398406982\\
};
\addplot [color=mycolor1,only marks,mark=*,mark options={solid},forget plot]
  table[row sep=crcr]{%
6	1.14870095252991\\
6	1.14779305458069\\
6	1.14860796928406\\
6	1.14767384529114\\
6	1.14760899543762\\
6	1.14768695831299\\
6	0\\
6	1.14863681793213\\
6	1.14810705184937\\
6	1.1475830078125\\
6	1.14868903160095\\
6	0\\
6	1.14762282371521\\
6	1.1486439704895\\
6	0\\
6	1.14873480796814\\
6	0\\
6	0\\
6	1.14763402938843\\
6	1.14762711524963\\
};
\addplot [color=mycolor1,only marks,mark=*,mark options={solid},forget plot]
  table[row sep=crcr]{%
7	1.14771604537964\\
7	0\\
7	1.14766097068787\\
7	1.14756989479065\\
7	0\\
7	1.14763617515564\\
7	1.1486930847168\\
7	0\\
7	0\\
7	0\\
7	1.14779114723206\\
7	1.14769601821899\\
7	1.14865589141846\\
7	1.14863991737366\\
7	1.14755702018738\\
7	1.1476879119873\\
7	0\\
7	1.14870691299438\\
7	1.14779496192932\\
7	1.14769411087036\\
};
\addplot [color=mycolor1,only marks,mark=*,mark options={solid},forget plot]
  table[row sep=crcr]{%
8	1.14765310287476\\
8	1.14868092536926\\
8	1.148521900177\\
8	1.14763808250427\\
8	0\\
8	1.14762806892395\\
8	1.14874792098999\\
8	0\\
8	1.14770197868347\\
8	1.14808297157288\\
8	0\\
8	1.14866805076599\\
8	0\\
8	1.14819407463074\\
8	1.14826083183289\\
8	1.14760708808899\\
8	1.14861392974854\\
8	1.1487021446228\\
8	1.14819502830505\\
8	1.14798307418823\\
};
\addplot [color=mycolor1,only marks,mark=*,mark options={solid},forget plot]
  table[row sep=crcr]{%
9	0\\
9	1.14850616455078\\
9	1.14752793312073\\
9	1.14864087104797\\
9	1.14869999885559\\
9	0\\
9	1.14872694015503\\
9	1.14868307113647\\
9	1.14765501022339\\
9	1.14763712882996\\
9	1.14769601821899\\
9	1.14774680137634\\
9	0\\
9	1.14756488800049\\
9	1.14765596389771\\
9	1.14865112304688\\
9	1.14868903160095\\
9	1.14756202697754\\
9	1.14766097068787\\
9	1.14769601821899\\
};
\addplot [color=mycolor1,only marks,mark=*,mark options={solid},forget plot]
  table[row sep=crcr]{%
10	0\\
10	1.14868307113647\\
10	1.14751195907593\\
10	1.14756798744202\\
10	0\\
10	0\\
10	0\\
10	1.14770293235779\\
10	0\\
10	0\\
10	1.14860892295837\\
10	1.1476309299469\\
10	1.14767909049988\\
10	0\\
10	0\\
10	1.14785599708557\\
10	1.14808678627014\\
10	1.1480119228363\\
10	0\\
10	0\\
};
\addplot [color=mycolor1,only marks,mark=*,mark options={solid},forget plot]
  table[row sep=crcr]{%
11	1.1487090587616\\
11	1.14756202697754\\
11	1.14763712882996\\
11	1.14790201187134\\
11	0\\
11	1.14756107330322\\
11	1.14760112762451\\
11	1.14776301383972\\
11	1.14790391921997\\
11	1.14773893356323\\
11	1.14760589599609\\
11	1.14763593673706\\
11	0\\
11	0\\
11	1.14833998680115\\
11	0\\
11	0\\
11	0\\
11	1.14766192436218\\
11	1.1485481262207\\
};
\addplot [color=mycolor1,only marks,mark=*,mark options={solid},forget plot]
  table[row sep=crcr]{%
12	1.147705078125\\
12	1.14687490463257\\
12	1.14872598648071\\
12	1.14797592163086\\
12	1.14834499359131\\
12	1.14762115478516\\
12	1.14752197265625\\
12	0\\
12	0\\
12	0\\
12	1.14772891998291\\
12	1.14748311042786\\
12	1.14785885810852\\
12	0\\
12	1.14766788482666\\
12	1.14836406707764\\
12	1.1478488445282\\
12	1.14873003959656\\
12	1.14742994308472\\
12	1.14833402633667\\
};
\addplot [color=mycolor1,only marks,mark=*,mark options={solid},forget plot]
  table[row sep=crcr]{%
13	0\\
13	0\\
13	0\\
13	0\\
13	1.14760804176331\\
13	0\\
13	0\\
13	0\\
13	0\\
13	0\\
13	0\\
13	0\\
13	0\\
13	0\\
13	1.1476571559906\\
13	0\\
13	0\\
13	0\\
13	0\\
13	0\\
};
\addplot [color=mycolor1,only marks,mark=*,mark options={solid},forget plot]
  table[row sep=crcr]{%
14	0\\
14	0\\
14	0\\
14	0\\
14	0\\
14	0\\
14	0\\
14	1.14835405349731\\
14	0\\
14	0\\
14	1.14746284484863\\
14	0\\
14	0\\
14	0\\
14	0\\
14	0\\
14	0\\
14	0\\
14	0\\
14	0\\
};
\addplot [color=mycolor1,only marks,mark=*,mark options={solid},forget plot]
  table[row sep=crcr]{%
15	1.12556910514832\\
15	0\\
15	0\\
15	1.14798092842102\\
15	0\\
15	1.15576195716858\\
15	0\\
15	0\\
15	0\\
15	0\\
15	0\\
15	0\\
15	1.16539812088013\\
15	0\\
15	0\\
15	0\\
15	0\\
15	0\\
15	0\\
15	0\\
};
\addplot [color=mycolor2,only marks,mark=*,mark options={solid},forget plot]
  table[row sep=crcr]
\caption{Execution time for the decoding the SIB2 step for different amount of devices and the baseline.}
\label{fig:MT_SIB2_Time}
\end{figure}

As seen in \autoref{fig:MT_SIB2_Time} do the baseline and changed code have the same tendency, beside for when there is emulated 15 devices. There is not a big spread for the measurements for all the other setups.

\section{NPRACH}
The execution time for the NPRACH step is measured dependably on the number of devices emulated and the baseline is measured as well, which gives the results seen in \autoref{fig:MT_Nprach_Time}. Here will the NPRACH error occur for some of the measurements and will effect the results. As the shown results is only for the first device, there are still measurement points for measurements where this error occurs, but no following RAR measurement.

\begin{figure}[H]
\tikzsetnextfilename{MT_Nprach_Time}
\centering
\resizebox{0.5\textwidth}{!}{
% This file was created by matlab2tikz.
%
%The latest updates can be retrieved from
%  http://www.mathworks.com/matlabcentral/fileexchange/22022-matlab2tikz-matlab2tikz
%where you can also make suggestions and rate matlab2tikz.
%
\definecolor{mycolor1}{rgb}{0.00000,0.44700,0.74100}%
\definecolor{mycolor2}{rgb}{0.85000,0.32500,0.09800}%
%
\begin{tikzpicture}

\begin{axis}[%
width=\textwidth,
height=.66\textwidth,
at={(0.758in,0.481in)},
scale only axis,
xmin=-0.5,
xmax=15.5,
xtick={0,1,2,3,4,5,6,7,8,9,10,11,12,13,14,15},
xticklabels={{Baseline},{1},{2},{3},{4},{5},{6},{7},{8},{9},{10},{11},{12},{13},{14},{15}},
xlabel={Number of devices},
ymin=0.3,
ymax=0.33,
ylabel={Time (s)},
axis background/.style={fill=white},
title style={font=\bfseries},
title={NPRACH execution time},
legend style={at={(0.03,0.97)},anchor=north west,legend cell align=left,align=left,draw=white!15!black}
]
\addplot [color=mycolor1,only marks,mark=*,mark options={solid}]
  table[row sep=crcr]{%
1	0.311680793762207\\
1	0.311604022979736\\
1	0.311627864837646\\
1	0.311679840087891\\
1	0.311601877212524\\
1	0.311666011810303\\
1	0.311678886413574\\
1	0\\
1	0.311702966690063\\
1	0\\
1	0\\
1	0.311586856842041\\
1	0.311655044555664\\
1	0.311685085296631\\
1	0.311557054519653\\
1	0.312576055526733\\
1	0.311637878417969\\
1	0.312717914581299\\
1	0.311601161956787\\
1	0\\
};
\addlegendentry{Change code};

\addplot [color=mycolor1,only marks,mark=*,mark options={solid},forget plot]
  table[row sep=crcr]{%
2	0\\
2	0.311550140380859\\
2	0.311643123626709\\
2	0.31199312210083\\
2	0.311596870422363\\
2	0.31159496307373\\
2	0\\
2	0\\
2	0.311676025390625\\
2	0.311542987823486\\
2	0.312494993209839\\
2	0\\
2	0\\
2	0\\
2	0.311656951904297\\
2	0\\
2	0.312523126602173\\
2	0.311601877212524\\
2	0.312467098236084\\
2	0\\
};


\addplot [color=mycolor1,only marks,mark=*,mark options={solid},forget plot]
  table[row sep=crcr]{%
3	0\\
3	0.311551809310913\\
3	0.311765193939209\\
3	0.311564922332764\\
3	0.311563014984131\\
3	0.311522006988525\\
3	0\\
3	0.311599016189575\\
3	0.311592102050781\\
3	0.312321901321411\\
3	0.312319040298462\\
3	0.311748027801514\\
3	0\\
3	0.311631202697754\\
3	0\\
3	0\\
3	0.31165599822998\\
3	0.311599969863892\\
3	0.311576128005981\\
3	0.311599969863892\\
};
\addplot [color=mycolor1,only marks,mark=*,mark options={solid},forget plot]
  table[row sep=crcr]{%
4	0.311563014984131\\
4	0.311460018157959\\
4	0.311659097671509\\
4	0\\
4	0\\
4	0.311501979827881\\
4	0.311540126800537\\
4	0.311670064926147\\
4	0.31163501739502\\
4	0\\
4	0.311506986618042\\
4	0.311635971069336\\
4	0.31148099899292\\
4	0.311570167541504\\
4	0.311554193496704\\
4	0.311676979064941\\
4	0.312494039535522\\
4	0.311719179153442\\
4	0.311642169952393\\
4	0.311550855636597\\
};
\addplot [color=mycolor1,only marks,mark=*,mark options={solid},forget plot]
  table[row sep=crcr]{%
5	0.311535835266113\\
5	0.311478137969971\\
5	0.312060117721558\\
5	0.311666011810303\\
5	0.311637878417969\\
5	0\\
5	0\\
5	0.311530113220215\\
5	0.311499834060669\\
5	0\\
5	0\\
5	0.31154990196228\\
5	0.311671018600464\\
5	0.311539888381958\\
5	0\\
5	0\\
5	0\\
5	0\\
5	0.311614990234375\\
5	0.311556816101074\\
};
\addplot [color=mycolor1,only marks,mark=*,mark options={solid},forget plot]
  table[row sep=crcr]{%
6	0.311546087265015\\
6	0.311470031738281\\
6	0.311733961105347\\
6	0.311630010604858\\
6	0.312007188796997\\
6	0.311566114425659\\
6	0\\
6	0.311662197113037\\
6	0.31212592124939\\
6	0.3122398853302\\
6	0.311693906784058\\
6	0\\
6	0.311550140380859\\
6	0.311647891998291\\
6	0\\
6	0.311639070510864\\
6	0\\
6	0\\
6	0.311668872833252\\
6	0.311506032943726\\
};
\addplot [color=mycolor1,only marks,mark=*,mark options={solid},forget plot]
  table[row sep=crcr]{%
7	0.312558889389038\\
7	0\\
7	0.311611890792847\\
7	0.311638116836548\\
7	0\\
7	0.311750888824463\\
7	0.31158185005188\\
7	0\\
7	0\\
7	0\\
7	0.311608791351318\\
7	0.311516046524048\\
7	0.31172513961792\\
7	0.311556100845337\\
7	0.312680959701538\\
7	0.311551094055176\\
7	0\\
7	0.311511039733887\\
7	0.311432123184204\\
7	0.312183856964111\\
};
\addplot [color=mycolor1,only marks,mark=*,mark options={solid},forget plot]
  table[row sep=crcr]{%
8	0.311455965042114\\
8	0.311665058135986\\
8	0.311743974685669\\
8	0.311501026153564\\
8	0\\
8	0.311676025390625\\
8	0.311566114425659\\
8	0\\
8	0.311695098876953\\
8	0.311681032180786\\
8	0\\
8	0.311643838882446\\
8	0\\
8	0.312039852142334\\
8	0.31199312210083\\
8	0.311548948287964\\
8	0.311679840087891\\
8	0.311714887619019\\
8	0.31155800819397\\
8	0.311631917953491\\
};
\addplot [color=mycolor1,only marks,mark=*,mark options={solid},forget plot]
  table[row sep=crcr]{%
9	0\\
9	0.311545848846436\\
9	0.311694145202637\\
9	0.31157112121582\\
9	0.311569929122925\\
9	0\\
9	0.31165599822998\\
9	0.311696767807007\\
9	0.311599969863892\\
9	0.311621904373169\\
9	0.311509132385254\\
9	0.311530113220215\\
9	0\\
9	0.312504053115845\\
9	0.311852216720581\\
9	0.311555862426758\\
9	0.311640977859497\\
9	0.311907052993774\\
9	0.311527013778687\\
9	0.311538934707642\\
};
\addplot [color=mycolor1,only marks,mark=*,mark options={solid},forget plot]
  table[row sep=crcr]{%
10	0\\
10	0.311673879623413\\
10	0.311647891998291\\
10	0.311652898788452\\
10	0\\
10	0\\
10	0\\
10	0.311555862426758\\
10	0\\
10	0\\
10	0.31167197227478\\
10	0.312138080596924\\
10	0.31160306930542\\
10	0\\
10	0\\
10	0.312062978744507\\
10	0.311703205108643\\
10	0.311650991439819\\
10	0\\
10	0\\
};
\addplot [color=mycolor1,only marks,mark=*,mark options={solid},forget plot]
  table[row sep=crcr]{%
11	0.311681032180786\\
11	0.31174898147583\\
11	0.311945915222168\\
11	0.312388181686401\\
11	0\\
11	0.311898946762085\\
11	0.311634063720703\\
11	0.312273979187012\\
11	0.31233811378479\\
11	0.311637878417969\\
11	0.311670064926147\\
11	0.311576128005981\\
11	0\\
11	0\\
11	0.312019824981689\\
11	0\\
11	0\\
11	0\\
11	0.311609029769897\\
11	0.311727046966553\\
};
\addplot [color=mycolor1,only marks,mark=*,mark options={solid},forget plot]
  table[row sep=crcr]{%
12	0.311507940292358\\
12	0.311978101730347\\
12	0.311573028564453\\
12	0.311161041259766\\
12	0.312034130096436\\
12	0.311545848846436\\
12	0.311670064926147\\
12	0\\
12	0\\
12	0\\
12	0.311485052108765\\
12	0.311722993850708\\
12	0.311711072921753\\
12	0\\
12	0.31154990196228\\
12	0.311701059341431\\
12	0.31179404258728\\
12	0.311655044555664\\
12	0.311732053756714\\
12	0.312110900878906\\
};
\addplot [color=mycolor1,only marks,mark=*,mark options={solid},forget plot]
  table[row sep=crcr]{%
13	0\\
13	0\\
13	0\\
13	0\\
13	0.315821886062622\\
13	0\\
13	0\\
13	0\\
13	0\\
13	0\\
13	0\\
13	0\\
13	0\\
13	0\\
13	0.323969841003418\\
13	0\\
13	0\\
13	0\\
13	0\\
13	0\\
};
\addplot [color=mycolor1,only marks,mark=*,mark options={solid},forget plot]
  table[row sep=crcr]{%
14	0\\
14	0\\
14	0\\
14	0\\
14	0\\
14	0\\
14	0\\
14	0.315366983413696\\
14	0\\
14	0\\
14	0.315733194351196\\
14	0\\
14	0\\
14	0\\
14	0\\
14	0\\
14	0\\
14	0\\
14	0\\
14	0\\
};
\addplot [color=mycolor1,only marks,mark=*,mark options={solid},forget plot]
  table[row sep=crcr]{%
15	0.309165954589844\\
15	0\\
15	0\\
15	0.321655988693237\\
15	0\\
15	0.306154012680054\\
15	0\\
15	0\\
15	0\\
15	0\\
15	0\\
15	0\\
15	0.321050882339478\\
15	0\\
15	0\\
15	0\\
15	0\\
15	0\\
15	0\\
15	0\\
};
\addplot [color=mycolor2,only marks,mark=*,mark options={solid}]
  table[row sep=crcr]
\caption{Execution time for the NPRACH step for different amount of devices and the baseline.}
\label{fig:MT_Nprach_Time}
\end{figure}

As seen in \autoref{fig:MT_Nprach_Time} does the baseline and changed code the same time values, but the changed code have a bigger spread of its measurement point. For all the measurements for the changed code over 12 devices, do not follow the other measurements tendency.

\section{RAR}
The execution time for the RAR step is measured dependably on the number of devices emulated and the baseline is measured as well, which gives the results seen in \autoref{fig:MT_Rar_Time}. Here does the msg2 error occurs and no msg2 will be received and no measurements can be made.

\begin{figure}[H]
\tikzsetnextfilename{MT_Rar_Time}
\centering
\resizebox{0.5\textwidth}{!}{
% This file was created by matlab2tikz.
%
%The latest updates can be retrieved from
%  http://www.mathworks.com/matlabcentral/fileexchange/22022-matlab2tikz-matlab2tikz
%where you can also make suggestions and rate matlab2tikz.
%
\definecolor{mycolor1}{rgb}{0.00000,0.44700,0.74100}%
\definecolor{mycolor2}{rgb}{0.85000,0.32500,0.09800}%
%
\begin{tikzpicture}

\begin{axis}[%
width=.5\textwidth,
height=.33\textwidth,
at={(0.758in,0.481in)},
scale only axis,
xmin=-0.5,
xmax=15.5,
xtick={0,1,2,3,4,5,6,7,8,9,10,11,12,13,14,15},
xticklabels={{Base},{1},{2},{3},{4},{5},{6},{7},{8},{9},{10},{11},{12},{13},{14},{15}},
xlabel={Number of devices},
ymin=0.03,
ymax=0.06,
ylabel={Time (s)},
axis background/.style={fill=white},
title style={font=\bfseries},
title={RAR execution time},
legend style={at={(0.03,0.97)},anchor=north west,legend cell align=left,align=left,draw=white!15!black}
]
\addplot [color=mycolor1,only marks,mark=*,mark options={solid}]
  table[row sep=crcr]{%
1	0.0341880321502686\\
1	0.034235954284668\\
1	0.0342159271240234\\
1	0.0342271327972412\\
1	0.0342719554901123\\
1	0.0341730117797852\\
1	0\\
1	0\\
1	0.0341699123382568\\
1	0\\
1	0\\
1	0.0342731475830078\\
1	0.0341849327087402\\
1	0.0341880321502686\\
1	0.0343189239501953\\
1	0.0333008766174316\\
1	0.0342051982879639\\
1	0.033297061920166\\
1	0.0342288017272949\\
1	0\\
};
\addlegendentry{Change code};

\addplot [color=mycolor1,only marks,mark=*,mark options={solid},forget plot]
  table[row sep=crcr]{%
2	0\\
2	0.0342118740081787\\
2	0.0341849327087402\\
2	0.0340640544891357\\
2	0.0342371463775635\\
2	0.0342140197753906\\
2	0\\
2	0\\
2	0.0341610908508301\\
2	0.0342671871185303\\
2	0.0333490371704102\\
2	0\\
2	0\\
2	0\\
2	0.0342299938201904\\
2	0\\
2	0.0333659648895264\\
2	0.0342400074005127\\
2	0.0333349704742432\\
2	0\\
};

\addplot [color=mycolor1,only marks,mark=*,mark options={solid},forget plot]
  table[row sep=crcr]{%
3	0\\
3	0.0342600345611572\\
3	0.0338799953460693\\
3	0.0342161655426025\\
3	0.0342400074005127\\
3	0.0343279838562012\\
3	0\\
3	0.0342199802398682\\
3	0.0342228412628174\\
3	0.0342550277709961\\
3	0.0343589782714844\\
3	0.0341391563415527\\
3	0\\
3	0.0343039035797119\\
3	0\\
3	0\\
3	0.0342159271240234\\
3	0.03420090675354\\
3	0.0342509746551514\\
3	0.0342059135437012\\
};
\addplot [color=mycolor1,only marks,mark=*,mark options={solid},forget plot]
  table[row sep=crcr]{%
4	0.0342638492584229\\
4	0.0343360900878906\\
4	0.0341908931732178\\
4	0\\
4	0\\
4	0.0342490673065186\\
4	0.0342738628387451\\
4	0.0342228412628174\\
4	0.0341949462890625\\
4	0\\
4	0.0342400074005127\\
4	0.0342440605163574\\
4	0.0342991352081299\\
4	0.0341489315032959\\
4	0.0341489315032959\\
4	0.0342459678649902\\
4	0.0341918468475342\\
4	0.0341000556945801\\
4	0.034153938293457\\
4	0.034203052520752\\
};
\addplot [color=mycolor1,only marks,mark=*,mark options={solid},forget plot]
  table[row sep=crcr]{%
5	0.0342710018157959\\
5	0.0341567993164063\\
5	0.0337579250335693\\
5	0.0341598987579346\\
5	0.0341479778289795\\
5	0\\
5	0\\
5	0.0342059135437012\\
5	0.0342340469360352\\
5	0\\
5	0\\
5	0.0342390537261963\\
5	0.0341389179229736\\
5	0.0341851711273193\\
5	0\\
5	0\\
5	0\\
5	0\\
5	0.034188985824585\\
5	0.0342321395874023\\
};
\addplot [color=mycolor1,only marks,mark=*,mark options={solid},forget plot]
  table[row sep=crcr]{%
6	0.0342798233032227\\
6	0.0341129302978516\\
6	0.0341250896453857\\
6	0.0342040061950684\\
6	0.0337188243865967\\
6	0.0342609882354736\\
6	0\\
6	0.0342237949371338\\
6	0.0336298942565918\\
6	0.0335140228271484\\
6	0.0341589450836182\\
6	0\\
6	0.0342769622802734\\
6	0.0342199802398682\\
6	0\\
6	0.0342271327972412\\
6	0\\
6	0\\
6	0.0341720581054688\\
6	0\\
};
\addplot [color=mycolor1,only marks,mark=*,mark options={solid},forget plot]
  table[row sep=crcr]{%
7	0.0336170196533203\\
7	0\\
7	0.0341181755065918\\
7	0.0341799259185791\\
7	0\\
7	0.0342490673065186\\
7	0.0342171192169189\\
7	0\\
7	0\\
7	0\\
7	0.0346660614013672\\
7	0.0343749523162842\\
7	0.0340960025787354\\
7	0.0341479778289795\\
7	0.0340781211853027\\
7	0.0341360569000244\\
7	0\\
7	0.0341110229492188\\
7	0.034153938293457\\
7	0.0335919857025146\\
};
\addplot [color=mycolor1,only marks,mark=*,mark options={solid},forget plot]
  table[row sep=crcr]{%
8	0.0365910530090332\\
8	0.0427010059356689\\
8	0.0370931625366211\\
8	0.0381538867950439\\
8	0\\
8	0.0364658832550049\\
8	0.0379340648651123\\
8	0\\
8	0.0374419689178467\\
8	0.0407381057739258\\
8	0\\
8	0.0341041088104248\\
8	0\\
8	0.0414822101593018\\
8	0.0361030101776123\\
8	0.0386650562286377\\
8	0.0383260250091553\\
8	0.0362110137939453\\
8	0.0341310501098633\\
8	0.037121057510376\\
};
\addplot [color=mycolor1,only marks,mark=*,mark options={solid},forget plot]
  table[row sep=crcr]{%
9	0\\
9	0.0370240211486816\\
9	0.0365219116210938\\
9	0.0378210544586182\\
9	0.038114070892334\\
9	0\\
9	0.037787914276123\\
9	0.0381491184234619\\
9	0.0389068126678467\\
9	0.0386679172515869\\
9	0.0383989810943604\\
9	0.0359559059143066\\
9	0\\
9	0.042572021484375\\
9	0.0371799468994141\\
9	0.0409531593322754\\
9	0.0386199951171875\\
9	0.0404210090637207\\
9	0.036876916885376\\
9	0.0371849536895752\\
};
\addplot [color=mycolor1,only marks,mark=*,mark options={solid},forget plot]
  table[row sep=crcr]{%
10	0\\
10	0.0420839786529541\\
10	0.0408451557159424\\
10	0.0439350605010986\\
10	0\\
10	0\\
10	0\\
10	0.0399701595306396\\
10	0\\
10	0\\
10	0.0381450653076172\\
10	0.0463569164276123\\
10	0.0390889644622803\\
10	0\\
10	0\\
10	0.0403990745544434\\
10	0.0427589416503906\\
10	0.0387308597564697\\
10	0\\
10	0\\
};
\addplot [color=mycolor1,only marks,mark=*,mark options={solid},forget plot]
  table[row sep=crcr]{%
11	0.0449898242950439\\
11	0.0432069301605225\\
11	0.0445609092712402\\
11	0.0408809185028076\\
11	0\\
11	0.0404579639434814\\
11	0.0427489280700684\\
11	0.0475912094116211\\
11	0.0448739528656006\\
11	0.0414700508117676\\
11	0.0436179637908936\\
11	0.0411219596862793\\
11	0\\
11	0\\
11	0.0407330989837646\\
11	0\\
11	0\\
11	0\\
11	0.0408778190612793\\
11	0.040151834487915\\
};
\addplot [color=mycolor1,only marks,mark=*,mark options={solid},forget plot]
  table[row sep=crcr]{%
12	0.0453028678894043\\
12	0.0516760349273682\\
12	0.0564429759979248\\
12	0.0475959777832031\\
12	0.0430538654327393\\
12	0.0450849533081055\\
12	0.0470409393310547\\
12	0\\
12	0\\
12	0\\
12	0.0429530143737793\\
12	0.0567960739135742\\
12	0.0497169494628906\\
12	0\\
12	0.0499541759490967\\
12	0.0428369045257568\\
12	0.0435919761657715\\
12	0.0445668697357178\\
12	0.0499989986419678\\
12	0.0418119430541992\\
};
\addplot [color=mycolor1,only marks,mark=*,mark options={solid},forget plot]
  table[row sep=crcr]{%
13	0\\
13	0\\
13	0\\
13	0\\
13	0\\
13	0\\
13	0\\
13	0\\
13	0\\
13	0\\
13	0\\
13	0\\
13	0\\
13	0\\
13	0\\
13	0\\
13	0\\
13	0\\
13	0\\
13	0\\
};
\addplot [color=mycolor1,only marks,mark=*,mark options={solid},forget plot]
  table[row sep=crcr]{%
14	0\\
14	0\\
14	0\\
14	0\\
14	0\\
14	0\\
14	0\\
14	0\\
14	0\\
14	0\\
14	0\\
14	0\\
14	0\\
14	0\\
14	0\\
14	0\\
14	0\\
14	0\\
14	0\\
14	0\\
};
\addplot [color=mycolor1,only marks,mark=*,mark options={solid},forget plot]
  table[row sep=crcr]{%
15	0\\
15	0\\
15	0\\
15	0\\
15	0\\
15	0\\
15	0\\
15	0\\
15	0\\
15	0\\
15	0\\
15	0\\
15	0\\
15	0\\
15	0\\
15	0\\
15	0\\
15	0\\
15	0\\
15	0\\
};
\addplot [color=mycolor2,only marks,mark=*,mark options={solid}]
  table[row sep=crcr]
\caption{Execution time for the decoding the MIB for different amount of devices and the baseline. A single measurement for the base line is placed at 5.0834 s, which is not shown on this figure.}
\label{fig:MT_Rar_Time}
\end{figure}

As seen in \autoref{fig:MT_Rar_Time} does the changed code starts out with following the tendency for the baseline, where it around eight devices begin to get a higher and higher execution time and getting a wider spread compared to the values at a low amount of devices.