\subsection{Massive IoT system overview}

The emulation of the massiveness scenarios uses the Massive IoT and eNB (Amarisoft) components. It is possible to connect and external NB-IoT device be combining its signal with the signals from the other two components. But as the focus for this part of the emulator is to stress test the protocol this feature will not be used. The Orchestrator is not used either in this part, as it has not been prioritized to set the two system together and have the same control unit. Both components is started up from a config file. While the Amarisoft software already fits to the purpose as an eNB, the SRS software code need to be change to emulate multiple UE. To understand this changes, there will be looked into the original code

\subsubsection{Original SRS code}
The code provide by SRS is still in development and is based on the code SRS develop to LTE, which was use in \todo{Ref to Charly and Mathias project}. The structure of the code is very similar and with that, some of the structure change and solutions can be copied over from the mention project.

\textbf{Structure}
The code is written in C++, with a little C in some of the lower level files. The code is divided up into two sections srsLTE, which contains all the lower layer functions and the radio layer, and srsUE, which emulate a UE with all the different communication layers and handles the protocol. The code is build up into an amount of classes for the different layers, as seen in \todo{ref to figure}, with the PHY and MAC layer being split up into even more classes.\todo{ref to mac and phy figures}

\todo{Insert classe diagram for all, mac and phy}

\todo{Explain short about what the different classes do}

As multiple task is needed to be handle at the same time int different layers, threading is used in multiple classes. \todo{list the classes with threads} Between these classes are there some interfaces, that handles the movement of data and grants.

To have the code emulate multiple UE, some parts of the code is need to duplicated. The goal will be to have the UE to be so separated, so must parts of the system still is done individual for the different UE, but the parts the surrounds the share resources, like the TX and RX channels, is done for all the UE together.