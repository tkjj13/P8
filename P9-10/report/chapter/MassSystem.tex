\chapter{Massive IoT system overview}

It has a high cost to test massiveness scenarios with a massive amount of hardware devices and will require a system to control them individually, especially if the desire is to test the impact of massiveness on the individual devices. A more optimal solution is instead of buying a massive amount of devices, to emulate them through software.


For the emulation of the eNB the Amarisoft LTE 100 code will be used as it can handle 1000 \gls{UE}s \citep{Amarisoft_solutions} and is tested to synchronize with the software based NB-IoT devices that will be used. As it is a commercial BSE, it is not open source and therefore harder to add new features and change all parameters, but as the focus is the devices, the software will still be used. A longer list of parameters for the Amarisoft LTE 100 code can be found in \appref{app:Amarisoft}.

For the \gls{MDE} the code from SRS which have been tested to work with the software eNB from Amarisoft is chosen. This is chosen primarily because access to the source code is available through Keysight Technologies.\todo{check if this is okay or keysight does not want to get mentioned in this connection} As the code is designed for a single device some change have to be implemented, which requires knowledge of the original code. 

The code will produce the infrastructure of the \gls{MDE} combining and transmitting the signals produced to the \gls{eNB}. The eNB will be emulated on secondary PC, as this frees up the PC running the \gls{MDE}. The NB-IoT spectrum overlaps with the LTE or GSM spectrum, meaning the spectrum is licensed, therefore a wire is used to connect \gls{MDE} and eNB. As the path loss in the wire is quite low, compared to transmission through the air, a 30 dB attenuator is added in between to increase the dynamic spectrum of the \gls{MDE} and eNB. To convert from digital signals to RF signals, a USRP B210 is connected to each the two emulators as these \gls{SDR}s are comparable with both softwares. The setup of the emulators can be seen on \autoref{fig:MassSetup}.

\begin{figure}[H]
\centering
\tikzsetnextfilename{MassSetup}
\resizebox{0.9\textwidth}{!}{
\input{figures/Mass_setup.tex}}
\caption{Overview of the emulator setup}
\label{fig:MassSetup}
\end{figure}



%The emulation of the massiveness scenarios uses the Massive IoT and eNB (Amarisoft) components. It is possible to connect and external NB-IoT device be combining its signal with the signals from the other two components. But as the focus for this part of the emulator is to stress test the protocol this feature will not be used. The Orchestrator is not used either in this part, as it has not been prioritized to set the two system together and have the same control unit. Both components is started up from a config file. While the Amarisoft software already fits to the purpose as an eNB, the SRS software code need to be change to emulate multiple UE. To understand this changes, there will be looked into the original code

%\section{eNB emulator}
%For the emulation of the eNB the Amarisoft LTE 100 code will be used as it can handle 1000 \gls{UE} \todo{Insert ref to Amarisoft side} and is tested to synchronize with the software based NB-IoT devices that will be used. As it is a commercial BSE, it is not open source and therefore harder to add new features and change all parameters, but as the focus is the devices, the software will still be used. A longer list of parameters for the Amarisoft LTE 100 code can be found \todo{Insert appendix for this}

%\section{Massive device emulator}
%For the emulation of the massive amount of NB-IoT devices the code provide by SRS which have been tested to work with the software eNB from Amarisoft. As the code is design for a single device some change have to be implemented, which requires knowledge of the original code.



\section{Original SRS code}
The code provide by SRS is still in development and is based on the code SRS developed to LTE, which was used in \citep{thesis_report}. The structure of the code is very similar and with that, some of the structure change and solutions can be copied over from the mentioned project.

Unfortunately as the code is still in development, the msg3 transmission does not work, as it seems that it gets transmitted in a wrong time slot. \todo{Check up on if this is what we want to say} This block the possibility to get a complete attach request, so the project will only work up to the reception of msg2. The plans for the changes and solution for the parts including the full attach procedure and data transmission, can be found in \autoref{ch:Future}.

\subsection{Structure}
The code is written in C++, with a little C in some of the lower level files. The code is divided up into two sections srsLTE, which contains all the lower layer functions and the radio layer, and srsUE, which emulate a device with all the different communication layers that handles the protocol. The code contains an amount of classes for the different layers known from the OSI model, as seen in \autoref{fig:MassClass}, with the PHY layer being split up into even more classes. All these classes is underclasses for an overall class called ue, which therefore contains all the parts needed for running a device.The structure of the PHY layer can be seen in \autoref{fig:PhyClass}. As multiple tasks is needed to be handled at the same time in the different layers, threading is used in multiple classes (PHY, PHCH recv, PHCH workers, MAC, RRC, GW).

\begin{figure}[H]
\centering
\tikzsetnextfilename{MassClass}
\resizebox{0.5\textwidth}{!}{
\usetikzlibrary{arrows}
\begin{tikzpicture}

\draw  (-3,1) rectangle (2,0);
\draw  (-3,1.5) rectangle (2,2.5);
\draw  (-3,3) rectangle (2,4);
\draw  (-3,4.5) rectangle (2,5.5);
\node at (-0.5,0.5) {Radio};
\node at (-0.5,2) {PHY};
\node at (-0.5,3.5) {MAC};
\node at (-0.5,5) {RLC};
\draw  (-3,6) rectangle (2,7);
\draw  (-5,7.5) rectangle (-1,8.5);
\node at (-0.5,6.5) {PDCP};
\node at (-3,8) {RRC};
\draw  (0,7.5) rectangle (2,8.5);
\node at (1,8) {GW};
\draw  (-5,10.5) rectangle (-1,11.5);
\draw  (-3,10) rectangle (2,9);
\node at (-3,11) {USIM};
\node at (-0.5,9.5) {NAS};

\draw [-triangle 60](-1,1) -- (-1,1.5);
\draw [-triangle 60](0,1.5) -- (0,1);
\draw [-triangle 60](-1,2.5) -- (-1,3);
\draw [-triangle 60](0,3) -- (0,2.5);
\draw [-triangle 60](-1,4) -- (-1,4.5);
\draw [-triangle 60](0,4.5) -- (0,4);
\draw [-triangle 60](-1,5.5) -- (-1,6);
\draw [-triangle 60](0,6) -- (0,5.5);
\draw [-triangle 60](-2.5,7) -- (-2.5,7.5);
\draw [-triangle 60](-1.5,7.5) -- (-1.5,7);
\draw [-triangle 60](0.5,7) -- (0.5,7.5);
\draw [-triangle 60](1.5,7.5) -- (1.5,7);
\draw [-triangle 60](0.5,8.5) -- (0.5,9);
\draw [-triangle 60](1.5,9) -- (1.5,8.5);
\draw [-triangle 60](-3,0.25) -- (-4.5,0.25) -- (-4.5,7.5);
\draw (-3,1.75) -- (-4.5,1.75);
\draw (-3,3.25) -- (-4.5,3.25);
\draw (-3,4.75) -- (-4.5,4.75);
\draw [-triangle 60](-3.5,7.5) -- (-3.5,5.25)  -- (-3,5.25);
\draw [-triangle 60](-3.5,7.5) -- (-3.5,3.75)  -- (-3,3.75);
\draw [-triangle 60](-3.5,7.5) -- (-3.5,2.25)  -- (-3,2.25);
\draw [-triangle 60](-3.5,7.5) -- (-3.5,0.75) -- (-3,0.75);
\draw [-triangle 60](-4.5,8.5) -- (-4.5,10.5);
\draw [-triangle 60](-3.5,10.5) -- (-3.5,8.5);
\draw [-triangle 60](-2.5,8.5) -- (-2.5,9);
\draw [-triangle 60](-1.5,9) -- (-1.5,8.5);
\draw [-triangle 60](-2.5,10) -- (-2.5,10.5);
\draw [-triangle 60](-1.5,10.5) -- (-1.5,10);
\end{tikzpicture}}
\caption{Class overview of SRS emulation code}
\label{fig:MassClass}
\end{figure}

\begin{figure}[H]
\centering
\tikzsetnextfilename{PhyClass}
\resizebox{0.5\textwidth}{!}{
\usetikzlibrary{arrows}
\begin{tikzpicture}

\draw  (-2.5,-1) rectangle (1.5,-3);
\draw  (-3,0) rectangle (2,-3.5);
\draw  (-7.5,-1) rectangle (-3.5,-3);
\draw  (-7.5,2.5) rectangle (2,0.5);
\draw  (-2.5,5) rectangle (1.5,3);
\node at (-1.75,-0.30) {Thread pool};
\node at (-0.5,-2) {PHCH workers};
\node at (-5.5,-2) {PHCH common};
\node at (-3.25,1.5) {PHCH recv};
\node at (-0.5,4) {NPRACH};
\draw [-triangle 60](-2.5,-2) -- (-3.5,-2);
\draw [-triangle 60](-7.5,1) -- (-8.5,1) -- (-8.5,-4.5);
\draw (-7.5,-2.5) -- (-8.5,-2.5);
\node at (-8.5,-4.75) {Radio};
\draw [-triangle 60](-0.5,0.5) -- (-0.5,0);
\draw [-triangle 60](-7.5,-1.5) -- (-8,-1.5)  -- (-8,5.5);
\draw (-7.5,2) -- (-8,2);
\node at (-8,5.75) {MAC};
\draw (-0.5,-3) -- (-0.5,-4) -- (-8,-4) -- (-8,-0);
\draw [-triangle 60](-0.5,2.5) -- (-0.5,3);
\draw [-triangle 60](-5.5,0.5) -- (-5.5,-1);
\draw [-triangle 60](-5.5,2.5) -- (-5.5,5.5);
\node at (-5.5,5.75) {RRC};
\draw (-2.5,4) -- (-8.5,4) -- (-8.5,0);
\end{tikzpicture}}
\caption{Class overview of the PHY layer from the SRS emulation code}
\label{fig:PhyClass}
\end{figure}

Even if the class layout look like the OSI model but it does not follow it on point especially the lower layers (PHY and MAC). This comes from that many of the MAC layers task is moved to the PHY class and normally the PHY layer is made up of physical components and do the task, that the SDR performs. The RLC, PDCP, GW, NAS and USIM classes is not used much, as their functionality is for the process after the RAP or at least msg4 have been received. Therefore have they not been though much development in the SRS code. Here following is a description of the different classes and their functionality:

\begin{itemize}
\item [Radio] The radio controls and sets all the parameters for the transmission and receiving. It is the only class connected to the SDR API and sends the data to it for transmission and receives data from it too. 
\item [PHY] The PHY class handles the synchronization, the receiving and decoding of MIB and SIBs messages and decoding and encoding of transmission and received data. As it have a high number of task, it is split up into an amount of underclasses.
	\begin{itemize}
	\item [PHCH recv] Controls the whole PHY class system. Keeps track on which state the system is in and activates the different underclasses in the PHY class.
	\item [PHCH common] Contains most parameters for the PHY underclasses, here under the RNTI's and metrics for UL and DL, and communication up to the MAC class and sends transmission data to the radio class.
	\item [PHCH workers] Handles all processing of data after decoding of MIB and SIBs messages, both encoding and decoding dependably on the grants granted by the MAC class. Their can be multiple workers for one device, but in these cases, their will only be used one.
	\item [Thread pool] Contains and controls the array of PHCH workers.	
	\item [NPRACH] Used to send msg1 in the RAP and keep track of the parameters set for this transmission, like preamble id.
	\end{itemize}
\item [MAC] Keep track of TTI and give grants to PHY class to decode and encode of DL and UL. Control flow from the PHY class to the RLC class. Take over the RAP from the NPRACH class for msg2 and onward.
%\item [RLC] Control data package sizes down to the MAC layer though a buffer, but this 
%\item [PDCP]
\item [RRC] Is the overall control class for the whole device. Here it control how far through the process the device is and if it got all the needed parameters, which is the reason for it being connected to all lower classes, besides the radio class.
%\item [GW]
%\item [NAS]
%\item [USIM]
\end{itemize}

Between these classes are there some interfaces, that handles the movement of data, grants, parameters and other information. These interfaces is build as a class for it self for each interface, with an overall interface for functionalities used multiple places.

\subsection{Execution of the code}
The code is build up into multiple section which runs at different times, which is shown in \autoref{fig:MassStep}. After the initialization, the steps is matching the steps needed for a device to connect to an eNB. The different threads will go to different state, depending on which step in the process the code have made it to.

\begin{figure}[H]
\centering
\tikzsetnextfilename{MassStep}
\resizebox{0.5\textwidth}{!}{
\usetikzlibrary{arrows}
\begin{tikzpicture}

\draw  (-5,5) rectangle (-2,3);
\draw  (-5,2.5) rectangle (-2,0.5);
\draw  (-1,2.5) rectangle (2,0.5);
\draw  (-1,0) rectangle (2,-2);
\draw  (-5,0) rectangle (-2,-2);
\draw  (-5,-2.5) rectangle (-2,-4.5);
\draw  (-1,-2.5) rectangle (2,-4.5);
\node at (-3.5,4) {Initialization};
\node at (-3.5,1.5) {Synchronization};
\node at (0.5,1.5) {MIB};
\node at (-3.5,-1) {SIB2};
\node at (0.5,-1) {SIB1};
\node at (-3.5,-3.5) {NPRACH};
\node at (0.5,-3.5) {RAP};
\draw [-triangle 60](-3.5,3) -- (-3.5,2.5);
\draw [-triangle 60](-2,1.5) -- (-1,1.5);
\draw [-triangle 60](0.5,0.5) -- (0.5,0);
\draw [-triangle 60](-1,-1) -- (-2,-1);
\draw [-triangle 60](-3.5,-2) -- (-3.5,-2.5);
\draw [-triangle 60](-2,-3.5) -- (-1,-3.5);
\end{tikzpicture}}
\caption{A simplification of the steps in the code process}
\label{fig:MassStep}
\end{figure}

\textbf{Initialization}
The initialization step starts up all the different classes and sets pointers to the different interfaces between them and read the config file, which includes a longer list of parameters, which can be found in \todo{Insert app ref to that}. The PHY have a number of underclasses and which are initialized in the PHY thread, which close down afterwards. The main thread will wait on the PHY to finish initialize, before sending an attach request, through the NAS class. This process is shown in \autoref{fig:InitStep}.

\begin{figure}[H]
\centering
\tikzsetnextfilename{InitStep}
\resizebox{0.5\textwidth}{!}{
\usetikzlibrary{arrows}
\begin{tikzpicture}


\draw  (-4,3) rectangle (-1,1.5);
\draw  (-8,3) rectangle (-5,1.5);
\draw  (-8,1) rectangle (-5,-0.5);
\draw  (-8,-1) rectangle (-5,-2.5);
\draw  (-8,-3) rectangle (-5,-4.5);
\draw  (-3.5,-1) rectangle (-0.5,-2.5);
\draw  (-3.5,-3) rectangle (-0.5,-4.5);
\draw  (-8,-5) rectangle (-5,-6.5);
\draw  (-4,-0.5) rectangle (0,-5);

\draw [-triangle 90](-4,2.25) -- (-5,2.25);
\draw [-triangle 90](-6.5,1.5) -- (-6.5,1);
\draw [-triangle 90](-6.5,-0.5) -- (-6.5,-1);
\draw [-triangle 90](-6.5,-2.5) -- (-6.5,-3);
\draw [-triangle 90](-6.5,-4.5) -- (-6.5,-5);
\draw [dash pattern=on 2pt off 3pt on 4pt off 4pt][-triangle 90](-5,0.25) -- (-2,0.25) -- (-2,-1);
\draw [dash pattern=on 2pt off 3pt on 4pt off 4pt][-triangle 90](-2,-4.5) -- (-2,-5.75) -- (-5,-5.75);
\draw [-triangle 90](-2,-2.5) -- (-2,-3);

\node at (-2.5,2.25) {Start};
\node at (-6.5,2.25) {Read config file};
\node at (-6.5,0.25) {Start PHY init};
\node at (-6.5,-1.75) {Radio init};
\node at (-6.5,-3.5) {Upper layers};
\node at (-6.5,-5.75) {Wait for PHY};
\node at (-2,-1.5) {Underclasses};
\node at (-2,-3.75) {PHCH recv init};
\draw [-triangle 60](-6.5,-6.5) -- (-6.5,-7.5);
\node at (-6.5,-7.75) {Attach request};
\node at (-0.50,-0.25) {PHY};
\node at (-6.5,-4) {init};
\node at (-2,-2) {init};
\end{tikzpicture}}
\caption{A simplification of the steps in the code process}
\label{fig:InitStep}
\end{figure}


\textbf{Synchronization}

\begin{figure}[H]
\centering
\tikzsetnextfilename{RecvState}
\resizebox{0.5\textwidth}{!}{
\usetikzlibrary{arrows}
\begin{tikzpicture}

\draw  (-4.5,1.5) ellipse (2 and 1);
\draw  (-4.5,4.5) ellipse (2 and 1);
\draw  (-4.5,7.5) ellipse (2 and 1);
\draw  (-4.5,10.5) ellipse (2 and 1);
\draw  (-11,6) ellipse (2 and 1);
\draw [-triangle 90](-4.5,9.5) -- (-4.5,8.5);
\draw [-triangle 90](-4.5,6.5) -- (-4.5,5.5);
\draw [-triangle 90](-4.5,3.5) -- (-4.5,2.5);
\draw [-triangle 90](-11,7) -- (-11,12.5) -- (-4.5,12.5) -- (-4.5,11.5);
\draw [-triangle 90](-2.75,8) -- (-1.5,8) -- (-1.5,10.5) -- (-2.5,10.5);
\draw [-triangle 90](-6.5,10.5) -- (-7.5,10.5) -- (-7.5,6) -- (-9,6);
\draw [-triangle 90](-2.75,7) -- (-1.5,7) -- (-1.5,1.5) -- (-2.5,1.5);
\draw (-6.5,7.5) -- (-7.5,7.5);
\draw (-6.5,1.5) -- (-7.5,1.5) -- (-7.5,8.5);
\node at (-11,6) {Ilde};
\node at (-4.5,10.5) {Cell search};
\node at (-4.5,7.5) {Cell select};
\node at (-4.5,4.5) {Cell measure};
\node at (-4.5,1.5) {Cell camp};
\draw (-6.5,4.5) -- (-7.5,4.5);
\end{tikzpicture}}
\caption{A simplification of the steps in the code process}
\label{fig:RecvState}
\end{figure}




















To have the code emulate multiple devices, some parts of the original device is needed to be duplicated. The goal of these changes will be to have a massive amount of individual devices emulated, without them effecting each other through their processing time and combining their signals and transmit it as one to the eNB.

\section{Changed code}

\textbf{Structure}\\
\textbf{Flow of the code}\\