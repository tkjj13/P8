\chapter{Conclusion}
In this project the focus has been to investigate LPWA technologies, with the main focus being the NB-IoT protocol. For this purpose the massiveness and energy consumption domains of the NB-IoT protocol has been investigated. To do this the focus is divided between designing a MDE and designing a measurement campaign regards battery lifetime for an existing NB-IoT device. 

The MDE was designed based on an existing device emulator, which was modified to emulate multiple devices simultaneously. The baseline emulator was not fully developed, which set certain limitations for the MDE. The baseline was modified based on a previous project, with some small changes for added flexibility. The MDE was tested with the Amarisoft LTE 100 basestation and during the testing phase the maximum number of devices achieved to receive msg2 was 12. This is somewhat lower than expected and cause of this was not discovered as the PC utilization does not seem to be the limiting factor. But the concept used for the previous project is proved to also work for other protocols and with the PC not being the limiting factor, the number of device possible to achieve, comparing to the previous project using LTE, should be higher for NB-IoT.

During the test campaign of the energy consumption, the focus was on four parameters: CP-format, frequency, operation-mode and transmit power. From this it was found that only the transmit power has any real influence on the battery lifetime. A few issues was discovered in the model used, which leads to some uncertainties about the result. The two main concerns is the merging of TX and RX in the measurement and the measurement of the attach procedure. The result from the test shows that with a transmit interval of 24 hours the maximum transmit duration must not exceed 0.468 seconds, which is quite unrealistic if a high number of repetitions is used. 

The extent to which each domain is investigated makes it near impossible to evaluate, if the NB-IoT protocol is optimal to become the primary protocol for LPWA devices. It was found that in order to do this further work is needed in both domains, for the MDE this is mainly concerning the fact that it is only able to get to msg2 of the NRAP. To get any real information about the feasibility of the NB-IoT protocol, when subjected to several thousands devices, it needs to be able to make a full attach and to use either the control plane or user plane for data transmission, preferably both. For the energy consumption domain, the issue is mainly the lack of information surrounding the transmit duration. The cause of this could be tracked back to the lack of investigation in to several cell parameters. Also the model should be re-evaluated and probably extended a bit to increase the reliability of the results. 




