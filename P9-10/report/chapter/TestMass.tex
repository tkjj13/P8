\chapter{Testing of Massive IoT system} \label{ch:mass_test}
\todo{could we instead of saying code all the time use emulator, baseline emulator and massive emulator}
The focus of this chapter is to showcase the emulator described in \autoref{ch:MassOver}. This is done in a series of test, where it is compared to the original code and to see if the changes made fulfil the goal set in \autoref{ch:MassOver}:

\textit{The goal of these changes will be to have a massive amount of individual devices emulated, without them effecting each other through their processing time and combining their signals and transmit it as one to the eNB.}
\todo{massive amount of devices is untestable, goal needs to be more precise, allways transmit as one or just using the same radio interface}

The is is done by comparing the changed code to the baseline code as well as testing the performance of the code at higher number of devices emulated.
The performance criteria that will be look into is:

\begin{itemize}
\item Error rate
\item Execution time
\item CPU usage
\item Memory usage
\end{itemize}

The error rate will showcase the stability of the code. This is done by running the code multiple times and to analyze the errors occurring. The errors can be analyzed from the log files.

The execution time is looked into to see if any processes in the code is taking longer, after the changes. This is measured by logging the time stamps, where the code goes from one step to another. 

CPU and memory usage is used to test for possible bottlenecks. CPU usage will be measure with the CPU stat tool, which can measure the CPU usage on the individual processes at a sample rate of 3 Hz. The memory usage is measured using the system monitor tool, as all bigger buffers is allocated in the initialization and the used memory therefore is nearly static. 

The parameters used for the eNB and SRS code is the default settings shown in \todo{Insert app ref to the two apps}. The emulator is executed 20 times each time the number of devices is increased until the emulator hit 100\% in error rate multiple times in a row. Besides testing the emulator with the changed code, the baseline code is also tested, this is what the code is matched against. Note that baseline code can only emulate one device, the results will be extrapolated for comparison with multiple devices though.

\section{Error rate}
\label{sec:MTerror}
The error rate is analyzed from the log messages coming from the emulator on its process through the code. These result can be seen in \autoref{fig:MT_error}.

\begin{figure}[H]
\tikzsetnextfilename{MT_error}
\centering
\resizebox{0.5\textwidth}{!}{
% This file was created by matlab2tikz.
%
%The latest updates can be retrieved from
%  http://www.mathworks.com/matlabcentral/fileexchange/22022-matlab2tikz-matlab2tikz
%where you can also make suggestions and rate matlab2tikz.
%
\definecolor{mycolor1}{rgb}{0.00000,0.44700,0.74100}%
\definecolor{mycolor2}{rgb}{0.85000,0.32500,0.09800}%
%
\begin{tikzpicture}

\begin{axis}[%
width=4.521in,
height=3.566in,
at={(0.758in,0.481in)},
scale only axis,
xmin=1,
xmax=15,
xlabel={Number of UEs},
ymin=0,
ymax=100,
ylabel={Error rate (\%)},
axis background/.style={fill=white},
title style={font=\bfseries},
title={Error rate},
legend style={at={(0.03,0.97)},anchor=north west,legend cell align=left,align=left,draw=white!15!black}
]
\addplot [color=mycolor1,solid]
  table[row sep=crcr]{%
1	25\\
2	40\\
3	25\\
4	15\\
5	40\\
6	30\\
7	30\\
8	20\\
9	15\\
10	50\\
11	30\\
12	20\\
13	100\\
14	100\\
15	100\\
};
\addlegendentry{Change code};

\addplot [color=mycolor2,solid]
  table[row sep=crcr]
\caption{Error rate for different amount of devices.}
\label{fig:MT_error}
\end{figure}
\todo{is it possible to order the errors in the order the could occur instead of what looks like a random order}

As it can be seen in \autoref{fig:MT_error}, the error rate is higher for the changed code, than the baseline that still is not perfect. Another important aspect is that when the number of devices hit 13, the error rate goes to 100\%, which shows the limit of the emulator. A more detailed overview of these error show that six different types of errors occur during the testing as shown in \autoref{fig:MT_error_dist}.

\begin{figure}[H]
\tikzsetnextfilename{MT_error_dist}
\centering
\resizebox{0.9\textwidth}{!}{
% This file was created by matlab2tikz.
%
%The latest updates can be retrieved from
%  http://www.mathworks.com/matlabcentral/fileexchange/22022-matlab2tikz-matlab2tikz
%where you can also make suggestions and rate matlab2tikz.
%
\definecolor{mycolor1}{rgb}{0.20810,0.16630,0.52920}%
\definecolor{mycolor2}{rgb}{0.01651,0.42660,0.87863}%
\definecolor{mycolor3}{rgb}{0.02650,0.61370,0.81350}%
\definecolor{mycolor4}{rgb}{0.21783,0.72504,0.61926}%
\definecolor{mycolor5}{rgb}{0.64730,0.74560,0.41880}%
\definecolor{mycolor6}{rgb}{0.99377,0.74546,0.24035}%
\definecolor{mycolor7}{rgb}{0.97630,0.98310,0.05380}%
%
\begin{tikzpicture}

\begin{axis}[%
width=.8\textwidth,
height=.533\textwidth,
at={(0.533in,0.481in)},
scale only axis,
bar width=0.5,
xmin=-0.5,
xmax=15.5,
xtick={0,1,2,3,4,5,6,7,8,9,10,11,12,13,14,15},
xticklabels={{Base},{1},{2},{3},{4},{5},{6},{7},{8},{9},{10},{11},{12},{13},{14},{15}},
xlabel={Number of devices},
ymin=0,
ymax=100,
ylabel={Trials},
axis background/.style={fill=white},
title style={font=\bfseries},
title={Error distrubution},
legend style={at={(1.03,1)},anchor=north west,legend cell align=left,align=left,draw=white!15!black}
]
\addplot[ybar stacked,draw=black,fill=mycolor1,area legend] plot table[row sep=crcr] {%
0	90\\
1	75\\
2	60\\
3	75\\
4	85\\
5	60\\
6	70\\
7	70\\
8	80\\
9	85\\
10	50\\
11	70\\
12	80\\
13	0\\
14	0\\
15	0\\
};
\addlegendentry{No errors};

\addplot[ybar stacked,draw=black,fill=mycolor2,area legend] plot table[row sep=crcr] {%
0	0\\
1	0\\
2	0\\
3	0\\
4	0\\
5	0\\
6	0\\
7	0\\
8	5\\
9	0\\
10	0\\
11	0\\
12	0\\
13	5\\
14	0\\
15	0\\
};
\addlegendentry{Cell sync error};

\addplot[ybar stacked,draw=black,fill=mycolor3,area legend] plot table[row sep=crcr] {%
0	10\\
1	10\\
2	10\\
3	10\\
4	0\\
5	15\\
6	15\\
7	5\\
8	10\\
9	5\\
10	15\\
11	15\\
12	10\\
13	0\\
14	15\\
15	5\\
};
\addlegendentry{Radio error};

\addplot[ybar stacked,draw=black,fill=mycolor4,area legend] plot table[row sep=crcr] {%
0	0\\
1	10\\
2	30\\
3	15\\
4	15\\
5	25\\
6	10\\
7	25\\
8	5\\
9	5\\
10	30\\
11	5\\
12	5\\
13	80\\
14	65\\
15	60\\
};
\addlegendentry{Idle after MIB-NB};

\addplot[ybar stacked,draw=black,fill=mycolor5,area legend] plot table[row sep=crcr] {%
0	0\\
1	0\\
2	0\\
3	0\\
4	0\\
5	0\\
6	0\\
7	0\\
8	0\\
9	5\\
10	5\\
11	10\\
12	5\\
13	5\\
14	10\\
15	20\\
};
\addlegendentry{Transmission after NB-SIB1};

\addplot[ybar stacked,draw=black,fill=mycolor6,area legend] plot table[row sep=crcr] {%
0	0\\
1	0\\
2	0\\
3	0\\
4	0\\
5	0\\
6	5\\
7	0\\
8	0\\
9	0\\
10	0\\
11	0\\
12	0\\
13	10\\
14	10\\
15	10\\
};
\addlegendentry{NPRACH error};

\addplot[ybar stacked,draw=black,fill=mycolor7,area legend] plot table[row sep=crcr] 
\caption{The distribution of different errors for different amount of devices.}
\label{fig:MT_error_dist}
\end{figure}

Radio error comes from miscommunication between the radio class and the API for the USRP B210. It occur when the process begins the search after \todo{for or after} SIB messages, where some radio parameters is changed. As the radio error is the only error occurring for the baseline, this should be the only error that is not produced by the changed made to the baseline.

The idle after MIB error occur in the same part of the process as the radio error, where the emulator gets stuck and runs without retrying or closing down. This error type is the most occurring type of the errors produced by the changes, which will be the most optimal place to improve the error rate, especially for higher amount of devices.

The msg2 not received error is when all devices have gone through the NPRACH step and waiting on the msg2, which is never received or not registered if received. This error is very rare and have no tendencies.

Cell sync error occur when the emulator shuts down before the device has synchronized to the cell and it goes to the MIB step. This error is very rare, but as the error occurs so early in the process, these test runs will not give any data for any other step, beside initialization.

Transmission after SIB1 is an error that occurs sometimes at the SIB1 step and the radio class transmit a signal, which causes the emulator to shut down. This error type have a tendency to occur with higher amount of devices and can indicate a bottleneck in the process to get a higher amount of devices.

NPRACH error is an error that occurs when some devices completes the NPRACH step, but other do not, as the system shuts down beforehand. This error type have the same tendency as the transmission after SIB1 error and also indicates a potential bottleneck.

To further test the emulator, the test will be split up into the different steps in the code process discussed in \autoref{sub:MassStruct}.

\todo{keep the structure from your list it helps the reader know how far we are}
\section{Initialization}
The execution time for the initialization step is measured dependably on the number of devices emulated and the baseline is measured as well, which gives the results seen in \autoref{fig:MT_Init_Time}.


\begin{figure}[H]
\tikzsetnextfilename{MT_Init_Time}
\centering
\resizebox{0.7\textwidth}{!}{
% This file was created by matlab2tikz.
%
%The latest updates can be retrieved from
%  http://www.mathworks.com/matlabcentral/fileexchange/22022-matlab2tikz-matlab2tikz
%where you can also make suggestions and rate matlab2tikz.
%
\definecolor{mycolor1}{rgb}{0.00000,0.44700,0.74100}%
\definecolor{mycolor2}{rgb}{0.85000,0.32500,0.09800}%
%
\begin{tikzpicture}

\begin{axis}[%
width=4.521in,
height=3.566in,
at={(0.758in,0.481in)},
scale only axis,
xmin=-0.5,
xmax=15.5,
xtick={0,1,2,3,4,5,6,7,8,9,10,11,12,13,14,15},
xticklabels={{Baseline},{1},{2},{3},{4},{5},{6},{7},{8},{9},{10},{11},{12},{13},{14},{15}},
xlabel={Number of UEs},
ymin=1.4,
ymax=3.2,
ylabel={Time (s)},
axis background/.style={fill=white},
title style={font=\bfseries},
title={Initialization time},
legend style={at={(0.03,0.97)},anchor=north west,legend cell align=left,align=left,draw=white!15!black}
]
\addplot [color=mycolor1,only marks,mark=*,mark options={solid}]
  table[row sep=crcr]{%
1	1.60657000541687\\
1	1.60206508636475\\
1	1.60402417182922\\
1	1.60998201370239\\
1	1.59828901290894\\
1	1.59890818595886\\
1	1.60678696632385\\
1	1.70481204986572\\
1	1.60345387458801\\
1	1.59799599647522\\
1	1.60137987136841\\
1	1.60408782958984\\
1	1.60380721092224\\
1	1.61764979362488\\
1	1.61315512657166\\
1	1.60408186912537\\
1	1.60214781761169\\
1	1.59820103645325\\
1	1.70323586463928\\
1	1.60239100456238\\
};
\addlegendentry{Change code};

\addplot [color=mycolor1,only marks,mark=*,mark options={solid}]
  table[row sep=crcr]{%
2	1.70376014709473\\
2	1.69752597808838\\
2	1.69991588592529\\
2	1.70224094390869\\
2	1.70335793495178\\
2	1.70019602775574\\
2	1.81223583221436\\
2	1.70259284973145\\
2	1.70058512687683\\
2	1.90431618690491\\
2	1.69746613502502\\
2	1.70056891441345\\
2	1.7057900428772\\
2	1.6926589012146\\
2	1.69660997390747\\
2	1.69284415245056\\
2	1.70735216140747\\
2	1.69585204124451\\
2	1.69405889511108\\
2	1.70744895935059\\
};
\addlegendentry{Baseline};

\addplot [color=mycolor1,only marks,mark=*,mark options={solid}]
  table[row sep=crcr]{%
3	1.79403805732727\\
3	1.80174088478088\\
3	1.80632400512695\\
3	1.79517197608948\\
3	1.7999119758606\\
3	1.80256199836731\\
3	1.79768395423889\\
3	1.79976677894592\\
3	1.79294800758362\\
3	1.80160784721375\\
3	1.81014108657837\\
3	1.8038489818573\\
3	1.79714298248291\\
3	1.78692388534546\\
3	1.80100011825562\\
3	1.79488515853882\\
3	1.81216788291931\\
3	1.79761385917664\\
3	1.80774402618408\\
3	1.79226994514465\\
};
\addlegendentry{Fitted line};

\addplot [color=mycolor1,only marks,mark=*,mark options={solid},forget plot]
  table[row sep=crcr]{%
4	1.98840284347534\\
4	1.98891401290894\\
4	1.90781497955322\\
4	1.90410208702087\\
4	1.89322519302368\\
4	1.89481401443481\\
4	1.89260292053223\\
4	1.89337205886841\\
4	1.89051008224487\\
4	1.99312591552734\\
4	1.89608120918274\\
4	1.98998594284058\\
4	1.9188380241394\\
4	1.88647603988647\\
4	1.89477205276489\\
4	1.89825892448425\\
4	1.90163683891296\\
4	1.89038610458374\\
4	1.91325998306274\\
4	1.88523697853088\\
};
\addplot [color=mycolor1,only marks,mark=*,mark options={solid},forget plot]
  table[row sep=crcr]{%
5	1.99140405654907\\
5	1.98323607444763\\
5	2.10928916931152\\
5	1.97859597206116\\
5	2.08593010902405\\
5	1.97863793373108\\
5	1.99062418937683\\
5	2.07845187187195\\
5	1.99165511131287\\
5	1.97789597511292\\
5	1.98802804946899\\
5	1.9954240322113\\
5	2.00401997566223\\
5	1.98706889152527\\
5	1.98496317863464\\
5	1.97689509391785\\
5	1.98235487937927\\
5	1.98183512687683\\
5	2.10645198822021\\
5	1.97618389129639\\
};
\addplot [color=mycolor1,only marks,mark=*,mark options={solid},forget plot]
  table[row sep=crcr]{%
6	2.07043600082397\\
6	2.08795213699341\\
6	2.08343291282654\\
6	2.09298491477966\\
6	2.07702803611755\\
6	2.07426118850708\\
6	2.1002459526062\\
6	2.09029412269592\\
6	2.08507609367371\\
6	2.07489895820618\\
6	2.08115887641907\\
6	2.09077501296997\\
6	2.08771395683289\\
6	2.07395005226135\\
6	2.07281303405762\\
6	2.1818699836731\\
6	2.08554911613464\\
6	2.0755889415741\\
6	2.09108996391296\\
6	2.08529710769653\\
};
\addplot [color=mycolor1,only marks,mark=*,mark options={solid},forget plot]
  table[row sep=crcr]{%
7	2.16757988929749\\
7	2.16262793540955\\
7	2.17564296722412\\
7	2.1759250164032\\
7	2.19155502319336\\
7	2.27151203155518\\
7	2.1676127910614\\
7	2.17567276954651\\
7	2.17186117172241\\
7	2.29317903518677\\
7	2.17238211631775\\
7	2.17540311813354\\
7	2.17659687995911\\
7	2.16815400123596\\
7	2.17439103126526\\
7	2.17733693122864\\
7	2.17026114463806\\
7	2.17389607429504\\
7	2.19961309432983\\
7	2.1946439743042\\
};
\addplot [color=mycolor1,only marks,mark=*,mark options={solid},forget plot]
  table[row sep=crcr]{%
8	2.27100706100464\\
8	2.26513481140137\\
8	2.26285195350647\\
8	2.2616548538208\\
8	2.27730703353882\\
8	2.27119493484497\\
8	2.26165103912354\\
8	2.25236415863037\\
8	2.26101899147034\\
8	2.36273288726807\\
8	2.25905799865723\\
8	2.27876305580139\\
8	2.29212021827698\\
8	2.26190590858459\\
8	2.27887606620789\\
8	2.25859999656677\\
8	2.26044893264771\\
8	2.26256895065308\\
8	2.26363897323608\\
8	2.26199913024902\\
};
\addplot [color=mycolor1,only marks,mark=*,mark options={solid},forget plot]
  table[row sep=crcr]{%
9	2.35985803604126\\
9	2.35758590698242\\
9	2.35490608215332\\
9	2.35805201530457\\
9	2.35813903808594\\
9	2.36619591712952\\
9	2.36291694641113\\
9	2.3615140914917\\
9	2.36494493484497\\
9	2.35645508766174\\
9	2.36352205276489\\
9	2.36592721939087\\
9	2.46607184410095\\
9	2.46248698234558\\
9	2.35874915122986\\
9	2.35979104042053\\
9	2.36403608322144\\
9	2.35437083244324\\
9	2.45976400375366\\
9	2.35649704933167\\
};
\addplot [color=mycolor1,only marks,mark=*,mark options={solid},forget plot]
  table[row sep=crcr]{%
10	2.4522500038147\\
10	2.45602107048035\\
10	2.45017790794373\\
10	2.46005892753601\\
10	2.4437301158905\\
10	2.45332884788513\\
10	2.45075607299805\\
10	2.44673299789429\\
10	2.44251298904419\\
10	2.45762705802917\\
10	2.45294904708862\\
10	2.48201680183411\\
10	2.45067620277405\\
10	2.57340097427368\\
10	2.44935178756714\\
10	2.45021200180054\\
10	2.55899405479431\\
10	2.55124402046204\\
10	2.45972394943237\\
10	2.4552149772644\\
};
\addplot [color=mycolor1,only marks,mark=*,mark options={solid},forget plot]
  table[row sep=crcr]{%
11	2.54409885406494\\
11	2.5580689907074\\
11	2.55131316184998\\
11	2.54538011550903\\
11	2.54991102218628\\
11	2.54586219787598\\
11	2.54519009590149\\
11	2.54685091972351\\
11	2.54958415031433\\
11	2.55115580558777\\
11	2.55340099334717\\
11	2.54899501800537\\
11	2.54859089851379\\
11	2.55042195320129\\
11	2.54939794540405\\
11	2.57132411003113\\
11	2.54307293891907\\
11	2.54036903381348\\
11	2.54952502250671\\
11	2.56598615646362\\
};
\addplot [color=mycolor1,only marks,mark=*,mark options={solid},forget plot]
  table[row sep=crcr]{%
12	2.63736891746521\\
12	2.6386501789093\\
12	2.63579487800598\\
12	2.76004385948181\\
12	2.64884686470032\\
12	2.63978505134583\\
12	2.65288496017456\\
12	2.6491858959198\\
12	2.64470791816711\\
12	2.65107917785645\\
12	2.83676886558533\\
12	2.63889098167419\\
12	2.65938687324524\\
12	2.6382257938385\\
12	2.63785290718079\\
12	2.65852999687195\\
12	2.74441695213318\\
12	2.64707398414612\\
12	2.63858604431152\\
12	2.65901207923889\\
};
\addplot [color=mycolor1,only marks,mark=*,mark options={solid},forget plot]
  table[row sep=crcr]{%
13	2.73925495147705\\
13	2.76844787597656\\
13	2.74468803405762\\
13	2.72794198989868\\
13	2.73288083076477\\
13	2.72759079933167\\
13	2.73981499671936\\
13	2.73808693885803\\
13	2.73037505149841\\
13	2.73060989379883\\
13	2.75299692153931\\
13	2.73150610923767\\
13	2.73392391204834\\
13	2.83988189697266\\
13	2.73668503761292\\
13	2.73319101333618\\
13	2.7269880771637\\
13	2.73366594314575\\
13	2.73546814918518\\
13	2.73145604133606\\
};
\addplot [color=mycolor1,only marks,mark=*,mark options={solid},forget plot]
  table[row sep=crcr]{%
14	2.93708896636963\\
14	2.8230938911438\\
14	2.82491183280945\\
14	2.81724190711975\\
14	2.83096098899841\\
14	2.82523608207703\\
14	2.92713284492493\\
14	2.83528399467468\\
14	2.81740093231201\\
14	2.94761514663696\\
14	2.84746098518372\\
14	2.92244791984558\\
14	2.82092189788818\\
14	2.82772517204285\\
14	2.82116103172302\\
14	2.8086531162262\\
14	2.82523202896118\\
14	2.82433700561523\\
14	2.85196495056152\\
14	2.82061409950256\\
};
\addplot [color=mycolor1,only marks,mark=*,mark options={solid},forget plot]
  table[row sep=crcr]{%
15	2.91355895996094\\
15	2.92956900596619\\
15	2.92799687385559\\
15	3.0291600227356\\
15	2.92018890380859\\
15	2.92904806137085\\
15	2.91579508781433\\
15	2.91569089889526\\
15	2.92359805107117\\
15	2.91793894767761\\
15	2.9355640411377\\
15	3.12653303146362\\
15	2.91362810134888\\
15	2.94397902488709\\
15	2.9135410785675\\
15	2.91744613647461\\
15	2.92062401771545\\
15	2.91158103942871\\
15	3.02637410163879\\
15	2.90584707260132\\
};
\addplot [color=mycolor2,only marks,mark=*,mark options={solid},forget plot]
  table[row sep=crcr]{%
0	1.51574110984802\\
0	1.60954785346985\\
0	1.61224007606506\\
0	1.51666712760925\\
0	1.51392078399658\\
0	1.51035499572754\\
0	1.50713300704956\\
0	1.51292109489441\\
0	1.51016092300415\\
0	1.5099790096283\\
0	1.51415991783142\\
0	1.51640319824219\\
0	1.51323890686035\\
0	1.51066589355469\\
0	1.5165901184082\\
0	1.50853300094604\\
0	1.50311899185181\\
0	1.60737919807434\\
0	1.51729917526245\\
0	1.50886988639832\\
};
\addplot [color=red,solid,forget plot]
  table[row sep=crcr]{%
0	1.52591009015129\\
0.015	1.52732309482467\\
0.03	1.52873609949805\\
0.045	1.53014910417142\\
0.06	1.5315621088448\\
0.075	1.53297511351818\\
0.09	1.53438811819156\\
0.105	1.53580112286494\\
0.12	1.53721412753832\\
0.135	1.5386271322117\\
0.15	1.54004013688508\\
0.165	1.54145314155845\\
0.18	1.54286614623183\\
0.195	1.54427915090521\\
0.21	1.54569215557859\\
0.225	1.54710516025197\\
0.24	1.54851816492535\\
0.255	1.54993116959873\\
0.27	1.55134417427211\\
0.285	1.55275717894549\\
0.3	1.55417018361886\\
0.315	1.55558318829224\\
0.33	1.55699619296562\\
0.345	1.558409197639\\
0.36	1.55982220231238\\
0.375	1.56123520698576\\
0.39	1.56264821165914\\
0.405	1.56406121633252\\
0.42	1.56547422100589\\
0.435	1.56688722567927\\
0.45	1.56830023035265\\
0.465	1.56971323502603\\
0.48	1.57112623969941\\
0.495	1.57253924437279\\
0.51	1.57395224904617\\
0.525	1.57536525371955\\
0.54	1.57677825839293\\
0.555	1.5781912630663\\
0.57	1.57960426773968\\
0.585	1.58101727241306\\
0.6	1.58243027708644\\
0.615	1.58384328175982\\
0.63	1.5852562864332\\
0.645	1.58666929110658\\
0.66	1.58808229577996\\
0.675	1.58949530045333\\
0.69	1.59090830512671\\
0.705	1.59232130980009\\
0.72	1.59373431447347\\
0.735	1.59514731914685\\
0.75	1.59656032382023\\
0.765	1.59797332849361\\
0.78	1.59938633316699\\
0.795	1.60079933784036\\
0.81	1.60221234251374\\
0.825	1.60362534718712\\
0.84	1.6050383518605\\
0.855	1.60645135653388\\
0.87	1.60786436120726\\
0.885	1.60927736588064\\
0.9	1.61069037055402\\
0.915	1.6121033752274\\
0.93	1.61351637990077\\
0.945	1.61492938457415\\
0.96	1.61634238924753\\
0.975	1.61775539392091\\
0.99	1.61916839859429\\
1.005	1.62058140326767\\
1.02	1.62199440794105\\
1.035	1.62340741261443\\
1.05	1.6248204172878\\
1.065	1.62623342196118\\
1.08	1.62764642663456\\
1.095	1.62905943130794\\
1.11	1.63047243598132\\
1.125	1.6318854406547\\
1.14	1.63329844532808\\
1.155	1.63471145000146\\
1.17	1.63612445467484\\
1.185	1.63753745934821\\
1.2	1.63895046402159\\
1.215	1.64036346869497\\
1.23	1.64177647336835\\
1.245	1.64318947804173\\
1.26	1.64460248271511\\
1.275	1.64601548738849\\
1.29	1.64742849206187\\
1.305	1.64884149673524\\
1.32	1.65025450140862\\
1.335	1.651667506082\\
1.35	1.65308051075538\\
1.365	1.65449351542876\\
1.38	1.65590652010214\\
1.395	1.65731952477552\\
1.41	1.6587325294489\\
1.425	1.66014553412227\\
1.44	1.66155853879565\\
1.455	1.66297154346903\\
1.47	1.66438454814241\\
1.485	1.66579755281579\\
1.5	1.66721055748917\\
1.515	1.66862356216255\\
1.53	1.67003656683593\\
1.545	1.67144957150931\\
1.56	1.67286257618268\\
1.575	1.67427558085606\\
1.59	1.67568858552944\\
1.605	1.67710159020282\\
1.62	1.6785145948762\\
1.635	1.67992759954958\\
1.65	1.68134060422296\\
1.665	1.68275360889634\\
1.68	1.68416661356971\\
1.695	1.68557961824309\\
1.71	1.68699262291647\\
1.725	1.68840562758985\\
1.74	1.68981863226323\\
1.755	1.69123163693661\\
1.77	1.69264464160999\\
1.785	1.69405764628337\\
1.8	1.69547065095674\\
1.815	1.69688365563012\\
1.83	1.6982966603035\\
1.845	1.69970966497688\\
1.86	1.70112266965026\\
1.875	1.70253567432364\\
1.89	1.70394867899702\\
1.905	1.7053616836704\\
1.92	1.70677468834378\\
1.935	1.70818769301715\\
1.95	1.70960069769053\\
1.965	1.71101370236391\\
1.98	1.71242670703729\\
1.995	1.71383971171067\\
2.01	1.71525271638405\\
2.025	1.71666572105743\\
2.04	1.71807872573081\\
2.055	1.71949173040418\\
2.07	1.72090473507756\\
2.085	1.72231773975094\\
2.1	1.72373074442432\\
2.115	1.7251437490977\\
2.13	1.72655675377108\\
2.145	1.72796975844446\\
2.16	1.72938276311784\\
2.175	1.73079576779121\\
2.19	1.73220877246459\\
2.205	1.73362177713797\\
2.22	1.73503478181135\\
2.235	1.73644778648473\\
2.25	1.73786079115811\\
2.265	1.73927379583149\\
2.28	1.74068680050487\\
2.295	1.74209980517825\\
2.31	1.74351280985162\\
2.325	1.744925814525\\
2.34	1.74633881919838\\
2.355	1.74775182387176\\
2.37	1.74916482854514\\
2.385	1.75057783321852\\
2.4	1.7519908378919\\
2.415	1.75340384256528\\
2.43	1.75481684723865\\
2.445	1.75622985191203\\
2.46	1.75764285658541\\
2.475	1.75905586125879\\
2.49	1.76046886593217\\
2.505	1.76188187060555\\
2.52	1.76329487527893\\
2.535	1.76470787995231\\
2.55	1.76612088462569\\
2.565	1.76753388929906\\
2.58	1.76894689397244\\
2.595	1.77035989864582\\
2.61	1.7717729033192\\
2.625	1.77318590799258\\
2.64	1.77459891266596\\
2.655	1.77601191733934\\
2.67	1.77742492201272\\
2.685	1.77883792668609\\
2.7	1.78025093135947\\
2.715	1.78166393603285\\
2.73	1.78307694070623\\
2.745	1.78448994537961\\
2.76	1.78590295005299\\
2.775	1.78731595472637\\
2.79	1.78872895939975\\
2.805	1.79014196407312\\
2.82	1.7915549687465\\
2.835	1.79296797341988\\
2.85	1.79438097809326\\
2.865	1.79579398276664\\
2.88	1.79720698744002\\
2.895	1.7986199921134\\
2.91	1.80003299678678\\
2.925	1.80144600146016\\
2.94	1.80285900613353\\
2.955	1.80427201080691\\
2.97	1.80568501548029\\
2.985	1.80709802015367\\
3	1.80851102482705\\
3.015	1.80992402950043\\
3.03	1.81133703417381\\
3.045	1.81275003884719\\
3.06	1.81416304352056\\
3.075	1.81557604819394\\
3.09	1.81698905286732\\
3.105	1.8184020575407\\
3.12	1.81981506221408\\
3.135	1.82122806688746\\
3.15	1.82264107156084\\
3.165	1.82405407623422\\
3.18	1.8254670809076\\
3.195	1.82688008558097\\
3.21	1.82829309025435\\
3.225	1.82970609492773\\
3.24	1.83111909960111\\
3.255	1.83253210427449\\
3.27	1.83394510894787\\
3.285	1.83535811362125\\
3.3	1.83677111829463\\
3.315	1.838184122968\\
3.33	1.83959712764138\\
3.345	1.84101013231476\\
3.36	1.84242313698814\\
3.375	1.84383614166152\\
3.39	1.8452491463349\\
3.405	1.84666215100828\\
3.42	1.84807515568166\\
3.435	1.84948816035503\\
3.45	1.85090116502841\\
3.465	1.85231416970179\\
3.48	1.85372717437517\\
3.495	1.85514017904855\\
3.51	1.85655318372193\\
3.525	1.85796618839531\\
3.54	1.85937919306869\\
3.555	1.86079219774207\\
3.57	1.86220520241544\\
3.585	1.86361820708882\\
3.6	1.8650312117622\\
3.615	1.86644421643558\\
3.63	1.86785722110896\\
3.645	1.86927022578234\\
3.66	1.87068323045572\\
3.675	1.8720962351291\\
3.69	1.87350923980247\\
3.705	1.87492224447585\\
3.72	1.87633524914923\\
3.735	1.87774825382261\\
3.75	1.87916125849599\\
3.765	1.88057426316937\\
3.78	1.88198726784275\\
3.795	1.88340027251613\\
3.81	1.88481327718951\\
3.825	1.88622628186288\\
3.84	1.88763928653626\\
3.855	1.88905229120964\\
3.87	1.89046529588302\\
3.885	1.8918783005564\\
3.9	1.89329130522978\\
3.915	1.89470430990316\\
3.93	1.89611731457654\\
3.945	1.89753031924991\\
3.96	1.89894332392329\\
3.975	1.90035632859667\\
3.99	1.90176933327005\\
4.005	1.90318233794343\\
4.02	1.90459534261681\\
4.035	1.90600834729019\\
4.05	1.90742135196357\\
4.065	1.90883435663694\\
4.08	1.91024736131032\\
4.095	1.9116603659837\\
4.11	1.91307337065708\\
4.125	1.91448637533046\\
4.14	1.91589938000384\\
4.155	1.91731238467722\\
4.17	1.9187253893506\\
4.185	1.92013839402397\\
4.2	1.92155139869735\\
4.215	1.92296440337073\\
4.23	1.92437740804411\\
4.245	1.92579041271749\\
4.26	1.92720341739087\\
4.275	1.92861642206425\\
4.29	1.93002942673763\\
4.305	1.93144243141101\\
4.32	1.93285543608438\\
4.335	1.93426844075776\\
4.35	1.93568144543114\\
4.365	1.93709445010452\\
4.38	1.9385074547779\\
4.395	1.93992045945128\\
4.41	1.94133346412466\\
4.425	1.94274646879804\\
4.44	1.94415947347141\\
4.455	1.94557247814479\\
4.47	1.94698548281817\\
4.485	1.94839848749155\\
4.5	1.94981149216493\\
4.515	1.95122449683831\\
4.53	1.95263750151169\\
4.545	1.95405050618507\\
4.56	1.95546351085845\\
4.575	1.95687651553182\\
4.59	1.9582895202052\\
4.605	1.95970252487858\\
4.62	1.96111552955196\\
4.635	1.96252853422534\\
4.65	1.96394153889872\\
4.665	1.9653545435721\\
4.68	1.96676754824548\\
4.695	1.96818055291885\\
4.71	1.96959355759223\\
4.725	1.97100656226561\\
4.74	1.97241956693899\\
4.755	1.97383257161237\\
4.77	1.97524557628575\\
4.785	1.97665858095913\\
4.8	1.97807158563251\\
4.815	1.97948459030588\\
4.83	1.98089759497926\\
4.845	1.98231059965264\\
4.86	1.98372360432602\\
4.875	1.9851366089994\\
4.89	1.98654961367278\\
4.905	1.98796261834616\\
4.92	1.98937562301954\\
4.935	1.99078862769292\\
4.95	1.99220163236629\\
4.965	1.99361463703967\\
4.98	1.99502764171305\\
4.995	1.99644064638643\\
5.01	1.99785365105981\\
5.025	1.99926665573319\\
5.04	2.00067966040657\\
5.055	2.00209266507995\\
5.07	2.00350566975332\\
5.085	2.0049186744267\\
5.1	2.00633167910008\\
5.115	2.00774468377346\\
5.13	2.00915768844684\\
5.145	2.01057069312022\\
5.16	2.0119836977936\\
5.175	2.01339670246698\\
5.19	2.01480970714035\\
5.205	2.01622271181373\\
5.22	2.01763571648711\\
5.235	2.01904872116049\\
5.25	2.02046172583387\\
5.265	2.02187473050725\\
5.28	2.02328773518063\\
5.295	2.02470073985401\\
5.31	2.02611374452739\\
5.325	2.02752674920076\\
5.34	2.02893975387414\\
5.355	2.03035275854752\\
5.37	2.0317657632209\\
5.385	2.03317876789428\\
5.4	2.03459177256766\\
5.415	2.03600477724104\\
5.43	2.03741778191442\\
5.445	2.03883078658779\\
5.46	2.04024379126117\\
5.475	2.04165679593455\\
5.49	2.04306980060793\\
5.505	2.04448280528131\\
5.52	2.04589580995469\\
5.535	2.04730881462807\\
5.55	2.04872181930145\\
5.565	2.05013482397483\\
5.58	2.0515478286482\\
5.595	2.05296083332158\\
5.61	2.05437383799496\\
5.625	2.05578684266834\\
5.64	2.05719984734172\\
5.655	2.0586128520151\\
5.67	2.06002585668848\\
5.685	2.06143886136186\\
5.7	2.06285186603523\\
5.715	2.06426487070861\\
5.73	2.06567787538199\\
5.745	2.06709088005537\\
5.76	2.06850388472875\\
5.775	2.06991688940213\\
5.79	2.07132989407551\\
5.805	2.07274289874889\\
5.82	2.07415590342226\\
5.835	2.07556890809564\\
5.85	2.07698191276902\\
5.865	2.0783949174424\\
5.88	2.07980792211578\\
5.895	2.08122092678916\\
5.91	2.08263393146254\\
5.925	2.08404693613592\\
5.94	2.0854599408093\\
5.955	2.08687294548267\\
5.97	2.08828595015605\\
5.985	2.08969895482943\\
6	2.09111195950281\\
6.015	2.09252496417619\\
6.03	2.09393796884957\\
6.045	2.09535097352295\\
6.06	2.09676397819633\\
6.075	2.0981769828697\\
6.09	2.09958998754308\\
6.105	2.10100299221646\\
6.12	2.10241599688984\\
6.135	2.10382900156322\\
6.15	2.1052420062366\\
6.165	2.10665501090998\\
6.18	2.10806801558336\\
6.195	2.10948102025674\\
6.21	2.11089402493011\\
6.225	2.11230702960349\\
6.24	2.11372003427687\\
6.255	2.11513303895025\\
6.27	2.11654604362363\\
6.285	2.11795904829701\\
6.3	2.11937205297039\\
6.315	2.12078505764377\\
6.33	2.12219806231714\\
6.345	2.12361106699052\\
6.36	2.1250240716639\\
6.375	2.12643707633728\\
6.39	2.12785008101066\\
6.405	2.12926308568404\\
6.42	2.13067609035742\\
6.435	2.1320890950308\\
6.45	2.13350209970417\\
6.465	2.13491510437755\\
6.48	2.13632810905093\\
6.495	2.13774111372431\\
6.51	2.13915411839769\\
6.525	2.14056712307107\\
6.54	2.14198012774445\\
6.555	2.14339313241783\\
6.57	2.14480613709121\\
6.585	2.14621914176458\\
6.6	2.14763214643796\\
6.615	2.14904515111134\\
6.63	2.15045815578472\\
6.645	2.1518711604581\\
6.66	2.15328416513148\\
6.675	2.15469716980486\\
6.69	2.15611017447824\\
6.705	2.15752317915161\\
6.72	2.15893618382499\\
6.735	2.16034918849837\\
6.75	2.16176219317175\\
6.765	2.16317519784513\\
6.78	2.16458820251851\\
6.795	2.16600120719189\\
6.81	2.16741421186527\\
6.825	2.16882721653865\\
6.84	2.17024022121202\\
6.855	2.1716532258854\\
6.87	2.17306623055878\\
6.885	2.17447923523216\\
6.9	2.17589223990554\\
6.915	2.17730524457892\\
6.93	2.1787182492523\\
6.945	2.18013125392568\\
6.96	2.18154425859905\\
6.975	2.18295726327243\\
6.99	2.18437026794581\\
7.005	2.18578327261919\\
7.02	2.18719627729257\\
7.035	2.18860928196595\\
7.05	2.19002228663933\\
7.065	2.19143529131271\\
7.08	2.19284829598608\\
7.095	2.19426130065946\\
7.11	2.19567430533284\\
7.125	2.19708731000622\\
7.14	2.1985003146796\\
7.155	2.19991331935298\\
7.17	2.20132632402636\\
7.185	2.20273932869974\\
7.2	2.20415233337312\\
7.215	2.20556533804649\\
7.23	2.20697834271987\\
7.245	2.20839134739325\\
7.26	2.20980435206663\\
7.275	2.21121735674001\\
7.29	2.21263036141339\\
7.305	2.21404336608677\\
7.32	2.21545637076015\\
7.335	2.21686937543352\\
7.35	2.2182823801069\\
7.365	2.21969538478028\\
7.38	2.22110838945366\\
7.395	2.22252139412704\\
7.41	2.22393439880042\\
7.425	2.2253474034738\\
7.44	2.22676040814718\\
7.455	2.22817341282055\\
7.47	2.22958641749393\\
7.485	2.23099942216731\\
7.5	2.23241242684069\\
7.515	2.23382543151407\\
7.53	2.23523843618745\\
7.545	2.23665144086083\\
7.56	2.23806444553421\\
7.575	2.23947745020759\\
7.59	2.24089045488096\\
7.605	2.24230345955434\\
7.62	2.24371646422772\\
7.635	2.2451294689011\\
7.65	2.24654247357448\\
7.665	2.24795547824786\\
7.68	2.24936848292124\\
7.695	2.25078148759462\\
7.71	2.25219449226799\\
7.725	2.25360749694137\\
7.74	2.25502050161475\\
7.755	2.25643350628813\\
7.77	2.25784651096151\\
7.785	2.25925951563489\\
7.8	2.26067252030827\\
7.815	2.26208552498165\\
7.83	2.26349852965503\\
7.845	2.2649115343284\\
7.86	2.26632453900178\\
7.875	2.26773754367516\\
7.89	2.26915054834854\\
7.905	2.27056355302192\\
7.92	2.2719765576953\\
7.935	2.27338956236868\\
7.95	2.27480256704206\\
7.965	2.27621557171543\\
7.98	2.27762857638881\\
7.995	2.27904158106219\\
8.01	2.28045458573557\\
8.025	2.28186759040895\\
8.04	2.28328059508233\\
8.055	2.28469359975571\\
8.07	2.28610660442909\\
8.085	2.28751960910246\\
8.1	2.28893261377584\\
8.115	2.29034561844922\\
8.13	2.2917586231226\\
8.145	2.29317162779598\\
8.16	2.29458463246936\\
8.175	2.29599763714274\\
8.19	2.29741064181612\\
8.205	2.2988236464895\\
8.22	2.30023665116287\\
8.235	2.30164965583625\\
8.25	2.30306266050963\\
8.265	2.30447566518301\\
8.28	2.30588866985639\\
8.295	2.30730167452977\\
8.31	2.30871467920315\\
8.325	2.31012768387653\\
8.34	2.3115406885499\\
8.355	2.31295369322328\\
8.37	2.31436669789666\\
8.385	2.31577970257004\\
8.4	2.31719270724342\\
8.415	2.3186057119168\\
8.43	2.32001871659018\\
8.445	2.32143172126356\\
8.46	2.32284472593693\\
8.475	2.32425773061031\\
8.49	2.32567073528369\\
8.505	2.32708373995707\\
8.52	2.32849674463045\\
8.535	2.32990974930383\\
8.55	2.33132275397721\\
8.565	2.33273575865059\\
8.58	2.33414876332397\\
8.595	2.33556176799734\\
8.61	2.33697477267072\\
8.625	2.3383877773441\\
8.64	2.33980078201748\\
8.655	2.34121378669086\\
8.67	2.34262679136424\\
8.685	2.34403979603762\\
8.7	2.345452800711\\
8.715	2.34686580538437\\
8.73	2.34827881005775\\
8.745	2.34969181473113\\
8.76	2.35110481940451\\
8.775	2.35251782407789\\
8.79	2.35393082875127\\
8.805	2.35534383342465\\
8.82	2.35675683809803\\
8.835	2.3581698427714\\
8.85	2.35958284744478\\
8.865	2.36099585211816\\
8.88	2.36240885679154\\
8.895	2.36382186146492\\
8.91	2.3652348661383\\
8.925	2.36664787081168\\
8.94	2.36806087548506\\
8.955	2.36947388015844\\
8.97	2.37088688483181\\
8.985	2.37229988950519\\
9	2.37371289417857\\
9.015	2.37512589885195\\
9.03	2.37653890352533\\
9.045	2.37795190819871\\
9.06	2.37936491287209\\
9.075	2.38077791754547\\
9.09	2.38219092221884\\
9.105	2.38360392689222\\
9.12	2.3850169315656\\
9.135	2.38642993623898\\
9.15	2.38784294091236\\
9.165	2.38925594558574\\
9.18	2.39066895025912\\
9.195	2.3920819549325\\
9.21	2.39349495960588\\
9.225	2.39490796427925\\
9.24	2.39632096895263\\
9.255	2.39773397362601\\
9.27	2.39914697829939\\
9.285	2.40055998297277\\
9.3	2.40197298764615\\
9.315	2.40338599231953\\
9.33	2.40479899699291\\
9.345	2.40621200166628\\
9.36	2.40762500633966\\
9.375	2.40903801101304\\
9.39	2.41045101568642\\
9.405	2.4118640203598\\
9.42	2.41327702503318\\
9.435	2.41469002970656\\
9.45	2.41610303437994\\
9.465	2.41751603905331\\
9.48	2.41892904372669\\
9.495	2.42034204840007\\
9.51	2.42175505307345\\
9.525	2.42316805774683\\
9.54	2.42458106242021\\
9.555	2.42599406709359\\
9.57	2.42740707176697\\
9.585	2.42882007644035\\
9.6	2.43023308111372\\
9.615	2.4316460857871\\
9.63	2.43305909046048\\
9.645	2.43447209513386\\
9.66	2.43588509980724\\
9.675	2.43729810448062\\
9.69	2.438711109154\\
9.705	2.44012411382738\\
9.72	2.44153711850075\\
9.735	2.44295012317413\\
9.75	2.44436312784751\\
9.765	2.44577613252089\\
9.78	2.44718913719427\\
9.795	2.44860214186765\\
9.81	2.45001514654103\\
9.825	2.45142815121441\\
9.84	2.45284115588779\\
9.855	2.45425416056116\\
9.87	2.45566716523454\\
9.885	2.45708016990792\\
9.9	2.4584931745813\\
9.915	2.45990617925468\\
9.93	2.46131918392806\\
9.945	2.46273218860144\\
9.96	2.46414519327482\\
9.975	2.46555819794819\\
9.99	2.46697120262157\\
10.005	2.46838420729495\\
10.02	2.46979721196833\\
10.035	2.47121021664171\\
10.05	2.47262322131509\\
10.065	2.47403622598847\\
10.08	2.47544923066185\\
10.095	2.47686223533522\\
10.11	2.4782752400086\\
10.125	2.47968824468198\\
10.14	2.48110124935536\\
10.155	2.48251425402874\\
10.17	2.48392725870212\\
10.185	2.4853402633755\\
10.2	2.48675326804888\\
10.215	2.48816627272226\\
10.23	2.48957927739563\\
10.245	2.49099228206901\\
10.26	2.49240528674239\\
10.275	2.49381829141577\\
10.29	2.49523129608915\\
10.305	2.49664430076253\\
10.32	2.49805730543591\\
10.335	2.49947031010929\\
10.35	2.50088331478266\\
10.365	2.50229631945604\\
10.38	2.50370932412942\\
10.395	2.5051223288028\\
10.41	2.50653533347618\\
10.425	2.50794833814956\\
10.44	2.50936134282294\\
10.455	2.51077434749632\\
10.47	2.51218735216969\\
10.485	2.51360035684307\\
10.5	2.51501336151645\\
10.515	2.51642636618983\\
10.53	2.51783937086321\\
10.545	2.51925237553659\\
10.56	2.52066538020997\\
10.575	2.52207838488335\\
10.59	2.52349138955673\\
10.605	2.5249043942301\\
10.62	2.52631739890348\\
10.635	2.52773040357686\\
10.65	2.52914340825024\\
10.665	2.53055641292362\\
10.68	2.531969417597\\
10.695	2.53338242227038\\
10.71	2.53479542694376\\
10.725	2.53620843161713\\
10.74	2.53762143629051\\
10.755	2.53903444096389\\
10.77	2.54044744563727\\
10.785	2.54186045031065\\
10.8	2.54327345498403\\
10.815	2.54468645965741\\
10.83	2.54609946433079\\
10.845	2.54751246900417\\
10.86	2.54892547367754\\
10.875	2.55033847835092\\
10.89	2.5517514830243\\
10.905	2.55316448769768\\
10.92	2.55457749237106\\
10.935	2.55599049704444\\
10.95	2.55740350171782\\
10.965	2.5588165063912\\
10.98	2.56022951106457\\
10.995	2.56164251573795\\
11.01	2.56305552041133\\
11.025	2.56446852508471\\
11.04	2.56588152975809\\
11.055	2.56729453443147\\
11.07	2.56870753910485\\
11.085	2.57012054377823\\
11.1	2.5715335484516\\
11.115	2.57294655312498\\
11.13	2.57435955779836\\
11.145	2.57577256247174\\
11.16	2.57718556714512\\
11.175	2.5785985718185\\
11.19	2.58001157649188\\
11.205	2.58142458116526\\
11.22	2.58283758583864\\
11.235	2.58425059051201\\
11.25	2.58566359518539\\
11.265	2.58707659985877\\
11.28	2.58848960453215\\
11.295	2.58990260920553\\
11.31	2.59131561387891\\
11.325	2.59272861855229\\
11.34	2.59414162322567\\
11.355	2.59555462789904\\
11.37	2.59696763257242\\
11.385	2.5983806372458\\
11.4	2.59979364191918\\
11.415	2.60120664659256\\
11.43	2.60261965126594\\
11.445	2.60403265593932\\
11.46	2.6054456606127\\
11.475	2.60685866528608\\
11.49	2.60827166995945\\
11.505	2.60968467463283\\
11.52	2.61109767930621\\
11.535	2.61251068397959\\
11.55	2.61392368865297\\
11.565	2.61533669332635\\
11.58	2.61674969799973\\
11.595	2.61816270267311\\
11.61	2.61957570734648\\
11.625	2.62098871201986\\
11.64	2.62240171669324\\
11.655	2.62381472136662\\
11.67	2.62522772604\\
11.685	2.62664073071338\\
11.7	2.62805373538676\\
11.715	2.62946674006014\\
11.73	2.63087974473351\\
11.745	2.63229274940689\\
11.76	2.63370575408027\\
11.775	2.63511875875365\\
11.79	2.63653176342703\\
11.805	2.63794476810041\\
11.82	2.63935777277379\\
11.835	2.64077077744717\\
11.85	2.64218378212054\\
11.865	2.64359678679392\\
11.88	2.6450097914673\\
11.895	2.64642279614068\\
11.91	2.64783580081406\\
11.925	2.64924880548744\\
11.94	2.65066181016082\\
11.955	2.6520748148342\\
11.97	2.65348781950758\\
11.985	2.65490082418095\\
12	2.65631382885433\\
12.015	2.65772683352771\\
12.03	2.65913983820109\\
12.045	2.66055284287447\\
12.06	2.66196584754785\\
12.075	2.66337885222123\\
12.09	2.66479185689461\\
12.105	2.66620486156798\\
12.12	2.66761786624136\\
12.135	2.66903087091474\\
12.15	2.67044387558812\\
12.165	2.6718568802615\\
12.18	2.67326988493488\\
12.195	2.67468288960826\\
12.21	2.67609589428164\\
12.225	2.67750889895502\\
12.24	2.67892190362839\\
12.255	2.68033490830177\\
12.27	2.68174791297515\\
12.285	2.68316091764853\\
12.3	2.68457392232191\\
12.315	2.68598692699529\\
12.33	2.68739993166867\\
12.345	2.68881293634205\\
12.36	2.69022594101542\\
12.375	2.6916389456888\\
12.39	2.69305195036218\\
12.405	2.69446495503556\\
12.42	2.69587795970894\\
12.435	2.69729096438232\\
12.45	2.6987039690557\\
12.465	2.70011697372908\\
12.48	2.70152997840245\\
12.495	2.70294298307583\\
12.51	2.70435598774921\\
12.525	2.70576899242259\\
12.54	2.70718199709597\\
12.555	2.70859500176935\\
12.57	2.71000800644273\\
12.585	2.71142101111611\\
12.6	2.71283401578949\\
12.615	2.71424702046286\\
12.63	2.71566002513624\\
12.645	2.71707302980962\\
12.66	2.718486034483\\
12.675	2.71989903915638\\
12.69	2.72131204382976\\
12.705	2.72272504850314\\
12.72	2.72413805317652\\
12.735	2.72555105784989\\
12.75	2.72696406252327\\
12.765	2.72837706719665\\
12.78	2.72979007187003\\
12.795	2.73120307654341\\
12.81	2.73261608121679\\
12.825	2.73402908589017\\
12.84	2.73544209056355\\
12.855	2.73685509523693\\
12.87	2.7382680999103\\
12.885	2.73968110458368\\
12.9	2.74109410925706\\
12.915	2.74250711393044\\
12.93	2.74392011860382\\
12.945	2.7453331232772\\
12.96	2.74674612795058\\
12.975	2.74815913262396\\
12.99	2.74957213729733\\
13.005	2.75098514197071\\
13.02	2.75239814664409\\
13.035	2.75381115131747\\
13.05	2.75522415599085\\
13.065	2.75663716066423\\
13.08	2.75805016533761\\
13.095	2.75946317001099\\
13.11	2.76087617468436\\
13.125	2.76228917935774\\
13.14	2.76370218403112\\
13.155	2.7651151887045\\
13.17	2.76652819337788\\
13.185	2.76794119805126\\
13.2	2.76935420272464\\
13.215	2.77076720739802\\
13.23	2.7721802120714\\
13.245	2.77359321674477\\
13.26	2.77500622141815\\
13.275	2.77641922609153\\
13.29	2.77783223076491\\
13.305	2.77924523543829\\
13.32	2.78065824011167\\
13.335	2.78207124478505\\
13.35	2.78348424945843\\
13.365	2.7848972541318\\
13.38	2.78631025880518\\
13.395	2.78772326347856\\
13.41	2.78913626815194\\
13.425	2.79054927282532\\
13.44	2.7919622774987\\
13.455	2.79337528217208\\
13.47	2.79478828684546\\
13.485	2.79620129151883\\
13.5	2.79761429619221\\
13.515	2.79902730086559\\
13.53	2.80044030553897\\
13.545	2.80185331021235\\
13.56	2.80326631488573\\
13.575	2.80467931955911\\
13.59	2.80609232423249\\
13.605	2.80750532890587\\
13.62	2.80891833357924\\
13.635	2.81033133825262\\
13.65	2.811744342926\\
13.665	2.81315734759938\\
13.68	2.81457035227276\\
13.695	2.81598335694614\\
13.71	2.81739636161952\\
13.725	2.8188093662929\\
13.74	2.82022237096627\\
13.755	2.82163537563965\\
13.77	2.82304838031303\\
13.785	2.82446138498641\\
13.8	2.82587438965979\\
13.815	2.82728739433317\\
13.83	2.82870039900655\\
13.845	2.83011340367993\\
13.86	2.83152640835331\\
13.875	2.83293941302668\\
13.89	2.83435241770006\\
13.905	2.83576542237344\\
13.92	2.83717842704682\\
13.935	2.8385914317202\\
13.95	2.84000443639358\\
13.965	2.84141744106696\\
13.98	2.84283044574034\\
13.995	2.84424345041371\\
14.01	2.84565645508709\\
14.025	2.84706945976047\\
14.04	2.84848246443385\\
14.055	2.84989546910723\\
14.07	2.85130847378061\\
14.085	2.85272147845399\\
14.1	2.85413448312737\\
14.115	2.85554748780075\\
14.13	2.85696049247412\\
14.145	2.8583734971475\\
14.16	2.85978650182088\\
14.175	2.86119950649426\\
14.19	2.86261251116764\\
14.205	2.86402551584102\\
14.22	2.8654385205144\\
14.235	2.86685152518778\\
14.25	2.86826452986115\\
14.265	2.86967753453453\\
14.28	2.87109053920791\\
14.295	2.87250354388129\\
14.31	2.87391654855467\\
14.325	2.87532955322805\\
14.34	2.87674255790143\\
14.355	2.87815556257481\\
14.37	2.87956856724818\\
14.385	2.88098157192156\\
14.4	2.88239457659494\\
14.415	2.88380758126832\\
14.43	2.8852205859417\\
14.445	2.88663359061508\\
14.46	2.88804659528846\\
14.475	2.88945959996184\\
14.49	2.89087260463522\\
14.505	2.89228560930859\\
14.52	2.89369861398197\\
14.535	2.89511161865535\\
14.55	2.89652462332873\\
14.565	2.89793762800211\\
14.58	2.89935063267549\\
14.595	2.90076363734887\\
14.61	2.90217664202225\\
14.625	2.90358964669562\\
14.64	2.905002651369\\
14.655	2.90641565604238\\
14.67	2.90782866071576\\
14.685	2.90924166538914\\
14.7	2.91065467006252\\
14.715	2.9120676747359\\
14.73	2.91348067940928\\
14.745	2.91489368408265\\
14.76	2.91630668875603\\
14.775	2.91771969342941\\
14.79	2.91913269810279\\
14.805	2.92054570277617\\
14.82	2.92195870744955\\
14.835	2.92337171212293\\
14.85	2.92478471679631\\
14.865	2.92619772146969\\
14.88	2.92761072614306\\
14.895	2.92902373081644\\
14.91	2.93043673548982\\
14.925	2.9318497401632\\
14.94	2.93326274483658\\
14.955	2.93467574950996\\
14.97	2.93608875418334\\
14.985	2.93750175885672\\
15	2.93891476353009\\
};
\end{axis}
\end{tikzpicture}%}
\caption{Execution time for the initialization for different amount of devices and the baseline. The fitted line is a linear approximation.}
\label{fig:MT_Init_Time}
\end{figure}

From \autoref{fig:MT_Init_Time} it can be seen that there is a linear tendency scaling with the number of devices. It is also seen that even if the baseline and the changed code emulates one device, the baseline have a lower execution time, with a estimated difference to be the same as a single step between different number of devices. The fitted line is estimated to be:

\begin{align}
&T_{init} (\text{NoD}) = 0.0942 \cdot \text{NoD} + 1.526 [s]
\end{align}
\todo{where list needed}

\section{Synchronization}
The execution time for the synchronization step is measured dependably on the number of devices emulated and the baseline is measured as well, which gives the results seen in \autoref{fig:MT_Sync_Time}. As the error type cell sync, mention in \autoref{sec:MTerror}, occurs in this step of the process, some measurements will be equal zero, as the execution time can not be calculated.

\captionsetup{belowskip=0em}
\begin{minipage}{0.48\textwidth}
\begin{figure}[H]
\tikzsetnextfilename{MT_Sync_Time}
\centering
\resizebox{0.9\textwidth}{!}{
% This file was created by matlab2tikz.
%
%The latest updates can be retrieved from
%  http://www.mathworks.com/matlabcentral/fileexchange/22022-matlab2tikz-matlab2tikz
%where you can also make suggestions and rate matlab2tikz.
%
\definecolor{mycolor1}{rgb}{0.00000,0.44700,0.74100}%
\definecolor{mycolor2}{rgb}{0.85000,0.32500,0.09800}%
%
\begin{tikzpicture}

\begin{axis}[%
width=\textwidth,
height=.66\textwidth,
at={(0.758in,0.481in)},
scale only axis,
xmin=-0.5,
xmax=15.5,
xtick={0,1,2,3,4,5,6,7,8,9,10,11,12,13,14,15},
xticklabels={{Base},{1},{2},{3},{4},{5},{6},{7},{8},{9},{10},{11},{12},{13},{14},{15}},
xlabel={Number of devices},
ymin=0,
ymax=1.8,
ylabel={Time [s]},
axis background/.style={fill=white},
title style={font=\bfseries},
title={Syncronation time},
legend style={at={(0.03,0.97)},anchor=north west,legend cell align=left,align=left,draw=white!15!black}
]
\addplot [color=mycolor1,only marks,mark=*,mark options={solid}]
  table[row sep=crcr]{%
1	0.650609970092773\\
1	0.653857946395874\\
1	0.640816926956177\\
1	0.642185926437378\\
1	0.645545959472656\\
1	1.06731581687927\\
1	0.649732112884521\\
1	0.653727054595947\\
1	1.06988906860352\\
1	0.648482799530029\\
1	0.650296926498413\\
1	0.653731107711792\\
1	0.650707960128784\\
1	0.656582117080688\\
1	0.651366949081421\\
1	0.653976202011108\\
1	0.653213024139404\\
1	0.650224924087524\\
1	0.648483037948608\\
1	0.654691934585571\\
};
\addlegendentry{Change code};

\addplot [color=mycolor1,only marks,mark=*,mark options={solid},forget plot]
  table[row sep=crcr]{%
2	0.646949052810669\\
2	0.649175882339478\\
2	0.650078058242798\\
2	0.658921957015991\\
2	0.644191026687622\\
2	1.06752800941467\\
2	0.652630090713501\\
2	0.649014949798584\\
2	0.650974988937378\\
2	0.660398006439209\\
2	0.641355037689209\\
2	0.652486085891724\\
2	0.646153926849365\\
2	0.645971059799194\\
2	0.647706985473633\\
2	0.652987003326416\\
2	0.651640892028809\\
2	0.64258599281311\\
2	0.644962072372437\\
2	0.649473905563354\\
};


\addplot [color=mycolor1,only marks,mark=*,mark options={solid},forget plot]
  table[row sep=crcr]{%
3	0.643748044967651\\
3	0.644491195678711\\
3	0.642645835876465\\
3	0.656013965606689\\
3	0.647945880889893\\
3	0.646684885025024\\
3	0.641174077987671\\
3	0.647907018661499\\
3	0.644546985626221\\
3	0.645722150802612\\
3	0.64906907081604\\
3	0.648266077041626\\
3	0.642682075500488\\
3	0.652300119400024\\
3	0.641947031021118\\
3	0.676903009414673\\
3	0.648000955581665\\
3	0.655172109603882\\
3	0.648041009902954\\
3	0.656927108764648\\
};
\addplot [color=mycolor1,only marks,mark=*,mark options={solid},forget plot]
  table[row sep=crcr]{%
4	0.648493051528931\\
4	0.64559006690979\\
4	0.654642105102539\\
4	0.6754150390625\\
4	0.646654844284058\\
4	0.642754077911377\\
4	0.639609098434448\\
4	0.653938055038452\\
4	0.650071144104004\\
4	0.656721115112305\\
4	0.643237829208374\\
4	0.651222229003906\\
4	0.652673006057739\\
4	0.65081000328064\\
4	0.672778844833374\\
4	0.651262998580933\\
4	0.649074077606201\\
4	0.647204875946045\\
4	0.646042108535767\\
4	0.64453911781311\\
};
\addplot [color=mycolor1,only marks,mark=*,mark options={solid},forget plot]
  table[row sep=crcr]{%
5	0.650202989578247\\
5	0.648781776428223\\
5	0.640866994857788\\
5	0.649247169494629\\
5	0.65114688873291\\
5	0.643033027648926\\
5	0.644155025482178\\
5	0.657757997512817\\
5	0.650070905685425\\
5	0.65401291847229\\
5	0.646816968917847\\
5	0.64539098739624\\
5	0.654246091842651\\
5	0.655587911605835\\
5	0.645053863525391\\
5	0.64438796043396\\
5	0.648882150650024\\
5	0.652736902236938\\
5	0.643465995788574\\
5	0.653812170028687\\
};
\addplot [color=mycolor1,only marks,mark=*,mark options={solid},forget plot]
  table[row sep=crcr]{%
6	0.646232128143311\\
6	0.648225069046021\\
6	1.40721011161804\\
6	0.651366949081421\\
6	0.652253866195679\\
6	0.650762796401978\\
6	0.648574829101563\\
6	0.640298843383789\\
6	0.653538942337036\\
6	0.645024061203003\\
6	0.661158084869385\\
6	0.647090911865234\\
6	0.642549991607666\\
6	0.642514944076538\\
6	0.643571853637695\\
6	0.65564489364624\\
6	1.07363796234131\\
6	0.647010087966919\\
6	0.646505117416382\\
6	0.6438148021698\\
};
\addplot [color=mycolor1,only marks,mark=*,mark options={solid},forget plot]
  table[row sep=crcr]{%
7	1.07072401046753\\
7	0.652054071426392\\
7	1.07058906555176\\
7	0.648339986801147\\
7	0.652436017990112\\
7	0.642619132995605\\
7	0.643392086029053\\
7	0.648621082305908\\
7	0.645454883575439\\
7	0.652534961700439\\
7	0.64702582359314\\
7	0.644365072250366\\
7	0.659003973007202\\
7	0.652891874313354\\
7	0.646991968154907\\
7	0.655962944030762\\
7	0.651911020278931\\
7	0.649963140487671\\
7	0.650946855545044\\
7	0.650607109069824\\
};
\addplot [color=mycolor1,only marks,mark=*,mark options={solid},forget plot]
  table[row sep=crcr]{%
8	0.649052143096924\\
8	0.655632972717285\\
8	0.653681039810181\\
8	0.651021003723145\\
8	0.645121097564697\\
8	0.651643991470337\\
8	0.644710063934326\\
8	0.643614053726196\\
8	0.641947984695435\\
8	0.648487091064453\\
8	0.652446985244751\\
8	0.648170948028564\\
8	0.646777153015137\\
8	0.644640922546387\\
8	0.652256011962891\\
8	0.646914958953857\\
8	0.647845983505249\\
8	0.647209167480469\\
8	0.65062403678894\\
8	0\\
};
\addplot [color=mycolor1,only marks,mark=*,mark options={solid},forget plot]
  table[row sep=crcr]{%
9	0.650902032852173\\
9	0.653230905532837\\
9	0.651026964187622\\
9	1.37894892692566\\
9	0.658339023590088\\
9	0.6616530418396\\
9	0.650697946548462\\
9	0.65686297416687\\
9	0.646430015563965\\
9	0.652285814285278\\
9	0.657356977462769\\
9	0.648330926895142\\
9	0.652124166488647\\
9	0.650056838989258\\
9	0.649123907089233\\
9	1.69176697731018\\
9	0.648163080215454\\
9	0.653515100479126\\
9	0.648416042327881\\
9	0.652014017105103\\
};
\addplot [color=mycolor1,only marks,mark=*,mark options={solid},forget plot]
  table[row sep=crcr]{%
10	0.652431964874268\\
10	0.653368949890137\\
10	0.653976917266846\\
10	0.647805213928223\\
10	0.651004791259766\\
10	0.657984972000122\\
10	0.652050971984863\\
10	0.653388023376465\\
10	0.645644187927246\\
10	0.653051853179932\\
10	0.655381917953491\\
10	0.653103113174438\\
10	0.649111986160278\\
10	0.650975942611694\\
10	0.646430015563965\\
10	0.657355070114136\\
10	0.655450105667114\\
10	0.647177934646606\\
10	0.656944990158081\\
10	0.651263952255249\\
};
\addplot [color=mycolor1,only marks,mark=*,mark options={solid},forget plot]
  table[row sep=crcr]{%
11	0.651514053344727\\
11	0.65205192565918\\
11	0.658270835876465\\
11	0.649635076522827\\
11	1.07549595832825\\
11	0.654492855072021\\
11	0.653318881988525\\
11	0.65693998336792\\
11	0.65533185005188\\
11	1.07469415664673\\
11	0.647542953491211\\
11	0.678914070129395\\
11	0.654986143112183\\
11	0.646422863006592\\
11	0.650253057479858\\
11	0.652107954025269\\
11	0.655264139175415\\
11	0.651924133300781\\
11	0.645050048828125\\
11	0.648700952529907\\
};
\addplot [color=mycolor1,only marks,mark=*,mark options={solid},forget plot]
  table[row sep=crcr]{%
12	0.65892505645752\\
12	0.650825977325439\\
12	0.649593114852905\\
12	0.650028944015503\\
12	0.647899150848389\\
12	0.658643007278442\\
12	0.649430990219116\\
12	0.666985988616943\\
12	0.659577131271362\\
12	0.653954982757568\\
12	0.65661096572876\\
12	0.647140026092529\\
12	1.07442212104797\\
12	0.653070211410522\\
12	0.676941156387329\\
12	0.655467987060547\\
12	0.646998167037964\\
12	0.653201103210449\\
12	0.649198055267334\\
12	0.652733087539673\\
};
\addplot [color=mycolor1,only marks,mark=*,mark options={solid},forget plot]
  table[row sep=crcr]{%
13	0.647382974624634\\
13	0.653558015823364\\
13	0.647633075714111\\
13	0.650227069854736\\
13	0.655040979385376\\
13	0.644393920898438\\
13	0.656842947006226\\
13	0.650371074676514\\
13	1.69415903091431\\
13	0.651180028915405\\
13	0.649358034133911\\
13	0.650844097137451\\
13	0.659465074539185\\
13	0.648830890655518\\
13	0.65994119644165\\
13	0.655570030212402\\
13	0.661641120910645\\
13	0.654593944549561\\
13	0.656475782394409\\
13	0\\
};
\addplot [color=mycolor1,only marks,mark=*,mark options={solid},forget plot]
  table[row sep=crcr]{%
14	0.643074035644531\\
14	0.657819986343384\\
14	0.644295215606689\\
14	0.660433053970337\\
14	0.653297185897827\\
14	0.658216953277588\\
14	0.653433084487915\\
14	0.649707078933716\\
14	0.653749942779541\\
14	0.649873971939087\\
14	0.647374868392944\\
14	0.680761098861694\\
14	0.653578996658325\\
14	0.646941900253296\\
14	0.657058954238892\\
14	0.654543876647949\\
14	0.644098997116089\\
14	0.649263143539429\\
14	0.653286933898926\\
14	1.39285397529602\\
};
\addplot [color=mycolor1,only marks,mark=*,mark options={solid},forget plot]
  table[row sep=crcr]{%
15	0.654828071594238\\
15	0.648122787475586\\
15	0.656240224838257\\
15	0.646528005599976\\
15	0.648042917251587\\
15	0.657634973526001\\
15	0.656691074371338\\
15	0.654543161392212\\
15	0.665755033493042\\
15	0.655173063278198\\
15	0.675770044326782\\
15	0.679379940032959\\
15	0.654716014862061\\
15	0.655852794647217\\
15	0.657128095626831\\
15	0.646656036376953\\
15	0.647743940353394\\
15	0.651691913604736\\
15	0.657409906387329\\
15	0.656056880950928\\
};
\addplot [color=mycolor2,only marks,mark=*,mark options={solid}]
  table[row sep=crcr]
\caption{Execution time for the synchronization for different amount of devices and the baseline}
\label{fig:MT_Sync_Time}
\end{figure}
\end{minipage}%
\hfill
\begin{minipage}{0.48\textwidth}
\begin{figure}[H]
\tikzsetnextfilename{MT_Sync_His}
\centering
\resizebox{0.9\textwidth}{!}{
% This file was created by matlab2tikz.
%
%The latest updates can be retrieved from
%  http://www.mathworks.com/matlabcentral/fileexchange/22022-matlab2tikz-matlab2tikz
%where you can also make suggestions and rate matlab2tikz.
%
\definecolor{mycolor1}{rgb}{0.00000,0.44700,0.74100}%
%
\begin{tikzpicture}

\begin{axis}[%
width=4.521in,
height=3.566in,
at={(0.758in,0.481in)},
scale only axis,
xmin=0.6,
xmax=1.8,
xlabel={Number of UEs},
ymin=0,
ymax=160,
ylabel={Trials},
axis background/.style={fill=white},
title style={font=\bfseries},
title={Histogram for Sync}
]
\addplot[fill=mycolor1,fill opacity=0.6,draw=black,ybar interval,area legend] plot table[row sep=crcr] {%
x	y\\
0.63	2\\
0.6407	157\\
0.6514	115\\
0.6621	3\\
0.6728	7\\
0.6835	0\\
0.6942	0\\
0.7049	0\\
0.7156	0\\
0.7263	0\\
0.737	0\\
0.7477	0\\
0.7584	0\\
0.7691	0\\
0.7798	0\\
0.7905	0\\
0.8012	0\\
0.8119	0\\
0.8226	0\\
0.8333	0\\
0.844	0\\
0.8547	0\\
0.8654	0\\
0.8761	0\\
0.8868	0\\
0.8975	0\\
0.9082	0\\
0.9189	0\\
0.9296	0\\
0.9403	0\\
0.951	0\\
0.9617	0\\
0.9724	0\\
0.9831	0\\
0.9938	0\\
1.0045	0\\
1.0152	0\\
1.0259	0\\
1.0366	0\\
1.0473	0\\
1.058	2\\
1.0687	7\\
1.0794	0\\
1.0901	0\\
1.1008	0\\
1.1115	0\\
1.1222	0\\
1.1329	0\\
1.1436	0\\
1.1543	0\\
1.165	0\\
1.1757	0\\
1.1864	0\\
1.1971	0\\
1.2078	0\\
1.2185	0\\
1.2292	0\\
1.2399	0\\
1.2506	0\\
1.2613	0\\
1.272	0\\
1.2827	0\\
1.2934	0\\
1.3041	0\\
1.3148	0\\
1.3255	0\\
1.3362	0\\
1.3469	0\\
1.3576	0\\
1.3683	1\\
1.379	0\\
1.3897	1\\
1.4004	1\\
1.4111	0\\
1.4218	0\\
1.4325	0\\
1.4432	0\\
1.4539	0\\
1.4646	0\\
1.4753	0\\
1.486	0\\
1.4967	0\\
1.5074	0\\
1.5181	0\\
1.5288	0\\
1.5395	0\\
1.5502	0\\
1.5609	0\\
1.5716	0\\
1.5823	0\\
1.593	0\\
1.6037	0\\
1.6144	0\\
1.6251	0\\
1.6358	0\\
1.6465	0\\
1.6572	0\\
1.6679	0\\
1.6786	0\\
1.6893	2\\
1.7	2\\
};
\end{axis}
\end{tikzpicture}%}
\caption{The distribution for the execution time for synchronization for all different amount of devices}
\label{fig:MT_Sync_His}
\end{figure}
\end{minipage}
\captionsetup{belowskip=-1.5em}
\todo{make more coluoms in histogram}

As seen in \autoref{fig:MT_Sync_Time} is the execution time for different amount of devices is behaving equal to each other. Another aspect seen on the figure is that some measurements have taken some extra time to execute, but is align at the same time values, which also indicated on the histogram in \autoref{fig:MT_Sync_His}. Here it is seen that most measurements is placed at 0.6 s to 0.8 s and the amount at the other points is much lower.



\section{MIB decoding}
\todo{this first line is the same for all the sections here move it to some introductory thing}
The execution time for the MIB decoding step is measured dependably on the number of devices emulated and the baseline is measured as well, which gives the results seen in \autoref{fig:MT_MIB_Time}. All these measurements is for a full decoding of MIB and all retries have been removed.
As all earlier occurred errors will infect that there is no measurement point for further steps in the process, the cell sync errors from the synchronization step will still infect the given measurements here and further on. \todo{what are you trying to say here, btw think you effect not infect}

\captionsetup{belowskip=0em}
\begin{minipage}{0.48\textwidth}
\begin{figure}[H]
\tikzsetnextfilename{MT_MIB_Time}
\centering
\resizebox{0.9\textwidth}{!}{
% This file was created by matlab2tikz.
%
%The latest updates can be retrieved from
%  http://www.mathworks.com/matlabcentral/fileexchange/22022-matlab2tikz-matlab2tikz
%where you can also make suggestions and rate matlab2tikz.
%
\definecolor{mycolor1}{rgb}{0.00000,0.44700,0.74100}%
\definecolor{mycolor2}{rgb}{0.85000,0.32500,0.09800}%
%
\begin{tikzpicture}

\begin{axis}[%
width=\textwidth,
height=.66\textwidth,
at={(0.758in,0.481in)},
scale only axis,
xmin=-0.5,
xmax=15.5,
xtick={0,1,2,3,4,5,6,7,8,9,10,11,12,13,14,15},
xticklabels={{Baseline},{1},{2},{3},{4},{5},{6},{7},{8},{9},{10},{11},{12},{13},{14},{15}},
xlabel={Number of devices},
ymin=0,
ymax=2,
ylabel={Time (s)},
axis background/.style={fill=white},
title style={font=\bfseries},
title={MIB decoding time},
legend style={at={(0.03,0.97)},anchor=north west,legend cell align=left,align=left,draw=white!15!black}
]
\addplot [color=mycolor1,only marks,mark=*,mark options={solid}]
  table[row sep=crcr]{%
1	1.18452787399292\\
1	1.02464199066162\\
1	1.16689705848694\\
1	0.904258966445923\\
1	0.843533992767334\\
1	1.13420486450195\\
1	0.824519872665405\\
1	0.961822032928467\\
1	0.891458988189697\\
1	1.3114321231842\\
1	0.72556209564209\\
1	1.26286196708679\\
1	1.25451898574829\\
1	1.12048602104187\\
1	1.02167701721191\\
1	1.09917998313904\\
1	0.873126983642578\\
1	1.14201807975769\\
1	1.21537899971008\\
1	1.05097317695618\\
};
\addlegendentry{Change code};

\addplot [color=mycolor1,only marks,mark=*,mark options={solid},forget plot]
  table[row sep=crcr]{%
2	0.742633819580078\\
2	1.12349700927734\\
2	0.862571001052856\\
2	1.09928011894226\\
2	1.06075310707092\\
2	0.841415166854858\\
2	0.824921131134033\\
2	1.05261611938477\\
2	1.0511691570282\\
2	1.34838080406189\\
2	1.16666889190674\\
2	0.892491817474365\\
2	0.863969087600708\\
2	0.991780996322632\\
2	1.16264510154724\\
2	1.15249800682068\\
2	1.24425196647644\\
2	0.853885173797607\\
2	1.44308090209961\\
2	0.932136058807373\\
};


\addplot [color=mycolor1,only marks,mark=*,mark options={solid},forget plot]
  table[row sep=crcr]{%
3	1.35272288322449\\
3	0.896404027938843\\
3	1.19529414176941\\
3	0.721312999725342\\
3	1.00501799583435\\
3	0.974997997283936\\
3	1.27371096611023\\
3	1.09565401077271\\
3	1.19446611404419\\
3	0.736949920654297\\
3	0.784852981567383\\
3	0.874388933181763\\
3	0.932243824005127\\
3	0.821744918823242\\
3	1.09404897689819\\
3	1.14502000808716\\
3	0.893469095230103\\
3	1.19056606292725\\
3	1.06648683547974\\
3	1.04886889457703\\
};
\addplot [color=mycolor1,only marks,mark=*,mark options={solid},forget plot]
  table[row sep=crcr]{%
4	1.21173596382141\\
4	0.894090890884399\\
4	0.950580835342407\\
4	1.31642508506775\\
4	0.83543586730957\\
4	0.925927877426147\\
4	0.976491928100586\\
4	1.28235483169556\\
4	0.803546905517578\\
4	0.790771007537842\\
4	1.27620601654053\\
4	1.01348495483398\\
4	1.07200884819031\\
4	1.01012992858887\\
4	0.839915037155151\\
4	0.962180137634277\\
4	1.07453393936157\\
4	1.29357409477234\\
4	1.40396094322205\\
4	1.24524092674255\\
};
\addplot [color=mycolor1,only marks,mark=*,mark options={solid},forget plot]
  table[row sep=crcr]{%
5	1.29511213302612\\
5	1.01446914672852\\
5	1.20752096176147\\
5	1.20313382148743\\
5	0.805004119873047\\
5	1.30468487739563\\
5	0.815110921859741\\
5	0.758973121643066\\
5	1.11016798019409\\
5	0.761851072311401\\
5	0.955629825592041\\
5	1.00391411781311\\
5	1.32148098945618\\
5	1.17293000221252\\
5	0.923401832580566\\
5	0.895086050033569\\
5	1.20391082763672\\
5	0.76114296913147\\
5	0.784970045089722\\
5	1.32375192642212\\
};
\addplot [color=mycolor1,only marks,mark=*,mark options={solid},forget plot]
  table[row sep=crcr]{%
6	1.12626886367798\\
6	0.784417867660522\\
6	0.973766088485718\\
6	1.34400105476379\\
6	1.03182911872864\\
6	1.00979208946228\\
6	1.18366503715515\\
6	0.977426052093506\\
6	0.831204891204834\\
6	1.34424090385437\\
6	1.26865386962891\\
6	0.815386056900024\\
6	0.72773003578186\\
6	1.09482884407043\\
6	0.776089191436768\\
6	1.34079313278198\\
6	0.740194082260132\\
6	0.825405120849609\\
6	1.03018093109131\\
6	1.05475306510925\\
};
\addplot [color=mycolor1,only marks,mark=*,mark options={solid},forget plot]
  table[row sep=crcr]{%
7	1.14058113098145\\
7	0.969480037689209\\
7	1.2895450592041\\
7	1.25373196601868\\
7	0.869355916976929\\
7	0.836460828781128\\
7	0.883985042572021\\
7	1.33135199546814\\
7	1.10523915290833\\
7	1.06331515312195\\
7	0.834944009780884\\
7	0.796628952026367\\
7	1.09889316558838\\
7	1.26145696640015\\
7	1.31127190589905\\
7	0.929186105728149\\
7	1.09361791610718\\
7	0.924177885055542\\
7	1.28277397155762\\
7	1.24324703216553\\
};
\addplot [color=mycolor1,only marks,mark=*,mark options={solid},forget plot]
  table[row sep=crcr]{%
8	1.21215677261353\\
8	0.811891078948975\\
8	1.35208201408386\\
8	1.25351810455322\\
8	1.24444103240967\\
8	0.96881103515625\\
8	1.32427191734314\\
8	1.24557089805603\\
8	0.985246181488037\\
8	1.10344481468201\\
8	0.947085857391357\\
8	1.03846478462219\\
8	0.940639019012451\\
8	1.27187204360962\\
8	0.932521820068359\\
8	1.20112204551697\\
8	0.952279090881348\\
8	0.981828927993774\\
8	1.12882900238037\\
8	0\\
};
\addplot [color=mycolor1,only marks,mark=*,mark options={solid},forget plot]
  table[row sep=crcr]{%
9	1.07991504669189\\
9	1.19017219543457\\
9	1.26013112068176\\
9	1.21215891838074\\
9	0.786477088928223\\
9	1.14683389663696\\
9	0.953114032745361\\
9	0.919471025466919\\
9	1.3333580493927\\
9	1.07340693473816\\
9	0.95987606048584\\
9	0.84372091293335\\
9	0.995597839355469\\
9	1.08392119407654\\
9	0.971073150634766\\
9	1.1183009147644\\
9	1.08447599411011\\
9	0.850917100906372\\
9	0.973115921020508\\
9	1.12152695655823\\
};
\addplot [color=mycolor1,only marks,mark=*,mark options={solid},forget plot]
  table[row sep=crcr]{%
10	1.25671696662903\\
10	1.05311989784241\\
10	0.929455041885376\\
10	1.19342279434204\\
10	1.05944609642029\\
10	1.01689720153809\\
10	0.967733860015869\\
10	1.21085119247437\\
10	1.22501087188721\\
10	1.08909916877747\\
10	0.941591024398804\\
10	1.1600980758667\\
10	1.24331402778625\\
10	0.889430046081543\\
10	1.32221412658691\\
10	1.01067590713501\\
10	0.999495983123779\\
10	1.3286120891571\\
10	1.03497505187988\\
10	0.809139966964722\\
};
\addplot [color=mycolor1,only marks,mark=*,mark options={solid},forget plot]
  table[row sep=crcr]{%
11	0.941027879714966\\
11	1.35822701454163\\
11	0.977467060089111\\
11	1.29018783569336\\
11	1.10762286186218\\
11	1.12313294410706\\
11	1.08049416542053\\
11	1.08900213241577\\
11	1.15175604820251\\
11	1.06713485717773\\
11	1.1618390083313\\
11	1.21526479721069\\
11	1.33760190010071\\
11	0.884803056716919\\
11	1.23175096511841\\
11	0.910167932510376\\
11	1.27922487258911\\
11	1.03915095329285\\
11	1.28432893753052\\
11	1.26249885559082\\
};
\addplot [color=mycolor1,only marks,mark=*,mark options={solid},forget plot]
  table[row sep=crcr]{%
12	0.858820915222168\\
12	1.21957588195801\\
12	0.880445003509521\\
12	1.05029010772705\\
12	1.23261284828186\\
12	1.30926299095154\\
12	0.879901170730591\\
12	1.27121210098267\\
12	1.13944888114929\\
12	0.958610057830811\\
12	1.08663201332092\\
12	1.01077389717102\\
12	1.08854699134827\\
12	1.2690908908844\\
12	1.22531986236572\\
12	1.09908199310303\\
12	1.27269697189331\\
12	1.25594997406006\\
12	1.10074591636658\\
12	0.738483905792236\\
};
\addplot [color=mycolor1,only marks,mark=*,mark options={solid},forget plot]
  table[row sep=crcr]{%
13	0.83938193321228\\
13	0.998347997665405\\
13	1.32503414154053\\
13	0.880541086196899\\
13	1.16189002990723\\
13	0.742758989334106\\
13	0.765993118286133\\
13	1.25955986976624\\
13	1.0792031288147\\
13	0.740605115890503\\
13	1.33235692977905\\
13	1.11989402770996\\
13	1.13771796226501\\
13	1.11200213432312\\
13	1.1292519569397\\
13	0.961405992507935\\
13	1.27758502960205\\
13	1.24095487594604\\
13	1.33164381980896\\
13	0\\
};
\addplot [color=mycolor1,only marks,mark=*,mark options={solid},forget plot]
  table[row sep=crcr]{%
14	0.835340976715088\\
14	1.21743702888489\\
14	1.1116509437561\\
14	0.764328956604004\\
14	1.3510730266571\\
14	1.34011006355286\\
14	0.861910104751587\\
14	1.09282898902893\\
14	1.14283895492554\\
14	1.34170484542847\\
14	0.792457103729248\\
14	0.864109992980957\\
14	0.750169038772583\\
14	1.22165703773499\\
14	0.819968938827515\\
14	0.823435068130493\\
14	0.993892908096313\\
14	0.913672924041748\\
14	0.91166090965271\\
14	0.859905958175659\\
};
\addplot [color=mycolor1,only marks,mark=*,mark options={solid},forget plot]
  table[row sep=crcr]{%
15	0.958617925643921\\
15	1.01214718818665\\
15	1.23916387557983\\
15	0.954987049102783\\
15	1.02038717269897\\
15	0.949368000030518\\
15	1.08639097213745\\
15	1.20632886886597\\
15	1.28288388252258\\
15	1.04977297782898\\
15	0.788140058517456\\
15	0.933758020401001\\
15	1.16064596176147\\
15	0.740518093109131\\
15	1.11970591545105\\
15	1.19137382507324\\
15	0.931004047393799\\
15	0.758590936660767\\
15	1.19861912727356\\
15	1.03083515167236\\
};
\addplot [color=mycolor2,only marks,mark=*,mark options={solid}]
  table[row sep=crcr]
\caption{Execution time for the decoding the MIB for different amount of devices and the baseline. A single measurement for the base line is placed at 5.0834 s, which is not shown on this figure.}
\label{fig:MT_MIB_Time}
\end{figure}
\end{minipage}%
\hfill
\begin{minipage}{0.48\textwidth}
\begin{figure}[H]
\tikzsetnextfilename{MT_MIB_His}
\centering
\resizebox{0.9\textwidth}{!}{
% This file was created by matlab2tikz.
%
%The latest updates can be retrieved from
%  http://www.mathworks.com/matlabcentral/fileexchange/22022-matlab2tikz-matlab2tikz
%where you can also make suggestions and rate matlab2tikz.
%
\definecolor{mycolor1}{rgb}{0.00000,0.44700,0.74100}%
%
\begin{tikzpicture}

\begin{axis}[%
width=4.521in,
height=3.566in,
at={(0.758in,0.481in)},
scale only axis,
xmin=0.7,
xmax=1.5,
xlabel={Number of UEs},
ymin=0,
ymax=25,
ylabel={Trials},
axis background/.style={fill=white},
title style={font=\bfseries},
title={Histogram for Mib}
]
\addplot[fill=mycolor1,fill opacity=0.6,draw=black,ybar interval,area legend] plot table[row sep=crcr] 
\caption{The distribution for the execution time for decoding the MIB for all different amount of devices.}
\label{fig:MT_MIB_His}
\end{figure}
\end{minipage}
\captionsetup{belowskip=-1.5em}

As seen in \autoref{fig:MT_MIB_Time} the execution time for the MIB decoding step have the same tendency across different amount of devices. The spread is bigger compared to the synchronazitation step, which also can be seen when comparing the histogram for the two steps, \autoref{fig:MT_Sync_His} and \autoref{fig:MT_MIB_His}. The baseline have the same tendency as the changed code. Mention before are these execution times only for a full decoding \todo{what is your intention with this sentence}. In \autoref{fig:MT_MIB_Tries} is it shown how many tries the different measurements needed before completing the MIB decoding step.

\begin{figure}[H]
\tikzsetnextfilename{MT_MIB_Tries}
\centering
\resizebox{0.5\textwidth}{!}{
% This file was created by matlab2tikz.
%
%The latest updates can be retrieved from
%  http://www.mathworks.com/matlabcentral/fileexchange/22022-matlab2tikz-matlab2tikz
%where you can also make suggestions and rate matlab2tikz.
%
\definecolor{mycolor1}{rgb}{0.20810,0.16630,0.52920}%
\definecolor{mycolor2}{rgb}{0.02650,0.61370,0.81350}%
\definecolor{mycolor3}{rgb}{0.64730,0.74560,0.41880}%
\definecolor{mycolor4}{rgb}{0.97630,0.98310,0.05380}%
%
\begin{tikzpicture}

\begin{axis}[%
width=0.5\textwidth,
height=.33\textwidth,
at={(0.607in,0.481in)},
scale only axis,
bar width=0.5,
xmin=-0.5,
xmax=15.5,
xtick={0,1,2,3,4,5,6,7,8,9,10,11,12,13,14,15},
xticklabels={{Base},{1},{2},{3},{4},{5},{6},{7},{8},{9},{10},{11},{12},{13},{14},{15}},
xlabel={Number of devices},
ymin=0,
ymax=1,
ylabel={Trials},
axis background/.style={fill=white},
title style={font=\bfseries},
title={Number of tries for decoding MIB},
legend style={at={(1.03,1)},anchor=north west,legend cell align=left,align=left,draw=white!15!black}
]
\addplot[ybar stacked,draw=black,fill=mycolor1,area legend] plot table[row sep=crcr] {%
0	0.65\\
1	0.8\\
2	0.85\\
3	0.75\\
4	0.9\\
5	0.65\\
6	0.85\\
7	0.85\\
8	0.736842105263158\\
9	0.9\\
10	0.95\\
11	0.95\\
12	1\\
13	0.842105263157895\\
14	0.95\\
15	0.95\\
};
\addlegendentry{1 Attempts};

\addplot[ybar stacked,draw=black,fill=mycolor2,area legend] plot table[row sep=crcr] {%
0	0.25\\
1	0.15\\
2	0.05\\
3	0.2\\
4	0.1\\
5	0.25\\
6	0\\
7	0.15\\
8	0.263157894736842\\
9	0.1\\
10	0.05\\
11	0.05\\
12	0\\
13	0.157894736842105\\
14	0.05\\
15	0.05\\
};
\addlegendentry{2 Attempts};

\addplot[ybar stacked,draw=black,fill=mycolor3,area legend] plot table[row sep=crcr] {%
0	0.1\\
1	0.05\\
2	0.05\\
3	0.05\\
4	0\\
5	0.05\\
6	0.15\\
7	0\\
8	0\\
9	0\\
10	0\\
11	0\\
12	0\\
13	0\\
14	0\\
15	0\\
};
\addlegendentry{3 Attempts};

\addplot[ybar stacked,draw=black,fill=mycolor4,area legend] plot table[row sep=crcr] 
\caption{The distribution for number of tries for decoding the MIB for different amount of devices.}
\label{fig:MT_MIB_Tries}
\end{figure}

It is seen in \autoref{fig:MT_MIB_Tries} that baseline code is not different from the changed code at a lower amount of devices. At a higher amount of devices it seems like the changed code is more efficient, as these levels is the amount of tries lower. \todo{call it retries or attemps instead of tries}

\section{SIB1}
The execution time for the SIB1 decoding step is measured dependably on the number of devices emulated and the baseline is measured as well, which gives the results seen in \autoref{fig:MT_SIB1_Time}. The test is executed with the different amount of devices, but the results shown in \autoref{fig:MT_SIB1_Time} is only for the first device which also will be the procedure for the following steps. \todo{an explanation is needed for why this is the case: as the sib is decoded for each device....} As both the radio error and idle after MIB error occurs in this step of the process, the amount of measurement points are lowered. \todo{shown time difference from first to last SIB mes}

\begin{figure}[H]
\tikzsetnextfilename{MT_SIB1_Time}
\centering
\resizebox{0.5\textwidth}{!}{
% This file was created by matlab2tikz.
%
%The latest updates can be retrieved from
%  http://www.mathworks.com/matlabcentral/fileexchange/22022-matlab2tikz-matlab2tikz
%where you can also make suggestions and rate matlab2tikz.
%
\definecolor{mycolor1}{rgb}{0.00000,0.44700,0.74100}%
\definecolor{mycolor2}{rgb}{0.85000,0.32500,0.09800}%
%
\begin{tikzpicture}

\begin{axis}[%
width=\textwidth,
height=.66\textwidth,
at={(0.758in,0.481in)},
scale only axis,
xmin=-0.5,
xmax=15.5,
xtick={0,1,2,3,4,5,6,7,8,9,10,11,12,13,14,15},
xticklabels={{Base},{1},{2},{3},{4},{5},{6},{7},{8},{9},{10},{11},{12},{13},{14},{15}},
xlabel={Number of devices},
ymin=0,
ymax=14,
ylabel={Time [s]},
axis background/.style={fill=white},
title style={font=\bfseries},
title={SIB1 execution time},
legend style={at={(0.03,0.97)},anchor=north west,legend cell align=left,align=left,draw=white!15!black}
]
\addplot [color=mycolor1,only marks,mark=*,mark options={solid}]
  table[row sep=crcr]{%
1	2.06326198577881\\
1	2.06440305709839\\
1	3.98324704170227\\
1	3.34370994567871\\
1	2.064288854599\\
1	2.06284713745117\\
1	3.34417414665222\\
1	0\\
1	3.98412108421326\\
1	0\\
1	0\\
1	3.34290599822998\\
1	3.98381996154785\\
1	2.70346093177795\\
1	2.7038140296936\\
1	3.34348487854004\\
1	2.70364594459534\\
1	3.34404802322388\\
1	2.70409893989563\\
1	0\\
};
\addlegendentry{MDE};

\addplot [color=mycolor1,only marks,mark=*,mark options={solid},forget plot]
  table[row sep=crcr]{%
2	0\\
2	2.06386804580688\\
2	3.9830470085144\\
2	2.70327496528625\\
2	3.98435401916504\\
2	3.34425902366638\\
2	0\\
2	0\\
2	3.34448194503784\\
2	3.98420310020447\\
2	3.98417806625366\\
2	0\\
2	0\\
2	0\\
2	2.06371593475342\\
2	0\\
2	3.98412013053894\\
2	6.54380798339844\\
2	3.34428119659424\\
2	0\\
};

\addplot [color=mycolor1,only marks,mark=*,mark options={solid}]
  table[row sep=crcr]{%
3	0\\
3	3.34311699867249\\
3	2.06403684616089\\
3	2.06435298919678\\
3	2.06442618370056\\
3	3.98329901695251\\
3	0\\
3	2.06374597549438\\
3	3.98314690589905\\
3	2.70358204841614\\
3	3.34318780899048\\
3	3.98370504379272\\
3	0\\
3	2.70299601554871\\
3	0\\
3	0\\
3	3.34412288665771\\
3	2.70408582687378\\
3	2.06382894515991\\
3	2.06484198570251\\
};
\addplot [color=mycolor1,only marks,mark=*,mark options={solid},forget plot]
  table[row sep=crcr]{%
4	3.34450602531433\\
4	2.70465612411499\\
4	3.9842541217804\\
4	0\\
4	0\\
4	2.70361804962158\\
4	3.34443497657776\\
4	3.3442051410675\\
4	3.98412203788757\\
4	0\\
4	2.70440602302551\\
4	2.70354795455933\\
4	3.98445606231689\\
4	3.34337210655212\\
4	2.7031729221344\\
4	2.06350088119507\\
4	2.7041220664978\\
4	3.98393082618713\\
4	2.7035870552063\\
4	2.06354808807373\\
};
\addplot [color=mycolor1,only marks,mark=*,mark options={solid},forget plot]
  table[row sep=crcr]{%
5	2.70302295684814\\
5	2.0638120174408\\
5	2.70368385314941\\
5	3.34371113777161\\
5	3.34336185455322\\
5	0\\
5	0\\
5	2.06377196311951\\
5	3.34426617622375\\
5	0\\
5	0\\
5	2.06288194656372\\
5	2.70433521270752\\
5	2.70352816581726\\
5	0\\
5	0\\
5	0\\
5	0\\
5	3.34436511993408\\
5	3.34376907348633\\
};
\addplot [color=mycolor1,only marks,mark=*,mark options={solid},forget plot]
  table[row sep=crcr]{%
6	2.70398306846619\\
6	3.98413896560669\\
6	2.70379996299744\\
6	2.06457901000977\\
6	3.34415984153748\\
6	2.06386399269104\\
6	0\\
6	3.98335313796997\\
6	2.06371712684631\\
6	3.34421420097351\\
6	3.98331713676453\\
6	0\\
6	3.98413610458374\\
6	2.06382918357849\\
6	0\\
6	3.34377503395081\\
6	0\\
6	0\\
6	3.34429097175598\\
6	2.70383501052856\\
};
\addplot [color=mycolor1,only marks,mark=*,mark options={solid},forget plot]
  table[row sep=crcr]{%
7	2.70354390144348\\
7	0\\
7	2.70460295677185\\
7	3.34423208236694\\
7	0\\
7	2.70423698425293\\
7	2.06377696990967\\
7	0\\
7	0\\
7	0\\
7	3.98403000831604\\
7	3.34456205368042\\
7	3.98315095901489\\
7	3.34323811531067\\
7	2.70326495170593\\
7	3.34363293647766\\
7	0\\
7	2.70313405990601\\
7	3.98417210578918\\
7	2.70344686508179\\
};
\addplot [color=mycolor1,only marks,mark=*,mark options={solid},forget plot]
  table[row sep=crcr]{%
8	3.34394001960754\\
8	3.9837920665741\\
8	3.98305892944336\\
8	2.06428194046021\\
8	0\\
8	2.06423997879028\\
8	3.9834680557251\\
8	0\\
8	2.70423197746277\\
8	3.34431982040405\\
8	0\\
8	2.7035391330719\\
8	0\\
8	3.34361791610718\\
8	2.06392502784729\\
8	3.98410606384277\\
8	2.70362114906311\\
8	3.34315776824951\\
8	3.98356890678406\\
8	3.98443794250488\\
};
\addplot [color=mycolor1,only marks,mark=*,mark options={solid},forget plot]
  table[row sep=crcr]{%
9	2.06394577026367\\
9	3.34412384033203\\
9	3.98399186134338\\
9	3.9841251373291\\
9	2.06348896026611\\
9	0\\
9	3.34316611289978\\
9	2.06286597251892\\
9	3.98410415649414\\
9	2.06404900550842\\
9	2.06381487846375\\
9	2.06389117240906\\
9	0\\
9	2.06410598754883\\
9	2.70342183113098\\
9	3.34323501586914\\
9	3.34321093559265\\
9	3.98393988609314\\
9	2.06367611885071\\
9	3.34430599212646\\
};
\addplot [color=mycolor1,only marks,mark=*,mark options={solid},forget plot]
  table[row sep=crcr]{%
10	0\\
10	3.98293900489807\\
10	3.34414911270142\\
10	2.06450819969177\\
10	0\\
10	0\\
10	0\\
10	3.34374499320984\\
10	0\\
10	0\\
10	3.98308205604553\\
10	2.06371092796326\\
10	2.06403684616089\\
10	2.06367516517639\\
10	0\\
10	2.7036759853363\\
10	2.06395602226257\\
10	2.70334911346436\\
10	0\\
10	0\\
};
\addplot [color=mycolor1,only marks,mark=*,mark options={solid},forget plot]
  table[row sep=crcr]{%
11	3.34377312660217\\
11	2.70394802093506\\
11	2.7042019367218\\
11	3.98395895957947\\
11	0\\
11	3.34369802474976\\
11	2.70436382293701\\
11	2.70322489738464\\
11	3.34422206878662\\
11	2.70361518859863\\
11	3.98409199714661\\
11	3.34430408477783\\
11	0\\
11	3.98311901092529\\
11	3.98332715034485\\
11	2.06374907493591\\
11	0\\
11	0\\
11	3.98366022109985\\
11	2.06314301490784\\
};
\addplot [color=mycolor1,only marks,mark=*,mark options={solid},forget plot]
  table[row sep=crcr]{%
12	2.7039110660553\\
12	2.06496000289917\\
12	2.70287799835205\\
12	2.7045431137085\\
12	3.98310804367065\\
12	2.70452785491943\\
12	3.34418082237244\\
12	0\\
12	3.98356199264526\\
12	0\\
12	3.34408807754517\\
12	3.34393095970154\\
12	3.98390102386475\\
12	0\\
12	3.98423719406128\\
12	3.34313797950745\\
12	2.70388698577881\\
12	3.34319090843201\\
12	2.06460905075073\\
12	2.06348609924316\\
};
\addplot [color=mycolor1,only marks,mark=*,mark options={solid},forget plot]
  table[row sep=crcr]{%
13	0\\
13	0\\
13	0\\
13	0\\
13	3.98369693756104\\
13	0\\
13	0\\
13	0\\
13	0\\
13	0\\
13	0\\
13	0\\
13	0\\
13	0\\
13	3.98429584503174\\
13	0\\
13	0\\
13	0\\
13	0\\
13	2.06700801849365\\
};
\addplot [color=mycolor1,only marks,mark=*,mark options={solid},forget plot]
  table[row sep=crcr]{%
14	0\\
14	2.06355404853821\\
14	0\\
14	0\\
14	0\\
14	0\\
14	0\\
14	3.34336185455322\\
14	0\\
14	0\\
14	3.34429407119751\\
14	0\\
14	0\\
14	0\\
14	0\\
14	0\\
14	12.9436860084534\\
14	0\\
14	0\\
14	0\\
};
\addplot [color=mycolor1,only marks,mark=*,mark options={solid},forget plot]
  table[row sep=crcr]{%
15	3.37162494659424\\
15	0\\
15	2.06366395950317\\
15	2.70392298698425\\
15	0\\
15	3.98395490646362\\
15	0\\
15	12.9445569515228\\
15	0\\
15	0\\
15	0\\
15	0\\
15	3.34374690055847\\
15	0\\
15	3.98381996154785\\
15	4.64827418327332\\
15	0\\
15	2.06406998634338\\
15	0\\
15	0\\
};
\addplot [color=mycolor2,only marks,mark=*,mark options={solid}]
  table[row sep=crcr]
\caption{Execution time for the decoding the SIB1 step for different amount of devices and the baseline.}
\label{fig:MT_SIB1_Time}
\end{figure}

As seen in \autoref{fig:MT_SIB1_Time} the baseline and changed code have the same tendency around four different time values, with some measurements way off. Compared to the previous steps, this step have long execution time and the steps between the time values is around 600 ms.
\todo{Maybe make a histogram here to, good idea :)}

\section{SIB2}
The execution time for the SIB2 decoding step is measured dependably on the number of devices emulated and the baseline is measured as well, which gives the results seen in \autoref{fig:MT_SIB2_Time}. The transmission after SIB1 error narrows the number of measurement points down even further for this step in the process.

\begin{figure}[H]
\tikzsetnextfilename{MT_SIB2_Time}
\centering
\resizebox{0.5\textwidth}{!}{
% This file was created by matlab2tikz.
%
%The latest updates can be retrieved from
%  http://www.mathworks.com/matlabcentral/fileexchange/22022-matlab2tikz-matlab2tikz
%where you can also make suggestions and rate matlab2tikz.
%
\definecolor{mycolor1}{rgb}{0.00000,0.44700,0.74100}%
\definecolor{mycolor2}{rgb}{0.85000,0.32500,0.09800}%
%
\begin{tikzpicture}

\begin{axis}[%
width=\textwidth,
height=.66\textwidth,
at={(0.758in,0.481in)},
scale only axis,
xmin=-0.5,
xmax=15.5,
xtick={0,1,2,3,4,5,6,7,8,9,10,11,12,13,14,15},
xticklabels={{Baseline},{1},{2},{3},{4},{5},{6},{7},{8},{9},{10},{11},{12},{13},{14},{15}},
xlabel={Number of devices},
ymin=1.12,
ymax=1.18,
ylabel={Time (s)},
axis background/.style={fill=white},
title style={font=\bfseries},
title={SIB2 execution time},
legend style={at={(0.03,0.97)},anchor=north west,legend cell align=left,align=left,draw=white!15!black}
]
\addplot [color=mycolor1,only marks,mark=*,mark options={solid}]
  table[row sep=crcr]{%
1	1.14861416816711\\
1	1.14764595031738\\
1	1.14869999885559\\
1	1.14867520332336\\
1	1.14768314361572\\
1	1.14871907234192\\
1	1.14760398864746\\
1	0\\
1	1.14761400222778\\
1	0\\
1	0\\
1	1.14873504638672\\
1	1.14866995811462\\
1	1.14762902259827\\
1	1.14844989776611\\
1	1.14772701263428\\
1	1.14866614341736\\
1	1.14762902259827\\
1	1.14766502380371\\
1	0\\
};
\addlegendentry{Change code};

\addplot [color=mycolor1,only marks,mark=*,mark options={solid},forget plot]
  table[row sep=crcr]{%
2	0\\
2	1.14869999885559\\
2	1.14872193336487\\
2	1.14743685722351\\
2	1.14759993553162\\
2	1.1476469039917\\
2	0\\
2	0\\
2	1.14758086204529\\
2	1.14762592315674\\
2	1.14773392677307\\
2	0\\
2	0\\
2	0\\
2	1.14763808250427\\
2	0\\
2	1.14763689041138\\
2	1.14846801757813\\
2	1.14778089523315\\
2	0\\
};

\addplot [color=mycolor1,only marks,mark=*,mark options={solid},forget plot]
  table[row sep=crcr]{%
3	0\\
3	1.1484739780426\\
3	1.14770102500916\\
3	1.14761209487915\\
3	1.14775395393372\\
3	1.14868903160095\\
3	0\\
3	1.14869093894959\\
3	1.14865207672119\\
3	1.14784407615662\\
3	1.14785313606262\\
3	1.14755582809448\\
3	0\\
3	1.14864993095398\\
3	0\\
3	0\\
3	1.14760112762451\\
3	1.14760613441467\\
3	1.14757800102234\\
3	1.14759707450867\\
};
\addplot [color=mycolor1,only marks,mark=*,mark options={solid},forget plot]
  table[row sep=crcr]{%
4	1.14768004417419\\
4	1.14766192436218\\
4	1.1485378742218\\
4	0\\
4	0\\
4	1.14876699447632\\
4	1.14763498306274\\
4	1.14748001098633\\
4	1.14754796028137\\
4	0\\
4	1.14770913124084\\
4	1.14860796928406\\
4	1.14766693115234\\
4	1.14847683906555\\
4	1.14886403083801\\
4	1.14751315116882\\
4	1.14781403541565\\
4	1.14758896827698\\
4	1.14863586425781\\
4	1.14876008033752\\
};
\addplot [color=mycolor1,only marks,mark=*,mark options={solid},forget plot]
  table[row sep=crcr]{%
5	1.14869904518127\\
5	1.14763689041138\\
5	1.14762902259827\\
5	1.14758992195129\\
5	1.14868807792664\\
5	0\\
5	0\\
5	1.14759087562561\\
5	1.14765095710754\\
5	0\\
5	0\\
5	1.14874196052551\\
5	1.14760994911194\\
5	1.14874196052551\\
5	0\\
5	0\\
5	0\\
5	0\\
5	1.1476788520813\\
5	1.14865398406982\\
};
\addplot [color=mycolor1,only marks,mark=*,mark options={solid},forget plot]
  table[row sep=crcr]{%
6	1.14870095252991\\
6	1.14779305458069\\
6	1.14860796928406\\
6	1.14767384529114\\
6	1.14760899543762\\
6	1.14768695831299\\
6	0\\
6	1.14863681793213\\
6	1.14810705184937\\
6	1.1475830078125\\
6	1.14868903160095\\
6	0\\
6	1.14762282371521\\
6	1.1486439704895\\
6	0\\
6	1.14873480796814\\
6	0\\
6	0\\
6	1.14763402938843\\
6	1.14762711524963\\
};
\addplot [color=mycolor1,only marks,mark=*,mark options={solid},forget plot]
  table[row sep=crcr]{%
7	1.14771604537964\\
7	0\\
7	1.14766097068787\\
7	1.14756989479065\\
7	0\\
7	1.14763617515564\\
7	1.1486930847168\\
7	0\\
7	0\\
7	0\\
7	1.14779114723206\\
7	1.14769601821899\\
7	1.14865589141846\\
7	1.14863991737366\\
7	1.14755702018738\\
7	1.1476879119873\\
7	0\\
7	1.14870691299438\\
7	1.14779496192932\\
7	1.14769411087036\\
};
\addplot [color=mycolor1,only marks,mark=*,mark options={solid},forget plot]
  table[row sep=crcr]{%
8	1.14765310287476\\
8	1.14868092536926\\
8	1.148521900177\\
8	1.14763808250427\\
8	0\\
8	1.14762806892395\\
8	1.14874792098999\\
8	0\\
8	1.14770197868347\\
8	1.14808297157288\\
8	0\\
8	1.14866805076599\\
8	0\\
8	1.14819407463074\\
8	1.14826083183289\\
8	1.14760708808899\\
8	1.14861392974854\\
8	1.1487021446228\\
8	1.14819502830505\\
8	1.14798307418823\\
};
\addplot [color=mycolor1,only marks,mark=*,mark options={solid},forget plot]
  table[row sep=crcr]{%
9	0\\
9	1.14850616455078\\
9	1.14752793312073\\
9	1.14864087104797\\
9	1.14869999885559\\
9	0\\
9	1.14872694015503\\
9	1.14868307113647\\
9	1.14765501022339\\
9	1.14763712882996\\
9	1.14769601821899\\
9	1.14774680137634\\
9	0\\
9	1.14756488800049\\
9	1.14765596389771\\
9	1.14865112304688\\
9	1.14868903160095\\
9	1.14756202697754\\
9	1.14766097068787\\
9	1.14769601821899\\
};
\addplot [color=mycolor1,only marks,mark=*,mark options={solid},forget plot]
  table[row sep=crcr]{%
10	0\\
10	1.14868307113647\\
10	1.14751195907593\\
10	1.14756798744202\\
10	0\\
10	0\\
10	0\\
10	1.14770293235779\\
10	0\\
10	0\\
10	1.14860892295837\\
10	1.1476309299469\\
10	1.14767909049988\\
10	0\\
10	0\\
10	1.14785599708557\\
10	1.14808678627014\\
10	1.1480119228363\\
10	0\\
10	0\\
};
\addplot [color=mycolor1,only marks,mark=*,mark options={solid},forget plot]
  table[row sep=crcr]{%
11	1.1487090587616\\
11	1.14756202697754\\
11	1.14763712882996\\
11	1.14790201187134\\
11	0\\
11	1.14756107330322\\
11	1.14760112762451\\
11	1.14776301383972\\
11	1.14790391921997\\
11	1.14773893356323\\
11	1.14760589599609\\
11	1.14763593673706\\
11	0\\
11	0\\
11	1.14833998680115\\
11	0\\
11	0\\
11	0\\
11	1.14766192436218\\
11	1.1485481262207\\
};
\addplot [color=mycolor1,only marks,mark=*,mark options={solid},forget plot]
  table[row sep=crcr]{%
12	1.147705078125\\
12	1.14687490463257\\
12	1.14872598648071\\
12	1.14797592163086\\
12	1.14834499359131\\
12	1.14762115478516\\
12	1.14752197265625\\
12	0\\
12	0\\
12	0\\
12	1.14772891998291\\
12	1.14748311042786\\
12	1.14785885810852\\
12	0\\
12	1.14766788482666\\
12	1.14836406707764\\
12	1.1478488445282\\
12	1.14873003959656\\
12	1.14742994308472\\
12	1.14833402633667\\
};
\addplot [color=mycolor1,only marks,mark=*,mark options={solid},forget plot]
  table[row sep=crcr]{%
13	0\\
13	0\\
13	0\\
13	0\\
13	1.14760804176331\\
13	0\\
13	0\\
13	0\\
13	0\\
13	0\\
13	0\\
13	0\\
13	0\\
13	0\\
13	1.1476571559906\\
13	0\\
13	0\\
13	0\\
13	0\\
13	0\\
};
\addplot [color=mycolor1,only marks,mark=*,mark options={solid},forget plot]
  table[row sep=crcr]{%
14	0\\
14	0\\
14	0\\
14	0\\
14	0\\
14	0\\
14	0\\
14	1.14835405349731\\
14	0\\
14	0\\
14	1.14746284484863\\
14	0\\
14	0\\
14	0\\
14	0\\
14	0\\
14	0\\
14	0\\
14	0\\
14	0\\
};
\addplot [color=mycolor1,only marks,mark=*,mark options={solid},forget plot]
  table[row sep=crcr]{%
15	1.12556910514832\\
15	0\\
15	0\\
15	1.14798092842102\\
15	0\\
15	1.15576195716858\\
15	0\\
15	0\\
15	0\\
15	0\\
15	0\\
15	0\\
15	1.16539812088013\\
15	0\\
15	0\\
15	0\\
15	0\\
15	0\\
15	0\\
15	0\\
};
\addplot [color=mycolor2,only marks,mark=*,mark options={solid}]
  table[row sep=crcr]
\caption{Execution time for the decoding the SIB2 step for different amount of devices and the baseline.}
\label{fig:MT_SIB2_Time}
\end{figure}

\todo{is this expected when each device decodes the SIB2 message or is this because you have picked device number 1.}

As seen in \autoref{fig:MT_SIB2_Time} do the baseline and changed code have the same tendency, beside for when there is emulated 15 devices. There is not a big spread for the measurements for all the other setups. \todo{any comment on why?}

\section{NPRACH}
The execution time for the NPRACH step is measured dependably on the number of devices emulated and the baseline is measured as well, which gives the results seen in \autoref{fig:MT_Nprach_Time}. Here will the NPRACH error occur for some of the measurements and will effect the results. As the shown results is only for the first device, there are still measurement points for measurements where this error occurs, but no following RAR measurement \todo{the last sentence is confusing and needs to be reformulated}.

\begin{figure}[H]
\tikzsetnextfilename{MT_Nprach_Time}
\centering
\resizebox{0.5\textwidth}{!}{
% This file was created by matlab2tikz.
%
%The latest updates can be retrieved from
%  http://www.mathworks.com/matlabcentral/fileexchange/22022-matlab2tikz-matlab2tikz
%where you can also make suggestions and rate matlab2tikz.
%
\definecolor{mycolor1}{rgb}{0.00000,0.44700,0.74100}%
\definecolor{mycolor2}{rgb}{0.85000,0.32500,0.09800}%
%
\begin{tikzpicture}

\begin{axis}[%
width=4.521in,
height=3.566in,
at={(0.758in,0.481in)},
scale only axis,
xmin=-0.5,
xmax=15.5,
xtick={0,1,2,3,4,5,6,7,8,9,10,11,12,13,14,15},
xticklabels={{Baseline},{1},{2},{3},{4},{5},{6},{7},{8},{9},{10},{11},{12},{13},{14},{15}},
xlabel={Number of UEs},
ymin=0.3,
ymax=0.33,
ylabel={Time (s)},
axis background/.style={fill=white},
title style={font=\bfseries},
title={NPRACH execution time},
legend style={at={(0.03,0.97)},anchor=north west,legend cell align=left,align=left,draw=white!15!black}
]
\addplot [color=mycolor1,only marks,mark=*,mark options={solid}]
  table[row sep=crcr]{%
1	0.311680793762207\\
1	0.311604022979736\\
1	0.311627864837646\\
1	0.311679840087891\\
1	0.311601877212524\\
1	0.311666011810303\\
1	0.311678886413574\\
1	0\\
1	0.311702966690063\\
1	0\\
1	0\\
1	0.311586856842041\\
1	0.311655044555664\\
1	0.311685085296631\\
1	0.311557054519653\\
1	0.312576055526733\\
1	0.311637878417969\\
1	0.312717914581299\\
1	0.311601161956787\\
1	0\\
};
\addlegendentry{Change code};

\addplot [color=mycolor1,only marks,mark=*,mark options={solid}]
  table[row sep=crcr]{%
2	0\\
2	0.311550140380859\\
2	0.311643123626709\\
2	0.31199312210083\\
2	0.311596870422363\\
2	0.31159496307373\\
2	0\\
2	0\\
2	0.311676025390625\\
2	0.311542987823486\\
2	0.312494993209839\\
2	0\\
2	0\\
2	0\\
2	0.311656951904297\\
2	0\\
2	0.312523126602173\\
2	0.311601877212524\\
2	0.312467098236084\\
2	0\\
};
\addlegendentry{Baseline};

\addplot [color=mycolor1,only marks,mark=*,mark options={solid},forget plot]
  table[row sep=crcr]{%
3	0\\
3	0.311551809310913\\
3	0.311765193939209\\
3	0.311564922332764\\
3	0.311563014984131\\
3	0.311522006988525\\
3	0\\
3	0.311599016189575\\
3	0.311592102050781\\
3	0.312321901321411\\
3	0.312319040298462\\
3	0.311748027801514\\
3	0\\
3	0.311631202697754\\
3	0\\
3	0\\
3	0.31165599822998\\
3	0.311599969863892\\
3	0.311576128005981\\
3	0.311599969863892\\
};
\addplot [color=mycolor1,only marks,mark=*,mark options={solid},forget plot]
  table[row sep=crcr]{%
4	0.311563014984131\\
4	0.311460018157959\\
4	0.311659097671509\\
4	0\\
4	0\\
4	0.311501979827881\\
4	0.311540126800537\\
4	0.311670064926147\\
4	0.31163501739502\\
4	0\\
4	0.311506986618042\\
4	0.311635971069336\\
4	0.31148099899292\\
4	0.311570167541504\\
4	0.311554193496704\\
4	0.311676979064941\\
4	0.312494039535522\\
4	0.311719179153442\\
4	0.311642169952393\\
4	0.311550855636597\\
};
\addplot [color=mycolor1,only marks,mark=*,mark options={solid},forget plot]
  table[row sep=crcr]{%
5	0.311535835266113\\
5	0.311478137969971\\
5	0.312060117721558\\
5	0.311666011810303\\
5	0.311637878417969\\
5	0\\
5	0\\
5	0.311530113220215\\
5	0.311499834060669\\
5	0\\
5	0\\
5	0.31154990196228\\
5	0.311671018600464\\
5	0.311539888381958\\
5	0\\
5	0\\
5	0\\
5	0\\
5	0.311614990234375\\
5	0.311556816101074\\
};
\addplot [color=mycolor1,only marks,mark=*,mark options={solid},forget plot]
  table[row sep=crcr]{%
6	0.311546087265015\\
6	0.311470031738281\\
6	0.311733961105347\\
6	0.311630010604858\\
6	0.312007188796997\\
6	0.311566114425659\\
6	0\\
6	0.311662197113037\\
6	0.31212592124939\\
6	0.3122398853302\\
6	0.311693906784058\\
6	0\\
6	0.311550140380859\\
6	0.311647891998291\\
6	0\\
6	0.311639070510864\\
6	0\\
6	0\\
6	0.311668872833252\\
6	0.311506032943726\\
};
\addplot [color=mycolor1,only marks,mark=*,mark options={solid},forget plot]
  table[row sep=crcr]{%
7	0.312558889389038\\
7	0\\
7	0.311611890792847\\
7	0.311638116836548\\
7	0\\
7	0.311750888824463\\
7	0.31158185005188\\
7	0\\
7	0\\
7	0\\
7	0.311608791351318\\
7	0.311516046524048\\
7	0.31172513961792\\
7	0.311556100845337\\
7	0.312680959701538\\
7	0.311551094055176\\
7	0\\
7	0.311511039733887\\
7	0.311432123184204\\
7	0.312183856964111\\
};
\addplot [color=mycolor1,only marks,mark=*,mark options={solid},forget plot]
  table[row sep=crcr]{%
8	0.311455965042114\\
8	0.311665058135986\\
8	0.311743974685669\\
8	0.311501026153564\\
8	0\\
8	0.311676025390625\\
8	0.311566114425659\\
8	0\\
8	0.311695098876953\\
8	0.311681032180786\\
8	0\\
8	0.311643838882446\\
8	0\\
8	0.312039852142334\\
8	0.31199312210083\\
8	0.311548948287964\\
8	0.311679840087891\\
8	0.311714887619019\\
8	0.31155800819397\\
8	0.311631917953491\\
};
\addplot [color=mycolor1,only marks,mark=*,mark options={solid},forget plot]
  table[row sep=crcr]{%
9	0\\
9	0.311545848846436\\
9	0.311694145202637\\
9	0.31157112121582\\
9	0.311569929122925\\
9	0\\
9	0.31165599822998\\
9	0.311696767807007\\
9	0.311599969863892\\
9	0.311621904373169\\
9	0.311509132385254\\
9	0.311530113220215\\
9	0\\
9	0.312504053115845\\
9	0.311852216720581\\
9	0.311555862426758\\
9	0.311640977859497\\
9	0.311907052993774\\
9	0.311527013778687\\
9	0.311538934707642\\
};
\addplot [color=mycolor1,only marks,mark=*,mark options={solid},forget plot]
  table[row sep=crcr]{%
10	0\\
10	0.311673879623413\\
10	0.311647891998291\\
10	0.311652898788452\\
10	0\\
10	0\\
10	0\\
10	0.311555862426758\\
10	0\\
10	0\\
10	0.31167197227478\\
10	0.312138080596924\\
10	0.31160306930542\\
10	0\\
10	0\\
10	0.312062978744507\\
10	0.311703205108643\\
10	0.311650991439819\\
10	0\\
10	0\\
};
\addplot [color=mycolor1,only marks,mark=*,mark options={solid},forget plot]
  table[row sep=crcr]{%
11	0.311681032180786\\
11	0.31174898147583\\
11	0.311945915222168\\
11	0.312388181686401\\
11	0\\
11	0.311898946762085\\
11	0.311634063720703\\
11	0.312273979187012\\
11	0.31233811378479\\
11	0.311637878417969\\
11	0.311670064926147\\
11	0.311576128005981\\
11	0\\
11	0\\
11	0.312019824981689\\
11	0\\
11	0\\
11	0\\
11	0.311609029769897\\
11	0.311727046966553\\
};
\addplot [color=mycolor1,only marks,mark=*,mark options={solid},forget plot]
  table[row sep=crcr]{%
12	0.311507940292358\\
12	0.311978101730347\\
12	0.311573028564453\\
12	0.311161041259766\\
12	0.312034130096436\\
12	0.311545848846436\\
12	0.311670064926147\\
12	0\\
12	0\\
12	0\\
12	0.311485052108765\\
12	0.311722993850708\\
12	0.311711072921753\\
12	0\\
12	0.31154990196228\\
12	0.311701059341431\\
12	0.31179404258728\\
12	0.311655044555664\\
12	0.311732053756714\\
12	0.312110900878906\\
};
\addplot [color=mycolor1,only marks,mark=*,mark options={solid},forget plot]
  table[row sep=crcr]{%
13	0\\
13	0\\
13	0\\
13	0\\
13	0.315821886062622\\
13	0\\
13	0\\
13	0\\
13	0\\
13	0\\
13	0\\
13	0\\
13	0\\
13	0\\
13	0.323969841003418\\
13	0\\
13	0\\
13	0\\
13	0\\
13	0\\
};
\addplot [color=mycolor1,only marks,mark=*,mark options={solid},forget plot]
  table[row sep=crcr]{%
14	0\\
14	0\\
14	0\\
14	0\\
14	0\\
14	0\\
14	0\\
14	0.315366983413696\\
14	0\\
14	0\\
14	0.315733194351196\\
14	0\\
14	0\\
14	0\\
14	0\\
14	0\\
14	0\\
14	0\\
14	0\\
14	0\\
};
\addplot [color=mycolor1,only marks,mark=*,mark options={solid},forget plot]
  table[row sep=crcr]{%
15	0.309165954589844\\
15	0\\
15	0\\
15	0.321655988693237\\
15	0\\
15	0.306154012680054\\
15	0\\
15	0\\
15	0\\
15	0\\
15	0\\
15	0\\
15	0.321050882339478\\
15	0\\
15	0\\
15	0\\
15	0\\
15	0\\
15	0\\
15	0\\
};
\addplot [color=mycolor2,only marks,mark=*,mark options={solid},forget plot]
  table[row sep=crcr]
\caption{Execution time for the NPRACH step for different amount of devices and the baseline.}
\label{fig:MT_Nprach_Time}
\end{figure}

As seen in \autoref{fig:MT_Nprach_Time} the baseline and changed code takes approximately the same time, but the changed code have a bigger spread of its measurement point. All the measurements of the changed code over 12 devices, however does not follow the tendency of the other measurements. \todo{any comment on why not?}

\section{RAR}
The execution time for the RAR step is measured dependably on the number of devices emulated and the baseline is measured as well, which gives the results seen in \autoref{fig:MT_Rar_Time}. Here does the msg2 error occurs if no msg2 is received and the measurements should therefore be excluded. \todo{maybe do a similar explanation in the introduction of the other errors}

\begin{figure}[H]
\tikzsetnextfilename{MT_Rar_Time}
\centering
\resizebox{0.5\textwidth}{!}{
% This file was created by matlab2tikz.
%
%The latest updates can be retrieved from
%  http://www.mathworks.com/matlabcentral/fileexchange/22022-matlab2tikz-matlab2tikz
%where you can also make suggestions and rate matlab2tikz.
%
\definecolor{mycolor1}{rgb}{0.00000,0.44700,0.74100}%
\definecolor{mycolor2}{rgb}{0.85000,0.32500,0.09800}%
%
\begin{tikzpicture}

\begin{axis}[%
width=\textwidth,
height=.66\textwidth,
at={(0.758in,0.481in)},
scale only axis,
xmin=-0.5,
xmax=15.5,
xtick={0,1,2,3,4,5,6,7,8,9,10,11,12,13,14,15},
xticklabels={{Baseline},{1},{2},{3},{4},{5},{6},{7},{8},{9},{10},{11},{12},{13},{14},{15}},
xlabel={Number of devices},
ymin=0.03,
ymax=0.06,
ylabel={Time (s)},
axis background/.style={fill=white},
title style={font=\bfseries},
title={RAR execution time},
legend style={at={(0.03,0.97)},anchor=north west,legend cell align=left,align=left,draw=white!15!black}
]
\addplot [color=mycolor1,only marks,mark=*,mark options={solid}]
  table[row sep=crcr]{%
1	0.0341880321502686\\
1	0.034235954284668\\
1	0.0342159271240234\\
1	0.0342271327972412\\
1	0.0342719554901123\\
1	0.0341730117797852\\
1	0\\
1	0\\
1	0.0341699123382568\\
1	0\\
1	0\\
1	0.0342731475830078\\
1	0.0341849327087402\\
1	0.0341880321502686\\
1	0.0343189239501953\\
1	0.0333008766174316\\
1	0.0342051982879639\\
1	0.033297061920166\\
1	0.0342288017272949\\
1	0\\
};
\addlegendentry{Change code};

\addplot [color=mycolor1,only marks,mark=*,mark options={solid},forget plot]
  table[row sep=crcr]{%
2	0\\
2	0.0342118740081787\\
2	0.0341849327087402\\
2	0.0340640544891357\\
2	0.0342371463775635\\
2	0.0342140197753906\\
2	0\\
2	0\\
2	0.0341610908508301\\
2	0.0342671871185303\\
2	0.0333490371704102\\
2	0\\
2	0\\
2	0\\
2	0.0342299938201904\\
2	0\\
2	0.0333659648895264\\
2	0.0342400074005127\\
2	0.0333349704742432\\
2	0\\
};

\addplot [color=mycolor1,only marks,mark=*,mark options={solid},forget plot]
  table[row sep=crcr]{%
3	0\\
3	0.0342600345611572\\
3	0.0338799953460693\\
3	0.0342161655426025\\
3	0.0342400074005127\\
3	0.0343279838562012\\
3	0\\
3	0.0342199802398682\\
3	0.0342228412628174\\
3	0.0342550277709961\\
3	0.0343589782714844\\
3	0.0341391563415527\\
3	0\\
3	0.0343039035797119\\
3	0\\
3	0\\
3	0.0342159271240234\\
3	0.03420090675354\\
3	0.0342509746551514\\
3	0.0342059135437012\\
};
\addplot [color=mycolor1,only marks,mark=*,mark options={solid},forget plot]
  table[row sep=crcr]{%
4	0.0342638492584229\\
4	0.0343360900878906\\
4	0.0341908931732178\\
4	0\\
4	0\\
4	0.0342490673065186\\
4	0.0342738628387451\\
4	0.0342228412628174\\
4	0.0341949462890625\\
4	0\\
4	0.0342400074005127\\
4	0.0342440605163574\\
4	0.0342991352081299\\
4	0.0341489315032959\\
4	0.0341489315032959\\
4	0.0342459678649902\\
4	0.0341918468475342\\
4	0.0341000556945801\\
4	0.034153938293457\\
4	0.034203052520752\\
};
\addplot [color=mycolor1,only marks,mark=*,mark options={solid},forget plot]
  table[row sep=crcr]{%
5	0.0342710018157959\\
5	0.0341567993164063\\
5	0.0337579250335693\\
5	0.0341598987579346\\
5	0.0341479778289795\\
5	0\\
5	0\\
5	0.0342059135437012\\
5	0.0342340469360352\\
5	0\\
5	0\\
5	0.0342390537261963\\
5	0.0341389179229736\\
5	0.0341851711273193\\
5	0\\
5	0\\
5	0\\
5	0\\
5	0.034188985824585\\
5	0.0342321395874023\\
};
\addplot [color=mycolor1,only marks,mark=*,mark options={solid},forget plot]
  table[row sep=crcr]{%
6	0.0342798233032227\\
6	0.0341129302978516\\
6	0.0341250896453857\\
6	0.0342040061950684\\
6	0.0337188243865967\\
6	0.0342609882354736\\
6	0\\
6	0.0342237949371338\\
6	0.0336298942565918\\
6	0.0335140228271484\\
6	0.0341589450836182\\
6	0\\
6	0.0342769622802734\\
6	0.0342199802398682\\
6	0\\
6	0.0342271327972412\\
6	0\\
6	0\\
6	0.0341720581054688\\
6	0\\
};
\addplot [color=mycolor1,only marks,mark=*,mark options={solid},forget plot]
  table[row sep=crcr]{%
7	0.0336170196533203\\
7	0\\
7	0.0341181755065918\\
7	0.0341799259185791\\
7	0\\
7	0.0342490673065186\\
7	0.0342171192169189\\
7	0\\
7	0\\
7	0\\
7	0.0346660614013672\\
7	0.0343749523162842\\
7	0.0340960025787354\\
7	0.0341479778289795\\
7	0.0340781211853027\\
7	0.0341360569000244\\
7	0\\
7	0.0341110229492188\\
7	0.034153938293457\\
7	0.0335919857025146\\
};
\addplot [color=mycolor1,only marks,mark=*,mark options={solid},forget plot]
  table[row sep=crcr]{%
8	0.0365910530090332\\
8	0.0427010059356689\\
8	0.0370931625366211\\
8	0.0381538867950439\\
8	0\\
8	0.0364658832550049\\
8	0.0379340648651123\\
8	0\\
8	0.0374419689178467\\
8	0.0407381057739258\\
8	0\\
8	0.0341041088104248\\
8	0\\
8	0.0414822101593018\\
8	0.0361030101776123\\
8	0.0386650562286377\\
8	0.0383260250091553\\
8	0.0362110137939453\\
8	0.0341310501098633\\
8	0.037121057510376\\
};
\addplot [color=mycolor1,only marks,mark=*,mark options={solid},forget plot]
  table[row sep=crcr]{%
9	0\\
9	0.0370240211486816\\
9	0.0365219116210938\\
9	0.0378210544586182\\
9	0.038114070892334\\
9	0\\
9	0.037787914276123\\
9	0.0381491184234619\\
9	0.0389068126678467\\
9	0.0386679172515869\\
9	0.0383989810943604\\
9	0.0359559059143066\\
9	0\\
9	0.042572021484375\\
9	0.0371799468994141\\
9	0.0409531593322754\\
9	0.0386199951171875\\
9	0.0404210090637207\\
9	0.036876916885376\\
9	0.0371849536895752\\
};
\addplot [color=mycolor1,only marks,mark=*,mark options={solid},forget plot]
  table[row sep=crcr]{%
10	0\\
10	0.0420839786529541\\
10	0.0408451557159424\\
10	0.0439350605010986\\
10	0\\
10	0\\
10	0\\
10	0.0399701595306396\\
10	0\\
10	0\\
10	0.0381450653076172\\
10	0.0463569164276123\\
10	0.0390889644622803\\
10	0\\
10	0\\
10	0.0403990745544434\\
10	0.0427589416503906\\
10	0.0387308597564697\\
10	0\\
10	0\\
};
\addplot [color=mycolor1,only marks,mark=*,mark options={solid},forget plot]
  table[row sep=crcr]{%
11	0.0449898242950439\\
11	0.0432069301605225\\
11	0.0445609092712402\\
11	0.0408809185028076\\
11	0\\
11	0.0404579639434814\\
11	0.0427489280700684\\
11	0.0475912094116211\\
11	0.0448739528656006\\
11	0.0414700508117676\\
11	0.0436179637908936\\
11	0.0411219596862793\\
11	0\\
11	0\\
11	0.0407330989837646\\
11	0\\
11	0\\
11	0\\
11	0.0408778190612793\\
11	0.040151834487915\\
};
\addplot [color=mycolor1,only marks,mark=*,mark options={solid},forget plot]
  table[row sep=crcr]{%
12	0.0453028678894043\\
12	0.0516760349273682\\
12	0.0564429759979248\\
12	0.0475959777832031\\
12	0.0430538654327393\\
12	0.0450849533081055\\
12	0.0470409393310547\\
12	0\\
12	0\\
12	0\\
12	0.0429530143737793\\
12	0.0567960739135742\\
12	0.0497169494628906\\
12	0\\
12	0.0499541759490967\\
12	0.0428369045257568\\
12	0.0435919761657715\\
12	0.0445668697357178\\
12	0.0499989986419678\\
12	0.0418119430541992\\
};
\addplot [color=mycolor1,only marks,mark=*,mark options={solid},forget plot]
  table[row sep=crcr]{%
13	0\\
13	0\\
13	0\\
13	0\\
13	0\\
13	0\\
13	0\\
13	0\\
13	0\\
13	0\\
13	0\\
13	0\\
13	0\\
13	0\\
13	0\\
13	0\\
13	0\\
13	0\\
13	0\\
13	0\\
};
\addplot [color=mycolor1,only marks,mark=*,mark options={solid},forget plot]
  table[row sep=crcr]{%
14	0\\
14	0\\
14	0\\
14	0\\
14	0\\
14	0\\
14	0\\
14	0\\
14	0\\
14	0\\
14	0\\
14	0\\
14	0\\
14	0\\
14	0\\
14	0\\
14	0\\
14	0\\
14	0\\
14	0\\
};
\addplot [color=mycolor1,only marks,mark=*,mark options={solid},forget plot]
  table[row sep=crcr]{%
15	0\\
15	0\\
15	0\\
15	0\\
15	0\\
15	0\\
15	0\\
15	0\\
15	0\\
15	0\\
15	0\\
15	0\\
15	0\\
15	0\\
15	0\\
15	0\\
15	0\\
15	0\\
15	0\\
15	0\\
};
\addplot [color=mycolor2,only marks,mark=*,mark options={solid}]
  table[row sep=crcr]
\caption{Execution time for the decoding the MIB for different amount of devices and the baseline. A single measurement for the base line is placed at 5.0834 s, which is not shown on this figure.}
\label{fig:MT_Rar_Time}
\end{figure}

As seen in \autoref{fig:MT_Rar_Time} the changed code starts out with following the tendency for the baseline, but around eight devices it begin to get a higher and higher execution time and getting a wider spread compared to the values at a low amount of devices. \todo{any comment on why?}

\todo{a summary of the time would be nice and also what have been found in this regards bottlenecks and so on}
