\chapter{Testing of Massive IoT system} \label{ch:mass_test}
The focus of this chapter is to showcase the emulator describe in \autoref{ch:MassOver}. This is done in a series of test, where it is compared to the original code and to see if the changed made for fill the goal set in \autoref{ch:MassOver}:

\textit{The goal of these changes will be to have a massive amount of individual devices emulated, without them effecting each other through their processing time and combining their signals and transmit it as one to the eNB.}

This will be done by comparing the changed code to the original code as well as testing the performance of the code at higher number of devices emulated.
The performance criteria that will be look into is CPU usage, memory usage, execution time and error rate.
CPU and memory usage is used to test where possible bottle necks can be compared to overloading the used computer. CPU usage will be measure with the CPU stat tool, which can measure the CPU usage on the individual kernels at a sample rate of 3 Hz. The memory usage is measured using the system monitor tool, as all bigger buffers is allocated in the initialization and the used memory therefore is nearly static. The execution time will be look into to see if any processes in code is taking longer, after the changes. This will be measured by inserting prints of time stamps into the code, when the code goes from one step to another. Lastly the error rate will showcase the stability of the code.
The parameters used for the eNB and SRS code is the default settings shown in \todo{Insert app ref to the two apps}.