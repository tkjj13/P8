\chapter{Battery Lifetime of a NB-IoT Device}
%Here should be an introduction of what we will test (the emulator and/or the protocol). 

The focus of this chapter is to provide an overview and an estimate of the power efficiency of the NB-IoT protocol. From \autoref{ch:NB-IoT} it is known that the desired performance of any IoT protocol is to achieve a battery life time of 10 years with a battery capacity of 5 Wh even in extreme situations. 

In \appref{app:bat_model} the model which will be used to estimate the battery lifetime is described, but to use it effectively several elements needs to be measured on an actual device, these are:

\begin{tabular}{ll}
$E_{sync}$ & the energy used for the modem to boot and synchronise to a cell. \\
$E_{attach}$ & the energy needed to attach the device to the cell. \\
$P_{tx}$ & the power consumption during the data transmission. \\
$E_{release}$ & the energy spent when disconnecting from the cell. \\
$P_{eDRX}$ & the power consumption during \gls{eDRX} idle mode. \\
$P_{PSM}$ & the power consumption during \gls{PSM} idle mode. \\
\end{tabular}

Multiple devices would of cause increase the accuracy of the model, but here only evaluation board for the Quectel BC95 modem is tested \citep{BC95}. Even with that a \gls{NB-IoT} system has a multitude of parameters which influence can the throughput and energy consumption during a transmission and thereby influence several of the states and transitions the device experience. To account for all of this is to huge a task for this project, therefore a couple of parameters is chosen to simplify this task. The downside of this is the reliability of the results as these can only be assume valid in cases where the other parameters are chosen to the same values as used in this project. The values chosen are CP format, operation mode, frequency and $P_{TX}$. The CP format and operation mode is chosen as these are some often mentioned parameters which also only has a few values, repetition could also be considered here however as four different channels can be repeated with each having its own specifying parameter it is chosen to set all repetition to 1 and leave this parameter up to future analysis. The frequency typically has a high influence on the hardware used it is therefore considered as well as the transmit power, $P_{TX}$, which also will have a significant influence on the power consumption when the device is active. Some of the main parameters that are not tested besides repetition is \gls{MCS}, \gls{TBS}, NPRACH start power, NPRACH step size, number of tones, sub-carrier spacing, number of antenna ports and LTE parameters in inband deployment. The influence of some of these parameters is tested by a NDS group at Aalborg University \todo{ref til NDS rapport}. However it will not be further considered here.

From a measurement perspective some of the elements can and should be measured together, these are $E_{sync}$ and $E_{attach}$. This is because these steps are depending on each other and can only be separated during the  post processing. Likewise the $E_{release}$ could be tested simultaneously as it requires the DUT to be connected and depends on most of the same parameters.  



\section{Test Setup}
%Here should be a description of the general setup (including figure) used in all test and a list of baseline values for all parameters. Including physical setup, BSE, UEE.
All the parameters can be tested with the same setup which can be seen in \autoref{fig:IPE_test_setup}.

\begin{figure}[H]
\centering
\includegraphics[width=0.5\textwidth]{figures/IPE_test_setup.pdf}
\caption{Setup used to test for the power efficiency of the \gls{DUT}'s}
\label{fig:IPE_test_setup}
\end{figure}
\todo{Make fig in tikz}


As seen in \autoref{fig:IPE_test_setup}, an orchestrator, in this case \gls{TAP}, maintains the system. The orchestrator has a \gls{LAN} connection with the \gls{PSU} and the UXM and in this case a serial connection with the DUT. The \gls{PSU}s connection with the DUT is wires used to power on and analyse the power consumption of the device. It should be noted that the \gls{DUT} in question have two power inputs, one for the device power and one for the RF modem power, here only the modem is connected to the power analyser. The DUT is connected to the UXM using RF SMA cables.

\gls{TAP} is a software developed by Keysight Technologies which enables quick and easy access to the computers external connection. It also allows for C\# features to be used in the test design, meaning test steps can and has been designed to enable functionalities of the instruments. This includes for instance cell activation of the UXM and data collection from the PSU among other key functionalities, this couple with the ability to use loops, parallel test steps, serial com ports and batch scripts make this a very strong tool in designing, controlling and performing measurements. \citep{TAP} 

The \gls{PSU} is a N6705C DC Power Analyzer with the internal module TBD it has a range of TBD. When getting data from the PSU it can be done in a couple different ways, but here it is stored in the internal RAM of the PSU before download. This though has a limitation of 512.000 data points per capture. \citep{PSU}

The base station is a E7515A UXM Wireless Test Set from Keysight Technologies which is able to emulate, both LTE and NB-IoT conditions. It features most of the functionalities in the NB-IoT protocol until release 13. It has a multitude of changeable parameters with easy access through a SCPI interface. \citep{UXM}

The initial settings of each component in the system can be seen in \autoref{tab:setup_parameters}.

\begin{table}[H]
\captionsetup{belowskip=0em}
\noindent
\centering
%\resizebox{!}{0.5\textheight}{
\begin{minipage}[t]{0.48\textwidth}
\begin{tabular}{|p{4cm}|p{2cm}|}
\hline
\multicolumn{2}{|c|}{\textbf{Power Supply/Analyser}}                         \\ \hline
Enable             & Off            \\ \hline
Volt               & 3.6 V          \\ \hline
Ampere             & 2.5 A          \\ \hline
Sample interval	   & 100 $\mu$s		\\ \hline
\multicolumn{2}{c}{}\\ \hline
\multicolumn{2}{|c|}{\textbf{Ekstern IoT device}}                            \\ \hline
Enable             & Off            \\ \hline
DL\_EARFCN         & 6240           \\ \hline
\end{tabular}
%\caption{Initial values of the parameters in the emulator.}
\end{minipage}% 
\hfill
\begin{minipage}[t]{0.48\textwidth}
\begin{tabular}{|p{4cm}|p{2cm}|} \hline
\multicolumn{2}{|c|}{\textbf{UXM \gls{BSE}}} \\ \hline
Cell type			 & NB-IoT         \\ \hline
Number of cells		 & 1              \\ \hline
Operation mode		 & Standalone     \\ \hline
Host cell DL\_EARFCN & 6240           \\ \hline
PRB offset			 & 0	          \\ \hline
Cell ID				 & 0              \\ \hline
Tx power			 & -80 dB/per 15 kHz \\ \hline
Repetition NPDSCH	 & 1	          \\ \hline
Max Repetition NPDCCH & 4	          \\ \hline
Repetition NPUSCH	 & 1	          \\ \hline
Repetition NPRACH	 & 1	          \\ \hline
CP format			 & Normal         \\ \hline
$P_{TX}$				 & 23 dBm         \\ \hline
MAC padding DL		 & off       	  \\ \hline
MAC padding UL		 & off       	  \\ \hline
\end{tabular}
%\caption{Initial values of the parameters in the emulator.}
%\label{tab:setup_parameters}
\end{minipage}
\caption{Initial values of the parameters in the emulator.}
\label{tab:setup_parameters}
\end{table}



%\subsubsection{Device Power Consumption}
%
%To measure the device power, the power input to the device is used in the setup. The test is performed using the following procedure:
%
%\textbf{Test Procedure}
%\vspace{-1.5em}
%\begin{enumerate}
%\item Setup the \gls{DUT} as shown on \autoref{fig:IPE_test_setup}
%\item Put in settings as described in \autoref{tab:setup_parameters} and \autoref{tab:UXM_initial_values} 
%\item Turn on power supply 
%\item Measure power output over 2 min
%\item Save measurements as "<device>\_Power\_consumption"
%\item Turn off power supply
%\item Change to next \gls{DUT}
%\item Repeat step 1-7 for all \gls{DUT}s.
%\end{enumerate}
%
%\textbf{Results}\\
%\begin{table}[H]
%\centering
%\begin{tabular}{|c|c|c|c|}\hline
%\textbf{Device}	& Quectel	& Telit & Ublox \\ \hline
%$\mathbf{P_{device}}$	& & & \\ \hline
%\end{tabular}
%\caption{Average power consumption of the \gls{DUT}s}
%\label{tab:device_power_results}
%\end{table}

\section{Energy to Connect and Disconnect the DUT to the Cell} \label{sec:performance_attach}

To measure the energy used to connect and disconnect the device to the cell the following procedure is used.

\subsection{Test Procedure}
\begin{enumerate}
\item Setup the \gls{DUT} as shown on \autoref{fig:IPE_test_setup}
\item Turn on power supply 
\item Input settings as described in \autoref{tab:setup_parameters}
\item Input chosen value of chosen parameter
\item Put device in disconnected state 
\item Input to log "start <Parameters used> <Parameter value>"
\item Turn on power analyser
\item Start up and attach of DUT
\item Verify connection to cell
\item Release DUT
\item Turn off power analyser
\item Save L3 log from UXM as "Attach\_<Parameters used>\_<Parameters value>.xml"
\item Turn off power supply
\item Change to next value
\item Repeat step 4-14 for all values
\item Save measurements as "Attach\_<Parameters used>\_MessageLog.csv"
\item Change to next parameter
\item Repeat step 3-17 for all parameters
\end{enumerate}


\subsection{Results}
\tikzsetnextfilename{Attach_raw}
\begin{figure}[H]
\centering
\begin{minipage}[tbp]{0.58\textwidth}
\resizebox{\textwidth}{!}{
% This file was created by matlab2tikz.
%
%The latest updates can be retrieved from
%  http://www.mathworks.com/matlabcentral/fileexchange/22022-matlab2tikz-matlab2tikz
%where you can also make suggestions and rate matlab2tikz.
%
\definecolor{mycolor1}{rgb}{0.00000,0.44700,0.74100}%
%
\begin{tikzpicture}
\begin{axis}[%
width=\textwidth,
height=0.66\textwidth,
at={(0.758in,0.481in)},
scale only axis,
xmin=0,
xmax=51.1998976,
xlabel={Time [s]},
ymin=-0.1,
ymax=0.8,
ylabel={Power Consumption [W]},
axis background/.style={fill=white}
]
\addplot [color=mycolor1,only marks,mark=*,mark options={solid},forget plot]
  table[row sep=crcr]{%
0	-0.00027465822\\
0.007168	-0.000137329092\\
0.014336	0\\
0.021504	-0.000137329092\\
0.028672	-0.000320435136\\
0.03584	-0.000160217568\\
0.043008	-0.000137329092\\
0.050176	-0.000137329092\\
0.057344	-4.5776088e-05\\
0.064512	-9.1553004e-05\\
0.07168	0.000205992828\\
0.078848	-0.00036621108\\
0.086016	-0.00027465822\\
0.093184	-0.000183106044\\
0.100352	-0.000251770572\\
0.10752	-0.00041198724\\
0.114688	-0.00057220488\\
0.121856	-0.000228882132\\
0.129024	-0.00041198724\\
0.136192	-4.5776088e-05\\
0.14336	4.5776088e-05\\
0.150528	-0.00059509332\\
0.157696	-0.00054931644\\
0.164864	-0.0006637572\\
0.172032	-0.000228882132\\
0.1792	-0.000114440652\\
0.186368	0.00034332192\\
0.193536	-0.000320435136\\
0.200704	-0.00018310518\\
0.207872	-4.5776916e-05\\
0.21504	-0.000183106044\\
0.222208	-0.00057220488\\
0.229376	0.000251769744\\
0.236544	-0.00048065184\\
0.243712	6.8664564e-05\\
0.25088	-0.00052642872\\
0.258048	-0.000228882132\\
0.265216	-2.28884616e-05\\
0.272384	-0.00043487568\\
0.279552	-0.000160217568\\
0.28672	-0.0004577634\\
0.293888	0.00018310518\\
0.301056	-0.000160217568\\
0.308224	-0.00054931644\\
0.315392	-0.000160217568\\
0.32256	-0.000205993656\\
0.329728	-4.5776088e-05\\
0.336896	-0.000228882132\\
0.344064	-0.000114440652\\
0.351232	-0.000160217568\\
0.3584	-0.000160217568\\
0.365568	-0.00045776412\\
0.372736	-0.00036621108\\
0.379904	-0.00061798104\\
0.387072	-0.00070953408\\
0.39424	-0.00043487568\\
0.401408	-9.1553004e-05\\
0.408576	-0.00027465822\\
0.415744	0\\
0.422912	-0.000160217568\\
0.43008	-0.000251769744\\
0.437248	-0.00057220488\\
0.444416	-0.00041198724\\
0.451584	-0.000160217568\\
0.458752	-0.000183106044\\
0.46592	-6.8664564e-05\\
0.473088	4.5776088e-05\\
0.480256	-0.000251769744\\
0.487424	0.0003890988\\
0.494592	-0.00018310518\\
0.50176	-0.000320434308\\
0.508928	2.28876264e-05\\
0.516096	-0.00043487568\\
0.523264	-0.00041198724\\
0.530432	-9.1553004e-05\\
0.5376	0.00016021674\\
0.544768	-0.00018310518\\
0.551936	-0.00029754666\\
0.559104	-0.0006637572\\
0.566272	0\\
0.57344	0.000114440652\\
0.580608	-0.000160217568\\
0.587776	-0.000160217568\\
0.594944	-0.0006637572\\
0.602112	-0.000183106044\\
0.60928	-0.00057220488\\
0.616448	-0.000137329092\\
0.623616	-0.000526428\\
0.630784	0.00018310518\\
0.637952	-6.8664564e-05\\
0.64512	9.1552176e-05\\
0.652288	0\\
0.659456	2.28876264e-05\\
0.666624	-0.000251770572\\
0.673792	-0.000343322748\\
0.68096	-0.0006637572\\
0.688128	-0.000160217568\\
0.695296	-0.000228882132\\
0.702464	0\\
0.709632	-0.0004577634\\
0.7168	-0.00029754666\\
0.723968	0.000114440652\\
0.731136	-0.0004577634\\
0.738304	-0.000343322748\\
0.745472	-0.000137329092\\
0.75264	4.5776088e-05\\
0.759808	0.000205993656\\
0.766976	-0.000343322748\\
0.774144	-0.000343322748\\
0.781312	4.5776088e-05\\
0.78848	-0.00029754666\\
0.795648	-0.00027465822\\
0.802816	4.5776088e-05\\
0.809984	-0.00041198724\\
0.817152	-9.1553004e-05\\
0.82432	2.28876264e-05\\
0.831488	-0.000228882132\\
0.838656	-0.0003890988\\
0.845824	-2.28884616e-05\\
0.852992	-0.000228882132\\
0.86016	-4.5776916e-05\\
0.867328	-0.000160217568\\
0.874496	-0.0004577634\\
0.881664	-0.000137329092\\
0.888832	-0.00029754666\\
0.896	0.00016021674\\
0.903168	-0.000343322748\\
0.910336	-0.0005950926\\
0.917504	-0.00027465822\\
0.924672	-0.00057220488\\
0.93184	-0.000114440652\\
0.939008	-4.5776916e-05\\
0.946176	9.1552176e-05\\
0.953344	0\\
0.960512	6.8664564e-05\\
0.96768	6.86637e-05\\
0.974848	-0.000160217568\\
0.982016	-4.5776088e-05\\
0.989184	-0.00054931644\\
0.996352	-0.00011444148\\
1.00352	-0.00050354028\\
1.010688	-0.000526428\\
1.017856	0.000137329092\\
1.025024	-0.000205993656\\
1.032192	0\\
1.03936	-0.00029754666\\
1.046528	-0.0005950926\\
1.053696	-0.00061798104\\
1.060864	-0.000205993656\\
1.068032	-0.000160217568\\
1.0752	-0.00011444148\\
1.082368	-0.000297545832\\
1.089536	-0.000343322748\\
1.096704	-0.0003890988\\
1.103872	-0.0003890988\\
1.11104	-0.00068664564\\
1.118208	0.00018310518\\
1.125376	0\\
1.132544	6.8664564e-05\\
1.139712	6.86637e-05\\
1.14688	-0.00036621108\\
1.154048	-0.00054931644\\
1.161216	-0.000343322748\\
1.168384	-0.000137329092\\
1.175552	-0.000228882132\\
1.18272	-0.00054931644\\
1.189888	-0.00068664564\\
1.197056	-0.00011444148\\
1.204224	-6.8664564e-05\\
1.211392	-0.00036621108\\
1.21856	-0.000343322748\\
1.225728	-0.00011444148\\
1.232896	-0.00018310518\\
1.240064	-0.000320435136\\
1.247232	-0.00027465822\\
1.2544	-0.000320434308\\
1.261568	0\\
1.268736	-0.000251770572\\
1.275904	-0.00027465822\\
1.283072	-0.00061798104\\
1.29024	0.0138244608\\
1.297408	0.0141448968\\
1.304576	0.0188598636\\
1.311744	0.0189056376\\
1.318912	0.023666382\\
1.32608	0.01966095\\
1.333248	0.0197753904\\
1.340416	0.104026788\\
1.347584	0.102653496\\
1.354752	0.102745044\\
1.36192	0.103019724\\
1.369088	0.102127068\\
1.376256	0.101989728\\
1.383424	0.103042584\\
1.390592	0.101966868\\
1.39776	0.101989728\\
1.404928	0.102287304\\
1.412096	0.104209884\\
1.419264	0.103248576\\
1.426432	0.103820796\\
1.4336	0.103729248\\
1.440768	0.104026788\\
1.447936	0.1036377\\
1.455104	0.103294368\\
1.462272	0.104301468\\
1.46944	0.105857856\\
1.476608	0.108352656\\
1.483776	0.105789168\\
1.490944	0.106361388\\
1.498112	0.105880752\\
1.50528	0.10583496\\
1.512448	0.106155396\\
1.519616	0.106979364\\
1.526784	0.10949706\\
1.533952	0.111213684\\
1.54112	0.108489996\\
1.548288	0.109909044\\
1.555456	0.110389716\\
1.562624	0.112197888\\
1.569792	0.11592864\\
1.57696	0.112907412\\
1.584128	0.11357118\\
1.591296	0.1131363\\
1.598464	0.11286162\\
1.605632	0.112472532\\
1.6128	0.115219116\\
1.619968	0.115173324\\
1.627136	0.114784236\\
1.634304	0.116249076\\
1.641472	0.116134632\\
1.64864	0.116455068\\
1.655808	0.11606598\\
1.662976	0.116546616\\
1.670144	0.118583676\\
1.677312	0.119773872\\
1.68448	0.119178756\\
1.691648	0.120826728\\
1.698816	0.12119292\\
1.705984	0.123046884\\
1.713152	0.123847956\\
1.72032	0.123870852\\
1.727488	0.124900812\\
1.734656	0.126319896\\
1.741824	0.12718962\\
1.748992	0.128585808\\
1.75616	0.128036484\\
1.763328	0.127967832\\
1.770496	0.128585808\\
1.777664	0.145133964\\
1.784832	0.1537857\\
1.792	0.153579708\\
1.799168	0.153396612\\
1.806336	0.1537857\\
1.813504	0.153099072\\
1.820672	0.15319062\\
1.82784	0.153396612\\
1.835008	0.151496892\\
1.842176	0.153327924\\
1.849344	0.0260238636\\
1.856512	0.0298919664\\
1.86368	0.0287704476\\
1.870848	0.0258865344\\
1.878016	0.028930662\\
1.885184	0.036781308\\
1.892352	0.0302352876\\
1.89952	0.037033092\\
1.906688	0.036186228\\
1.913856	0.0296630856\\
1.921024	0.0331649784\\
1.928192	0.042068484\\
1.93536	0.047904984\\
1.942528	0.0323638884\\
1.949696	0.037078848\\
1.956864	0.076766976\\
1.964032	0.0332336412\\
1.9712	0.039550788\\
1.978368	0.043601976\\
1.985536	0.03609468\\
1.992704	0.03588867\\
1.999872	0.0334854108\\
2.00704	0.0326614356\\
2.014208	0.03872682\\
2.021376	0.0359573364\\
2.028544	0.03362274\\
2.035712	0.044036856\\
2.04288	0.0303955056\\
2.050048	0.0336456288\\
2.057216	0.0338745096\\
2.064384	0.0335540772\\
2.071552	0.043395984\\
2.07872	0.038795472\\
2.085888	0.03362274\\
2.093056	0.036483768\\
2.100224	0.030830382\\
2.107392	0.0330047604\\
2.11456	0.0351333612\\
2.121728	0.0334396368\\
2.128896	0.0325927728\\
2.136064	0.034103394\\
2.143232	0.0331420896\\
2.1504	0.03552246\\
2.157568	0.0335998512\\
2.164736	0.0337371804\\
2.171904	0.0333938592\\
2.179072	0.03325653\\
2.18624	0.0359802252\\
2.193408	0.033782958\\
2.200576	0.0332336412\\
2.207744	0.0330505344\\
2.214912	0.033576966\\
2.22208	0.03588867\\
2.229248	0.0330505344\\
2.236416	0.033576966\\
2.243584	0.0338745096\\
2.250752	0.041152968\\
2.25792	0.0339660648\\
2.265088	0.051269544\\
2.272256	0.044403084\\
2.279424	0.031402584\\
2.286592	0.038314836\\
2.29376	0.0336456288\\
2.300928	0.038658132\\
2.308096	0.03918456\\
2.315264	0.0352706904\\
2.322432	0.0335082996\\
2.3296	0.0336685176\\
2.336768	0.042137136\\
2.343936	0.0346755996\\
2.351104	0.042503364\\
2.358272	0.038818368\\
2.36544	0.0344238264\\
2.372608	0.0319290156\\
2.379776	0.037147536\\
2.386944	0.0300292956\\
2.394112	0.0344924892\\
2.40128	0.0334396368\\
2.408448	0.0336456288\\
2.415616	0.0326614356\\
2.422784	0.0337142952\\
2.429952	0.0308761596\\
2.43712	0.03289032\\
2.444288	0.039367656\\
2.451456	0.03325653\\
2.458624	0.0337142952\\
2.465792	0.03362274\\
2.47296	0.048568716\\
2.480128	0.0314712504\\
2.487296	0.044471736\\
2.494464	0.0350189208\\
2.501632	0.036506664\\
2.5088	0.0321578964\\
2.515968	0.0257949828\\
2.523136	0.0204162588\\
2.530304	0.0205306992\\
2.537472	0.020553588\\
2.54464	0.0205078104\\
2.551808	0.0208053576\\
2.558976	0.0205306992\\
2.566144	0.0203704812\\
2.573312	0.0203247072\\
2.58048	0.02039337\\
2.587648	0.0206222508\\
2.594816	0.020553588\\
2.601984	0.020874024\\
2.609152	0.0201187116\\
2.61632	0.020553588\\
2.623488	0.0205078104\\
2.630656	0.0204162588\\
2.637824	0.0205306992\\
2.644992	0.0205764768\\
2.65216	0.0207366948\\
2.659328	0.0204162588\\
2.666496	0.0201644892\\
2.673664	0.0207366948\\
2.680832	0.0208511352\\
2.688	0.0205764768\\
2.695168	0.0206909172\\
2.702336	0.02039337\\
2.709504	0.020874024\\
2.716672	0.0205306992\\
2.72384	0.0202789296\\
2.731008	0.0207824688\\
2.738176	0.0205306992\\
2.745344	0.0209426868\\
2.752512	0.0203704812\\
2.75968	0.0204620364\\
2.766848	0.020347596\\
2.774016	0.020874024\\
2.781184	0.0204620364\\
2.788352	0.0246505752\\
2.79552	0.02039337\\
2.802688	0.0204620364\\
2.809856	0.0210113532\\
2.817024	0.020187378\\
2.824192	0.0206680284\\
2.83136	0.0205993656\\
2.838528	0.020553588\\
2.845696	0.0203018184\\
2.852864	0.0205993656\\
2.860032	0.0206222544\\
2.8672	0.020713806\\
2.874368	0.0203475924\\
2.881536	0.0208053576\\
2.888704	0.0203247072\\
2.895872	0.0201644892\\
2.90304	0.0205764768\\
2.910208	0.0208282464\\
2.917376	0.0208053576\\
2.924544	0.0207595836\\
2.931712	0.0205078104\\
2.93888	0.0204162588\\
2.946048	0.020553588\\
2.953216	0.0204620364\\
2.960384	0.0206222508\\
2.967552	0.0205306992\\
2.97472	0.0205993656\\
2.981888	0.0204620364\\
2.989056	0.0207366948\\
2.996224	0.020347596\\
3.003392	0.0204849252\\
3.01056	0.0204162588\\
3.017728	0.0203475924\\
3.024896	0.0203018184\\
3.032064	0.0202560408\\
3.039232	0.0204620364\\
3.0464	0.020713806\\
3.053568	0.0204162588\\
3.060736	0.0203018184\\
3.067904	0.0208969128\\
3.075072	0.020553588\\
3.08224	0.0205306992\\
3.089408	0.0204849252\\
3.096576	0.02039337\\
3.103744	0.0204162588\\
3.110912	0.0201416004\\
3.11808	0.0206909172\\
3.125248	0.0205764768\\
3.132416	0.0203247072\\
3.139584	0.020347596\\
3.146752	0.0207595836\\
3.15392	0.0200500488\\
3.161088	0.020713806\\
3.168256	0.0200042712\\
3.175424	0.0203247072\\
3.182592	0.0205764768\\
3.18976	0.0202560408\\
3.196928	0.020713806\\
3.204096	0.0208053576\\
3.211264	0.020187378\\
3.218432	0.0204391476\\
3.2256	0.0205306992\\
3.232768	0.02039337\\
3.239936	0.0203247072\\
3.247104	0.0207366948\\
3.254272	0.020713806\\
3.26144	0.020553588\\
3.268608	0.0203475924\\
3.275776	0.0203247072\\
3.282944	0.020072934\\
3.290112	0.0204849216\\
3.29728	0.0205993656\\
3.304448	0.0206451396\\
3.311616	0.0208053576\\
3.318784	0.024513246\\
3.325952	0.0209655756\\
3.33312	0.0205764768\\
3.340288	0.0202102632\\
3.347456	0.0200729376\\
3.354624	0.0202102668\\
3.361792	0.0203247072\\
3.36896	0.0202102668\\
3.376128	0.0206451396\\
3.383296	0.02002716\\
3.390464	0.0204620364\\
3.397632	0.0208282464\\
3.4048	0.0204391476\\
3.411968	0.020713806\\
3.419136	0.020233152\\
3.426304	0.0207595836\\
3.433472	0.0203704812\\
3.44064	0.0203704812\\
3.447808	0.0206909172\\
3.454976	0.0207824688\\
3.462144	0.020347596\\
3.469312	0.0204849252\\
3.47648	0.0205993656\\
3.483648	0.020553588\\
3.490816	0.0208053576\\
3.497984	0.0204162588\\
3.505152	0.0206680284\\
3.51232	0.0204162588\\
3.519488	0.0199584936\\
3.526656	0.0200729376\\
3.533824	0.0206222544\\
3.540992	0.02039337\\
3.54816	0.0205078104\\
3.555328	0.0204391476\\
3.562496	0.0204849216\\
3.569664	0.0209655756\\
3.576832	0.0204391476\\
3.584	0.0202789296\\
3.591168	0.020233152\\
3.598336	0.0204620364\\
3.605504	0.020874024\\
3.612672	0.020713806\\
3.61984	0.020713806\\
3.627008	0.0206909172\\
3.634176	0.0202789296\\
3.641344	0.0205306992\\
3.648512	0.0203018184\\
3.65568	0.0205306992\\
3.662848	0.0204391476\\
3.670016	0.0208511352\\
3.677184	0.0204620364\\
3.684352	0.0202789296\\
3.69152	0.0205993656\\
3.698688	0.0203475924\\
3.705856	0.0201187116\\
3.713024	0.0200729376\\
3.720192	0.0206680284\\
3.72736	0.0205764768\\
3.734528	0.0207824688\\
3.741696	0.020874024\\
3.748864	0.0204849216\\
3.756032	0.0201644892\\
3.7632	0.0202102668\\
3.770368	0.0208969128\\
3.777536	0.0204391476\\
3.784704	0.0200958228\\
3.791872	0.0199356084\\
3.79904	0.0204849252\\
3.806208	0.0232086168\\
3.813376	0.024513246\\
3.820544	0.0223388676\\
3.827712	0.020233152\\
3.83488	0.020553588\\
3.842048	0.0206222544\\
3.849216	0.0205306992\\
3.856384	0.02002716\\
3.863552	0.0206222544\\
3.87072	0.0202560408\\
3.877888	0.0202789296\\
3.885056	0.0205078104\\
3.892224	0.020187378\\
3.899392	0.0203018184\\
3.90656	0.0206451396\\
3.913728	0.02039337\\
3.920896	0.0202560408\\
3.928064	0.020919798\\
3.935232	0.0206222544\\
3.9424	0.0206909172\\
3.949568	0.0204620364\\
3.956736	0.0204162588\\
3.963904	0.020233152\\
3.971072	0.020553588\\
3.97824	0.0202560408\\
3.985408	0.0204391476\\
3.992576	0.0203247072\\
3.999744	0.0204391476\\
4.006912	0.020553588\\
4.01408	0.0207366948\\
4.021248	0.0205306992\\
4.028416	0.0204391476\\
4.035584	0.0203704812\\
4.042752	0.0203018184\\
4.04992	0.0201416004\\
4.057088	0.020713806\\
4.064256	0.020187378\\
4.071424	0.0204849216\\
4.078592	0.020187378\\
4.08576	0.0202102668\\
4.092928	0.0205764768\\
4.100096	0.0205993656\\
4.107264	0.0205764768\\
4.114432	0.0208053576\\
4.1216	0.0201416004\\
4.128768	0.0203247072\\
4.135936	0.0208282464\\
4.143104	0.0205993656\\
4.150272	0.0210571272\\
4.15744	0.0202102668\\
4.164608	0.0205306992\\
4.171776	0.0206909172\\
4.178944	0.0211715676\\
4.186112	0.020553588\\
4.19328	0.0206680284\\
4.200448	0.020553588\\
4.207616	0.0206680284\\
4.214784	0.020187378\\
4.221952	0.0204620364\\
4.22912	0.0197753904\\
4.236288	0.0206222544\\
4.243456	0.0205993656\\
4.250624	0.02039337\\
4.257792	0.0205764768\\
4.26496	0.0208511352\\
4.272128	0.0206222544\\
4.279296	0.02002716\\
4.286464	0.0207595836\\
4.293632	0.0204162588\\
4.3008	0.0206451396\\
4.307968	0.0205993656\\
4.315136	0.028244016\\
4.322304	0.200042712\\
4.329472	0.22570038\\
4.33664	0.243667584\\
4.343808	0.21283722\\
4.350976	0.254035944\\
4.358144	0.233001684\\
4.365312	0.245521512\\
4.37248	0.22190094\\
4.379648	0.253761264\\
4.386816	0.24895476\\
4.393984	0.246894804\\
4.401152	0.0249481188\\
4.40832	0.020553588\\
4.415488	0.149894712\\
4.422656	0.19953918\\
4.429824	0.232017516\\
4.436992	0.152389512\\
4.44416	0.14955138\\
4.451328	0.201919536\\
4.458496	0.155158992\\
4.465664	0.155754072\\
4.472832	0.200958264\\
4.48	0.19983672\\
4.487168	0.0242843616\\
4.494336	0.153282168\\
4.501504	0.201553344\\
4.508672	0.152458164\\
4.51584	0.159553512\\
4.523008	0.201095568\\
4.530176	0.197937\\
4.537344	0.151908876\\
4.544512	0.160995492\\
4.55168	0.201416004\\
4.558848	0.150627132\\
4.566016	0.0281753532\\
4.573184	0.153671256\\
4.580352	0.201782232\\
4.58752	0.155342088\\
4.594688	0.15319062\\
4.601856	0.200386044\\
4.609024	0.201347352\\
4.616192	0.15541074\\
4.62336	0.156967164\\
4.630528	0.201507552\\
4.637696	0.151062012\\
4.644864	0.0321807852\\
4.652032	0.0286331184\\
4.6592	0.0251998884\\
4.666368	0.0207366948\\
4.673536	0.0247421268\\
4.680704	0.020233152\\
4.687872	0.02039337\\
4.69504	0.024513246\\
4.702208	0.0201416004\\
4.709376	0.219451896\\
4.716544	0.024765012\\
4.723712	0.0205764768\\
4.73088	0.202812192\\
4.738048	0.0247879044\\
4.745216	0.0208969128\\
4.752384	0.0208053576\\
4.759552	0.0202789296\\
4.76672	0.0206680284\\
4.773888	0.02039337\\
4.781056	0.0206222544\\
4.788224	0.0207366948\\
4.795392	0.0206222508\\
4.80256	0.149734476\\
4.809728	0.149322492\\
4.816896	0.14984892\\
4.824064	0.153831456\\
4.831232	0.2372589\\
4.8384	0.028083798\\
4.845568	0.0246963492\\
4.852736	0.02419281\\
4.859904	0.0205306992\\
4.867072	0.0204849252\\
4.87424	0.0248336784\\
4.881408	0.0208511352\\
4.888576	0.020713806\\
4.895744	0.0246963492\\
4.902912	0.0204620364\\
4.91008	0.208671552\\
4.917248	0.0247421268\\
4.924416	0.020233152\\
4.931584	0.0205993656\\
4.938752	0.0211715676\\
4.94592	0.0205764768\\
4.953088	0.0205306992\\
4.960256	0.0203018184\\
4.967424	0.150489792\\
4.974592	0.15232086\\
4.98176	0.150260904\\
4.988928	0.150398244\\
4.996096	0.150306696\\
5.003264	0.150100704\\
5.010432	0.1502838\\
5.0176	0.150329592\\
5.024768	0.221717844\\
5.031936	0.150970464\\
5.039104	0.150489792\\
5.046272	0.151268004\\
5.05344	0.150352488\\
5.060608	0.151359552\\
5.067776	0.14998626\\
5.074944	0.151657092\\
5.082112	0.149528484\\
5.08928	0.151794432\\
5.096448	0.149734476\\
5.103616	0.15188598\\
5.110784	0.0278091432\\
5.117952	0.0489807\\
5.12512	0.153259272\\
5.132288	0.249572736\\
5.139456	0.0277862544\\
5.146624	0.0290222172\\
5.153792	0.0274429296\\
5.16096	0.0280609128\\
5.168128	0.024513246\\
5.175296	0.0252914436\\
5.182464	0.0249481188\\
5.189632	0.0333709704\\
5.1968	0.0303268428\\
5.203968	0.0296630856\\
5.211136	0.274566636\\
5.218304	0.246734604\\
5.225472	0.155136096\\
5.23264	0.221008284\\
5.239808	0.024879456\\
5.246976	0.0249023412\\
5.254144	0.207778932\\
5.261312	0.050148\\
5.26848	0.214508052\\
5.275648	0.213203448\\
5.282816	0.0295257564\\
5.289984	0.44702892\\
5.297152	0.4543992\\
5.30432	0.44274888\\
5.311488	0.4561614\\
5.318656	0.44068896\\
5.325824	0.151176456\\
5.332992	0.201942432\\
5.34016	0.250602696\\
5.347328	0.0230255136\\
5.354496	0.0205306992\\
5.361664	0.0205764768\\
5.368832	0.0204391476\\
5.376	0.131629932\\
5.383168	0.224876376\\
5.390336	0.0251312256\\
5.397504	0.219383244\\
5.404672	0.0239181516\\
5.41184	0.030670164\\
5.419008	0.71953596\\
5.426176	0.72981252\\
5.433344	0.254631024\\
5.440512	0.15158844\\
5.44768	0.218284596\\
5.454848	0.21416472\\
5.462016	0.216911304\\
5.469184	0.198188784\\
5.476352	0.0205078104\\
5.48352	0.0205306992\\
5.490688	0.0206222508\\
5.497856	0.020553588\\
5.505024	0.020553588\\
5.512192	0.203590404\\
5.51936	0.0284271228\\
5.526528	0.71548452\\
5.533696	0.73143756\\
5.540864	0.7099686\\
5.548032	0.72892008\\
5.5552	0.70976268\\
5.562368	0.73123164\\
5.569536	0.71083836\\
5.576704	0.72988128\\
5.583872	0.71276076\\
5.59104	0.0250167816\\
5.598208	0.254241936\\
5.605376	0.0242843616\\
5.612544	0.0203704812\\
5.619712	0.0208969128\\
5.62688	0.0204391476\\
5.634048	0.209701548\\
5.641216	0.215057376\\
5.648384	0.0272827152\\
5.655552	0.72972108\\
5.66272	0.7143174\\
5.669888	0.72569268\\
5.677056	0.709236\\
5.684224	0.72921744\\
5.691392	0.71058636\\
5.69856	0.72985824\\
5.705728	0.70925904\\
5.712896	0.72990432\\
5.720064	0.204917904\\
5.727232	0.0245819088\\
5.7344	0.0204391476\\
5.741568	0.0202789296\\
5.748736	0.020553588\\
5.755904	0.0202789296\\
5.763072	0.020347596\\
5.77024	0.0204849252\\
5.777408	0.205810524\\
5.784576	0.197364816\\
5.791744	0.208534248\\
5.798912	0.213386544\\
5.80608	0.0312652548\\
5.813248	0.72809604\\
5.820416	0.72649368\\
5.827584	0.0285873408\\
5.834752	0.218719476\\
5.84192	0.153579708\\
5.849088	0.16181946\\
5.856256	0.213203448\\
5.863424	0.249435396\\
5.870592	0.157241808\\
5.87776	0.039253248\\
5.884928	0.0331192008\\
5.892096	0.216224676\\
5.899264	0.258567804\\
5.906432	0.230712876\\
5.9136	0.265663152\\
5.920768	0.172828692\\
5.927936	0.21196746\\
5.935104	0.268157952\\
5.942272	0.224601732\\
5.94944	0.168914808\\
5.956608	0.21986388\\
5.963776	0.238609296\\
5.970944	0.17191314\\
5.978112	0.209976192\\
5.98528	0.0334854108\\
5.992448	0.041770944\\
5.999616	0.043533324\\
6.006784	0.043373124\\
6.013952	0.0349731432\\
6.02112	0.20670318\\
6.028288	0.235450728\\
6.035456	0.161132796\\
6.042624	0.237464892\\
6.049792	0.223205544\\
6.05696	0.235656756\\
6.064128	0.208557108\\
6.071296	0.231445296\\
6.078464	0.227508552\\
6.085632	0.150581376\\
6.0928	0.20100402\\
6.099968	0.225082404\\
6.107136	0.197410572\\
6.114304	0.0249481188\\
6.121472	0.02039337\\
6.12864	0.0207595836\\
6.135808	0.0204391476\\
6.142976	0.0201644892\\
6.150144	0.151130664\\
6.157312	0.244239804\\
6.16448	0.149574276\\
6.171648	0.210227976\\
6.178816	0.215835552\\
6.185984	0.185485824\\
6.193152	0.253417968\\
6.20032	0.207206712\\
6.207488	0.22279356\\
6.214656	0.249435396\\
6.221824	0.224853516\\
6.228992	0.202995288\\
6.23616	0.202926636\\
6.243328	0.024513246\\
6.250496	0.0206222544\\
6.257664	0.0208511352\\
6.264832	0.0202560408\\
6.272	0.131446836\\
6.279168	0.22583772\\
6.286336	0.202239972\\
6.293504	0.221946696\\
6.300672	0.150260904\\
6.30784	0.21956634\\
6.315008	0.204849252\\
6.322176	0.22000122\\
6.329344	0.20803068\\
6.336512	0.214210512\\
6.34368	0.211257936\\
6.350848	0.1501236\\
6.358016	0.213912972\\
6.365184	0.0280151352\\
6.372352	0.0207366948\\
6.37952	0.0204162588\\
6.386688	0.0204162588\\
6.393856	0.0203018184\\
6.401024	0.0207366948\\
6.408192	0.0245361312\\
6.41536	0.223457328\\
6.422528	0.200798028\\
6.429696	0.147811896\\
6.436864	0.20070648\\
6.444032	0.225791928\\
6.4512	0.195350652\\
6.458368	0.248085036\\
6.465536	0.150833124\\
6.472704	0.223022448\\
6.479872	0.202857948\\
6.48704	0.15042114\\
6.494208	0.206085204\\
6.501376	0.217620828\\
6.508544	0.207344052\\
6.515712	0.21372984\\
6.52288	0.0246505752\\
6.530048	0.149574276\\
6.537216	0.216247536\\
6.544384	0.249755868\\
6.551552	0.217918404\\
6.55872	0.206886276\\
6.565888	0.151542648\\
6.573056	0.205169688\\
6.580224	0.225494388\\
6.587392	0.2039337\\
6.59456	0.22586058\\
6.601728	0.201782232\\
6.608896	0.254104596\\
6.616064	0.192695616\\
6.623232	0.0247879044\\
6.6304	0.0204162588\\
6.637568	0.0204391476\\
6.644736	0.0203247072\\
6.651904	0.0206222544\\
6.659072	0.201187116\\
6.66624	0.219406104\\
6.673408	0.245841984\\
6.680576	0.148338324\\
6.687744	0.207115164\\
6.694912	0.215515116\\
6.70208	0.205078104\\
6.709248	0.229248036\\
6.716416	0.21503448\\
6.723584	0.229980456\\
6.730752	0.151496892\\
6.73792	0.207778932\\
6.745088	0.151542648\\
6.752256	0.02455902\\
6.759424	0.0204162588\\
6.766592	0.02039337\\
6.77376	0.0207366948\\
6.780928	0.0209884644\\
6.788096	0.201210012\\
6.795264	0.215309124\\
6.802432	0.2419281\\
6.8096	0.151863084\\
6.816768	0.225173952\\
6.823936	0.201667788\\
6.831104	0.196723944\\
6.838272	0.207412704\\
6.84544	0.149963364\\
6.852608	0.204574572\\
6.859776	0.217323288\\
6.866944	0.206932068\\
6.874112	0.195991524\\
6.88128	0.0306472752\\
6.888448	0.282531708\\
6.895616	0.0248107896\\
6.902784	0.0204162588\\
6.909952	0.02039337\\
6.91712	0.19397736\\
6.924288	0.0274200444\\
6.931456	0.20553588\\
6.938624	0.212745672\\
6.945792	0.0310134888\\
6.95296	0.46089936\\
6.960128	0.4420854\\
6.967296	0.4585878\\
6.974464	0.43782804\\
6.981632	0.027717588\\
6.9888	0.200775132\\
6.995968	0.177406308\\
7.003136	0.256919832\\
7.010304	0.0212860116\\
7.017472	0.0211486824\\
7.02464	0.0204849252\\
7.031808	0.0206451396\\
7.038976	0.0203704812\\
7.046144	0.0206451396\\
7.053312	0.221054076\\
7.06048	0.149322492\\
7.067648	0.0280151352\\
7.074816	0.72404496\\
7.081984	0.72248832\\
7.089152	0.72246564\\
7.09632	0.71784216\\
7.103488	0.72447984\\
7.110656	0.71514144\\
7.117824	0.72576144\\
7.124992	0.71040348\\
7.13216	0.72846216\\
7.139328	0.206382744\\
7.146496	0.226203912\\
7.153664	0.201644892\\
7.160832	0.0249710076\\
7.168	0.132179256\\
7.175168	0.226203912\\
7.182336	0.0253601064\\
7.189504	0.1476288\\
7.196672	0.212470992\\
7.20384	0.0308990484\\
7.211008	0.72843948\\
7.218176	0.73599228\\
7.225344	0.0274429296\\
7.232512	0.242546076\\
7.23968	0.213455196\\
7.246848	0.218124396\\
7.254016	0.216407772\\
7.261184	0.199310292\\
7.268352	0.0247879044\\
7.27552	0.0277404768\\
7.282688	0.0309677112\\
7.289856	0.029617308\\
7.297024	0.230781564\\
7.304192	0.215538012\\
7.31136	0.22952268\\
7.318528	0.239524848\\
7.325696	0.150695784\\
7.332864	0.201782232\\
7.340032	0.226181016\\
7.3472	0.202766436\\
7.354368	0.225425736\\
7.361536	0.200958264\\
7.368704	0.223869312\\
7.375872	0.214416504\\
7.38304	0.22235868\\
7.390208	0.065071116\\
7.397376	0.0250625628\\
7.404544	0.0207824688\\
7.411712	0.0207824688\\
7.41888	0.0207595836\\
7.426048	0.13286592\\
7.433216	0.213340752\\
7.440384	0.190452564\\
7.447552	0.215423568\\
7.45472	0.208328256\\
7.461888	0.218559276\\
7.469056	0.2062683\\
7.476224	0.221992488\\
7.483392	0.20436858\\
7.49056	0.151039116\\
7.497728	0.202880844\\
7.504896	0.21503448\\
7.512064	0.201965328\\
7.519232	0.200180052\\
7.5264	0.0208053576\\
7.533568	0.0211029048\\
7.540736	0.0208053576\\
7.547904	0.0206909172\\
7.555072	0.200683584\\
7.56224	0.237762432\\
7.569408	0.16197966\\
7.576576	0.218719476\\
7.583744	0.205879212\\
7.590912	0.149803164\\
7.59808	0.254013048\\
7.605248	0.149894712\\
7.612416	0.212860116\\
7.619584	0.230575572\\
7.626752	0.215583804\\
7.63392	0.208785996\\
7.641088	0.195510852\\
7.648256	0.0242614728\\
7.655424	0.0206909172\\
7.662592	0.0206222544\\
7.66976	0.0206222544\\
7.676928	0.0208282464\\
7.684096	0.0203704812\\
7.691264	0.0206909172\\
7.698432	0.24179076\\
7.7056	0.149528484\\
7.712768	0.22586058\\
7.719936	0.200546244\\
7.727104	0.199928268\\
7.734272	0.206886276\\
7.74144	0.22263336\\
7.748608	0.203155524\\
7.755776	0.147880548\\
7.762944	0.151336656\\
7.770112	0.149917608\\
7.77728	0.2170944\\
7.784448	0.149528484\\
7.791616	0.211326588\\
7.798784	0.0280838016\\
7.805952	0.0205993656\\
7.81312	0.194457996\\
7.820288	0.219612096\\
7.827456	0.207344052\\
7.834624	0.221214276\\
7.841792	0.205284132\\
7.84896	0.223297128\\
7.856128	0.191390976\\
7.863296	0.229591368\\
7.870464	0.149871816\\
7.877632	0.247032144\\
7.8848	0.149894712\\
7.891968	0.226272564\\
7.899136	0.194503788\\
7.906304	0.0242843616\\
7.913472	0.0205078104\\
7.92064	0.0208282464\\
7.927808	0.0203018184\\
7.934976	0.0203018184\\
7.942144	0.20086668\\
7.949312	0.218353284\\
7.95648	0.207984924\\
7.963648	0.233184816\\
7.970816	0.15174864\\
7.977984	0.21256254\\
7.985152	0.15204618\\
7.99232	0.211280832\\
7.999488	0.21839904\\
8.006656	0.207458496\\
8.013824	0.220573404\\
8.020992	0.186195384\\
8.02816	0.25069428\\
8.035328	0.0242843616\\
8.042496	0.020874024\\
8.049664	0.0206451396\\
8.056832	0.0206451396\\
8.064	0.0204620364\\
8.071168	0.200546244\\
8.078336	0.200431836\\
8.085504	0.150306696\\
8.092672	0.205787664\\
8.09984	0.254470824\\
8.107008	0.202514652\\
8.114176	0.235267632\\
8.121344	0.206291196\\
8.128512	0.238151556\\
8.13568	0.148704516\\
8.142848	0.214988724\\
8.150016	0.151863084\\
8.157184	0.215263368\\
8.164352	0.0243759168\\
8.17152	0.0206222544\\
8.178688	0.0205764768\\
8.185856	0.0203018184\\
8.193024	0.22263336\\
8.200192	0.207481392\\
8.20736	0.224166852\\
8.214528	0.2395935\\
8.221696	0.224807724\\
8.228864	0.207092268\\
8.236032	0.193290696\\
8.2432	0.203018184\\
8.250368	0.150810228\\
8.257536	0.201095568\\
8.264704	0.147399912\\
8.271872	0.201164256\\
8.27904	0.224327088\\
8.286208	0.0246734604\\
8.293376	0.020553588\\
8.300544	0.0205078104\\
8.307712	0.0198898308\\
8.31488	0.0201416004\\
8.322048	0.030464172\\
8.329216	0.27321624\\
8.336384	0.0249710076\\
8.343552	0.232154856\\
8.35072	0.149711616\\
8.357888	0.216911304\\
8.365056	0.149894712\\
8.372224	0.22162626\\
8.379392	0.250900272\\
8.38656	0.223571772\\
8.393728	0.203521716\\
8.400896	0.214485156\\
8.408064	0.149734476\\
8.415232	0.230758632\\
8.4224	0.201805128\\
8.429568	0.188644392\\
8.436736	0.195259104\\
8.443904	0.0249252336\\
8.451072	0.1932678\\
8.45824	0.235908504\\
8.465408	0.154975896\\
8.472576	0.21986388\\
8.479744	0.0273742668\\
8.486912	0.205261236\\
8.49408	0.44709768\\
8.501248	0.45444492\\
8.508416	0.44561016\\
8.515584	0.45256788\\
8.522752	0.216842652\\
8.52992	0.149963364\\
8.537088	0.206062308\\
8.544256	0.0251083368\\
8.551424	0.0208511352\\
8.558592	0.0203247072\\
8.56576	0.0206909172\\
8.572928	0.0205306992\\
8.580096	0.203659056\\
8.587264	0.244514448\\
8.594432	0.0300979584\\
8.6016	0.7146378\\
8.608768	0.73457352\\
8.615936	0.70882416\\
8.623104	0.72889704\\
8.630272	0.70953372\\
8.63744	0.73075104\\
8.644608	0.70985412\\
8.651776	0.72990432\\
8.658944	0.7124634\\
8.666112	0.19324494\\
8.67328	0.024513246\\
8.680448	0.0209426868\\
8.687616	0.0202102668\\
8.694784	0.0207366948\\
8.701952	0.0208511352\\
8.70912	0.194915772\\
8.716288	0.2170944\\
8.723456	0.215698248\\
8.730624	0.150970464\\
8.737792	0.0276260364\\
8.74496	0.187293996\\
8.752128	0.7160112\\
8.759296	0.73287972\\
8.766464	0.209747304\\
8.773632	0.225379944\\
8.7808	0.149528484\\
8.787968	0.226615896\\
8.795136	0.193954464\\
8.802304	0.0248107896\\
8.809472	0.0201187116\\
8.81664	0.0205078104\\
8.823808	0.0202789296\\
8.830976	0.020187378\\
8.838144	0.198623664\\
8.845312	0.149757372\\
8.85248	0.205490124\\
8.859648	0.235427832\\
8.866816	0.20800782\\
8.873984	0.214096068\\
8.881152	0.203567508\\
8.88832	0.244926468\\
8.895488	0.149642928\\
8.902656	0.20919798\\
8.909824	0.151245108\\
8.916992	0.206909172\\
8.92416	0.198783864\\
8.931328	0.0209884644\\
8.938496	0.0201187116\\
8.945664	0.0206680284\\
8.952832	0.0202560408\\
8.96	0.0208053576\\
8.967168	0.0250625628\\
8.974336	0.204551712\\
8.981504	0.225105264\\
8.988672	0.200729376\\
8.99584	0.213935832\\
9.003008	0.201553344\\
9.010176	0.150398244\\
9.017344	0.203956596\\
9.024512	0.147743208\\
9.03168	0.205558776\\
9.038848	0.217712412\\
9.046016	0.151702884\\
9.053184	0.196632396\\
9.060352	0.212814324\\
9.06752	0.211578372\\
9.074688	0.214942932\\
9.081856	0.0248565672\\
9.089024	0.221397408\\
9.096192	0.194686884\\
9.10336	0.222175584\\
9.110528	0.150100704\\
9.117696	0.225952128\\
9.124864	0.150077808\\
9.132032	0.225631728\\
9.1392	0.20407104\\
9.146368	0.227371212\\
9.153536	0.20084382\\
9.160704	0.150146496\\
9.167872	0.200935368\\
9.17504	0.225196812\\
9.182208	0.0309448224\\
9.189376	0.0205078104\\
9.196544	0.0205078104\\
9.203712	0.0204849252\\
9.21088	0.02039337\\
9.218048	0.216522216\\
9.225216	0.151176456\\
9.232384	0.214714044\\
9.239552	0.252433764\\
9.24672	0.21166992\\
9.253888	0.214416504\\
9.261056	0.253761264\\
9.268224	0.219657888\\
9.275392	0.181800828\\
9.28256	0.221214276\\
9.289728	0.146575908\\
9.296896	0.22350312\\
9.304064	0.203384412\\
9.311232	0.0249023412\\
9.3184	0.020919798\\
9.325568	0.0205764768\\
9.332736	0.0205764768\\
9.339904	0.020347596\\
9.347072	0.204391476\\
9.35424	0.233001684\\
9.361408	0.20173644\\
9.368576	0.229270932\\
9.375744	0.150398244\\
9.382912	0.221534712\\
9.39008	0.150901776\\
9.397248	0.240509016\\
9.404416	0.160606368\\
9.411584	0.214828488\\
9.418752	0.210456864\\
9.42592	0.149780268\\
9.433088	0.206176752\\
9.440256	0.0272140488\\
9.447424	0.0205993656\\
9.454592	0.0206222544\\
9.46176	0.0204849252\\
9.468928	0.0205306992\\
9.476096	0.193748472\\
9.483264	0.209381112\\
9.490432	0.150535584\\
9.4976	0.20187378\\
9.504768	0.213890076\\
9.511936	0.20130156\\
9.519104	0.249778728\\
9.526272	0.2026062\\
9.53344	0.225036612\\
9.540608	0.152229312\\
9.547776	0.223411572\\
9.554944	0.202835088\\
9.562112	0.198326088\\
9.56928	0.020233152\\
9.576448	0.0208053576\\
9.583616	0.0206909172\\
9.590784	0.0205764768\\
9.597952	0.0208969128\\
9.60512	0.0294570936\\
9.612288	0.0208511352\\
9.619456	0.20874024\\
9.626624	0.21752928\\
9.633792	0.207344052\\
9.64096	0.16692354\\
9.648128	0.219245904\\
9.655296	0.227050776\\
9.662464	0.20363616\\
9.669632	0.206314092\\
9.6768	0.202331556\\
9.683968	0.151336656\\
9.691136	0.149963364\\
9.698304	0.244583136\\
9.705472	0.150444036\\
9.71264	0.225425736\\
9.719808	0.0246276828\\
9.726976	0.020874024\\
9.734144	0.197296128\\
9.741312	0.221855148\\
9.74848	0.204048144\\
9.755648	0.148635864\\
9.762816	0.250648488\\
9.769984	0.153831456\\
9.777152	0.207389808\\
9.78432	0.245269764\\
9.791488	0.214233408\\
9.798656	0.221122728\\
9.805824	0.152503956\\
9.812992	0.208419804\\
9.82016	0.251129124\\
9.827328	0.0271682712\\
9.834496	0.0233917236\\
9.841664	0.0232086168\\
9.848832	0.27804564\\
9.856	0.0251769996\\
9.863168	0.225173952\\
9.870336	0.149711616\\
9.877504	0.22613526\\
9.884672	0.152572644\\
9.89184	0.225425736\\
9.899008	0.0283126824\\
9.906176	0.4684068\\
9.913344	0.44332128\\
9.920512	0.45909108\\
9.92768	0.44048304\\
9.934848	0.45790092\\
9.942016	0.20990754\\
9.949184	0.198348984\\
9.956352	0.020233152\\
9.96352	0.0204849252\\
9.970688	0.0205078104\\
9.977856	0.0204162588\\
9.985024	0.0203247072\\
9.992192	0.2109375\\
9.99936	0.0274429296\\
10.006528	0.7204284\\
10.013696	0.7308198\\
10.020864	0.7136766\\
10.028032	0.72617328\\
10.0352	0.71067816\\
10.042368	0.72937764\\
10.049536	0.70960248\\
10.056704	0.73043064\\
10.063872	0.70912152\\
10.07104	0.025085448\\
10.078208	0.254035944\\
10.085376	0.0250396704\\
10.092544	0.0204620364\\
10.099712	0.0204162588\\
10.10688	0.0204391476\\
10.114048	0.219062808\\
10.121216	0.203361516\\
10.128384	0.243736236\\
10.135552	0.148269636\\
10.14272	0.0309677112\\
10.149888	0.72344988\\
10.157056	0.72251136\\
10.164224	0.72152712\\
10.171392	0.212882976\\
10.17856	0.218421936\\
10.185728	0.149871816\\
10.192896	0.253532412\\
10.200064	0.205009452\\
10.207232	0.0252227772\\
10.2144	0.020874024\\
10.221568	0.020874024\\
10.228736	0.0204849252\\
10.235904	0.0205078104\\
10.243072	0.0211029048\\
10.25024	0.0206222544\\
10.257408	0.200637828\\
10.264576	0.197868348\\
10.271744	0.205490124\\
10.278912	0.212516784\\
10.28608	0.0310821516\\
10.293248	0.73697652\\
10.300416	0.72063432\\
10.307584	0.0283126824\\
10.314752	0.211463928\\
10.32192	0.154289232\\
10.329088	0.1596222\\
10.336256	0.216155988\\
10.343424	0.248840316\\
10.350592	0.153877248\\
10.35776	0.0261154152\\
10.364928	0.0247421232\\
10.372096	0.209724408\\
10.379264	0.243873576\\
10.386432	0.22439574\\
10.3936	0.254013048\\
10.400768	0.150489792\\
10.407936	0.202102668\\
10.415104	0.251403804\\
10.422272	0.211143492\\
10.42944	0.152412408\\
10.436608	0.20173644\\
10.443776	0.224304192\\
10.450944	0.151817328\\
10.458112	0.19866942\\
10.46528	0.024398802\\
10.472448	0.0205993656\\
10.479616	0.0206680284\\
10.486784	0.02039337\\
10.493952	0.0204849252\\
10.50112	0.192581172\\
10.508288	0.212745672\\
10.515456	0.149230944\\
10.522624	0.214279164\\
10.529792	0.214508052\\
10.53696	0.21752928\\
10.544128	0.194778432\\
10.551296	0.22263336\\
10.558464	0.239227308\\
10.565632	0.151130664\\
10.5728	0.203727708\\
10.579968	0.225334152\\
10.587136	0.193153392\\
10.594304	0.024398802\\
10.601472	0.0205078104\\
10.60864	0.02039337\\
10.615808	0.020347596\\
10.622976	0.0205306992\\
10.630144	0.150215148\\
10.637312	0.244445796\\
10.64448	0.149093604\\
10.651648	0.220092768\\
10.658816	0.204917904\\
10.665984	0.182235708\\
10.673152	0.252410904\\
10.68032	0.216590868\\
10.687488	0.2110977\\
10.694656	0.250534044\\
10.701824	0.2144394\\
10.708992	0.21020508\\
10.71616	0.205101\\
10.723328	0.0248565672\\
10.730496	0.0206680284\\
10.737664	0.020553588\\
10.744832	0.0206222544\\
10.752	0.131332392\\
10.759168	0.224853516\\
10.766336	0.202926636\\
10.773504	0.226066572\\
10.780672	0.149002092\\
10.78784	0.226478556\\
10.795008	0.201095568\\
10.802176	0.226478556\\
10.809344	0.201072672\\
10.816512	0.2229309\\
10.82368	0.202812192\\
10.830848	0.149803164\\
10.838016	0.205032348\\
10.845184	0.028198242\\
10.852352	0.020233152\\
10.85952	0.0202789296\\
10.866688	0.0209884644\\
10.873856	0.020553588\\
10.881024	0.0205078104\\
10.888192	0.024719238\\
10.89536	0.21723174\\
10.902528	0.206520084\\
10.909696	0.148544316\\
10.916864	0.205444332\\
10.924032	0.222152688\\
10.9312	0.194457996\\
10.938368	0.247604364\\
10.945536	0.14925384\\
10.952704	0.226364112\\
10.959872	0.201599136\\
10.96704	0.150810228\\
10.974208	0.202629096\\
10.981376	0.225906372\\
10.988544	0.201370248\\
10.995712	0.213706944\\
11.00288	0.0246734604\\
11.010048	0.150077808\\
11.017216	0.205467228\\
11.024384	0.248931864\\
11.031552	0.20684052\\
11.03872	0.216430668\\
11.045888	0.151840188\\
11.053056	0.213615432\\
11.060224	0.214942932\\
11.067392	0.211830156\\
11.07456	0.216247536\\
11.081728	0.20800782\\
11.088896	0.253303524\\
11.096064	0.192787164\\
11.103232	0.024879456\\
11.1104	0.0207595836\\
11.117568	0.0205306992\\
11.124736	0.020713806\\
11.131904	0.0205306992\\
11.139072	0.200889576\\
11.14624	0.229476924\\
11.153408	0.249046308\\
11.160576	0.148338324\\
11.167744	0.201393144\\
11.174912	0.224075304\\
11.18208	0.20157624\\
11.189248	0.23519898\\
11.196416	0.205078104\\
11.203584	0.230163588\\
11.210752	0.152069076\\
11.21792	0.216384876\\
11.225088	0.152252172\\
11.232256	0.0249023412\\
11.239424	0.0206680284\\
11.246592	0.0201416004\\
11.25376	0.0207366948\\
11.260928	0.020553588\\
11.268096	0.205352784\\
11.275264	0.226547244\\
11.282432	0.242546076\\
11.2896	0.153442368\\
11.296768	0.224647524\\
11.303936	0.202651956\\
11.311104	0.199905372\\
11.318272	0.208053576\\
11.32544	0.151382448\\
11.332608	0.201347352\\
11.339776	0.22439574\\
11.346944	0.20130156\\
11.354112	0.199996956\\
11.36128	0.030303954\\
11.368448	0.281616192\\
11.375616	0.0249481188\\
11.382784	0.020713806\\
11.389952	0.0211944564\\
11.39712	0.195945732\\
11.404288	0.0276031512\\
11.411456	0.213546744\\
11.418624	0.212356548\\
11.425792	0.0308074932\\
11.43296	0.45433044\\
11.440128	0.44945532\\
11.447296	0.45309456\\
11.454464	0.44414532\\
11.461632	0.0273971556\\
11.4688	0.20434572\\
11.475968	0.170562744\\
11.483136	0.257057172\\
11.490304	0.0215835552\\
11.497472	0.0207595836\\
11.50464	0.0205306992\\
11.511808	0.0204391476\\
11.518976	0.020347596\\
11.526144	0.02039337\\
11.533312	0.229934664\\
11.54048	0.148841856\\
11.547648	0.0283355712\\
11.554816	0.71603388\\
11.561984	0.73118592\\
11.569152	0.71312724\\
11.57632	0.72665388\\
11.583488	0.7154388\\
11.590656	0.7244568\\
11.597824	0.71839152\\
11.604992	0.7189866\\
11.61216	0.72081756\\
11.619328	0.213317856\\
11.626496	0.219886776\\
11.633664	0.206085204\\
11.640832	0.02455902\\
11.648	0.131904612\\
11.655168	0.223113996\\
11.662336	0.0251083368\\
11.669504	0.148933404\\
11.676672	0.21283722\\
11.68384	0.0325241064\\
11.691008	0.72496044\\
11.698176	0.73020168\\
11.705344	0.024513246\\
11.712512	0.236984256\\
11.71968	0.201782232\\
11.726848	0.22263336\\
11.734016	0.202995288\\
11.741184	0.198280332\\
11.748352	0.0206451396\\
11.75552	0.0207366948\\
11.762688	0.020553588\\
11.769856	0.0208282464\\
11.777024	0.214393608\\
11.784192	0.21416472\\
11.79136	0.216476424\\
11.798528	0.239021316\\
11.805696	0.151359552\\
11.812864	0.20670318\\
11.820032	0.220573404\\
11.8272	0.20713806\\
11.834368	0.22396086\\
11.841536	0.203430168\\
11.848704	0.225288396\\
11.855872	0.214347852\\
11.86304	0.22586058\\
11.870208	0.030464172\\
11.877376	0.0202102668\\
11.884544	0.020874024\\
11.891712	0.0204620364\\
11.89888	0.0207595836\\
11.906048	0.131607036\\
11.913216	0.203659056\\
11.920384	0.210159288\\
11.927552	0.204528816\\
11.93472	0.218353284\\
11.941888	0.20700072\\
11.949056	0.21546936\\
11.956224	0.209060676\\
11.963392	0.212722776\\
11.97056	0.151359552\\
11.977728	0.209976192\\
11.984896	0.215057376\\
11.992064	0.207664488\\
11.999232	0.200683584\\
12.0064	0.0208282464\\
12.013568	0.0202102668\\
12.020736	0.0206680284\\
12.027904	0.0206680284\\
12.035072	0.201324456\\
12.04224	0.231216408\\
12.049408	0.168640128\\
12.056576	0.225883476\\
12.063744	0.200592036\\
12.070912	0.15042114\\
12.07808	0.252365112\\
12.085248	0.150398244\\
12.092416	0.203315724\\
12.099584	0.23023224\\
12.106752	0.204826356\\
12.11392	0.217918404\\
12.121088	0.194778432\\
12.128256	0.024032592\\
12.135424	0.0202789296\\
12.142592	0.0207366948\\
12.14976	0.0248336784\\
12.156928	0.02039337\\
12.164096	0.025405884\\
12.171264	0.0205078104\\
12.178432	0.242912268\\
12.1856	0.148933404\\
12.192768	0.222679152\\
12.199936	0.203956596\\
12.207104	0.204711912\\
12.214272	0.209266668\\
12.22144	0.226341252\\
12.228608	0.201484656\\
12.235776	0.14868162\\
12.242944	0.150192252\\
12.250112	0.150787368\\
12.25728	0.213340752\\
12.264448	0.1514511\\
12.271616	0.20274354\\
12.278784	0.028564452\\
12.285952	0.0207824688\\
12.29312	0.196632396\\
12.300288	0.209106432\\
12.307456	0.215423568\\
12.314624	0.209884644\\
12.321792	0.213111864\\
12.32896	0.212745672\\
12.336128	0.191528316\\
12.343296	0.223136892\\
12.350464	0.149620068\\
12.357632	0.243804924\\
12.3648	0.154541016\\
12.371968	0.222747804\\
12.379136	0.19367982\\
12.386304	0.024719238\\
12.393472	0.0204849252\\
12.40064	0.0204391476\\
12.407808	0.025245666\\
12.414976	0.0244674684\\
12.422144	0.197525016\\
12.429312	0.23155974\\
12.43648	0.209243772\\
12.443648	0.241882308\\
12.450816	0.15071868\\
12.457984	0.222610464\\
12.465152	0.150833124\\
12.47232	0.220916736\\
12.479488	0.206588736\\
12.486656	0.215812692\\
12.493824	0.209312424\\
12.500992	0.183952332\\
12.50816	0.252639756\\
12.515328	0.0246047976\\
12.522496	0.0202102668\\
12.529664	0.020187378\\
12.536832	0.0205306992\\
12.544	0.0203704812\\
12.551168	0.203361516\\
12.558336	0.202651956\\
12.565504	0.151336656\\
12.572672	0.208785996\\
12.57984	0.253807056\\
12.587008	0.201690648\\
12.594176	0.245338452\\
12.601344	0.201347352\\
12.608512	0.236297592\\
12.61568	0.14765166\\
12.622848	0.224990856\\
12.630016	0.150741576\\
12.637184	0.225036612\\
12.644352	0.0248336784\\
12.65152	0.020187378\\
12.658688	0.0205993656\\
12.665856	0.0206222544\\
12.673024	0.211944564\\
12.680192	0.216728208\\
12.68736	0.213363648\\
12.694528	0.239044176\\
12.701696	0.216155988\\
12.708864	0.208557108\\
12.716032	0.19354248\\
12.7232	0.208534248\\
12.730368	0.15172578\\
12.737536	0.204528816\\
12.744704	0.14852142\\
12.751872	0.20246886\\
12.75904	0.226272564\\
12.766208	0.02455902\\
12.773376	0.020713806\\
12.780544	0.0202102668\\
12.787712	0.0203704812\\
12.79488	0.0207595836\\
12.802048	0.0206222508\\
12.809216	0.0246505752\\
12.816384	0.22119138\\
12.823552	0.236412036\\
12.83072	0.150146496\\
12.837888	0.205719012\\
12.845056	0.150169356\\
12.852224	0.209976192\\
12.859392	0.249755868\\
12.86656	0.212928768\\
12.873728	0.211738572\\
12.880896	0.214485156\\
12.888064	0.149620068\\
12.895232	0.220756536\\
12.9024	0.207092268\\
12.909568	0.191047644\\
12.916736	0.172851552\\
12.923904	0.22835538\\
12.931072	0.192581172\\
12.93824	0.0310821516\\
12.945408	0.71392824\\
12.952576	0.73493964\\
12.959744	0.70852644\\
12.966912	0.225929268\\
12.97408	0.201232908\\
12.981248	0.22483062\\
12.988416	0.20203398\\
12.995584	0.148338324\\
13.002752	0.202583304\\
13.00992	0.150077808\\
13.017088	0.203704812\\
13.024256	0.0243759168\\
13.031424	0.0207366948\\
13.038592	0.0210571272\\
13.04576	0.02039337\\
13.052928	0.0204391476\\
13.060096	0.20553588\\
13.067264	0.24339294\\
13.074432	0.218421936\\
13.0816	0.206520084\\
13.088768	0.221397408\\
13.095936	0.149665824\\
13.103104	0.205306992\\
13.110272	0.150054912\\
13.11744	0.225769032\\
13.124608	0.152664192\\
13.131776	0.226478556\\
13.138944	0.201553344\\
13.146112	0.19400022\\
13.15328	0.0247421268\\
13.160448	0.02039337\\
13.167616	0.019866942\\
13.174784	0.0204849252\\
13.181952	0.0206680284\\
13.18912	0.196746804\\
13.196288	0.206291196\\
13.203456	0.216728208\\
13.210624	0.151817328\\
13.217792	0.214782732\\
13.22496	0.15158844\\
13.232128	0.195190416\\
13.239296	0.219932532\\
13.246464	0.209060676\\
13.253632	0.218765268\\
13.2608	0.150169356\\
13.267968	0.221717844\\
13.275136	0.193794228\\
13.282304	0.0250167852\\
13.289472	0.0204391476\\
13.29664	0.0205764768\\
13.303808	0.0205993656\\
13.310976	0.0205078104\\
13.318144	0.192718512\\
13.325312	0.15071868\\
13.33248	0.200935368\\
13.339648	0.24279786\\
13.346816	0.201370248\\
13.353984	0.223709112\\
13.361152	0.197662356\\
13.36832	0.245819088\\
13.375488	0.149368284\\
13.382656	0.217918404\\
13.389824	0.151817328\\
13.396992	0.215446464\\
13.40416	0.198577872\\
13.411328	0.0206451396\\
13.418496	0.0207824688\\
13.425664	0.020553588\\
13.432832	0.020874024\\
13.44	0.020713806\\
13.447168	0.0246276828\\
13.454336	0.208694448\\
13.461504	0.22805784\\
13.468672	0.203018184\\
13.47584	0.21533202\\
13.483008	0.202102668\\
13.490176	0.150650028\\
13.497344	0.201965328\\
13.504512	0.148132332\\
13.51168	0.201370248\\
13.518848	0.224922168\\
13.526016	0.150924672\\
13.533184	0.199150092\\
13.540352	0.203727708\\
13.54752	0.2204361\\
13.554688	0.204895008\\
13.561856	0.0278091432\\
13.569024	0.272941596\\
13.576192	0.0243759168\\
13.58336	0.211212144\\
13.590528	0.14982606\\
13.597696	0.216842652\\
13.604864	0.14968872\\
13.612032	0.216590868\\
13.6192	0.210708612\\
13.626368	0.22206114\\
13.633536	0.205101\\
13.640704	0.15071868\\
13.647872	0.20391084\\
13.65504	0.22423554\\
13.662208	0.027877806\\
13.669376	0.0207366948\\
13.676544	0.0205764768\\
13.683712	0.0204162588\\
13.69088	0.0208282464\\
13.698048	0.22469328\\
13.705216	0.149917608\\
13.712384	0.242912268\\
13.719552	0.201164256\\
13.72672	0.0281295756\\
13.733888	0.44906616\\
13.741056	0.45787824\\
13.748224	0.44785296\\
13.755392	0.4589766\\
13.76256	0.4453812\\
13.769728	0.0282211308\\
13.776896	0.0245361312\\
13.784064	0.0243759168\\
13.791232	0.0246734604\\
13.7984	0.020713806\\
13.805568	0.0211257936\\
13.812736	0.0206451396\\
13.819904	0.0203018184\\
13.827072	0.0205306992\\
13.83424	0.0207366948\\
13.841408	0.178024284\\
13.848576	0.150833124\\
13.855744	0.149757372\\
13.862912	0.15087888\\
13.87008	0.19324494\\
13.877248	0.0170059176\\
13.884416	0.0178527816\\
13.891584	0.017074584\\
13.898752	0.0136642464\\
13.90592	0.0136184688\\
13.913088	-0.000251770572\\
13.920256	-0.000160217568\\
13.927424	-0.00061798104\\
13.934592	-4.5776088e-05\\
13.94176	-0.00075530988\\
13.948928	-0.00018310518\\
13.956096	-0.000114440652\\
13.963264	-0.00054931644\\
13.970432	0.00036621036\\
13.9776	-0.00041198724\\
13.984768	9.1552176e-05\\
13.991936	-6.8664564e-05\\
13.999104	-0.000320435136\\
14.006272	-0.000137329092\\
14.01344	-0.000137329092\\
14.020608	2.28876264e-05\\
14.027776	-2.28884616e-05\\
14.034944	-0.000137329092\\
14.042112	-0.000137329092\\
14.04928	0.000205992828\\
14.056448	-0.00041198724\\
14.063616	6.8664564e-05\\
14.070784	2.28876264e-05\\
14.077952	-0.00029754666\\
14.08512	-0.00029754666\\
14.092288	-9.1553004e-05\\
14.099456	-2.28884616e-05\\
14.106624	-0.000343322748\\
14.113792	-0.000137329092\\
14.12096	-6.8664564e-05\\
14.128128	-0.00027465822\\
14.135296	6.8664564e-05\\
14.142464	-0.00029754666\\
14.149632	-0.000251769744\\
14.1568	-0.000137329092\\
14.163968	-0.00036621108\\
14.171136	-0.000343322748\\
14.178304	-8.3819016e-10\\
14.185472	-0.0003890988\\
14.19264	-2.28884616e-05\\
14.199808	-0.0006637572\\
14.206976	-0.00070953408\\
14.214144	-0.00011444148\\
14.221312	-0.000251770572\\
14.22848	-0.000320434308\\
14.235648	-9.1553004e-05\\
14.242816	-9.1553004e-05\\
14.249984	-2.28884616e-05\\
14.257152	-0.00077819868\\
14.26432	-0.00057220488\\
14.271488	-0.000160217568\\
14.278656	-0.000205993656\\
14.285824	0.000114440652\\
14.292992	-0.000205993656\\
14.30016	-0.00043487496\\
14.307328	-0.00011444148\\
14.314496	0.000320434308\\
14.321664	-0.00041198724\\
14.328832	0.00036621036\\
14.336	2.28876264e-05\\
14.343168	6.86637e-05\\
14.350336	-0.00054931644\\
14.357504	-9.1553004e-05\\
14.364672	-0.000320435136\\
14.37184	-0.00050354028\\
14.379008	-0.000526428\\
14.386176	-6.8664564e-05\\
14.393344	-0.000228882132\\
14.400512	6.86637e-05\\
14.40768	-0.000137329092\\
14.414848	-0.000205993656\\
14.422016	-0.00043487568\\
14.429184	-0.0003890988\\
14.436352	0.000114440652\\
14.44352	-0.000320434308\\
14.450688	-0.0003890988\\
14.457856	-0.00050353956\\
14.465024	-0.00048065184\\
14.472192	-2.28884616e-05\\
14.47936	-0.000251769744\\
14.486528	-0.00080108604\\
14.493696	-0.00038909952\\
14.500864	-0.000228882132\\
14.508032	-0.00070953408\\
14.5152	-0.000137329092\\
14.522368	-6.8664564e-05\\
14.529536	-0.000137329092\\
14.536704	-0.000205993656\\
14.543872	-0.000228882132\\
14.55104	-0.000160217568\\
14.558208	-0.00073242144\\
14.565376	0.000320434308\\
14.572544	-4.5776916e-05\\
14.579712	0.00016021674\\
14.58688	-6.8664564e-05\\
14.594048	0\\
14.601216	-0.00048065184\\
14.608384	-9.1553004e-05\\
14.615552	-0.000228882132\\
14.62272	-0.00018310518\\
14.629888	-0.000137329092\\
14.637056	6.86637e-05\\
14.644224	9.1552176e-05\\
14.651392	-6.8665392e-05\\
14.65856	-6.8664564e-05\\
14.665728	-0.000160217568\\
14.672896	-0.00107574444\\
14.680064	-0.00043487568\\
14.687232	-0.000343322748\\
14.6944	-0.00029754666\\
14.701568	-4.5776916e-05\\
14.708736	-4.5776916e-05\\
14.715904	-6.8664564e-05\\
14.723072	-0.00029754666\\
14.73024	-0.0004577634\\
14.737408	-0.00018310518\\
14.744576	-2.28884616e-05\\
14.751744	-0.000343322748\\
14.758912	-0.000251769744\\
14.76608	-9.1553004e-05\\
14.773248	0.000114440652\\
14.780416	-0.00029754666\\
14.787584	-0.000228882132\\
14.794752	-0.000343322748\\
14.80192	-0.00054931644\\
14.809088	0\\
14.816256	-0.00068664564\\
14.823424	-0.000160217568\\
14.830592	-0.00036621036\\
14.83776	-0.000343322748\\
14.844928	-0.000343322748\\
14.852096	-0.00054931644\\
14.859264	-0.000160217568\\
14.866432	-0.000114440652\\
14.8736	-2.28884616e-05\\
14.880768	0\\
14.887936	-0.00064086948\\
14.895104	9.1552176e-05\\
14.902272	-0.00029754666\\
14.90944	-0.000160217568\\
14.916608	6.86637e-05\\
14.923776	-0.00018310518\\
14.930944	-0.000320434308\\
14.938112	-4.5776088e-05\\
14.94528	2.28876264e-05\\
14.952448	-0.00043487568\\
14.959616	0.000114439788\\
14.966784	-0.000251770572\\
14.973952	-0.00011444148\\
14.98112	9.1552176e-05\\
14.988288	-0.00027465822\\
14.995456	0.00048065112\\
14.995456	0.00048065112\\
15.149056	-0.000228882132\\
15.302656	-0.000251769744\\
15.456256	-0.00027465822\\
15.609856	2.28876264e-05\\
15.763456	-0.000205993656\\
15.917056	4.5776088e-05\\
16.070656	0.000274657356\\
16.224256	0.000114440652\\
16.377856	0.000114440652\\
16.531456	-0.00036621108\\
16.685056	-0.000320434308\\
16.838656	-0.00082397448\\
16.992256	-0.000160217568\\
17.145856	-0.00041198724\\
17.299456	-0.00036621108\\
17.453056	-0.000114440652\\
17.606656	6.86637e-05\\
17.760256	-0.0004577634\\
17.913856	0.000320434308\\
18.067456	-0.000320434308\\
18.221056	-0.00061798104\\
18.374656	-0.000160217568\\
18.528256	-0.000251770572\\
18.681856	-0.00041198724\\
18.835456	0.000114440652\\
18.989056	-0.00011444148\\
19.142656	-0.0005950926\\
19.296256	-0.00036621108\\
19.449856	-0.00027465822\\
19.603456	-0.000343322748\\
19.757056	0\\
19.910656	-0.0004577634\\
20.064256	0.00016021674\\
20.217856	-0.000183106044\\
20.371456	2.28876264e-05\\
20.525056	-0.000251769744\\
20.678656	-9.1553004e-05\\
20.832256	0.00016021674\\
20.985856	-9.1553004e-05\\
21.139456	-0.000297545832\\
21.293056	0.00016021674\\
21.446656	-0.000205993656\\
21.600256	-0.000137329092\\
21.753856	-9.1553004e-05\\
21.907456	-0.00029754666\\
22.061056	-0.000160217568\\
22.214656	-0.00041198724\\
22.368256	-0.00029754666\\
22.521856	9.1552176e-05\\
22.675456	-0.00029754666\\
22.829056	-0.00050354028\\
22.982656	-0.000228882132\\
23.136256	-9.1553004e-05\\
23.289856	-4.5776916e-05\\
23.443456	-0.000343322748\\
23.597056	-0.000320434308\\
23.750656	-4.5776916e-05\\
23.904256	-0.00057220488\\
24.057856	-0.0005950926\\
24.211456	-0.000228882132\\
24.365056	-0.00050354028\\
24.518656	-0.00048065184\\
24.672256	-0.0004577634\\
24.825856	-0.000160217568\\
24.979456	-0.000228882132\\
25.133056	-0.00057220488\\
25.286656	-6.8664564e-05\\
25.440256	-0.0004577634\\
25.593856	-0.00029754666\\
25.747456	-0.000228882132\\
25.901056	-0.000228882132\\
26.054656	-0.00048065184\\
26.208256	4.5776088e-05\\
26.361856	-0.00018310518\\
26.515456	0.00027465822\\
26.669056	-9.1553004e-05\\
26.822656	0.000114440652\\
26.976256	-0.000205993656\\
27.129856	-0.00043487568\\
27.283456	-0.000137329092\\
27.437056	-0.00061798104\\
27.590656	-0.000137329092\\
27.744256	-0.00061798104\\
27.897856	2.28876264e-05\\
28.051456	-0.000343322748\\
28.205056	-0.000205993656\\
28.358656	-6.8664564e-05\\
28.512256	-9.1553004e-05\\
28.665856	0.000251769744\\
28.819456	9.1552176e-05\\
28.973056	-0.000228882132\\
29.126656	-0.000137329092\\
29.280256	-0.00027465822\\
29.433856	-9.1553004e-05\\
29.587456	-0.000228882132\\
29.741056	2.28876264e-05\\
29.894656	-0.00043487568\\
30.048256	-0.00029754666\\
30.201856	-0.000183106044\\
30.355456	-0.00018310518\\
30.509056	0\\
30.662656	-0.00036621108\\
30.816256	-9.1553004e-05\\
30.969856	-0.000183106044\\
31.123456	-0.000137329092\\
31.277056	-6.8664564e-05\\
31.430656	-0.00041198724\\
31.584256	0.000114440652\\
31.737856	-6.8664564e-05\\
31.891456	-0.000137329092\\
32.045056	-0.0005950926\\
32.198656	-0.000205993656\\
32.352256	-0.000205993656\\
32.505856	-6.8664564e-05\\
32.659456	-0.00011444148\\
32.813056	0.00016021674\\
32.966656	-0.00057220488\\
33.120256	2.28884616e-05\\
33.273856	-0.00054931644\\
33.427456	-0.00011444148\\
33.581056	-9.1553004e-05\\
33.734656	-0.00057220488\\
33.888256	-0.000343322748\\
34.041856	9.1552176e-05\\
34.195456	-0.00029754666\\
34.349056	-0.00061798104\\
34.502656	-0.00064086948\\
34.656256	-0.000160217568\\
34.809856	-0.00057220488\\
34.963456	-0.00029754666\\
35.117056	-0.000343322748\\
35.270656	9.1552176e-05\\
35.424256	-0.000251769744\\
35.577856	-0.00029754666\\
35.731456	-0.00011444148\\
35.885056	9.1552176e-05\\
36.038656	-0.00043487568\\
36.192256	-0.000320435136\\
36.345856	-0.0003890988\\
36.499456	-9.1553004e-05\\
36.653056	-0.00043487568\\
36.806656	-0.000251769744\\
36.960256	-0.0003890988\\
37.113856	-0.00100707984\\
37.267456	-0.00029754666\\
37.421056	0.00043487496\\
37.574656	-9.1553004e-05\\
37.728256	-0.00064086948\\
37.881856	-0.00011444148\\
38.035456	-0.00018310518\\
38.189056	-0.00068664564\\
38.342656	-0.00041198724\\
38.496256	-0.000526428\\
38.649856	-0.000137329092\\
38.803456	4.5776088e-05\\
38.957056	-0.00029754666\\
39.110656	-0.00029754666\\
39.264256	-0.000205993656\\
39.417856	-0.000205993656\\
39.571456	-9.1553004e-05\\
39.725056	-6.8664564e-05\\
39.878656	-0.000205993656\\
40.032256	2.28876264e-05\\
40.185856	-0.000205993656\\
40.339456	0.000228881268\\
40.493056	-0.0005950926\\
40.646656	-0.00070953408\\
40.800256	-0.00054931644\\
40.953856	-0.00054931644\\
41.107456	2.28876264e-05\\
41.261056	-0.000320434308\\
41.414656	-0.0004577634\\
41.568256	0.000297545832\\
41.721856	-0.000251769744\\
41.875456	-0.000251769744\\
42.029056	-0.0003890988\\
42.182656	-0.00041198724\\
42.336256	-0.0005950926\\
42.489856	-0.0003890988\\
42.643456	-4.5776916e-05\\
42.797056	-0.0004577634\\
42.950656	0.000251769744\\
43.104256	0\\
43.257856	-0.00043487568\\
43.411456	-0.000251770572\\
43.565056	-0.00011444148\\
43.718656	-0.00027465822\\
43.872256	-0.00043487568\\
44.025856	-0.000137329092\\
44.179456	2.28876264e-05\\
44.333056	9.1552176e-05\\
44.486656	0.000297545832\\
44.640256	-0.00057220488\\
44.793856	0\\
44.947456	2.28876264e-05\\
45.101056	-0.00061798104\\
45.254656	-0.0003890988\\
45.408256	-0.000160217568\\
45.561856	-0.00029754666\\
45.715456	-0.00018310518\\
45.869056	-0.00073242252\\
46.022656	-0.0005950926\\
46.176256	-4.5776916e-05\\
46.329856	-0.0003890988\\
46.483456	-0.00029754666\\
46.637056	-0.00011444148\\
46.790656	-0.00029754666\\
46.944256	0.00018310518\\
47.097856	-0.000183106044\\
47.251456	-0.00043487568\\
47.405056	-6.8664564e-05\\
47.558656	-0.000343322748\\
47.712256	-0.000320434308\\
47.865856	9.1552176e-05\\
48.019456	-0.000160217568\\
48.173056	-0.00018310518\\
48.326656	-0.00036621108\\
48.480256	-0.00041198724\\
48.633856	0\\
48.787456	0.000251768916\\
48.941056	-0.000251769744\\
49.094656	-0.00018310518\\
49.248256	-0.00048065184\\
49.401856	0.00016021674\\
49.555456	-0.00018310518\\
49.709056	-0.0005950926\\
49.862656	-0.000526428\\
50.016256	-0.00011444148\\
50.169856	-9.1553004e-05\\
50.323456	-0.000137329092\\
50.477056	-0.000205993656\\
50.630656	-0.000228882132\\
50.784256	-0.000114440652\\
50.937856	-0.000526428\\
51.091456	-0.00036621108\\
};
\addplot [color=black,dashed,forget plot]
  table[row sep=crcr]{%
5.718	-0.1\\
5.718	0.8\\
};
\addplot [color=black,dashed,forget plot]
  table[row sep=crcr]{%
7.138	-0.1\\
7.138	0.8\\
};
\addplot [color=black,dashed,forget plot]
  table[row sep=crcr]{%
7.14	-0.1\\
7.14	0.8\\
};
\addplot [color=black,dashed,forget plot]
  table[row sep=crcr]{%
8.662	-0.1\\
8.662	0.8\\
};
\addplot [color=black,dashed,forget plot]
  table[row sep=crcr]{%
10.105	-0.1\\
10.105	0.8\\
};
\addplot [color=black,dashed,forget plot]
  table[row sep=crcr]{%
11.619	-0.1\\
11.619	0.8\\
};

\addplot[area legend,solid,draw=black,fill=black,fill opacity=0.1,forget plot]
table[row sep=crcr] 
\end{minipage}%
\begin{minipage}[tbp]{0.39\textwidth}
\resizebox{\textwidth}{!}{
\begin{tabular}{|c|p{6.5cm}|} \hline
\multicolumn{2}{|c|}{\textbf{Log messages}} \\ \hline
\textbf{Line number} & \textbf{Message} \\ \hline
1 & Security Protected NAS Message (Attach Request) \\ \hline
2 & Security Protected NAS Message (Authentication Response) \\ \hline
3 & Security Protected NAS Message (Security Mode Command) \\ \hline
4 & Security Protected NAS Message (Security Mode Complete) \\ \hline
5 & Security Protected NAS Message (Attach Accept) \\ \hline
6 & Security Protected NAS Message (Attach Complete) \\ \hline
\end{tabular}}
\end{minipage}
\caption{Example of raw data from measurement. Area A is boot up and cell synchronization, area B is attach procedure and Area C is Cell release and idle period. Dashed line indicates log messages.}
\label{fig:Attach_raw}
\end{figure}

The next step is to get the four needed elements. The elements are found as:

\begin{align}
E_{sync} &= \int_{A_{start}}^{A_{end}} f(x) dx \\
P_{attach} &= \mathbf{E}(f(x)) \quad for \, B_{start} \leq x \leq B_{end} \label{eq:mean_over_phase} \\
T_{attach} &=  B_{end} - B_{start} \\
E_{release} &= \int_{C_{start}}^{C_{end}} f(x) dx \\
\end{align}
\begin{where}
\va{$E_{sync}$}{is the energy to boot up and synchronize to the cell}{J}
\va{$P_{attach}$}{is the average power consumption during the attach procedure}{W} 
\va{$T_{attach}$}{is the time it take to attach to the network}{s}
\va{$\mathbf{E}$(•)}{is the mean function}{1}
\va{f(x)}{is the data point at time x}{W}
\va{$A_{start}$}{is the start time of area A}{s}
\va{$A_{end}$}{is the end time of area A}{s}
\va{$B_{start}$}{is the start time of area B}{s}
\va{$B_{end}$}{is the end time of area B}{s}
\va{$C_{start}$}{is the start time of area C}{s}
\va{$C_{end}$}{is the end time of area C}{s}
\end{where}


This is done for all measurements and the results are presented along with their statistical properties in \autoref{fig:Sync_Points}, \autoref{fig:Attach_Power_Points}, \autoref{fig:Attach_Time_Points} \autoref{fig:Release_Points} respectively.

\begin{figure}[H]
\centering
\begin{minipage}{0.48\textwidth}
\tikzsetnextfilename{Sync_Points}
\resizebox{\textwidth}{!}{
% This file was created by matlab2tikz.
%
%The latest updates can be retrieved from
%  http://www.mathworks.com/matlabcentral/fileexchange/22022-matlab2tikz-matlab2tikz
%where you can also make suggestions and rate matlab2tikz.
%
\definecolor{mycolor1}{rgb}{0.00000,0.44700,0.74100}%
\definecolor{mycolor2}{rgb}{0.85000,0.32500,0.09800}%
\definecolor{mycolor3}{rgb}{0.92900,0.69400,0.12500}%
\definecolor{mycolor4}{rgb}{0.49400,0.18400,0.55600}%
%
\begin{tikzpicture}

\begin{axis}[%
width=0.951\textwidth,
height=0.66\textwidth,
at={(0\textwidth,0\textwidth)},
scale only axis,
xmin=1,
xmax=209,
xlabel={Data Points},
ymin=0,
ymax=1.4e-05,
ylabel={Energy [J]},
axis background/.style={fill=white},
title style={font=\bfseries},
title={Synchronization},
legend style={legend cell align=left,align=left,draw=white!15!black},
y tick label style={/pgf/number format/fixed}
]
\addplot [color=mycolor1,only marks,mark=*,mark options={solid}]
  table[row sep=crcr]{%
1	1.34067425208712e-06\\
2	1.12242602877281e-06\\
};
\addlegendentry{CP format};

\addplot [color=mycolor2,only marks,mark=*,mark options={solid}]
  table[row sep=crcr]{%
3	1.0756823088186e-06\\
4	9.97196253118895e-07\\
5	1.07530106959878e-06\\
6	1.10324330048733e-06\\
7	1.04014870921076e-06\\
8	1.08995478703403e-06\\
9	1.1088101599878e-06\\
10	1.0574919265968e-06\\
11	1.01933949858063e-06\\
12	1.0491033254808e-06\\
13	1.06707578286328e-06\\
14	1.63460769812375e-06\\
15	1.76965847945324e-06\\
16	1.6738210613843e-06\\
17	1.60690932864828e-06\\
18	1.72305997763063e-06\\
19	1.6508115391229e-06\\
20	1.56544019032926e-06\\
21	1.38199369240671e-06\\
22	1.09175487242403e-06\\
23	1.08166052595419e-06\\
24	1.71707936829757e-06\\
25	1.2924199967427e-06\\
26	1.47677538448511e-06\\
27	1.05464451047743e-06\\
28	1.26633908447658e-05\\
29	1.10713299878423e-06\\
30	1.05110069743594e-06\\
31	1.04509796344584e-06\\
32	7.85685380289262e-06\\
33	1.03941008462375e-06\\
34	1.04673374011435e-06\\
35	1.08389679662242e-06\\
36	1.08989320264378e-06\\
37	1.2232857822079e-05\\
38	1.71367546181616e-06\\
39	1.75188602710752e-06\\
40	1.05099817414129e-05\\
41	1.03102716006802e-05\\
42	1.00874506887433e-06\\
43	1.10039982398377e-06\\
44	1.72631780144864e-06\\
45	1.58841923592161e-06\\
46	1.67590984722733e-06\\
47	1.65918676629418e-06\\
48	1.12585857014324e-06\\
49	1.21532498016198e-05\\
50	1.0639670648209e-06\\
51	7.90220087917637e-06\\
52	1.10689124879218e-06\\
53	1.0676077696025e-06\\
54	1.02055968155225e-06\\
55	1.03395358962369e-06\\
56	1.02284176756657e-06\\
57	1.00698099365716e-06\\
58	1.01379266752505e-06\\
59	1.03784467029782e-06\\
60	1.04450143716487e-06\\
61	1.03554181243738e-06\\
62	1.06368443105705e-06\\
63	1.82842144943951e-06\\
64	1.62838648123534e-06\\
65	1.59678592550239e-06\\
66	1.62725781321621e-06\\
67	1.65902004362397e-06\\
68	1.67276848918022e-06\\
69	1.68009520248227e-06\\
70	1.65913762147219e-06\\
71	1.66958674466518e-06\\
72	1.03778659447381e-05\\
73	9.84603352986803e-07\\
74	9.56331062781806e-07\\
75	1.12956665075211e-06\\
76	1.05250931326862e-06\\
77	9.84810532273154e-07\\
78	1.03202459334288e-06\\
79	1.12447645031274e-06\\
80	1.1195701483476e-06\\
81	1.05549610825046e-06\\
82	1.03780730834437e-06\\
83	1.01451348392339e-06\\
84	9.7639080900736e-07\\
85	1.10388877159849e-06\\
86	1.46699178243079e-06\\
87	1.05528627769586e-06\\
88	1.06458895815028e-06\\
89	1.04739165217311e-06\\
90	1.06948595729843e-06\\
91	1.03704030611977e-06\\
92	1.06632948749806e-06\\
93	1.03752014683528e-06\\
94	1.78180333669874e-06\\
95	1.07831439734552e-06\\
96	1.15076575264746e-06\\
97	1.08720724352209e-06\\
98	9.88482488761432e-07\\
99	1.6120315796871e-06\\
100	1.6108189148589e-06\\
101	1.71780986353568e-06\\
102	1.14097621006086e-05\\
103	1.12966905992261e-06\\
104	1.07873402619562e-06\\
105	1.57303628227537e-06\\
106	1.25109337911685e-05\\
107	1.06306033639251e-06\\
108	1.01303664525916e-06\\
109	1.75385325235212e-06\\
110	1.60377716479687e-06\\
111	1.10900335278275e-06\\
112	1.08714248594812e-06\\
113	1.06173094639215e-06\\
114	1.06015971416799e-06\\
115	1.09544315069156e-06\\
116	1.02519762185331e-06\\
117	1.01783592260399e-06\\
118	1.0317569162224e-06\\
119	1.04889689721377e-06\\
120	1.06989439617546e-06\\
121	1.02707285658087e-06\\
122	1.81571982604701e-06\\
123	1.74532069170229e-06\\
124	1.64413315255404e-06\\
125	1.74222146043301e-06\\
126	1.60430704818907e-06\\
127	1.59930515569438e-06\\
128	1.02043187892461e-06\\
129	1.59417255244855e-06\\
130	1.04117727640757e-05\\
131	1.63276247008971e-06\\
132	1.76376031377767e-06\\
133	1.68926103201192e-06\\
134	1.02136090696712e-05\\
135	1.55646972188926e-06\\
136	1.62787210816775e-06\\
137	1.60662818064574e-06\\
138	1.66435120525633e-06\\
139	1.06309302326035e-06\\
140	1.09014975930545e-06\\
141	1.02966227181924e-06\\
142	1.0298157208698e-06\\
143	1.03261820054816e-06\\
144	1.01021929908402e-06\\
145	1.01759878321502e-06\\
146	1.03589330412151e-06\\
147	1.02904015412003e-06\\
148	1.0388339664431e-06\\
149	1.02325681731463e-06\\
150	1.06454234831571e-06\\
151	1.08137970610969e-06\\
152	1.2792154864867e-06\\
};
\addlegendentry{Frequency};

\addplot [color=mycolor3,only marks,mark=*,mark options={solid}]
  table[row sep=crcr]{%
153	1.71863070186303e-06\\
154	1.01456363868033e-06\\
155	1.33120240444451e-06\\
};
\addlegendentry{Operation mode};

\addplot [color=mycolor4,only marks,mark=*,mark options={solid}]
  table[row sep=crcr]{%
156	1.0671797625832e-06\\
157	1.0588456898191e-06\\
158	1.05795589484699e-06\\
159	1.08055848780402e-06\\
160	1.04762808644952e-06\\
161	1.1013140820616e-06\\
162	1.07759871557913e-06\\
163	1.08160697361213e-06\\
164	1.04025231904184e-06\\
165	1.09816496570356e-06\\
166	1.08714269056602e-06\\
167	1.08761153797267e-06\\
168	1.07365285652857e-06\\
169	1.10125425405216e-06\\
170	1.07156572489788e-06\\
171	1.0759174643657e-06\\
172	1.03783488430604e-06\\
173	1.12183899012571e-06\\
174	1.0201216620912e-06\\
175	1.03660357243892e-06\\
176	1.01556518712642e-06\\
177	1.09429710451097e-06\\
178	1.0796650157043e-06\\
179	1.07937842520173e-06\\
180	1.08979312304457e-06\\
181	1.09333904615246e-06\\
182	1.07734439096524e-06\\
183	1.07069784829799e-06\\
184	1.06280968742978e-06\\
185	1.08881266457677e-06\\
186	1.05410891640532e-06\\
187	1.09146145961626e-06\\
188	1.07078720758496e-06\\
189	1.08727910212997e-06\\
190	1.09437331228688e-06\\
191	1.10180616008859e-06\\
192	1.05744390923426e-06\\
193	1.08339901023109e-06\\
194	1.08311458846742e-06\\
195	1.07909301411026e-06\\
196	1.12340040319467e-06\\
197	1.05293692062927e-06\\
198	1.06402922366403e-06\\
199	1.05680910463732e-06\\
200	1.06535147640935e-06\\
201	1.10944323021434e-06\\
202	1.51935088635014e-06\\
203	1.12051665665269e-06\\
204	1.09551654623524e-06\\
205	1.05557387096436e-06\\
206	1.06804892364889e-06\\
207	1.09046565872179e-06\\
208	1.06742440177378e-06\\
209	1.0907300220633e-06\\
};
\addlegendentry{Pmax};

\end{axis}
\end{tikzpicture}%}
\end{minipage}
\hfill
\begin{minipage}{0.48\textwidth}
\tikzsetnextfilename{Sync_Stat}
\resizebox{\textwidth}{!}{
% This file was created by matlab2tikz.
%
%The latest updates can be retrieved from
%  http://www.mathworks.com/matlabcentral/fileexchange/22022-matlab2tikz-matlab2tikz
%where you can also make suggestions and rate matlab2tikz.
%
%Point distribution 
%mean = 0.000001759911629 
%var = 0.000000000005108
%
\definecolor{mycolor1}{rgb}{0.00000,0.44700,0.74100}%
\definecolor{mycolor2}{rgb}{0.85000,0.32500,0.09800}%
%
\begin{tikzpicture}

\begin{axis}[%
width=0.951\textwidth,
height=0.66\textwidth,
at={(0\textwidth,0\textwidth)},
scale only axis,
xmin=0,
xmax=1.4e-05,
xlabel={Energy [J]},
ymin=0,
ymax=0.8,
ylabel={Probability},
axis background/.style={fill=white},
title style={font=\bfseries},
title={Synchronization},
legend style={legend cell align=left,align=left,draw=white!15!black},
y tick label style={/pgf/number format/fixed}
]
\addplot[fill=mycolor1,fill opacity=0.6,draw=black,ybar interval,area legend] plot table[row sep=crcr] {%
x	y\\
9e-07	0.076555023923445\\
1.018e-06	0.607655502392344\\
1.136e-06	0.00478468899521531\\
1.254e-06	0.0191387559808612\\
1.372e-06	0.0143540669856459\\
1.49e-06	0.0574162679425837\\
1.608e-06	0.114832535885167\\
1.726e-06	0.0478468899521531\\
1.844e-06	0\\
1.962e-06	0\\
2.08e-06	0\\
2.198e-06	0\\
2.316e-06	0\\
2.434e-06	0\\
2.552e-06	0\\
2.67e-06	0\\
2.788e-06	0\\
2.906e-06	0\\
3.024e-06	0\\
3.142e-06	0\\
3.26e-06	0\\
3.378e-06	0\\
3.496e-06	0\\
3.614e-06	0\\
3.732e-06	0\\
3.85e-06	0\\
3.968e-06	0\\
4.086e-06	0\\
4.204e-06	0\\
4.322e-06	0\\
4.44e-06	0\\
4.558e-06	0\\
4.676e-06	0\\
4.794e-06	0\\
4.912e-06	0\\
5.03e-06	0\\
5.148e-06	0\\
5.266e-06	0\\
5.384e-06	0\\
5.502e-06	0\\
5.62e-06	0\\
5.738e-06	0\\
5.856e-06	0\\
5.974e-06	0\\
6.092e-06	0\\
6.21e-06	0\\
6.328e-06	0\\
6.446e-06	0\\
6.564e-06	0\\
6.682e-06	0\\
6.8e-06	0\\
6.918e-06	0\\
7.036e-06	0\\
7.154e-06	0\\
7.272e-06	0\\
7.39e-06	0\\
7.508e-06	0\\
7.626e-06	0\\
7.744e-06	0.00478468899521531\\
7.862e-06	0.00478468899521531\\
7.98e-06	0\\
8.098e-06	0\\
8.216e-06	0\\
8.334e-06	0\\
8.452e-06	0\\
8.57e-06	0\\
8.688e-06	0\\
8.806e-06	0\\
8.924e-06	0\\
9.042e-06	0\\
9.16e-06	0\\
9.278e-06	0\\
9.396e-06	0\\
9.514e-06	0\\
9.632e-06	0\\
9.75e-06	0\\
9.868e-06	0\\
9.986e-06	0\\
1.0104e-05	0.00478468899521531\\
1.0222e-05	0.00478468899521531\\
1.034e-05	0.00956937799043062\\
1.0458e-05	0.00478468899521531\\
1.0576e-05	0\\
1.0694e-05	0\\
1.0812e-05	0\\
1.093e-05	0\\
1.1048e-05	0\\
1.1166e-05	0\\
1.1284e-05	0\\
1.1402e-05	0.00478468899521531\\
1.152e-05	0\\
1.1638e-05	0\\
1.1756e-05	0\\
1.1874e-05	0\\
1.1992e-05	0\\
1.211e-05	0.00478468899521531\\
1.2228e-05	0.00478468899521531\\
1.2346e-05	0\\
1.2464e-05	0.00478468899521531\\
1.2582e-05	0.00478468899521531\\
1.27e-05	0.00478468899521531\\
};
\addlegendentry{Data points};

\addplot[ycomb,color=mycolor2,solid,mark=o,mark options={solid}] plot table[row sep=crcr] 
\end{minipage}
\caption{Energy consumed during boot up and cell synchronization.}
\label{fig:Sync_Points}
\end{figure}

\begin{figure}[H]
\centering
\begin{minipage}{0.48\textwidth}
\tikzsetnextfilename{Attach_Power_Points}
\resizebox{\textwidth}{!}{
% This file was created by matlab2tikz.
%
%The latest updates can be retrieved from
%  http://www.mathworks.com/matlabcentral/fileexchange/22022-matlab2tikz-matlab2tikz
%where you can also make suggestions and rate matlab2tikz.
%
\definecolor{mycolor1}{rgb}{0.00000,0.44700,0.74100}%
\definecolor{mycolor2}{rgb}{0.85000,0.32500,0.09800}%
\definecolor{mycolor3}{rgb}{0.92900,0.69400,0.12500}%
%
\begin{tikzpicture}

\begin{axis}[%
width=\textwidth,
height=0.66\textwidth,
at={(2.08in,0.858in)},
scale only axis,
xmin=0,
xmax=150,
xlabel={Data Points},
ymin=0.25,
ymax=0.3,
ylabel={$P_{attach}$ [W]},
axis background/.style={fill=white},
title style={font=\bfseries},
title={Measurement overview},
legend style={at={(0.03,0.97)},anchor=north west,legend cell align=left,align=left,draw=white!15!black}
]
\addplot [color=mycolor1,only marks,mark=*,mark options={solid}]
  table[row sep=crcr]{%
1	0.255801781387079\\
2	0.251421193899255\\
};
\addlegendentry{CP format};

\addplot [color=mycolor2,only marks,mark=*,mark options={solid}]
  table[row sep=crcr]{%
1	0.265445452572333\\
2	0.266954209300766\\
3	0.264797928614294\\
4	0.272704083459804\\
5	0.267746363195588\\
6	0.273178340361388\\
7	0.266368901171886\\
8	0.268619684665784\\
9	0.26708039540221\\
10	0.26655998303708\\
11	0.266845320410119\\
12	0.26698328534778\\
13	0.274736472721134\\
14	0.26539051637896\\
15	0.26820485827082\\
16	0.269552843355458\\
17	0.270038646899647\\
18	0.267200278116755\\
19	0.267458859897132\\
20	0.266468679143444\\
21	0.26471865694658\\
22	0.271832738092186\\
23	0.265124819513339\\
24	0.271922464341057\\
25	0.26576713342322\\
26	0.297821655391698\\
27	0.266692748192563\\
28	0.27185355957331\\
29	0.27109167349431\\
30	0.274880327160046\\
31	0.272695441318171\\
32	0.269091895735741\\
33	0.272947558238\\
34	0.264727815075479\\
35	0.269121646325412\\
36	0.273982228971079\\
37	0.268740328073609\\
38	0.271002884242907\\
39	0.270969670950985\\
40	0.276056888833248\\
41	0.273177349384088\\
42	0.26891830134814\\
43	0.275789507233496\\
44	0.276163342958147\\
45	0.267687699991572\\
46	0.27275713803948\\
47	0.276798360841021\\
48	0.274031090363989\\
49	0.273568020346507\\
50	0.266483755723602\\
51	0.275306897683914\\
52	0.266156483990352\\
53	0.275860617369944\\
54	0.272732960453597\\
55	0.275832360931848\\
56	0.265790729613348\\
57	0.264866706743245\\
58	0.27268646654511\\
59	0.264753886733427\\
60	0.272355979523188\\
61	0.264634276307107\\
62	0.263899661419284\\
63	0.262340020610494\\
64	0.262979411732452\\
65	0.261931117679891\\
66	0.26357937047217\\
67	0.26556369307208\\
68	0.270640483343679\\
69	0.263361875135284\\
70	0.266352652929479\\
71	0.270079331364635\\
72	0.270463024741068\\
73	0.264301440365561\\
74	0.262465009871972\\
75	0.269909229745724\\
76	0.263830255992203\\
77	0.26312089127998\\
78	0.265610113533262\\
79	0.264752943169823\\
80	0.271741014345849\\
81	0.266602782621202\\
82	0.270223435545716\\
83	0.270346941149791\\
84	0.267448254432004\\
85	0.27000486018762\\
86	0.272588367329993\\
87	0.263662673885227\\
88	0.267540780880364\\
89	0.264446563400098\\
90	0.272814771249928\\
91	0.273135136900738\\
92	0.267048611657137\\
93	0.269776935214174\\
94	0.273145441050165\\
95	0.268382709252621\\
96	0.27325746904602\\
97	0.264877676225099\\
98	0.273111618218773\\
99	0.268504508185846\\
100	0.26981930830676\\
101	0.26635946567318\\
102	0.26475272191475\\
103	0.266291505024782\\
104	0.298721808531873\\
105	0.265683446153979\\
106	0.272693837948645\\
107	0.273133025449803\\
108	0.27343760359363\\
109	0.269099368354071\\
110	0.266543572066331\\
111	0.274539036010476\\
112	0.265681406612183\\
113	0.264117585450568\\
114	0.2638563551989\\
115	0.265457769724398\\
116	0.264650225933215\\
117	0.272462026303984\\
118	0.272541196122011\\
119	0.264688911717546\\
120	0.27185822484575\\
121	0.266344679318229\\
122	0.265120973142423\\
123	0.264044325882864\\
124	0.274519937915519\\
125	0.267497846746046\\
126	0.26532517894989\\
127	0.266108968347231\\
128	0.271165338108506\\
129	0.26541109379603\\
130	0.265506580313867\\
131	0.268428618015063\\
132	0.268303887513969\\
133	0.267725430056234\\
134	0.265800497472174\\
135	0.269969688927248\\
136	0.267988154047674\\
137	0.267691948327441\\
138	0.268813192576425\\
139	0.278376814778973\\
140	0.271446959287438\\
141	0.271177312223487\\
142	0.276294566737266\\
143	0.279817661513477\\
144	0.271924769209123\\
145	0.27728984238318\\
146	0.278348619230541\\
147	0.26977093324355\\
148	0.270057690768147\\
149	0.272815459477205\\
150	0.27838604219041\\
};
\addlegendentry{Frequency};

\addplot [color=mycolor3,only marks,mark=*,mark options={solid}]
  table[row sep=crcr]
\end{minipage}
\hfill
\begin{minipage}{0.48\textwidth}
\tikzsetnextfilename{Attach_Power_Stat}
\resizebox{\textwidth}{!}{
% This file was created by matlab2tikz.
%
%The latest updates can be retrieved from
%  http://www.mathworks.com/matlabcentral/fileexchange/22022-matlab2tikz-matlab2tikz
%where you can also make suggestions and rate matlab2tikz.
%
%Lognormal distribution 
%mean = -1.312861377596010 
%var = 0.000442742141377
%
\definecolor{mycolor1}{rgb}{0.00000,0.44700,0.74100}%
\definecolor{mycolor2}{rgb}{0.85000,0.32500,0.09800}%
%
\begin{tikzpicture}

\begin{axis}[%
width=0.951\textwidth,
height=0.66\textwidth,
at={(0\textwidth,0\textwidth)},
scale only axis,
xmin=0.25,
xmax=0.3,
xlabel={Avg. power consumption [W]},
ymin=0,
ymax=120,
ylabel={PDF},
axis background/.style={fill=white},
title style={font=\bfseries},
title={Statistical Overview},
legend style={legend cell align=left,align=left,draw=white!15!black},
y tick label style={/pgf/number format/fixed}
]
\addplot[fill=mycolor1,fill opacity=0.6,draw=black,ybar interval,area legend] plot table[row sep=crcr] {%
x	y\\
0.251	12.9032258064516\\
0.252	0\\
0.253	0\\
0.254	0\\
0.255	6.4516129032258\\
0.256	0\\
0.257	0\\
0.258	0\\
0.259	6.4516129032258\\
0.26	0\\
0.261	6.4516129032258\\
0.262	25.8064516129032\\
0.263	45.1612903225806\\
0.264	96.774193548387\\
0.265	96.774193548387\\
0.266	103.225806451613\\
0.267	77.4193548387096\\
0.268	58.0645161290322\\
0.269	58.0645161290322\\
0.27	58.0645161290322\\
0.271	70.9677419354838\\
0.272	83.8709677419354\\
0.273	64.516129032258\\
0.274	32.258064516129\\
0.275	25.8064516129032\\
0.276	25.8064516129032\\
0.277	6.4516129032258\\
0.278	19.3548387096774\\
0.279	6.4516129032258\\
0.28	0\\
0.281	0\\
0.282	0\\
0.283	0\\
0.284	0\\
0.285	0\\
0.286	0\\
0.287	0\\
0.288	0\\
0.289	0\\
0.29	0\\
0.291	0\\
0.292	0\\
0.293	0\\
0.294	0\\
0.295	0\\
0.296	0\\
0.297	6.4516129032258\\
0.298	6.4516129032258\\
0.299	6.4516129032258\\
};
\addlegendentry{Data points};

\addplot [color=mycolor2,solid]
  table[row sep=crcr]{%
0.25	0.171845223870984\\
0.25025	0.202399430433348\\
0.2505	0.237810851550256\\
0.25075	0.27874538217441\\
0.251	0.32594204360666\\
0.25125	0.380218131614393\\
0.2515	0.442474283858895\\
0.25175	0.513699382980346\\
0.252	0.594975201347595\\
0.25225	0.687480683304638\\
0.2525	0.792495750975073\\
0.25275	0.911404510579902\\
0.253	1.04569772807084\\
0.25325	1.19697443597815\\
0.2535	1.36694252804082\\
0.25375	1.55741819474601\\
0.254	1.77032405167894\\
0.25425	2.00768581388706\\
0.2545	2.27162737359284\\
0.25475	2.56436414581347\\
0.255	2.88819455700698\\
0.25525	3.24548956593147\\
0.2555	3.63868012363672\\
0.25575	4.07024250092882\\
0.256	4.54268143679494\\
0.25625	5.05851108999705\\
0.2565	5.62023380821046\\
0.25675	6.23031676436176\\
0.257	6.8911665478748\\
0.25725	7.6051018388539\\
0.2575	8.37432433527333\\
0.25775	9.20088814630254\\
0.258	10.0866679082699\\
0.25825	11.0333259225855\\
0.2585	12.0422786563597\\
0.25875	13.1146629855087\\
0.259	14.251302595908\\
0.25925	15.4526749896376\\
0.2595	16.718879569671\\
0.25975	18.049607296526\\
0.26	19.444112423636\\
0.26025	20.9011868237368\\
0.2605	22.4191374157584\\
0.26075	23.9957671900685\\
0.261	25.628360309104\\
0.26125	27.3136717302079\\
0.2615	29.0479217579943\\
0.26175	30.826795884836\\
0.262	32.6454502207078\\
0.26225	34.4985227480648\\
0.2625	36.3801505646406\\
0.26275	38.2839931979449\\
0.263	40.2032619910344\\
0.26325	42.1307554711782\\
0.2635	44.0589005227978\\
0.26375	45.9797990951194\\
0.264	47.8852800850038\\
0.26425	49.7669559480777\\
0.2645	51.6162835083292\\
0.26475	53.4246283593318\\
0.265	55.1833321809165\\
0.26525	56.8837822347645\\
0.2655	58.517482252517\\
0.26575	60.0761238915919\\
0.266	61.5516579080271\\
0.26625	62.9363641829331\\
0.2665	64.222919740083\\
0.26675	65.4044639068683\\
0.267	66.4746597993442\\
0.26725	67.4277513538758\\
0.2675	68.2586151824778\\
0.26775	68.962806595305\\
0.268	69.5365992108802\\
0.26825	69.9770176611054\\
0.2685	70.2818629924392\\
0.26875	70.4497304651151\\
0.269	70.4800195571528\\
0.26925	70.3729360873009\\
0.2695	70.1294864790465\\
0.26975	69.7514642945422\\
0.27	69.2414292708743\\
0.27025	68.6026791897826\\
0.2705	67.8392150040735\\
0.27075	66.9556997280941\\
0.271	65.9574116744\\
0.27125	64.8501926831172\\
0.2715	63.6403920435241\\
0.27175	62.3348068485042\\
0.272	60.9406195512668\\
0.27225	59.465333510043\\
0.2725	57.916707310352\\
0.27275	56.302688646281\\
0.273	54.6313485225445\\
0.27325	52.9108165086296\\
0.2735	51.1492177360156\\
0.27375	49.3546122803347\\
0.274	47.5349375135889\\
0.27425	45.6979539484794\\
0.2745	43.8511950287846\\
0.27475	42.0019212480285\\
0.275	40.1570789046672\\
0.27525	38.3232637271064\\
0.2755	36.5066895273072\\
0.27575	34.7131619687698\\
0.276	32.9480574643891\\
0.27625	31.2163071531178\\
0.2765	29.522385842391\\
0.27675	27.8703057465758\\
0.277	26.2636148010197\\
0.27725	24.7053992868722\\
0.2775	23.1982904642394\\
0.27775	21.7444748803849\\
0.278	20.3457079958193\\
0.27825	19.0033307539685\\
0.2785	17.7182887096151\\
0.27875	16.4911533270284\\
0.279	15.3221450603451\\
0.27925	14.2111578357656\\
0.2795	13.1577845670416\\
0.27975	12.1613433518946\\
0.28	11.2209040168847\\
0.28025	10.3353147011627\\
0.2805	9.5032281949243\\
0.28075	8.72312777556484\\
0.281	7.9933523129895\\
0.28125	7.31212044463474\\
0.2815	6.67755365005487\\
0.28175	6.08769808387722\\
0.282	5.54054505417151\\
0.28225	5.03405006039098\\
0.2825	4.56615033074434\\
0.28275	4.13478082286445\\
0.283	3.73788867377045\\
0.28325	3.37344610519386\\
0.2835	3.03946180829223\\
0.28375	2.73399084751225\\
0.284	2.45514313691701\\
0.28425	2.20109055365431\\
0.2845	1.97007276250164\\
0.28475	1.76040183265436\\
0.285	1.57046573325377\\
0.28525	1.39873079770297\\
0.2855	1.24374324875627\\
0.28575	1.10412987683429\\
0.286	0.978597963189774\\
0.28625	0.865934537589105\\
0.2865	0.765005057248215\\
0.28675	0.674751590032798\\
0.287	0.594190580553818\\
0.28725	0.522410272908554\\
0.2875	0.458567858570735\\
0.28775	0.401886412444545\\
0.288	0.351651674481697\\
0.28825	0.30720872861487\\
0.2885	0.267958625175911\\
0.28875	0.233354987514721\\
0.289	0.20290063827961\\
0.28925	0.176144275811011\\
0.2895	0.152677226379283\\
0.28975	0.132130293591465\\
0.29	0.114170722223781\\
0.29025	0.0984992900159874\\
0.2905	0.0848475375971341\\
0.29075	0.072975143696206\\
0.291	0.0626674501198255\\
0.29125	0.0537331386397327\\
0.2915	0.0460020599109428\\
0.29175	0.039323212818203\\
0.292	0.0335628712043782\\
0.29225	0.0286028537474146\\
0.2925	0.0243389318007153\\
0.29275	0.0206793692722316\\
0.293	0.0175435880679838\\
0.29325	0.0148609522440249\\
0.2935	0.0125696637760976\\
0.29375	0.0106157627481245\\
0.294	0.00895222476061181\\
0.29425	0.00753814845039744\\
0.2945	0.00633802617798225\\
0.29475	0.00532109116340113\\
0.295	0.00446073462309082\\
0.29525	0.0037339867669241\\
0.2955	0.00312105584628309\\
0.29575	0.00260491979185399\\
0.296	0.00217096533617669\\
0.29625	0.00180666987442564\\
0.2965	0.00150132167210304\\
0.29675	0.0012457743759302\\
0.297	0.00103223212075256\\
0.29725	0.000854061848032544\\
0.2975	0.000705629758496274\\
0.29775	0.000582159111319884\\
0.298	0.000479606854005204\\
0.29825	0.000394556820349153\\
0.2985	0.000324127468573277\\
0.29875	0.000265892347956714\\
0.299	0.000217811680661702\\
0.29925	0.000178173626488333\\
0.2995	0.000145543962828507\\
0.29975	0.000118723060983074\\
0.3	9.67091742132514e-05\\
};
\addlegendentry{Approx.};

\end{axis}
\end{tikzpicture}%}
\end{minipage}
\caption{Average power consumption during attach procedure.}
\label{fig:Attach_Power_Points}
\end{figure}

\begin{figure}[H]
\centering
\begin{minipage}{0.48\textwidth}
\tikzsetnextfilename{Attach_Time_Points}
\resizebox{\textwidth}{!}{
% This file was created by matlab2tikz.
%
%The latest updates can be retrieved from
%  http://www.mathworks.com/matlabcentral/fileexchange/22022-matlab2tikz-matlab2tikz
%where you can also make suggestions and rate matlab2tikz.
%
\definecolor{mycolor1}{rgb}{0.00000,0.44700,0.74100}%
\definecolor{mycolor2}{rgb}{0.85000,0.32500,0.09800}%
\definecolor{mycolor3}{rgb}{0.92900,0.69400,0.12500}%
\definecolor{mycolor4}{rgb}{0.49400,0.18400,0.55600}%
%
\begin{tikzpicture}

\begin{axis}[%
width=12.4in,
height=6.357in,
at={(2.08in,0.858in)},
scale only axis,
xmin=0,
xmax=150,
xlabel={Data Points},
ymin=2.2,
ymax=2.8,
ylabel={Time [s]},
axis background/.style={fill=white},
title style={font=\bfseries},
title={Attach procedure},
legend style={legend cell align=left,align=left,draw=white!15!black}
]
\addplot [color=mycolor1,only marks,mark=*,mark options={solid}]
  table[row sep=crcr]{%
1	2.47\\
2	2.611\\
};
\addlegendentry{CP format};

\addplot [color=mycolor2,only marks,mark=*,mark options={solid}]
  table[row sep=crcr]{%
1	2.345\\
2	2.345\\
3	2.348\\
4	2.307\\
5	2.345\\
6	2.214\\
7	2.346\\
8	2.344\\
9	2.313\\
10	2.336\\
11	2.334\\
12	2.345\\
13	2.214\\
14	2.345\\
15	2.345\\
16	2.344\\
17	2.344\\
18	2.347\\
19	2.344\\
20	2.344\\
21	2.404\\
22	2.214\\
23	2.307\\
24	2.216\\
25	2.31\\
26	2.30299999999999\\
27	2.345\\
28	2.219\\
29	2.215\\
30	2.31099999999999\\
31	2.215\\
32	2.273\\
33	2.214\\
34	2.344\\
35	2.32199999999999\\
36	2.215\\
37	2.345\\
38	2.328\\
39	2.33\\
40	2.215\\
41	2.219\\
42	2.344\\
43	2.214\\
44	2.216\\
45	2.347\\
46	2.214\\
47	2.25699999999999\\
48	2.214\\
49	2.32199999999999\\
50	2.345\\
51	2.217\\
52	2.348\\
53	2.215\\
54	2.22\\
55	2.215\\
56	2.345\\
57	2.344\\
58	2.214\\
59	2.344\\
60	2.214\\
61	2.344\\
62	2.344\\
63	2.344\\
64	2.347\\
65	2.344\\
66	2.344\\
67	2.345\\
68	2.215\\
69	2.35\\
70	2.256\\
71	2.214\\
72	2.218\\
73	2.345\\
74	2.344\\
75	2.214\\
76	2.345\\
77	2.345\\
78	2.345\\
79	2.344\\
80	2.214\\
81	2.344\\
82	2.214\\
83	2.214\\
84	2.261\\
85	2.243\\
86	2.219\\
87	2.345\\
88	2.345\\
89	2.344\\
90	2.217\\
91	2.214\\
92	2.246\\
93	2.308\\
94	2.214\\
95	2.344\\
96	2.215\\
97	2.344\\
98	2.214\\
99	2.345\\
100	2.34\\
101	2.344\\
102	2.347\\
103	2.34\\
104	2.31399999999999\\
105	2.346\\
106	2.218\\
107	2.214\\
108	2.218\\
109	2.345\\
110	2.344\\
111	2.214\\
112	2.345\\
113	2.345\\
114	2.345\\
115	2.345\\
116	2.344\\
117	2.214\\
118	2.214\\
119	2.344\\
120	2.214\\
121	2.344\\
122	2.344\\
123	2.345\\
124	2.215\\
125	2.345\\
126	2.344\\
127	2.344\\
128	2.28\\
129	2.345\\
130	2.346\\
131	2.345\\
132	2.309\\
133	2.345\\
134	2.345\\
135	2.344\\
136	2.345\\
137	2.346\\
138	2.349\\
139	2.214\\
140	2.348\\
141	2.344\\
142	2.218\\
143	2.215\\
144	2.349\\
145	2.215\\
146	2.217\\
147	2.346\\
148	2.343\\
149	2.345\\
150	2.214\\
};
\addlegendentry{Frequency};

\addplot [color=mycolor3,only marks,mark=*,mark options={solid}]
  table[row sep=crcr]{%
1	2.61\\
2	2.346\\
3	2.344\\
};
\addlegendentry{Operation mode};

\addplot [color=mycolor4,only marks,mark=*,mark options={solid}]
  table[row sep=crcr]{%
1	2.469\\
2	2.469\\
3	2.726\\
4	2.613\\
5	2.61\\
6	2.47\\
7	2.47\\
8	2.726\\
9	2.611\\
10	2.611\\
11	2.469\\
12	2.47\\
13	2.61\\
14	2.47\\
15	2.611\\
16	2.611\\
17	2.611\\
18	2.471\\
19	2.611\\
20	2.611\\
21	2.469\\
22	2.61\\
23	2.611\\
24	2.61\\
25	2.47\\
26	2.619\\
27	2.61\\
28	2.471\\
29	2.611\\
30	2.469\\
31	2.612\\
32	2.726\\
33	2.47\\
34	2.611\\
35	2.61\\
36	2.726\\
37	2.471\\
38	2.469\\
39	2.611\\
40	2.61\\
41	2.614\\
42	2.469\\
43	2.473\\
44	2.726\\
45	2.612\\
46	2.469\\
47	2.216\\
48	2.61\\
49	2.614\\
50	2.61\\
51	2.611\\
52	2.726\\
53	2.474\\
54	2.61\\
};
\addlegendentry{Pmax};

\end{axis}
\end{tikzpicture}%}
\end{minipage}
\hfill
\begin{minipage}{0.48\textwidth}
\tikzsetnextfilename{Attach_Time_Stat}
\resizebox{\textwidth}{!}{
% This file was created by matlab2tikz.
%
%The latest updates can be retrieved from
%  http://www.mathworks.com/matlabcentral/fileexchange/22022-matlab2tikz-matlab2tikz
%where you can also make suggestions and rate matlab2tikz.
%
%Point distribution 
%mean = 2.371081339712919 
%var = 0.019530805852043
%
\definecolor{mycolor1}{rgb}{0.00000,0.44700,0.74100}%
\definecolor{mycolor2}{rgb}{0.85000,0.32500,0.09800}%
%
\begin{tikzpicture}

\begin{axis}[%
width=0.951\textwidth,
height=0.66\textwidth,
at={(0\textwidth,0\textwidth)},
scale only axis,
xmin=2.2,
xmax=2.8,
xlabel={Time [s]},
ymin=0,
ymax=0.45,
ylabel={Probability},
axis background/.style={fill=white},
title style={font=\bfseries},
title={Attach procedure},
legend style={legend cell align=left,align=left,draw=white!15!black},
y tick label style={/pgf/number format/fixed}
]
\addplot[fill=mycolor1,fill opacity=0.6,draw=black,ybar interval,area legend] plot table[row sep=crcr] {%
x	y\\
2.21	0.229665071770335\\
2.22	0.00478468899521531\\
2.23	0\\
2.24	0.00956937799043062\\
2.25	0.00956937799043062\\
2.26	0.00478468899521531\\
2.27	0.00478468899521531\\
2.28	0.00478468899521531\\
2.29	0\\
2.3	0.0239234449760766\\
2.31	0.0191387559808612\\
2.32	0.0143540669856459\\
2.33	0.0143540669856459\\
2.34	0.38755980861244\\
2.35	0\\
2.36	0\\
2.37	0\\
2.38	0\\
2.39	0\\
2.4	0.00478468899521531\\
2.41	0\\
2.42	0\\
2.43	0\\
2.44	0\\
2.45	0\\
2.46	0.0430622009569378\\
2.47	0.0526315789473684\\
2.48	0\\
2.49	0\\
2.5	0\\
2.51	0\\
2.52	0\\
2.53	0\\
2.54	0\\
2.55	0\\
2.56	0\\
2.57	0\\
2.58	0\\
2.59	0\\
2.6	0.00478468899521531\\
2.61	0.138755980861244\\
2.62	0\\
2.63	0\\
2.64	0\\
2.65	0\\
2.66	0\\
2.67	0\\
2.68	0\\
2.69	0\\
2.7	0\\
2.71	0\\
2.72	0.0287081339712919\\
2.73	0.0287081339712919\\
};
\addlegendentry{Data points};

\addplot[ycomb,color=mycolor2,solid,mark=o,mark options={solid}] plot table[row sep=crcr] 
\end{minipage}
\caption{Time spent during attach procedure}
\label{fig:Attach_Time_Points}
\end{figure}

\begin{figure}[H]
\centering
\begin{minipage}{0.48\textwidth}
\tikzsetnextfilename{Release_Points}
\resizebox{\textwidth}{!}{
% This file was created by matlab2tikz.
%
%The latest updates can be retrieved from
%  http://www.mathworks.com/matlabcentral/fileexchange/22022-matlab2tikz-matlab2tikz
%where you can also make suggestions and rate matlab2tikz.
%
\definecolor{mycolor1}{rgb}{0.00000,0.44700,0.74100}%
\definecolor{mycolor2}{rgb}{0.85000,0.32500,0.09800}%
\definecolor{mycolor3}{rgb}{0.92900,0.69400,0.12500}%
\definecolor{mycolor4}{rgb}{0.49400,0.18400,0.55600}%
%
\begin{tikzpicture}

\begin{axis}[%
width=0.951\textwidth,
height=0.66\textwidth,
at={(0\textwidth,0\textwidth)},
scale only axis,
xmin=1,
xmax=209,
xlabel={Data Points},
ymin=0,
ymax=0.16,
ylabel={Energy [J]},
axis background/.style={fill=white},
title style={font=\bfseries},
title={Measurement Overview},
legend style={legend cell align=left,align=left,draw=white!15!black},
y tick label style={/pgf/number format/fixed}
]
\addplot [color=mycolor1,only marks,mark=*,mark options={solid}]
  table[row sep=crcr]{%
1	0.00994643674206147\\
2	0.0105962969702692\\
};
\addlegendentry{CP format};

\addplot [color=mycolor2,only marks,mark=*,mark options={solid}]
  table[row sep=crcr]{%
3	0.0301573014724813\\
4	0.0476739666827834\\
5	0.0369806975842273\\
6	0.0302127860754506\\
7	0.031429934947553\\
8	0.0312430958394074\\
9	0.0388080608645223\\
10	0.0415286524453647\\
11	0.0346616187367609\\
12	0.032985371215972\\
13	0.03677990655978\\
14	0.0240184511951105\\
15	0.0214482887235071\\
16	0.0235789089457138\\
17	0.0252496566335599\\
18	0.0219902408076722\\
19	0.0236230842634184\\
20	0.0248031155978056\\
21	0.0246638611896653\\
22	0.0303144746126213\\
23	0.0300979722987182\\
24	0.0217394466170055\\
25	0.0252719020886292\\
26	0.0273962993708698\\
27	0.0349042517026061\\
28	0.0289077768\\
29	0.0418298721412264\\
30	0.0432300065855512\\
31	0.0427895542272352\\
32	0.0308532708\\
33	0.0424758968846703\\
34	0.0323016383440109\\
35	0.0327525684718638\\
36	0.028421662313484\\
37	0.150638995305388\\
38	0.0242683208317086\\
39	0.0212560336821989\\
40	0.0484779682866675\\
41	0.0494706372756239\\
42	0.047641411982072\\
43	0.0442140421604536\\
44	0.022142816465737\\
45	0.023377053388956\\
46	0.0227670077238219\\
47	0.0236926187808507\\
48	0.0311431444109122\\
49	0.0312194808\\
50	0.0416885795543528\\
51	0.031196592\\
52	0.0351579249229107\\
53	0.0421188185525581\\
54	0.0445006186374646\\
55	0.0429764023419857\\
56	0.0454428675747027\\
57	0.0338490992021966\\
58	0.0463084374569551\\
59	0.0396001252641951\\
60	0.0410025597991693\\
61	0.0428515606684853\\
62	0.0386916371975134\\
63	0.0198587760001\\
64	0.0216802435924764\\
65	0.0192769592038102\\
66	0.0234723867125537\\
67	0.0228497276689089\\
68	0.0234490009380129\\
69	0.0214983237698915\\
70	0.0233115223792541\\
71	0.0203821370115099\\
72	0.0583271098224916\\
73	0.0481864636753437\\
74	0.0392019014041474\\
75	0.0290774416333613\\
76	0.0316347354775427\\
77	0.0373109621216969\\
78	0.0430630880362546\\
79	0.0368025918263015\\
80	0.0371096797138465\\
81	0.0417347395292866\\
82	0.0443138313368579\\
83	0.0343677611598686\\
84	0.0369234107098707\\
85	0.0294826405176199\\
86	0.0228095771877725\\
87	0.0328371941103613\\
88	0.0418733793188408\\
89	0.0410804711541898\\
90	0.044163628231653\\
91	0.0428773604302311\\
92	0.0402453283530561\\
93	0.0447655421533779\\
94	0.0184683616801084\\
95	0.0331746907181822\\
96	0.0306443902772821\\
97	0.0282307480775131\\
98	0.0401920149160178\\
99	0.0246769054318894\\
100	0.0255149496468721\\
101	0.0239499000627468\\
102	0.105556945981192\\
103	0.0311732215781298\\
104	0.0312567668089809\\
105	0.0215987728127363\\
106	0.029045106\\
107	0.0404921019581184\\
108	0.0453014004167779\\
109	0.0205118098572748\\
110	0.0256244534710191\\
111	0.029976276972839\\
112	0.029997539329442\\
113	0.0335589510331619\\
114	0.0313474007638388\\
115	0.041854556170765\\
116	0.0414178967337897\\
117	0.0440620543391526\\
118	0.042806526701038\\
119	0.0467704200363513\\
120	0.0453159316376013\\
121	0.0412425279202057\\
122	0.0215703960956637\\
123	0.0205876372817572\\
124	0.0227212802089646\\
125	0.0208266313686899\\
126	0.0230468412963926\\
127	0.0243681175138924\\
128	0.031154041555864\\
129	0.0244065361710705\\
130	0.0583874971071656\\
131	0.0239662234771132\\
132	0.0195808625445741\\
133	0.0246720697418699\\
134	0.0555509063989803\\
135	0.0244960451423602\\
136	0.0237984181473131\\
137	0.0212839139938878\\
138	0.0215885086916623\\
139	0.0426531530573094\\
140	0.0396117766387705\\
141	0.0441558894297013\\
142	0.0429818337947965\\
143	0.0430406371074319\\
144	0.0477479102012606\\
145	0.0455132734561443\\
146	0.0396237726778813\\
147	0.044236155259222\\
148	0.041301456369265\\
149	0.0458625592441825\\
150	0.0325020552800921\\
151	0.0314654176895867\\
152	0.0322851415923038\\
};
\addlegendentry{Frequency};

\addplot [color=mycolor3,only marks,mark=*,mark options={solid}]
  table[row sep=crcr]{%
153	0.010212791649414\\
154	0.0115551225453926\\
155	0.00998554706395161\\
};
\addlegendentry{Operation mode};

\addplot [color=mycolor4,only marks,mark=*,mark options={solid}]
  table[row sep=crcr]{%
156	0.0219160191967968\\
157	0.0249456366845058\\
158	0.0254198523644307\\
159	0.0230796595439017\\
160	0.0256430307198872\\
161	0.0187815478039183\\
162	0.0225169286113658\\
163	0.023193142251799\\
164	0.0251978942233934\\
165	0.0179904299221518\\
166	0.0201096900007778\\
167	0.0211275199228428\\
168	0.0210609035215946\\
169	0.0249984972626615\\
170	0.0227834355078076\\
171	0.0237427448261522\\
172	0.0254040092175111\\
173	0.0219673072278129\\
174	0.0298528913106717\\
175	0.0224095684638939\\
176	0.0235149851026889\\
177	0.0207478717299931\\
178	0.0236309313514829\\
179	0.0234580329655271\\
180	0.021049627228871\\
181	0.0184816188723577\\
182	0.0236662303722607\\
183	0.0222337455022364\\
184	0.0256993907568842\\
185	0.0203310464168308\\
186	0.0258835203189179\\
187	0.020016720373758\\
188	0.0240306728153272\\
189	0.0229568522829869\\
190	0.0233453999223019\\
191	0.0276938479429201\\
192	0.0277825191036915\\
193	0.0230184581161349\\
194	0.0263889867159425\\
195	0.0253005212954955\\
196	0.0260988682322384\\
197	0.0300702231680592\\
198	0.0254723909256918\\
199	0.0260432154077976\\
200	0.0261230307748304\\
201	0.0264162549129624\\
202	0.0193381061081033\\
203	0.0229394622677526\\
204	0.0189131845166264\\
205	0.0262889021166067\\
206	0.0264485596375074\\
207	0.0203052875904694\\
208	0.026770069919362\\
209	0.0251456601870061\\
};
\addlegendentry{Pmax};

\end{axis}
\end{tikzpicture}%}
\end{minipage}
\hfill
\begin{minipage}{0.48\textwidth}
\tikzsetnextfilename{Release_Stat}
\resizebox{\textwidth}{!}{
% This file was created by matlab2tikz.
%
%The latest updates can be retrieved from
%  http://www.mathworks.com/matlabcentral/fileexchange/22022-matlab2tikz-matlab2tikz
%where you can also make suggestions and rate matlab2tikz.
%
\definecolor{mycolor1}{rgb}{0.00000,0.44700,0.74100}%
\definecolor{mycolor2}{rgb}{0.85000,0.32500,0.09800}%
%
\begin{tikzpicture}

\begin{axis}[%
width=\textwidth,
height=0.66\textwidth,
at={(0.758in,0.481in)},
scale only axis,
x tick label style={/pgf/number format/fixed},
xmin=0,
xmax=0.16,
xlabel={Energy [J]},
ymin=0,
ymax=70,
ylabel={PDF},
axis background/.style={fill=white},
title style={font=\bfseries},
title={Statistical Overview},
legend style={legend cell align=left,align=left,draw=white!15!black}
]
\addplot[fill=mycolor1,fill opacity=0.6,draw=black,ybar interval,area legend] plot table[row sep=crcr] {%
x	y\\
0.005	1.91387559808612\\
0.01	2.87081339712919\\
0.015	8.61244019138756\\
0.02	64.1148325358852\\
0.025	31.578947368421\\
0.03	28.7081339712919\\
0.035	12.4401913875598\\
0.04	32.5358851674641\\
0.045	12.4401913875598\\
0.05	0\\
0.055	2.87081339712919\\
0.06	0\\
0.065	0\\
0.07	0\\
0.075	0\\
0.08	0\\
0.085	0\\
0.09	0\\
0.095	0\\
0.1	0\\
0.105	0.956937799043064\\
0.11	0\\
0.115	0\\
0.12	0\\
0.125	0\\
0.13	0\\
0.135	0\\
0.14	0\\
0.145	0\\
0.15	0.956937799043061\\
0.155	0.956937799043061\\
};
\addlegendentry{Data points};

\addplot [color=mycolor2,solid]
  table[row sep=crcr]{%
0.0001	1.37108217944147e-51\\
0.00035	1.37472822791907e-30\\
0.0006	3.39727921670846e-23\\
0.00085	6.06496048131467e-19\\
0.0011	4.61587269206167e-16\\
0.00135	6.19185336810283e-14\\
0.0016	2.80918534740367e-12\\
0.00185	6.11649355562522e-11\\
0.0021	7.86955320774355e-10\\
0.00235	6.83423698345826e-09\\
0.0026	4.38237696110035e-08\\
0.00285	2.21032804310277e-07\\
0.0031	9.18162622704505e-07\\
0.00335	3.25135684325233e-06\\
0.0036	1.00776806224247e-05\\
0.00385	2.7910697249072e-05\\
0.0041	7.02133001625104e-05\\
0.00435	0.00016257486147984\\
0.0046	0.000350241956438618\\
0.00485	0.000708343957052096\\
0.0051	0.0013549432364362\\
0.00535	0.00246676760987226\\
0.0056	0.00429718408330676\\
0.00585	0.00719568614779679\\
0.0061	0.011627927124476\\
0.00635	0.0181951667958628\\
0.0066	0.0276519241410247\\
0.00685	0.0409206501023571\\
0.0071	0.0591023450209066\\
0.00735	0.0834822314524254\\
0.0076	0.115529834856582\\
0.00785	0.156893099958749\\
0.0081	0.209386457289973\\
0.00835	0.27497303244344\\
0.0086	0.355741443287687\\
0.00885	0.453877845242556\\
0.0091	0.571634053602845\\
0.00935	0.71129269077632\\
0.0096	0.875130374817354\\
0.00985	1.06537998638675\\
0.0101	1.2841930290886\\
0.01035	1.5336030393702\\
0.0106	1.81549091400655\\
0.01085	2.13155291308402\\
0.0111	2.48327197162275\\
0.01135	2.87189282027564\\
0.0116	3.29840128089377\\
0.01185	3.76350797122724\\
0.0121	4.26763652874284\\
0.01235	4.81091634963183\\
0.0126	5.39317973778175\\
0.01285	6.01396327119781\\
0.0131	6.6725131207567\\
0.01335	7.36779399831973\\
0.0136	8.09850136768769\\
0.01385	8.86307652182681\\
0.0141	9.65972411212754\\
0.01435	10.4864317088827\\
0.0146	11.3409909752875\\
0.01485	12.2210200486157\\
0.0151	13.1239867403889\\
0.01535	14.0472321909336\\
0.0156	14.9879946414284\\
0.01585	15.9434330171833\\
0.0161	16.9106500483982\\
0.01635	17.8867146880856\\
0.0166	18.8686836203932\\
0.01685	19.853621685559\\
0.0171	20.8386210796086\\
0.01735	21.8208192172205\\
0.0176	22.7974151746057\\
0.01785	23.7656846555218\\
0.0181	24.7229934475171\\
0.01835	25.6668093570726\\
0.0186	26.5947126314675\\
0.01885	27.5044048919438\\
0.0191	28.3937166171642\\
0.01935	29.260613228124\\
0.0196	30.1031998357385\\
0.01985	30.9197247203927\\
0.0201	31.70858161899\\
0.02035	32.4683108996065\\
0.0206	33.1975997069295\\
0.02085	33.8952811633813\\
0.0211	34.560332711366\\
0.02135	35.1918736815829\\
0.0216	35.7891621709673\\
0.02185	36.3515913116822\\
0.0221	36.8786850098325\\
0.02235	37.3700932293039\\
0.0226	37.8255868924721\\
0.02285	38.245052465557\\
0.0231	38.6284862922226\\
0.02335	38.9759887346972\\
0.0236	39.2877581773012\\
0.02385	39.5640849428634\\
0.0241	39.8053451681374\\
0.02435	40.0119946800415\\
0.0246	40.1845629103707\\
0.02485	40.3236468825951\\
0.0251	40.4299053004965\\
0.02535	40.50405276471\\
0.0256	40.5468541397544\\
0.02585	40.5591190908559\\
0.0261	40.5416968068011\\
0.02635	40.4954709222038\\
0.0266	40.4213546499294\\
0.02685	40.3202861319906\\
0.0271	40.1932240150086\\
0.02735	40.0411432543111\\
0.0276	39.8650311489102\\
0.02785	39.6658836079636\\
0.0281	39.4447016478533\\
0.02835	39.2024881177217\\
0.0286	38.9402446501606\\
0.02885	38.6589688327592\\
0.0291	38.3596515953568\\
0.02935	38.0432748071253\\
0.0296	37.7108090769892\\
0.02985	37.3632117503969\\
0.0301	37.0014250950508\\
0.03035	36.6263746678945\\
0.0306	36.2389678554274\\
0.03085	35.840092579259\\
0.0311	35.4306161587304\\
0.03135	35.0113843223972\\
0.0316	34.5832203601937\\
0.03185	34.1469244081661\\
0.0321	33.7032728577712\\
0.03235	33.2530178818823\\
0.0326	32.796887069817\\
0.03285	32.3355831639014\\
0.0331	31.8697838903038\\
0.03335	31.4001418771106\\
0.0336	30.9272846528665\\
0.03385	30.451814719062\\
0.0341	29.9743096903228\\
0.03435	29.4953224963279\\
0.0346	29.0153816397615\\
0.03485	28.5349915048822\\
0.0351	28.0546327115702\\
0.03535	27.5747625099887\\
0.0356	27.0958152112661\\
0.03585	26.6182026498743\\
0.0361	26.142314673636\\
0.03635	25.6685196575517\\
0.0366	25.197165037881\\
0.03685	24.7285778631528\\
0.0371	24.2630653590098\\
0.03735	23.8009155040147\\
0.0376	23.3423976137581\\
0.03785	22.8877629308126\\
0.0381	22.4372452182718\\
0.03835	21.9910613548004\\
0.0386	21.5494119292961\\
0.03885	21.112481833433\\
0.0391	20.6804408505147\\
0.03935	20.2534442392155\\
0.0396	19.8316333109288\\
0.03985	19.4151359995753\\
0.0401	19.0040674228499\\
0.04035	18.5985304340013\\
0.0406	18.1986161633493\\
0.04085	17.8044045488469\\
0.0411	17.4159648550904\\
0.04135	17.0333561802674\\
0.0416	16.6566279506177\\
0.04185	16.2858204020553\\
0.0421	15.9209650486731\\
0.04235	15.5620851379141\\
0.0426	15.2091960922538\\
0.04285	14.8623059372931\\
0.0431	14.521415716211\\
0.04335	14.1865198905703\\
0.0436	13.8576067275133\\
0.04385	13.5346586734205\\
0.0441	13.2176527141387\\
0.04435	12.9065607219147\\
0.0446	12.6013497892015\\
0.04485	12.3019825495215\\
0.0451	12.0084174855985\\
0.04535	11.7206092249855\\
0.0456	11.4385088234317\\
0.04585	11.1620640362477\\
0.0461	10.8912195779367\\
0.04635	10.6259173703739\\
0.0466	10.3660967798185\\
0.04685	10.1116948430548\\
0.0471	9.86264648296051\\
0.04735	9.61888471380462\\
0.0476	9.38034083658001\\
0.04785	9.14694462467723\\
0.0481	8.9186245002052\\
0.04835	8.69530770126479\\
0.0486	8.4769204404791\\
0.04885	8.26338805508171\\
0.0491	8.05463514886186\\
0.04935	7.85058572626103\\
0.0496	7.65116331891161\\
0.04985	7.45629110490373\\
0.0501	7.26589202106127\\
0.05035	7.07988886850205\\
0.0506	6.89820441175257\\
0.05085	6.7207614716805\\
0.0511	6.54748301250275\\
0.05135	6.37829222312034\\
0.0516	6.21311259302428\\
0.05185	6.05186798301107\\
0.0521	5.89448269093861\\
0.05235	5.74088151274734\\
0.0526	5.59098979896443\\
0.05285	5.44473350690175\\
0.0531	5.30203924875232\\
0.05335	5.16283433578238\\
0.0536	5.02704681881025\\
0.05385	4.89460552515616\\
0.0541	4.76544009224092\\
0.05435	4.63948099800482\\
0.0546	4.51665958831184\\
0.05485	4.3969081014981\\
0.0551	4.28015969021738\\
0.05535	4.1663484407306\\
0.0556	4.05540938978041\\
0.05585	3.94727853918616\\
0.0561	3.84189286828933\\
0.05635	3.73919034437373\\
0.0566	3.63910993117989\\
0.05685	3.54159159562758\\
0.0571	3.44657631285599\\
0.05735	3.35400606968562\\
0.0576	3.26382386660189\\
0.05785	3.17597371835564\\
0.0581	3.09040065327141\\
0.05835	3.00705071135016\\
0.0586	2.92587094124913\\
0.05885	2.84680939621739\\
0.0591	2.76981512906209\\
0.05935	2.69483818621666\\
0.0596	2.62182960097851\\
0.05985	2.55074138598086\\
0.0601	2.48152652495949\\
0.06035	2.41413896387255\\
0.0606	2.34853360142839\\
0.06085	2.28466627907329\\
0.0611	2.22249377048857\\
0.06135	2.16197377064362\\
0.0616	2.10306488444897\\
0.06185	2.04572661505097\\
0.0621	1.98991935180754\\
0.06235	1.93560435798192\\
0.0626	1.88274375818944\\
0.06285	1.83130052563018\\
0.0631	1.78123846913851\\
0.06335	1.73252222007851\\
0.0636	1.6851172191127\\
0.06385	1.63898970286959\\
0.0641	1.59410669053406\\
0.06435	1.5504359703831\\
0.0646	1.50794608628786\\
0.06485	1.46660632420158\\
0.0651	1.42638669865168\\
0.06535	1.38725793925306\\
0.0656	1.34919147725842\\
0.06585	1.3121594321602\\
0.0661	1.27613459835779\\
0.06635	1.24109043190262\\
0.0666	1.20700103733257\\
0.06685	1.17384115460662\\
0.0671	1.14158614614925\\
0.06735	1.11021198401384\\
0.0676	1.07969523717308\\
0.06785	1.05001305894398\\
0.0681	1.02114317455411\\
0.06835	0.993063868855351\\
0.0686	0.965753974190449\\
0.06885	0.939192858417471\\
0.0691	0.913360413096351\\
0.06935	0.888237041841482\\
0.0696	0.863803648843616\\
0.06985	0.840041627564043\\
0.0701	0.816932849603437\\
0.07035	0.794459653747493\\
0.0706	0.772604835190984\\
0.07085	0.751351634941621\\
0.0711	0.730683729404649\\
0.07135	0.710585220148953\\
0.0716	0.691040623854999\\
0.07185	0.672034862444805\\
0.0721	0.653553253393787\\
0.07235	0.635581500224182\\
0.0726	0.618105683179419\\
0.07285	0.60111225007877\\
0.0731	0.584588007351266\\
0.07335	0.568520111247799\\
0.0736	0.552896059230146\\
0.07385	0.537703681535494\\
0.0741	0.522931132914929\\
0.07435	0.508566884544245\\
0.0746	0.494599716105258\\
0.07485	0.481018708035831\\
0.0751	0.467813233946568\\
0.07535	0.454972953202212\\
0.0756	0.442487803665588\\
0.07585	0.430347994601941\\
0.0761	0.418543999741438\\
0.07635	0.407066550497552\\
0.0766	0.395906629338987\\
0.07685	0.385055463312834\\
0.0771	0.374504517716504\\
0.07735	0.364245489916084\\
0.0776	0.354270303308632\\
0.07785	0.344571101425996\\
0.0781	0.335140242177673\\
0.07835	0.325970292230248\\
0.0786	0.317054021520934\\
0.07885	0.308384397902752\\
0.0791	0.299954581918867\\
0.07935	0.291757921703637\\
0.0796	0.283787948007901\\
0.07985	0.276038369346082\\
0.0801	0.268503067262676\\
0.08035	0.261176091715708\\
0.0806	0.254051656574779\\
0.08085	0.247124135231323\\
0.0811	0.240388056318735\\
0.08135	0.233838099540034\\
0.0816	0.227469091600784\\
0.08185	0.221276002244971\\
0.0821	0.215253940391623\\
0.08235	0.209398150369937\\
0.0826	0.203704008250734\\
0.08285	0.198167018272082\\
0.0831	0.192782809356978\\
0.08335	0.187547131720967\\
0.0836	0.182455853567673\\
0.08385	0.177504957870182\\
0.0841	0.172690539236301\\
0.08435	0.168008800855735\\
0.0846	0.163456051527226\\
0.08485	0.1590287027638\\
0.0851	0.154723265974224\\
0.08535	0.150536349718875\\
0.0856	0.146464657038206\\
0.08585	0.14250498285206\\
0.0861	0.138654211428116\\
0.08635	0.134909313917751\\
0.0866	0.131267345957683\\
0.08685	0.127725445335754\\
0.0871	0.124280829719269\\
0.08735	0.12093079444433\\
0.0876	0.117672710364632\\
0.08785	0.114504021758237\\
0.0881	0.111422244290846\\
0.08835	0.10842496303416\\
0.0886	0.105509830537904\\
0.08885	0.10267456495416\\
0.0891	0.0999169482126714\\
0.08935	0.0972348242457939\\
0.0896	0.0946260972618255\\
0.08985	0.092088730065457\\
0.0901	0.0896207424241262\\
0.09035	0.0872202094790763\\
0.0906	0.0848852601999511\\
0.09085	0.082614075881793\\
0.0911	0.080404888683323\\
0.09135	0.078255980205423\\
0.0916	0.0761656801087561\\
0.09185	0.0741323647694889\\
0.0921	0.0721544559721096\\
0.09235	0.0702304196383515\\
0.0926	0.0683587645912603\\
0.09285	0.0665380413534721\\
0.0931	0.0647668409787812\\
0.09335	0.0630437939161057\\
0.0936	0.061367568904983\\
0.09385	0.0597368719017473\\
0.0941	0.0581504450355547\\
0.09435	0.0566070655934595\\
0.0946	0.0551055450337461\\
0.09485	0.0536447280267563\\
0.0951	0.0522234915224625\\
0.09535	0.05084074384406\\
0.0956	0.0494954238068682\\
0.09585	0.0481864998618511\\
0.0961	0.046912969263086\\
0.09635	0.0456738572585191\\
0.0966	0.0444682163033759\\
0.09685	0.0432951252956003\\
0.0971	0.0421536888327204\\
0.09735	0.0410430364895469\\
0.0976	0.0399623221161339\\
0.09785	0.0389107231554419\\
0.0981	0.037887439980157\\
0.09835	0.0368916952481396\\
0.0986	0.0359227332759833\\
0.09885	0.0349798194301868\\
0.0991	0.0340622395354435\\
0.09935	0.0331692992995808\\
0.0996	0.0323003237546804\\
0.09985	0.0314546567139327\\
0.1001	0.030631660243785\\
0.10035	0.029830714150959\\
0.1006	0.0290512154839217\\
0.10085	0.0282925780484063\\
0.1011	0.0275542319365897\\
0.10135	0.0268356230695469\\
0.1016	0.0261362127526076\\
0.10185	0.0254554772432569\\
0.1021	0.0247929073312253\\
0.10235	0.0241480079304287\\
0.1026	0.0235202976824251\\
0.10285	0.0229093085710625\\
0.1031	0.0223145855480078\\
0.10335	0.0217356861688463\\
0.1036	0.0211721802394577\\
0.10385	0.020623649472377\\
0.1041	0.0200896871528616\\
0.10435	0.019569897814388\\
0.1046	0.0190638969233151\\
0.10485	0.0185713105724546\\
0.1051	0.0180917751832963\\
0.10535	0.0176249372166451\\
0.1056	0.0171704528914324\\
0.10585	0.0167279879114691\\
0.1061	0.0162972171999188\\
0.10635	0.01587782464127\\
0.1066	0.015469502830597\\
0.10685	0.0150719528299026\\
0.1071	0.0146848839313431\\
0.10735	0.0143080134271396\\
0.1076	0.0139410663859865\\
0.10785	0.0135837754357749\\
0.1081	0.0132358805524496\\
0.10835	0.0128971288548271\\
0.1086	0.0125672744052065\\
0.10885	0.0122460780156074\\
0.1091	0.0119333070594766\\
0.10935	0.0116287352887078\\
0.1096	0.0113321426558246\\
0.10985	0.0110433151411786\\
0.1101	0.0107620445850225\\
0.11035	0.0104881285243185\\
0.1106	0.0102213700341482\\
0.11085	0.00996157757359258\\
0.1111	0.00970856483595715\\
0.11135	0.00946215060321664\\
0.1116	0.0092221586045614\\
0.11185	0.00898841737892766\\
0.1121	0.0087607601413998\\
0.11235	0.00853902465337387\\
0.1126	0.00832305309637593\\
0.11285	0.00811269194943162\\
0.1131	0.00790779186988593\\
0.11335	0.0077082075775754\\
0.1136	0.00751379774225754\\
0.11385	0.00732442487420513\\
0.1141	0.00713995521787533\\
0.11435	0.00696025864856701\\
0.1146	0.00678520857198044\\
0.11485	0.00661468182659817\\
0.1151	0.00644855858880613\\
0.11535	0.00628672228067793\\
0.1156	0.00612905948034619\\
0.11585	0.00597545983488819\\
0.1161	0.0058258159756539\\
0.11635	0.00568002343596763\\
0.1166	0.00553798057113578\\
0.11685	0.00539958848069543\\
0.1171	0.00526475093284012\\
0.11735	0.0051333742909614\\
0.1176	0.00500536744224573\\
0.11785	0.00488064172826927\\
0.1181	0.00475911087753306\\
0.11835	0.0046406909398844\\
0.1186	0.00452530022277073\\
0.11885	0.0044128592292741\\
0.1191	0.00430329059787586\\
0.11935	0.00419651904390284\\
0.1196	0.00409247130260713\\
0.11985	0.00399107607383352\\
0.1201	0.00389226396822958\\
0.12035	0.00379596745495484\\
0.1206	0.00370212081084661\\
0.12085	0.00361066007100154\\
0.1211	0.00352152298073254\\
0.12135	0.00343464894886271\\
0.1216	0.00334997900231796\\
0.12185	0.00326745574198228\\
0.1221	0.00318702329977964\\
0.12235	0.00310862729694793\\
0.1226	0.00303221480347165\\
0.12285	0.00295773429864026\\
0.1231	0.00288513563270079\\
0.12335	0.00281436998957386\\
0.1236	0.00274538985060287\\
0.12385	0.00267814895930765\\
0.1241	0.00261260228711419\\
0.12435	0.00254870600003268\\
0.1246	0.00248641742625779\\
0.12485	0.00242569502466466\\
0.1251	0.00236649835417584\\
0.12535	0.00230878804397449\\
0.1256	0.0022525257645402\\
0.12585	0.00219767419948418\\
0.1261	0.00214419701816167\\
0.12635	0.00209205884903941\\
0.1266	0.00204122525379723\\
0.12685	0.00199166270214319\\
0.1271	0.00194333854732204\\
0.12735	0.00189622100229798\\
0.1276	0.00185027911659239\\
0.12785	0.00180548275375855\\
0.1281	0.00176180256947528\\
0.12835	0.00171920999024245\\
0.1286	0.00167767719266119\\
0.12885	0.00163717708328289\\
0.1291	0.00159768327901066\\
0.12935	0.00155917008803814\\
0.1296	0.00152161249131053\\
0.12985	0.00148498612449335\\
0.1301	0.0014492672604346\\
0.13035	0.00141443279210682\\
0.1306	0.00138046021601542\\
0.13085	0.00134732761606056\\
0.1311	0.00131501364783964\\
0.13135	0.00128349752337853\\
0.1316	0.00125275899627913\\
0.13185	0.00122277834727221\\
0.1321	0.00119353637016388\\
0.13235	0.00116501435816486\\
0.1326	0.00113719409059207\\
0.13285	0.00111005781993195\\
0.1331	0.00108358825925577\\
0.13335	0.00105776856997682\\
0.1336	0.00103258234994043\\
0.13385	0.00100801362183718\\
0.1341	0.000984046821930637\\
0.13435	0.000960666789090836\\
0.1346	0.00093785875412498\\
0.13485	0.000915608329397264\\
0.1351	0.00089390149872969\\
0.13535	0.00087272460757626\\
0.1356	0.000852064353462912\\
0.13585	0.000831907776685858\\
0.1361	0.000812242251261251\\
0.13635	0.000793055476119227\\
0.1366	0.000774335466535584\\
0.13685	0.000756070545794541\\
0.1371	0.000738249337076278\\
0.13735	0.000720860755562995\\
0.1376	0.000703894000757513\\
0.13785	0.000687338549008618\\
0.1381	0.000671184146237401\\
0.13835	0.000655420800859095\\
0.1386	0.000640038776895103\\
0.13885	0.000625028587269907\\
0.1391	0.000610380987287858\\
0.13935	0.000596086968284894\\
0.1396	0.000582137751450406\\
0.13985	0.000568524781814576\\
0.1401	0.000555239722396686\\
0.14035	0.000542274448509983\\
0.1406	0.000529621042218813\\
0.14085	0.000517271786943893\\
0.1411	0.000505219162211645\\
0.14135	0.000493455838543697\\
0.1416	0.000481974672482693\\
0.14185	0.000470768701750711\\
0.1421	0.000459831140536701\\
0.14235	0.000449155374909401\\
0.1426	0.000438734958352308\\
0.14285	0.000428563607417435\\
0.1431	0.000418635197494564\\
0.14335	0.000408943758692913\\
0.1436	0.000399483471832116\\
0.14385	0.000390248664539596\\
0.1441	0.000381233807451404\\
0.14435	0.000372433510513757\\
0.1446	0.000363842519382506\\
0.14485	0.00035545571191794\\
0.1451	0.000347268094772277\\
0.14535	0.000339274800067396\\
0.1456	0.000331471082160335\\
0.14585	0.000323852314494204\\
0.1461	0.000316413986532181\\
0.14635	0.000309151700772385\\
0.1466	0.000302061169841425\\
0.14685	0.00029513821366449\\
0.1471	0.000288378756709956\\
0.14735	0.000281778825306464\\
0.1476	0.00027533454503055\\
0.14785	0.000269042138162916\\
0.1481	0.00026289792121149\\
0.14835	0.000256898302499503\\
0.1486	0.000251039779816812\\
0.14885	0.000245318938132795\\
0.1491	0.00023973244736915\\
0.14935	0.000234277060231005\\
0.1496	0.000228949610094769\\
0.14985	0.000223747008951208\\
0.1501	0.000218666245402272\\
0.15035	0.000213704382710228\\
0.1506	0.000208858556897714\\
0.15085	0.00020412597489734\\
0.1511	0.000199503912749522\\
0.15135	0.000194989713847265\\
0.1516	0.000190580787226631\\
0.15185	0.000186274605901697\\
0.1521	0.000182068705242785\\
0.15235	0.000177960681396856\\
0.1526	0.0001739481897489\\
0.15285	0.000170028943423273\\
0.1531	0.000166200711823896\\
0.15335	0.000162461319212284\\
0.1536	0.000158808643322421\\
0.15385	0.000155240614011484\\
0.1541	0.000151755211945469\\
0.15435	0.000148350467318811\\
0.1546	0.000145024458607071\\
0.15485	0.000141775311351846\\
0.1551	0.00013860119697702\\
0.15535	0.000135500331635554\\
0.1556	0.000132470975085992\\
0.15585	0.000129511429597904\\
0.1561	0.000126620038885505\\
0.15635	0.0001237951870687\\
0.1566	0.000121035297660851\\
0.15685	0.000118338832582532\\
0.1571	0.000115704291200618\\
0.15735	0.000113130209392027\\
0.1576	0.000110615158631468\\
0.15785	0.000108157745102567\\
0.1581	0.000105756608831759\\
0.15835	0.000103410422844344\\
0.1586	0.000101117892342131\\
0.15885	9.88777539021037e-05\\
0.1591	9.66887746955478e-05\\
0.15935	9.45497517271256e-05\\
0.1596	9.24595110933533e-05\\
0.15985	9.04169072599892e-05\\
};
\addlegendentry{Approx.};

\end{axis}
\end{tikzpicture}%}
\end{minipage}
\caption{Energy spent during release sequence.}
\label{fig:Release_Points}
\end{figure}

The parameters can be modelled in several different ways, it is chosen to use a point distribution for both the $E_{sync}$ and $T_{attach}$, a normal distribution for $P_{attach}$ and a lognormal distribution for $E_{release}$ based on the the statistical properties of the parameters. 

It can also be seen from this that the parameters are uninfluenced by CP format, frequency, operation mode and $P_{TX}$ except in the case of $P_{attach}$ where $P_{TX}$ influences the mean value. The parameters can therefore be described as follows:

\begin{align}
E_{sync} &\sim\begin{cases} 1.05\cdot 10^{-6} \quad p = 0.758\\
1.76\cdot 10^{-6} \quad p = 0.242
\end{cases}\\
P_{attach} &\sim\mathcal{N}(0.269,0.173\cdot 10^{-6}) \\
T_{attach} &\sim\begin{cases} 2.215 \quad p = 0.2584\\
2.345 \quad p = 0.4382\\
2.470 \quad p = 0.1124\\
2.615 \quad p = 0.1573\\
2.725 \quad p = 0.0337
\end{cases}\\
E_{release} &\sim \text{Lognormal}(-3.5287,0.1277)
\end{align}

As mentioned the parameter $P_{TX}$ influences the mean value of the power consumption during the attach procedure, it is therefore looked upon separately in \autoref{fig:Attach_Pmax}. 

\begin{figure}[H]
\centering
\tikzsetnextfilename{Attach_Pmax}
\resizebox{0.7\textwidth}{!}{
% This file was created by matlab2tikz.
%
%The latest updates can be retrieved from
%  http://www.mathworks.com/matlabcentral/fileexchange/22022-matlab2tikz-matlab2tikz
%where you can also make suggestions and rate matlab2tikz.
%
\definecolor{mycolor1}{rgb}{0.00000,0.44700,0.74100}%
\definecolor{mycolor2}{rgb}{0.85000,0.32500,0.09800}%
\definecolor{mycolor3}{rgb}{0.92900,0.69400,0.12500}%
%
\begin{tikzpicture}

\begin{axis}[%
width=12.4in,
height=6.357in,
at={(2.08in,0.858in)},
scale only axis,
xmin=-30,
xmax=30,
xlabel={Pmax [dBm]},
ymin=0.12,
ymax=0.28,
ylabel={Avg. Power Consumption [W]},
axis background/.style={fill=white},
title style={font=\bfseries},
title={Attach Procedure},
legend style={at={(0.03,0.97)},anchor=north west,legend cell align=left,align=left,draw=white!15!black}
]
\addplot [color=mycolor1,only marks,mark=*,mark options={solid}]
  table[row sep=crcr]{%
10	0.151047313371487\\
11	0.154761231210888\\
12	0.156907209211033\\
13	0.160684296068248\\
14	0.162507067874023\\
15	0.168046628044659\\
16	0.167349824554413\\
17	0.171283150343803\\
18	0.180161933269978\\
19	0.189439361249306\\
20	0.190898508743685\\
21	0.201872122219901\\
22	0.219567639872928\\
23	0.272516970950456\\
9	0.150974118861972\\
-1	0.130660121937841\\
-10	0.130575324810667\\
-11	0.132150467709027\\
-12	0.130605932962615\\
-13	0.129078875349553\\
-14	0.130599994705029\\
-15	0.129996957966701\\
-16	0.131599561535464\\
-17	0.130325487903084\\
-18	0.129413615846293\\
-19	0.13048045065209\\
-2	0.133708928603574\\
-20	0.127151660592556\\
-21	0.129485852333684\\
-22	0.13064634706093\\
-23	0.129580782637227\\
-24	0.129435597322756\\
-25	0.129270377248005\\
-26	0.125087904466061\\
-27	0.125727964840569\\
-28	0.125627487672513\\
-29	0.12924531898976\\
-3	0.131838548678704\\
-30	0.12863679811696\\
-4	0.132531210544604\\
-5	0.133192324600549\\
-6	0.131820841838912\\
-7	0.132364387132469\\
-8	0.130477571591827\\
-9	0.13151969584952\\
0	0.135203216563572\\
1	0.136439430941167\\
2	0.136422353003321\\
3	0.135367589970859\\
4	0.136472125157741\\
5	0.138547714791761\\
6	0.140637304886408\\
7	0.14273115545425\\
8	0.142493961491374\\
};
\addlegendentry{Data Points};

\addplot [color=mycolor2,solid]
  table[row sep=crcr]{%
-30	0.127404890355152\\
-29	0.127581064519652\\
-28	0.127757585925784\\
-27	0.127934481801407\\
-26	0.128111786279127\\
-25	0.128289542168819\\
-24	0.128467803185214\\
-23	0.128646636747371\\
-22	0.128826127496846\\
-21	0.129006381719082\\
-20	0.129187532899891\\
-19	0.12936974870845\\
-18	0.129553239773022\\
-17	0.129738270709663\\
-16	0.129925173982303\\
-15	0.130114367321117\\
-14	0.130306375612679\\
-13	0.130501858409943\\
-12	0.130701644504829\\
-11	0.130906775376563\\
-10	0.131118559794461\\
-9	0.13133864243879\\
-8	0.131569090138571\\
-7	0.131812500249081\\
-6	0.132072136852947\\
-5	0.13235210192795\\
-4	0.132657550458482\\
-3	0.132994960772249\\
-2	0.13337247428011\\
-1	0.133800322436827\\
0	0.134291363314813\\
1	0.134861755931651\\
2	0.135531807696734\\
3	0.136327039421605\\
4	0.137279523748831\\
5	0.13842956719373\\
6	0.139827824014172\\
7	0.141537952771039\\
8	0.143639954903506\\
};
\addlegendentry{y = 0.1328*exp(0.0014*x) + 1.51E-03*exp(0.229*x) RMSE = 0.00136};

\addplot [color=mycolor3,solid]
  table[row sep=crcr]
\caption{Average power consumption as a function of $P_{TX}$ during the attach procedure.}
\label{fig:Attach_Pmax}
\end{figure}

From \autoref{fig:Attach_Pmax} it can be seen that the power consumption can be approximated using two functions based on the value of Pmax this can be contributed to the usage of a power amplifier. 



\section{Transmit Power Consumption}

To measure the transmit power consumption the UXM needs to enable the MAC padding this is done in both UL and DL to ensure the maximum utilization of the device. 

\subsection{Test Procedure}
\begin{enumerate}
\item Setup the \gls{DUT} as shown on \autoref{fig:IPE_test_setup}
\item Turn on power supply 
\item Put in settings as described in \autoref{tab:setup_parameters} 
\item Enable MAC padding in UL and DL
\item Input chosen value of chosen parameter
\item Input to log "start <Parameter used> <Parameter value>"
\item Put device in connected state
\item Measure power output over 35 s
\item Change to next value
\item Repeat step 5-9 for all values.
\item Save measurements as "Transmit/<Parameter used>/Messagelog.csv"
\item Change to next parameter
\item Repeat step 5-12 for all parameters.
\item Turn off power supply
\end{enumerate}

\subsection{Results}
\begin{figure}[H]
\centering
\tikzsetnextfilename{Transmit_raw}
\resizebox{0.6\textwidth}{!}{
% This file was created by matlab2tikz.
%
%The latest updates can be retrieved from
%  http://www.mathworks.com/matlabcentral/fileexchange/22022-matlab2tikz-matlab2tikz
%where you can also make suggestions and rate matlab2tikz.
%
\definecolor{mycolor1}{rgb}{0.00000,0.44700,0.74100}%
%
\begin{tikzpicture}

\begin{axis}[%
width=\textwidth,
height=0.66\textwidth,
at={(0.758in,0.481in)},
scale only axis,
xmin=0,
xmax=36,
ymin=0,
ymax=0.8,
xlabel={Time [s]},
ylabel={Power Consumption [W]},
axis background/.style={fill=white}
]
\addplot [color=mycolor1,only marks,mark=*,mark options={solid},forget plot]
  table[row sep=crcr]{%
0	0.719055\\
0.03072	0.70951068\\
0.06144	0.70463556\\
0.09216	0.0247879044\\
0.12288	0.0235748268\\
0.1536	0.0277404768\\
0.18432	0.7072218\\
0.21504	0.703926\\
0.24576	0.0277633656\\
0.27648	0.048843396\\
0.3072	0.72319788\\
0.33792	0.7089156\\
0.36864	0.023300172\\
0.39936	0.0246963492\\
0.43008	0.72358704\\
0.4608	0.72303768\\
0.49152	0.023345946\\
0.52224	0.70211772\\
0.55296	0.70761096\\
0.58368	0.72496044\\
0.6144	0.0274887072\\
0.64512	0.7139052\\
0.67584	0.70339968\\
0.70656	0.70806888\\
0.73728	0.256484988\\
0.768	0.72571572\\
0.79872	0.7208406\\
0.82944	0.7034454\\
0.86016	0.0271911636\\
0.89088	0.0232543944\\
0.9216	0.73672488\\
0.95232	0.71871192\\
0.98304	0.70399476\\
1.01376	0.0230483988\\
1.04448	0.02835846\\
1.0752	0.72752364\\
1.10592	0.7196274\\
1.13664	0.0232086168\\
1.16736	0.024765012\\
1.19808	0.7135848\\
1.2288	0.72756972\\
1.25952	0.22613526\\
1.29024	0.70930476\\
1.32096	0.70044696\\
1.35168	0.7126236\\
1.3824	0.0271911636\\
1.41312	0.72452556\\
1.44384	0.71186832\\
1.47456	0.70182036\\
1.50528	0.0243759168\\
1.536	0.72241956\\
1.56672	0.72956088\\
1.59744	0.71424864\\
1.62816	0.024765012\\
1.65888	0.0228652956\\
1.6896	0.0276260364\\
1.72032	0.72624204\\
1.75104	0.71610264\\
1.78176	0.0317001348\\
1.81248	0.073722852\\
1.8432	0.72035964\\
1.87392	0.72489168\\
1.90464	0.0232315056\\
1.93536	0.024879456\\
1.96608	0.7016832\\
1.9968	0.71997084\\
2.02752	0.0234146124\\
2.05824	0.7214814\\
2.08896	0.70598592\\
2.11968	0.70236972\\
2.1504	0.030670164\\
2.18112	0.72752364\\
2.21184	0.72230544\\
2.24256	0.70877844\\
2.27328	0.254150388\\
2.304	0.70992288\\
2.33472	0.73228464\\
2.36544	0.72468576\\
2.39616	0.0248565672\\
2.42688	0.0234603864\\
2.4576	0.71939844\\
2.48832	0.72473148\\
2.51904	0.72486864\\
2.54976	0.0230712876\\
2.58048	0.075759876\\
2.6112	0.70783992\\
2.64192	0.72049716\\
2.67264	0.0236206044\\
2.70336	0.024398802\\
2.73408	0.70298784\\
2.7648	0.7059402\\
2.79552	0.152114868\\
2.82624	0.72740952\\
2.85696	0.7179336\\
2.88768	0.70479576\\
2.9184	0.0279006948\\
2.94912	0.71951292\\
2.97984	0.727272\\
3.01056	0.72008532\\
3.04128	0.2121048\\
3.072	0.70040124\\
3.10272	0.72111528\\
3.13344	0.72745524\\
3.16416	0.0246963492\\
3.19488	0.150764472\\
3.2256	0.71168508\\
3.25632	0.7126236\\
3.28704	0.7249374\\
3.31776	0.0310363776\\
3.34848	0.202720644\\
3.3792	0.70133976\\
3.40992	0.7067412\\
3.44064	0.0234603864\\
3.47136	0.0250625628\\
3.50208	0.71395092\\
3.5328	0.70193484\\
3.56352	0.0231628392\\
3.59424	0.72280872\\
3.62496	0.72534924\\
3.65568	0.71582796\\
3.6864	0.0272140488\\
3.71712	0.70555104\\
3.74784	0.72054288\\
3.77856	0.72699732\\
3.80928	0.253898604\\
3.84	0.704727\\
3.87072	0.70909884\\
3.90144	0.71880336\\
3.93216	0.0241699212\\
3.96288	0.0236206044\\
3.9936	0.027671814\\
4.02432	0.70298784\\
4.05504	0.71768196\\
4.08576	0.0230941764\\
4.11648	0.0668106\\
4.1472	0.70822908\\
4.17792	0.70033248\\
4.20864	0.0234603864\\
4.23936	0.02455902\\
4.27008	0.72427356\\
4.3008	0.71099856\\
4.33152	0.200546244\\
4.36224	0.71024328\\
4.39296	0.72207648\\
4.42368	0.72463968\\
4.4544	0.0275344848\\
4.48512	0.70285032\\
4.51584	0.70820604\\
4.54656	0.72452556\\
4.57728	0.253372176\\
4.608	0.71699508\\
4.63872	0.70783992\\
4.66944	0.70564284\\
4.70016	0.0271224972\\
4.73088	0.023506164\\
4.7616	0.7263792\\
4.79232	0.70461288\\
4.82304	0.70202628\\
4.85376	0.0274200444\\
4.88448	0.0286560036\\
4.9152	0.72120672\\
4.94592	0.70662672\\
4.97664	0.023185728\\
5.00736	0.0247879044\\
5.03808	0.72702036\\
5.0688	0.72207648\\
5.09952	0.0231170652\\
5.13024	0.702072\\
5.16096	0.70925904\\
5.19168	0.72706608\\
5.2224	0.0276260364\\
5.25312	0.71257788\\
5.28384	0.70170588\\
5.31456	0.7122114\\
5.34528	0.0246734604\\
5.376	0.726219\\
5.40672	0.7179336\\
5.43744	0.702072\\
5.46816	0.0249481188\\
5.49888	0.0233230572\\
5.5296	0.0271224972\\
5.56032	0.71582796\\
5.59104	0.70214076\\
5.62176	0.023506164\\
5.65248	0.07308198\\
5.6832	0.72779832\\
5.71392	0.71804808\\
5.74464	0.0237579336\\
5.77536	0.0247879044\\
5.80608	0.7186662\\
5.8368	0.72674568\\
5.86752	0.2229309\\
5.89824	0.70664976\\
5.92896	0.7003098\\
5.95968	0.71710956\\
5.9904	0.0311050404\\
6.02112	0.72328932\\
6.05184	0.70854948\\
6.08256	0.70220952\\
6.11328	0.251975988\\
6.144	0.72560124\\
6.17472	0.72800424\\
6.20544	0.71056368\\
6.23616	0.0246963492\\
6.26688	0.0236434932\\
6.2976	0.73320012\\
6.32832	0.72553248\\
6.35904	0.71314992\\
6.38976	0.0273971556\\
6.42048	0.075187692\\
6.4512	0.72351828\\
6.48192	0.72468576\\
6.51264	0.0230712876\\
6.54336	0.0249481188\\
6.57408	0.7056198\\
6.6048	0.7210008\\
6.63552	0.0232315056\\
6.66624	0.71813952\\
6.69696	0.70378884\\
6.72768	0.70493328\\
6.7584	0.0275802588\\
6.78912	0.72770688\\
6.81984	0.71990208\\
6.85056	0.70644384\\
6.88128	0.207183816\\
6.912	0.714798\\
6.94272	0.73191816\\
6.97344	0.72251136\\
7.00416	0.025451658\\
7.03488	0.307983384\\
7.0656	0.72173304\\
7.09632	0.72720324\\
7.12704	0.7244568\\
7.15776	0.0230483988\\
7.18848	0.201484656\\
7.2192	0.71070084\\
7.24992	0.72095472\\
7.28064	0.0236434932\\
7.31136	0.0249481188\\
7.34208	0.70218648\\
7.3728	0.70774848\\
7.40352	0.21913146\\
7.43424	0.7269516\\
7.46496	0.71580492\\
7.49568	0.70319376\\
7.5264	0.0270996084\\
7.55712	0.72118368\\
7.58784	0.72528084\\
7.61856	0.71754444\\
7.64928	0.216545112\\
7.68	0.70259868\\
7.71072	0.72450252\\
7.74144	0.7275924\\
7.77216	0.0246734604\\
7.80288	0.0234603864\\
7.8336	0.027717588\\
7.86432	0.71630856\\
7.89504	0.7314606\\
7.92576	0.0276031512\\
7.95648	0.074478132\\
7.9872	0.70220952\\
8.01792	0.70941924\\
8.04864	0.0232086168\\
8.07936	0.0245819088\\
8.11008	0.7115706\\
8.1408	0.70161444\\
8.17152	0.0234603864\\
8.20224	0.72576144\\
8.23296	0.72567\\
8.26368	0.712944\\
8.2944	0.0274200444\\
8.32512	0.70758828\\
8.35584	0.72157284\\
8.38656	0.72656244\\
8.41728	0.24632262\\
8.448	0.70314768\\
8.47872	0.71143344\\
8.50944	0.72191628\\
8.54016	0.0274200444\\
8.57088	0.0236206044\\
8.6016	0.71049492\\
8.63232	0.70447536\\
8.66304	0.71864316\\
8.69376	0.0231628428\\
8.72448	0.0282211308\\
8.7552	0.70616916\\
8.78592	0.6998976\\
8.81664	0.023185728\\
8.84736	0.0244903536\\
8.87808	0.7222824\\
8.9088	0.70834356\\
8.93952	0.200180052\\
8.97024	0.71456904\\
9.00096	0.72502884\\
9.03168	0.7232436\\
9.0624	0.0272827152\\
9.09312	0.7017516\\
9.12384	0.70802316\\
9.15456	0.72544104\\
9.18528	0.024765012\\
9.216	0.71379072\\
9.24672	0.70687872\\
9.27744	0.70898436\\
9.30816	0.0249481188\\
9.33888	0.0230026248\\
9.3696	0.0277404768\\
9.40032	0.70289604\\
9.43104	0.70630632\\
9.46176	0.0312881472\\
9.49248	0.073333728\\
9.5232	0.71832276\\
9.55392	0.7036056\\
9.58464	0.0230941764\\
9.61536	0.0246047976\\
9.64608	0.72738648\\
9.6768	0.71953596\\
9.70752	0.0233230572\\
9.73824	0.70301052\\
9.76896	0.71477496\\
9.79968	0.72770688\\
9.8304	0.0311050404\\
9.86112	0.70864092\\
9.89184	0.70099632\\
9.92256	0.71392824\\
9.95328	0.242042544\\
9.984	0.72436536\\
10.01472	0.71523288\\
10.04544	0.702072\\
10.07616	0.0245819088\\
10.10688	0.0238037112\\
10.1376	0.73001844\\
10.16832	0.71285256\\
10.19904	0.70209504\\
10.22976	0.0232086168\\
10.26048	0.07278444\\
10.2912	0.72667692\\
10.32192	0.71589672\\
10.35264	0.0230483988\\
10.38336	0.0249023412\\
10.41408	0.72159588\\
10.4448	0.7255782\\
10.47552	0.156806928\\
10.50624	0.70388028\\
10.53696	0.70339968\\
10.56768	0.72077184\\
10.5984	0.0272598264\\
10.62912	0.721161\\
10.65984	0.70603164\\
10.69056	0.70319376\\
10.72128	0.219314592\\
10.752	0.72809604\\
10.78272	0.72596736\\
10.81344	0.70841232\\
10.84416	0.0245361312\\
10.87488	0.320503212\\
10.9056	0.7319412\\
10.93632	0.7239762\\
10.96704	0.7107696\\
10.99776	0.0309677112\\
11.02848	0.200592036\\
11.0592	0.72521208\\
11.08992	0.72447984\\
11.12064	0.0235519416\\
11.15136	0.0249938964\\
11.18208	0.70841232\\
11.2128	0.72093204\\
11.24352	0.0230255136\\
11.27424	0.71589672\\
11.30496	0.7029648\\
11.33568	0.70724484\\
11.3664	0.0276260364\\
11.39712	0.72798156\\
11.42784	0.71795664\\
11.45856	0.70442964\\
11.48928	0.209381112\\
11.52	0.71992476\\
11.55072	0.73125468\\
11.58144	0.71948988\\
11.61216	0.0243301392\\
11.64288	0.0233917236\\
11.6736	0.027717588\\
11.70432	0.72814176\\
11.73504	0.72708876\\
11.76576	0.0234603864\\
11.79648	0.073448172\\
11.8272	0.71468352\\
11.85792	0.72393048\\
11.88864	0.0231628428\\
11.91936	0.0242843616\\
11.95008	0.70147692\\
11.9808	0.70832052\\
12.01152	0.223319988\\
12.04224	0.7257384\\
12.07296	0.71376804\\
12.10368	0.70234668\\
12.1344	0.0272369376\\
12.16512	0.72351828\\
12.19584	0.72473148\\
12.22656	0.71443188\\
12.25728	0.252433764\\
12.288	0.70738236\\
12.31872	0.72669996\\
12.34944	0.72718056\\
12.38016	0.0279006948\\
12.41088	0.0235519416\\
12.4416	0.71026596\\
12.47232	0.72081756\\
12.50304	0.72665388\\
12.53376	0.027465822\\
12.56448	0.028564452\\
12.5952	0.70285032\\
12.62592	0.71443188\\
12.65664	0.0231170652\\
12.68736	0.0246047976\\
12.71808	0.70854948\\
12.7488	0.70044696\\
12.77952	0.0230712876\\
12.81024	0.72704304\\
12.84096	0.72374724\\
12.87168	0.71038044\\
12.9024	0.0279464724\\
12.93312	0.7109298\\
12.96384	0.72280872\\
12.99456	0.72452556\\
13.02528	0.0241012584\\
13.056	0.7026444\\
13.08672	0.71353908\\
13.11744	0.72530352\\
13.14816	0.025085448\\
13.17888	0.0230712876\\
13.2096	0.02755737\\
13.24032	0.706833\\
13.27104	0.72086328\\
13.30176	0.022819518\\
13.33248	0.067222584\\
13.3632	0.70397172\\
13.39392	0.7022322\\
13.42464	0.0234375012\\
13.45536	0.0250167852\\
13.48608	0.7208406\\
13.5168	0.70557408\\
13.54752	0.201965328\\
13.57824	0.71710956\\
13.60896	0.72674568\\
13.63968	0.7213896\\
13.6704	0.030670164\\
13.70112	0.70198056\\
13.73184	0.71017452\\
13.76256	0.72738648\\
13.79328	0.257766732\\
13.824	0.71179956\\
13.85472	0.70664976\\
13.88544	0.71216568\\
13.91616	0.0249252336\\
13.94688	0.0229339584\\
13.9776	0.71710956\\
14.00832	0.70225524\\
14.03904	0.708435\\
14.06976	0.0276489252\\
14.10048	0.07411194\\
14.1312	0.7152786\\
14.16192	0.70177464\\
14.19264	0.0230255136\\
14.22336	0.02455902\\
14.25408	0.72754668\\
14.2848	0.71747604\\
14.31552	0.0233230572\\
14.34624	0.70461288\\
14.37696	0.7202682\\
14.40768	0.72681444\\
14.4384	0.0277862544\\
14.46912	0.70623792\\
14.49984	0.70042428\\
14.53056	0.7172928\\
14.56128	0.235404972\\
14.592	0.72315216\\
14.62272	0.71280684\\
14.65344	0.70209504\\
14.68416	0.0242385876\\
14.71488	0.60896304\\
14.7456	0.72782136\\
14.77632	0.70946496\\
14.80704	0.70069896\\
14.83776	0.0234603864\\
14.86848	0.200431836\\
14.8992	0.72496044\\
14.92992	0.71255484\\
14.96064	0.0232086168\\
14.99136	0.0249481188\\
15.02208	0.72425088\\
15.0528	0.72431928\\
15.08352	0.215835552\\
15.11424	0.70250688\\
15.14496	0.70687872\\
15.17568	0.72168732\\
15.2064	0.0279235836\\
15.23712	0.7179336\\
15.26784	0.70362864\\
15.29856	0.70573428\\
15.32928	0.215560908\\
15.36	0.72809604\\
15.39072	0.7234038\\
15.42144	0.70523064\\
15.45216	0.0245361312\\
15.48288	0.0233917236\\
15.5136	0.0273284928\\
15.54432	0.72068004\\
15.57504	0.70777116\\
15.60576	0.0272827152\\
15.63648	0.057403548\\
15.6672	0.72738648\\
15.69792	0.72354132\\
15.72864	0.0232543944\\
15.75936	0.0247879044\\
15.79008	0.71168508\\
15.8208	0.72198468\\
15.85152	0.0232772832\\
15.88224	0.71179956\\
15.91296	0.70225524\\
15.94368	0.70829784\\
15.9744	0.0272369376\\
16.00512	0.72651672\\
16.03584	0.71472924\\
16.06656	0.70282728\\
16.09728	0.255477888\\
16.128	0.72184752\\
16.15872	0.73036188\\
16.18944	0.71690364\\
16.22016	0.028038024\\
16.25088	0.0234603864\\
16.2816	0.72821052\\
16.31232	0.72754668\\
16.34304	0.71990208\\
16.37376	0.0232315056\\
16.40448	0.0280838016\\
16.4352	0.71781912\\
16.46592	0.72720324\\
16.49664	0.0235519416\\
16.52736	0.0249252336\\
16.55808	0.70250688\\
16.5888	0.71065512\\
16.61952	0.225654588\\
16.65024	0.7239762\\
16.68096	0.71056368\\
16.71168	0.701271\\
16.7424	0.0273971556\\
16.77312	0.72594432\\
16.80384	0.72500616\\
16.83456	0.71191404\\
16.86528	0.0213775596\\
16.896	0.7090758\\
16.92672	0.7263792\\
16.95744	0.72596736\\
16.98816	0.0248107896\\
17.01888	0.0229568472\\
17.0496	0.0274429296\\
17.08032	0.72301464\\
17.11104	0.72720324\\
17.14176	0.0310821516\\
17.17248	0.069786072\\
17.2032	0.70486452\\
17.23392	0.71990208\\
17.26464	0.0229110696\\
17.29536	0.0249938964\\
17.32608	0.70600896\\
17.3568	0.70033248\\
17.38752	0.0232772832\\
17.41824	0.72830196\\
17.44896	0.7218018\\
17.47968	0.70724484\\
17.5104	0.0313796988\\
17.54112	0.71445456\\
17.57184	0.72505188\\
17.60256	0.72219096\\
17.63328	0.257926932\\
17.664	0.70198056\\
17.69472	0.71385948\\
17.72544	0.7258986\\
17.75616	0.0246505752\\
17.78688	0.0230941764\\
17.8176	0.70724484\\
17.84832	0.70944228\\
17.87904	0.72443376\\
17.90976	0.0230483988\\
17.94048	0.074226384\\
17.9712	0.70211772\\
18.00192	0.7066728\\
18.03264	0.0232086168\\
18.06336	0.0251541144\\
18.09408	0.71756748\\
18.1248	0.70291908\\
18.15552	0.155273436\\
18.18624	0.72051984\\
18.21696	0.7279128\\
18.24768	0.7183458\\
18.2784	0.0273742668\\
18.30912	0.70372008\\
18.33984	0.71530164\\
18.37056	0.72656244\\
18.40128	0.234489456\\
18.432	0.70783992\\
18.46272	0.70536816\\
18.49344	0.71530164\\
18.52416	0.0250625628\\
18.55488	0.7644654\\
18.5856	0.71498124\\
18.61632	0.70172892\\
18.64704	0.71234892\\
18.67776	0.0309677112\\
18.70848	0.072990432\\
18.7392	0.71218872\\
18.76992	0.7018434\\
18.80064	0.0236206044\\
18.83136	0.0246963492\\
18.86208	0.72603612\\
18.8928	0.7142256\\
18.92352	0.0232543944\\
18.95424	0.70662672\\
18.98496	0.72177876\\
19.01568	0.72626508\\
19.0464	0.0273056004\\
19.07712	0.7042464\\
19.10784	0.70369704\\
19.13856	0.72168732\\
19.16928	0.216819756\\
19.2	0.71981064\\
19.23072	0.70909884\\
19.26144	0.70337664\\
19.29216	0.0249252336\\
19.32288	0.0237350448\\
19.3536	0.0271682712\\
19.38432	0.7067412\\
19.41504	0.70655832\\
19.44576	0.0232772832\\
19.47648	0.07337952\\
19.5072	0.72354132\\
19.53792	0.70983108\\
19.56864	0.0230026248\\
19.59936	0.024879456\\
19.63008	0.72610488\\
19.6608	0.72431928\\
19.69152	0.212081904\\
19.72224	0.7019118\\
19.75296	0.7095564\\
19.78368	0.72152712\\
19.8144	0.0273971556\\
19.84512	0.71456904\\
19.87584	0.70202628\\
19.90656	0.70770276\\
19.93728	0.255157488\\
19.968	0.7275924\\
19.99872	0.72177876\\
20.02944	0.703125\\
20.06016	0.0276489252\\
20.09088	0.0232543944\\
20.1216	0.73118592\\
20.15232	0.71852868\\
20.18304	0.7051392\\
20.21376	0.0277404768\\
20.24448	0.0286331184\\
20.2752	0.72770688\\
20.30592	0.72150408\\
20.33664	0.023506164\\
20.36736	0.024719238\\
20.39808	0.71587368\\
20.4288	0.72594432\\
20.45952	0.0228881844\\
20.49024	0.70935048\\
20.52096	0.70163712\\
20.55168	0.70930476\\
20.5824	0.0277633656\\
20.61312	0.72518904\\
20.64384	0.71253216\\
20.67456	0.70273584\\
20.70528	0.0204391476\\
20.736	0.72454824\\
20.76672	0.72983556\\
20.79744	0.71385948\\
20.82816	0.0246734604\\
20.85888	0.0230712876\\
20.8896	0.179122932\\
20.92032	0.7269516\\
20.95104	0.71674344\\
20.98176	0.023300172\\
21.01248	0.0352249128\\
21.0432	0.72072612\\
21.07392	0.72740952\\
21.10464	0.0232086168\\
21.13536	0.024719238\\
21.16608	0.7040862\\
21.1968	0.7160112\\
21.22752	0.228652956\\
21.25824	0.72113796\\
21.28896	0.7073136\\
21.31968	0.70090488\\
21.3504	0.0308990484\\
21.38112	0.72724896\\
21.41184	0.72377028\\
21.44256	0.70925904\\
21.47328	0.256073004\\
21.504	0.71207424\\
21.53472	0.7275924\\
21.56544	0.72443376\\
21.59616	0.024765012\\
21.62688	0.0236892708\\
21.6576	0.71466048\\
21.68832	0.72518904\\
21.71904	0.72576144\\
21.74976	0.0273284928\\
21.78048	0.07278444\\
21.8112	0.70758828\\
21.84192	0.72191628\\
21.87264	0.0241241436\\
21.90336	0.0246963492\\
21.93408	0.70436088\\
21.9648	0.70337664\\
21.99552	0.023345946\\
22.02624	0.7275924\\
22.05696	0.7192152\\
22.08768	0.70458984\\
22.1184	0.0277862544\\
22.14912	0.7185744\\
22.17984	0.72718056\\
22.21056	0.72049716\\
22.24128	0.215492256\\
22.272	0.70243848\\
22.30272	0.71594244\\
22.33344	0.72718056\\
22.36416	0.0247421268\\
22.39488	0.74338524\\
22.4256	0.7066728\\
22.45632	0.71344764\\
22.48704	0.72592164\\
22.51776	0.022979736\\
22.54848	0.022819518\\
22.5792	0.70188912\\
22.60992	0.70930476\\
22.64064	0.0233230572\\
22.67136	0.024513246\\
22.70208	0.7136766\\
22.7328	0.70149996\\
22.76352	0.205558776\\
22.79424	0.72290052\\
22.82496	0.727272\\
22.85568	0.71658324\\
22.8864	0.026779176\\
22.91712	0.70502472\\
22.94784	0.72008532\\
22.97856	0.7267914\\
23.00928	0.21649932\\
23.04	0.70568856\\
23.07072	0.70475004\\
23.10144	0.71912376\\
23.13216	0.0247879044\\
23.16288	0.0227966292\\
23.1936	0.0276031512\\
23.22432	0.70282728\\
23.25504	0.71569044\\
23.28576	0.0274200444\\
23.31648	0.072486864\\
23.3472	0.70971696\\
23.37792	0.7014312\\
23.40864	0.023185728\\
23.43936	0.024513246\\
23.47008	0.72454824\\
23.5008	0.71170812\\
23.53152	0.023300172\\
23.56224	0.70978536\\
23.59296	0.72482292\\
23.62368	0.72466272\\
23.6544	0.0274429296\\
23.68512	0.70225524\\
23.71584	0.70717608\\
23.74656	0.72198468\\
23.77728	0.254173248\\
23.808	0.71733852\\
23.83872	0.70774848\\
23.86944	0.70607772\\
23.90016	0.0281524644\\
23.93088	0.0235519416\\
23.9616	0.72209916\\
23.99232	0.70509348\\
24.02304	0.70349112\\
24.05376	0.023139954\\
24.08448	0.028038024\\
24.1152	0.72040572\\
24.14592	0.70726788\\
24.17664	0.0236434932\\
24.20736	0.0248107896\\
24.23808	0.72734076\\
24.2688	0.72331236\\
24.29952	0.208374012\\
24.33024	0.70186608\\
24.36096	0.71335584\\
24.39168	0.72239688\\
24.4224	0.0276260364\\
24.45312	0.711891\\
24.48384	0.70227828\\
24.51456	0.7098084\\
24.54528	0.020347596\\
24.576	0.72628776\\
24.60672	0.71884908\\
24.63744	0.7023924\\
24.66816	0.024353028\\
24.69888	0.0232086168\\
24.7296	0.226570104\\
24.76032	0.71589672\\
24.79104	0.70353684\\
24.82176	0.0306930528\\
24.85248	0.0293884272\\
24.8832	0.72722628\\
24.91392	0.71955864\\
24.94464	0.0230483988\\
24.97536	0.0249938964\\
25.00608	0.7191468\\
25.0368	0.72738648\\
25.06752	0.0232543944\\
25.09824	0.70648956\\
25.12896	0.7026444\\
25.15968	0.71115876\\
25.1904	0.0306930528\\
25.22112	0.7226028\\
25.25184	0.70960248\\
25.28256	0.70163712\\
25.31328	0.251564004\\
25.344	0.72681444\\
25.37472	0.72926316\\
25.40544	0.7122114\\
25.43616	0.0249023412\\
25.46688	0.0234603864\\
25.4976	0.72683712\\
25.52832	0.72473148\\
25.55904	0.71381376\\
25.58976	0.023300172\\
25.62048	0.062828064\\
25.6512	0.72351828\\
25.68192	0.72665388\\
25.71264	0.0229110696\\
25.74336	0.0246505752\\
25.77408	0.70559676\\
25.8048	0.72072612\\
25.83552	0.157081608\\
25.86624	0.71818524\\
25.89696	0.70438392\\
25.92768	0.70065324\\
25.9584	0.0276947028\\
25.98912	0.7277526\\
26.01984	0.721161\\
26.05056	0.70644384\\
26.08128	0.209426868\\
26.112	0.71580492\\
26.14272	0.7314606\\
26.17344	0.72200772\\
26.20416	0.0250396704\\
26.23488	0.72715752\\
26.2656	0.7146378\\
26.29632	0.7266312\\
26.32704	0.72475416\\
26.35776	0.0307388304\\
26.38848	0.0198211644\\
26.4192	0.71113572\\
26.44992	0.72496044\\
26.48064	0.0231628428\\
26.51136	0.0246734604\\
26.54208	0.70225524\\
26.5728	0.70804584\\
26.60352	0.023139954\\
26.63424	0.72683712\\
26.66496	0.71651448\\
26.69568	0.7019118\\
26.7264	0.0272369376\\
26.75712	0.72070308\\
26.78784	0.72683712\\
26.81856	0.71752176\\
26.84928	0.215126028\\
26.88	0.70399476\\
26.91072	0.72045144\\
26.94144	0.72772992\\
26.97216	0.0249938964\\
27.00288	0.0231628428\\
27.0336	0.027511596\\
27.06432	0.71596512\\
27.09504	0.72967536\\
27.12576	0.023666382\\
27.15648	0.072280872\\
27.1872	0.70259868\\
27.21792	0.71289828\\
27.24864	0.0227966292\\
27.27936	0.0249023412\\
27.31008	0.71095284\\
27.3408	0.70156872\\
27.37152	0.207618696\\
27.40224	0.72564696\\
27.43296	0.72560124\\
27.46368	0.71337888\\
27.4944	0.0271682712\\
27.52512	0.70800012\\
27.55584	0.72235116\\
27.58656	0.72512064\\
27.61728	0.25522614\\
27.648	0.70326216\\
27.67872	0.70912152\\
27.70944	0.72216792\\
27.74016	0.0281524644\\
27.77088	0.0230712876\\
27.8016	0.70896132\\
27.83232	0.70390332\\
27.86304	0.71836848\\
27.89376	0.027465822\\
27.92448	0.0283813452\\
27.9552	0.70639812\\
27.98592	0.7025526\\
28.01664	0.0230941764\\
28.04736	0.0244445796\\
28.07808	0.72225936\\
28.1088	0.70832052\\
28.13952	0.0235519416\\
28.17024	0.71452332\\
28.20096	0.72642528\\
28.23168	0.723564\\
28.2624	0.0279006948\\
28.29312	0.70129404\\
28.32384	0.71003736\\
28.35456	0.72203076\\
28.38528	0.0208282464\\
28.416	0.71376804\\
28.44672	0.70582572\\
28.47744	0.7085952\\
28.50816	0.0251541144\\
28.53888	0.0232086168\\
28.5696	0.321739164\\
28.60032	0.70285032\\
28.63104	0.70534512\\
28.66176	0.0232772832\\
28.69248	0.0294113124\\
28.7232	0.71813952\\
28.75392	0.70493328\\
28.78464	0.0232086168\\
28.81536	0.0247421268\\
28.84608	0.72809604\\
28.8768	0.72081756\\
28.90752	0.205764768\\
28.93824	0.70275888\\
28.96896	0.71644608\\
28.99968	0.7250976\\
29.0304	0.0313339212\\
29.06112	0.7090758\\
29.09184	0.70202628\\
29.12256	0.7107696\\
29.15328	0.251953128\\
29.184	0.72452556\\
29.21472	0.71610264\\
29.24544	0.70241544\\
29.27616	0.0246047976\\
29.30688	0.0233917236\\
29.3376	0.72956088\\
29.36832	0.71301276\\
29.39904	0.7023924\\
29.42976	0.027511596\\
29.46048	0.073562616\\
29.4912	0.72612756\\
29.52192	0.7159194\\
29.55264	0.0231628428\\
29.58336	0.0246963492\\
29.61408	0.72166428\\
29.6448	0.7266312\\
29.67552	0.0232543944\\
29.70624	0.70426944\\
29.73696	0.70454412\\
29.76768	0.71571348\\
29.7984	0.027832032\\
29.82912	0.72061164\\
29.85984	0.70664976\\
29.89056	0.70099632\\
29.92128	0.217002852\\
29.952	0.7281648\\
29.98272	0.72789012\\
30.01344	0.7085952\\
30.04416	0.024879456\\
30.07488	0.73123164\\
30.1056	0.72841644\\
30.13632	0.72351828\\
30.16704	0.7104492\\
30.19776	0.0234374976\\
30.22848	0.0211029048\\
30.2592	0.72617328\\
30.28992	0.72523512\\
30.32064	0.0232772832\\
30.35136	0.0246963492\\
30.38208	0.70866396\\
30.4128	0.72198468\\
30.44352	0.23023224\\
30.47424	0.71598816\\
30.50496	0.7032852\\
30.53568	0.70394904\\
30.5664	0.0275802588\\
30.59712	0.72763812\\
30.62784	0.71916948\\
30.65856	0.70404048\\
30.68928	0.215194716\\
30.72	0.71953596\\
30.75072	0.73148328\\
30.78144	0.72019944\\
30.81216	0.0246047976\\
30.84288	0.0235748268\\
30.8736	0.027717588\\
30.90432	0.72779832\\
30.93504	0.72187056\\
30.96576	0.027465822\\
30.99648	0.0728073\\
31.0272	0.71486676\\
31.05792	0.72601308\\
31.08864	0.023345946\\
31.11936	0.0246047976\\
31.15008	0.70218648\\
31.1808	0.71022024\\
31.21152	0.0235748268\\
31.24224	0.72610488\\
31.27296	0.71335584\\
31.30368	0.70104204\\
31.3344	0.027717588\\
31.36512	0.723816\\
31.39584	0.72702036\\
31.42656	0.71523288\\
31.45728	0.254539476\\
31.488	0.70598592\\
31.51872	0.7263108\\
31.54944	0.72658548\\
31.58016	0.0282211308\\
31.61088	0.0235748268\\
31.6416	0.70591752\\
31.67232	0.72013104\\
31.70304	0.72763812\\
31.73376	0.0235290528\\
31.76448	0.0281753532\\
31.7952	0.70339968\\
31.82592	0.7160112\\
31.85664	0.0237350448\\
31.88736	0.0250167852\\
31.91808	0.70802316\\
31.9488	0.701271\\
31.97952	0.210662856\\
32.01024	0.72736344\\
32.04096	0.724617\\
32.07168	0.71120448\\
32.1024	0.0271911636\\
32.13312	0.71099856\\
32.16384	0.72528084\\
32.19456	0.72436536\\
32.22528	0.0202560408\\
32.256	0.70236972\\
32.28672	0.7135848\\
32.31744	0.7237242\\
32.34816	0.0249023412\\
32.37888	0.023185728\\
32.4096	0.343482948\\
32.44032	0.70603164\\
32.47104	0.72152712\\
32.50176	0.0312423696\\
32.53248	0.0292739832\\
32.5632	0.7046586\\
32.59392	0.70463556\\
32.62464	0.0233230572\\
32.65536	0.0243759168\\
32.68608	0.72051984\\
32.7168	0.7063524\\
32.74752	0.0236434932\\
32.77824	0.71651448\\
32.80896	0.72798156\\
32.83968	0.72221364\\
32.8704	0.0274429296\\
32.90112	0.7021638\\
32.93184	0.71417988\\
32.96256	0.72331236\\
32.99328	0.252571104\\
33.024	0.71056368\\
33.05472	0.7062606\\
33.08544	0.71063244\\
33.11616	0.025085448\\
33.14688	0.0232315056\\
33.1776	0.7186662\\
33.20832	0.70259868\\
33.23904	0.70866396\\
33.26976	0.0233230572\\
33.30048	0.0309219336\\
33.3312	0.7151184\\
33.36192	0.7025526\\
33.39264	0.0234146124\\
33.42336	0.0251541144\\
33.45408	0.72724896\\
33.4848	0.71818524\\
33.51552	0.198692316\\
33.54624	0.70410924\\
33.57696	0.71951292\\
33.60768	0.727272\\
33.6384	0.0274887072\\
33.66912	0.70580304\\
33.69984	0.70321644\\
33.73056	0.711891\\
33.76128	0.232658388\\
33.792	0.72301464\\
33.82272	0.71335584\\
33.85344	0.70163712\\
33.88416	0.0246734604\\
33.91488	0.73720548\\
33.9456	0.72898848\\
33.97632	0.71081532\\
34.00704	0.70172892\\
34.03776	0.0306930528\\
34.06848	0.0235519416\\
34.0992	0.72521208\\
34.12992	0.71248608\\
34.16064	0.023506164\\
34.19136	0.0244445796\\
34.22208	0.72491436\\
34.2528	0.72624204\\
34.28352	0.0230255136\\
34.31424	0.70298784\\
34.34496	0.70630632\\
34.37568	0.72145836\\
34.4064	0.0276947028\\
34.43712	0.71781912\\
34.46784	0.70390332\\
34.49856	0.70101936\\
34.52928	0.197799696\\
34.56	0.72800424\\
34.59072	0.72521208\\
34.62144	0.70591752\\
34.65216	0.0244903536\\
34.68288	0.0231170652\\
34.7136	0.0268478388\\
34.74432	0.72065736\\
34.77504	0.70767972\\
34.80576	0.0232315056\\
34.83648	0.069099408\\
34.8672	0.72722628\\
34.89792	0.7231752\\
34.92864	0.0233230572\\
34.95936	0.0241470324\\
34.99008	0.7115022\\
35.0208	0.72546372\\
35.05152	0.229385376\\
35.08224	0.71241768\\
35.11296	0.7023924\\
35.14368	0.70880112\\
35.1744	0.027030942\\
35.20512	0.72898848\\
35.23584	0.7161714\\
35.26656	0.70161444\\
35.29728	0.253257732\\
35.328	0.72251136\\
35.35872	0.73175796\\
35.38944	0.7172928\\
35.42016	0.0286102296\\
35.45088	0.0235977156\\
35.4816	0.72141264\\
35.51232	0.72763812\\
35.54304	0.71955864\\
35.57376	0.0275344848\\
35.60448	0.0282897936\\
35.6352	0.71685792\\
35.66592	0.72720324\\
35.69664	0.023139954\\
35.72736	0.048912048\\
35.75808	0.70186608\\
35.7888	0.71440884\\
35.81952	0.0230255136\\
};
\end{axis}
\end{tikzpicture}%}
\caption{Example of raw measurements of the transmit case.}
\label{fig:Transmit_raw}
\end{figure}

The next step is to find the average power consumption as:
\begin{equation}
P_{transmit} = E(f(x))
\end{equation}
\begin{where}
\va{$P_{transmit}$}{is the average power consumption during transmit state}{W}
\end{where}

This is done for all parameters and the result can be seen in \autoref{fig:Transmit_Points}.

\begin{figure}[H]
\centering
\begin{minipage}{0.48\textwidth}
\tikzsetnextfilename{Transmit_Points}
\resizebox{\textwidth}{!}{
% This file was created by matlab2tikz.
%
%The latest updates can be retrieved from
%  http://www.mathworks.com/matlabcentral/fileexchange/22022-matlab2tikz-matlab2tikz
%where you can also make suggestions and rate matlab2tikz.
%
\definecolor{mycolor1}{rgb}{0.00000,0.44700,0.74100}%
\definecolor{mycolor2}{rgb}{0.85000,0.32500,0.09800}%
\definecolor{mycolor3}{rgb}{0.92900,0.69400,0.12500}%
%
\begin{tikzpicture}

\begin{axis}[%
width=12.4in,
height=6.357in,
at={(2.08in,0.858in)},
scale only axis,
xmin=0,
xmax=100,
xlabel={Data Points},
ymin=0.131,
ymax=0.137,
ylabel={Avg. Power consumption [W]},
axis background/.style={fill=white},
title style={font=\bfseries},
title={Transmit},
legend style={at={(0.97,0.03)},anchor=south east,legend cell align=left,align=left,draw=white!15!black}
]
\addplot [color=mycolor1,only marks,mark=*,mark options={solid}]
  table[row sep=crcr]{%
1	0.131034879691784\\
2	0.135206269300117\\
};
\addlegendentry{CP format};

\addplot [color=mycolor2,only marks,mark=*,mark options={solid}]
  table[row sep=crcr]{%
1	0.135829821049459\\
2	0.135795684929074\\
3	0.135548569987852\\
4	0.135577564808571\\
5	0.135534178338524\\
6	0.135727405853401\\
7	0.135572171857743\\
8	0.13576021115347\\
9	0.135577925402099\\
10	0.136152962699817\\
11	0.135964314501334\\
12	0.135971211199521\\
13	0.136002139090922\\
14	0.136176454410933\\
15	0.1359879811862\\
16	0.136168580140498\\
17	0.136183723981932\\
18	0.136181740560122\\
19	0.13616678569156\\
20	0.136161475038809\\
21	0.136174058866293\\
22	0.135906947053042\\
23	0.13615405677175\\
24	0.136150442623051\\
25	0.135925289935805\\
26	0.135927153368611\\
27	0.13591955162778\\
28	0.135902203870294\\
29	0.136102630855924\\
30	0.136092075747546\\
31	0.136087729696951\\
32	0.13560293284348\\
33	0.135818827502946\\
34	0.135751801086845\\
35	0.1357953285655\\
36	0.135791248474108\\
37	0.135961356928112\\
38	0.135776763525742\\
39	0.135922330158771\\
40	0.13590407367665\\
41	0.135574757414205\\
42	0.135673077262854\\
43	0.135512156332663\\
44	0.135818839069838\\
45	0.135781754964222\\
46	0.135575646789595\\
47	0.135402692797852\\
48	0.13568474838069\\
49	0.135392093380107\\
50	0.135671437428649\\
51	0.135294028343203\\
52	0.135600166083035\\
53	0.135599444034098\\
54	0.135299883105007\\
55	0.135286355166246\\
56	0.135261102066647\\
57	0.135240394499234\\
58	0.135555665658007\\
59	0.13519196612858\\
60	0.135103591432874\\
61	0.135213524386535\\
62	0.135113598273484\\
63	0.135110539597205\\
64	0.135290143197378\\
65	0.135107973018762\\
66	0.135271780414257\\
67	0.135066044937166\\
68	0.135260596885485\\
69	0.135226596522249\\
70	0.135395855440633\\
71	0.135611631138825\\
72	0.135622955103172\\
73	0.135594409106628\\
74	0.135397213890925\\
75	0.135581507604902\\
76	0.135635085247642\\
77	0.135434263265754\\
78	0.135419162633024\\
79	0.135414966008461\\
80	0.135378056396111\\
81	0.135471137663363\\
82	0.13542695815948\\
83	0.135441789644027\\
84	0.135489011808238\\
85	0.135496424566135\\
86	0.135642494969293\\
87	0.135498924887968\\
88	0.135709610142296\\
89	0.135546033303295\\
90	0.135432140484162\\
91	0.135462030749248\\
92	0.135488953040129\\
93	0.135821924450509\\
94	0.135821571098135\\
95	0.1358277489622\\
96	0.135561639609378\\
97	0.135597077745581\\
98	0.135734557842914\\
99	0.135969635728828\\
100	0.135990533251521\\
};
\addlegendentry{Frequency};

\addplot [color=mycolor3,only marks,mark=*,mark options={solid}]
  table[row sep=crcr]
\end{minipage}
\hfill
\begin{minipage}{0.48\textwidth}
\tikzsetnextfilename{Transmit_Stat}
\resizebox{\textwidth}{!}{
% This file was created by matlab2tikz.
%
%The latest updates can be retrieved from
%  http://www.mathworks.com/matlabcentral/fileexchange/22022-matlab2tikz-matlab2tikz
%where you can also make suggestions and rate matlab2tikz.
%
%Lognormal distribution 
%mean = -1.998143976645835 
%var = 0.000006003978845
%
\definecolor{mycolor1}{rgb}{0.00000,0.44700,0.74100}%
\definecolor{mycolor2}{rgb}{0.85000,0.32500,0.09800}%
%
\begin{tikzpicture}

\begin{axis}[%
width=0.951\textwidth,
height=0.66\textwidth,
at={(0\textwidth,0\textwidth)},
scale only axis,
xmin=0.13,
xmax=0.142,
xlabel={Avg. power consumption [W]},
ymin=0,
ymax=1500,
ylabel={PDF},
axis background/.style={fill=white},
title style={font=\bfseries},
title={Transmit},
legend style={legend cell align=left,align=left,draw=white!15!black},
y tick label style={/pgf/number format/fixed}
]
\addplot[fill=mycolor1,fill opacity=0.6,draw=black,ybar interval,area legend] plot table[row sep=crcr] {%
x	y\\
0.131	95.2380952381057\\
0.1311	0\\
0.1312	0\\
0.1313	0\\
0.1314	0\\
0.1315	0\\
0.1316	0\\
0.1317	0\\
0.1318	0\\
0.1319	0\\
0.132	0\\
0.1321	0\\
0.1322	0\\
0.1323	0\\
0.1324	0\\
0.1325	0\\
0.1326	0\\
0.1327	0\\
0.1328	0\\
0.1329	0\\
0.133	0\\
0.1331	0\\
0.1332	0\\
0.1333	0\\
0.1334	0\\
0.1335	0\\
0.1336	0\\
0.1337	0\\
0.1338	0\\
0.1339	0\\
0.134	0\\
0.1341	0\\
0.1342	0\\
0.1343	0\\
0.1344	0\\
0.1345	95.2380952381057\\
0.1346	95.2380952380793\\
0.1347	0\\
0.1348	0\\
0.1349	0\\
0.135	95.2380952381057\\
0.1351	476.190476190396\\
0.1352	1047.61904761916\\
0.1353	380.952380952423\\
0.1354	1333.33333333311\\
0.1355	1428.57142857159\\
0.1356	857.142857142714\\
0.1357	952.380952381057\\
0.1358	571.428571428634\\
0.1359	1238.09523809503\\
0.136	285.714285714317\\
0.1361	1047.61904761887\\
0.1362	1047.61904761887\\
};
\addlegendentry{Data points};

\addplot [color=mycolor2,solid]
  table[row sep=crcr]{%
0.13	1.16164330468779e-61\\
0.13005	1.69804894357985e-60\\
0.1301	2.41930020161846e-59\\
0.13015	3.3597281090425e-58\\
0.1302	4.54784826235889e-57\\
0.13025	6.00078258551592e-56\\
0.1303	7.71831544066605e-55\\
0.13035	9.67748388134173e-54\\
0.1304	1.18287885453742e-52\\
0.13045	1.4095111604456e-51\\
0.1305	1.63741871983569e-50\\
0.13055	1.8544985154435e-49\\
0.1306	2.04777031471524e-48\\
0.13065	2.2046339968818e-47\\
0.1307	2.31422048905479e-46\\
0.13075	2.36863640788135e-45\\
0.1308	2.36390454958215e-44\\
0.13085	2.30044451518551e-43\\
0.1309	2.18301280328119e-42\\
0.13095	2.02011401724298e-41\\
0.131	1.82298430431415e-40\\
0.13105	1.60431548764211e-39\\
0.1311	1.37692015234563e-38\\
0.13115	1.15252986230421e-37\\
0.1312	9.40875950661919e-37\\
0.13125	7.49137576867985e-36\\
0.1313	5.81771034609314e-35\\
0.13135	4.40672708376657e-34\\
0.1314	3.25586203505543e-33\\
0.13145	2.34646362448573e-32\\
0.1315	1.64957417726612e-31\\
0.13155	1.13123346583997e-30\\
0.1316	7.56775694273406e-30\\
0.13165	4.93888356930682e-29\\
0.1317	3.14448272655002e-28\\
0.13175	1.95317233492781e-27\\
0.1318	1.18362720077887e-26\\
0.13185	6.99816876481493e-26\\
0.1319	4.03702139029959e-25\\
0.13195	2.27225412396104e-24\\
0.132	1.2479120845314e-23\\
0.13205	6.68735430905601e-23\\
0.1321	3.49687353475123e-22\\
0.13215	1.78432001883225e-21\\
0.1322	8.88474079149813e-21\\
0.13225	4.31725712324858e-20\\
0.1323	2.04726535148873e-19\\
0.13235	9.47447906404855e-19\\
0.1324	4.27920826926179e-18\\
0.13245	1.88629784299474e-17\\
0.1325	8.11535900452811e-17\\
0.13255	3.40774961970891e-16\\
0.1326	1.39669610969155e-15\\
0.13265	5.58756114278225e-15\\
0.1327	2.18192878686015e-14\\
0.13275	8.31703553591299e-14\\
0.1328	3.09469611683633e-13\\
0.13285	1.12408895649855e-12\\
0.1329	3.98591749362826e-12\\
0.13295	1.3797887478074e-11\\
0.133	4.66299839046657e-11\\
0.13305	1.53850173531201e-10\\
0.1331	4.95589895904161e-10\\
0.13315	1.55865600890377e-09\\
0.1332	4.78622523809113e-09\\
0.13325	1.43503544503932e-08\\
0.1333	4.20117111637523e-08\\
0.13335	1.20095836323467e-07\\
0.1334	3.35233137514502e-07\\
0.13345	9.13774221001799e-07\\
0.1335	2.43229010627129e-06\\
0.13355	6.32248609149947e-06\\
0.1336	1.60497377540098e-05\\
0.13365	3.97893680969636e-05\\
0.1337	9.63377870964212e-05\\
0.13375	0.000227807635591979\\
0.1338	0.000526130308642955\\
0.13385	0.00118681555874011\\
0.1339	0.00261486591507528\\
0.13395	0.00562734187543001\\
0.134	0.0118292237261419\\
0.13405	0.0242895589808875\\
0.1341	0.0487197225251715\\
0.13415	0.0954603702487431\\
0.1342	0.182719912107211\\
0.13425	0.341668077190623\\
0.1343	0.624151451316813\\
0.13435	1.11391973428943\\
0.1344	1.94226017459425\\
0.13445	3.3087332257364\\
0.1345	5.50716604757163\\
0.13455	8.9560781946207\\
0.1346	14.2312179417941\\
0.13465	22.0959121830284\\
0.1347	33.5226732072197\\
0.13475	49.6973634450438\\
0.1348	71.995804976101\\
0.13485	101.922788346009\\
0.1349	141.005733447711\\
0.13495	190.640300063607\\
0.135	251.89305790091\\
0.13505	325.276204194278\\
0.1351	410.519741008545\\
0.13515	506.375274513852\\
0.1352	610.490132593015\\
0.13525	719.388551552925\\
0.1353	828.586972983726\\
0.13535	932.853317155876\\
0.1354	1026.59763458587\\
0.13545	1104.357648008\\
0.1355	1161.32223686272\\
0.13555	1193.82363022633\\
0.1356	1199.72837546879\\
0.13565	1178.66912405818\\
0.1357	1132.08222393826\\
0.13575	1063.04580808117\\
0.1358	975.943718806079\\
0.13585	876.006200767805\\
0.1359	768.79404887912\\
0.13595	659.696331280676\\
0.136	553.503194511955\\
0.13605	454.097352904335\\
0.1361	364.285041865513\\
0.13615	285.76430321396\\
0.1362	219.209620747567\\
0.13625	164.439803978804\\
0.1363	120.631490182927\\
0.13635	86.5429151206067\\
0.1364	60.7197674552852\\
0.13645	41.6645873879736\\
0.1365	27.9610268148267\\
0.13655	18.352651445999\\
0.1366	11.7818854912787\\
0.13665	7.39795448564029\\
0.1367	4.54359874987214\\
0.13675	2.72954534971232\\
0.1368	1.60395890605571\\
0.13685	0.921975286923165\\
0.1369	0.51841668201362\\
0.13695	0.285156195996194\\
0.137	0.15344096127324\\
0.13705	0.0807727641731888\\
0.1371	0.0415972084060117\\
0.13715	0.0209579831888183\\
0.1372	0.0103307366647475\\
0.13725	0.00498218699964029\\
0.1373	0.00235085638881494\\
0.13735	0.00108532527314777\\
0.1374	0.000490266094980445\\
0.13745	0.000216696702368948\\
0.1375	9.37198534606748e-05\\
0.13755	3.9662501294735e-05\\
0.1376	1.64251039834183e-05\\
0.13765	6.6561944732527e-06\\
0.1377	2.63963589445018e-06\\
0.13775	1.02440687847721e-06\\
0.1378	3.89064547660069e-07\\
0.13785	1.44611158530605e-07\\
0.1379	5.2604533467373e-08\\
0.13795	1.87281981891042e-08\\
0.138	6.52575095403185e-09\\
0.13805	2.22554754650012e-09\\
0.1381	7.42891405545798e-10\\
0.13815	2.42720269943892e-10\\
0.1382	7.7622777789328e-11\\
0.13825	2.42988034453303e-11\\
0.1383	7.44566185538678e-12\\
0.13835	2.23333858079733e-12\\
0.1384	6.55765823507397e-13\\
0.13845	1.88493352558548e-13\\
0.1385	5.3040356442708e-14\\
0.13855	1.46113396383166e-14\\
0.1386	3.9405505545264e-15\\
0.13865	1.04044009967842e-15\\
0.1387	2.6895560743613e-16\\
0.13875	6.80702720070597e-17\\
0.1388	1.68677725285311e-17\\
0.13885	4.09252503093772e-18\\
0.1389	9.72228372637252e-19\\
0.13895	2.26151015014094e-19\\
0.139	5.15100590579558e-20\\
0.13905	1.14883788800727e-20\\
0.1391	2.50904571246684e-21\\
0.13915	5.36600944372126e-22\\
0.1392	1.12382121543359e-22\\
0.13925	2.30491992421088e-23\\
0.1393	4.62953267251391e-24\\
0.13935	9.10649013407372e-25\\
0.1394	1.75431385708442e-25\\
0.13945	3.30990710767499e-26\\
0.1395	6.11626618942095e-27\\
0.13955	1.10695208140941e-27\\
0.1396	1.96224364017326e-28\\
0.13965	3.40697140309891e-29\\
0.1397	5.7940903853467e-30\\
0.13975	9.65190458259587e-31\\
0.1398	1.57493086853462e-31\\
0.13985	2.51733157616806e-32\\
0.1399	3.94148178023837e-33\\
0.13995	6.04544815538351e-34\\
0.14	9.0835800955027e-35\\
};
\addlegendentry{Approx.};

\end{axis}
\end{tikzpicture}%}
\end{minipage}
\caption{Average power consumption when in transmit state.}
\label{fig:Transmit_Points}
\end{figure}

As $P_{transmit}$ shares several traits with $P_{attach}$ it can also be modelled using a Gaussian distribution with $P_{TX}$ setting the mean value this can be from \autoref{fig:Transmit_Points} and \autoref{fig:Transmit_Pmax}. The mean is split into two models for the transmit part as expected from the attach measurements. 

\begin{figure}[H]
\centering
\tikzsetnextfilename{Transmit_Pmax}
\resizebox{0.7\textwidth}{!}{
% This file was created by matlab2tikz.
%
%The latest updates can be retrieved from
%  http://www.mathworks.com/matlabcentral/fileexchange/22022-matlab2tikz-matlab2tikz
%where you can also make suggestions and rate matlab2tikz.
%
\definecolor{mycolor1}{rgb}{0.00000,0.44700,0.74100}%
\definecolor{mycolor2}{rgb}{0.85000,0.32500,0.09800}%
\definecolor{mycolor3}{rgb}{0.92900,0.69400,0.12500}%
%
\begin{tikzpicture}

\begin{axis}[%
width=0.951\textwidth,
height=0.66\textwidth,
at={(0\textwidth,0\textwidth)},
scale only axis,
xmin=-30,
xmax=30,
xlabel={Pmax [dBm]},
ymin=0.04,
ymax=0.22,
ylabel={Avg. Power Consumption [W]},
axis background/.style={fill=white},
title style={font=\bfseries},
title={Transmit},
legend style={at={(0.03,0.97)},anchor=north west,legend cell align=left,align=left,draw=white!15!black},
y tick label style={/pgf/number format/fixed}
]
\addplot [color=mycolor1,only marks,mark=*,mark options={solid}]
  table[row sep=crcr]{%
10	0.0637950254237825\\
11	0.064713243745866\\
12	0.0666659683567463\\
13	0.0704764637679604\\
14	0.0723517382545409\\
15	0.077964939639652\\
16	0.079334950628774\\
17	0.0878082519401085\\
18	0.0936253392352084\\
19	0.101816405487039\\
20	0.111331959596792\\
21	0.123535456484453\\
22	0.136128742064438\\
23	0.174972950689174\\
9	0.0617877150990408\\
-1	0.0470602737490459\\
-10	0.0446109538683134\\
-11	0.0442200313078414\\
-12	0.0440716559725411\\
-13	0.0441436351842956\\
-14	0.0440310218032299\\
-15	0.0437250232069267\\
-16	0.043861312456332\\
-17	0.0438432049965383\\
-18	0.0435259843807757\\
-19	0.0436726853298729\\
-2	0.0464200132118457\\
-20	0.0434029278993332\\
-21	0.0435548718544184\\
-22	0.0434999814497639\\
-23	0.0432525264618958\\
-24	0.0432345150410041\\
-25	0.0433854386156806\\
-26	0.0431660444350098\\
-27	0.0433332959821664\\
-28	0.0431155547810735\\
-29	0.0430935271540902\\
-3	0.0460319501445317\\
-30	0.0432671370442985\\
-4	0.0456832887367053\\
-5	0.0456148403706944\\
-6	0.045118962743008\\
-7	0.0449127933362612\\
-8	0.0449036143685845\\
-9	0.0447768047382843\\
0	0.0473341788384111\\
1	0.0480683479919247\\
2	0.0486490752697266\\
3	0.0490368064453516\\
4	0.0498296655741014\\
5	0.0507689571429081\\
6	0.0517899366621504\\
7	0.0527056931949284\\
8	0.0537705239438872\\
};
\addlegendentry{Data Points};

\addplot [color=mycolor2,solid]
  table[row sep=crcr]{%
-30	0.0431292666676032\\
-29	0.0431542082548141\\
-28	0.0431810736294623\\
-27	0.0432100996121427\\
-26	0.0432415522070661\\
-25	0.0432757301983898\\
-24	0.0433129691897273\\
-23	0.0433536461414516\\
-22	0.0433981844671359\\
-21	0.0434470597580337\\
-20	0.0435008062129939\\
-19	0.043560023860741\\
-18	0.0436253866721654\\
-17	0.0436976516723004\\
-16	0.0437776691751785\\
-15	0.0438663942799402\\
-14	0.0439648997836222\\
-13	0.0440743906852025\\
-12	0.0441962204769953\\
-11	0.0443319094436535\\
-10	0.0444831652161776\\
-9	0.044651905858819\\
-8	0.0448402858010076\\
-7	0.0450507249649014\\
-6	0.045285941482356\\
-5	0.0455489884436444\\
-4	0.0458432951747622\\
-3	0.0461727136013829\\
-2	0.0465415703262953\\
-1	0.0469547251244026\\
0	0.0474176366461276\\
1	0.0479364362175212\\
2	0.0485180107348439\\
3	0.0491700957743394\\
4	0.0499013801760316\\
5	0.0507216235155014\\
6	0.0516417880518442\\
7	0.0526741869357233\\
8	0.05383265068127\\
};
\addlegendentry{y = 0.0433*exp(0.0002*x) + 4.13E-03*exp(0.116*x) RMSE = 0.00012};

\addplot [color=mycolor3,solid]
  table[row sep=crcr]
\caption{Transmit Pmax}
\label{fig:Transmit_Pmax}
\end{figure}

This means that $P_{transmit}$ can be modelled as:
\begin{align}
&P_{transmit} \sim \mathcal{N}(\mu_{Pmax},11.034\cdot 10^{-6}) \\ \nonumber
&\mu_{Pmax} = \begin{cases} 0.0433\cdot\exp{(0.0002\cdot x)} + 4.13\cdot10^{-3}\cdot\exp{(0.116\cdot x)} \quad for x \leq 8 dBm \\
0.0399\cdot\exp{(0.0442\cdot x)} + 22.3\cdot10^{-6}\cdot\exp{(0.545\cdot x)} \quad for x > 8 dBm \end{cases}
\end{align}


\section{Idle Mode Power Consumption}
Four different kinds of idle mode exist for a NB-IoT device: \gls{cDRX}, \gls{DRX}, \gls{eDRX} and \gls{PSM}. It is chosen not to look into \gls{cDRX} due to an assumption of the device transmitting its data quickly and then going into deep sleep. The setup to test this can be seen in \autoref{fig:IPE_test_setup}.

When the device is in idle mode most parameters lose their influence entirely. Because of this a whole new set of parameters should be considered. For the \gls{DRX} case the cycle period is important, while for \gls{eDRX} however a few parameter more should be considered such as the repetition of the DRX cycle and the eDRX cycle period. For \gls{PSM} its is also the DRX cycle period as well as the repetition of the DRX cycle as well as the PSM cycle time. This can be summed up to two different types of parameters: period lengths and repetition numbers. It is assumed that both of these parameters have a proportional relation to the energy consumption of the idle modes, this means that only a single instance of the power consumption is needed for these different idle modes to extrapolate from.

Because of this a new set of parameters need to be defined for the UXM, as can be seen in \autoref{tab:UXM_idle_values}.

\begin{table}[H]
\centering
\begin{tabular}{|c|c|} \hline
\multicolumn{2}{|c|}{\textbf{DRX}}  	 \\ \hline
Long DRX cycle     	& 1024 subframes 	 \\ \hline
onDuration Timer   	& 4 NPDCCH subframes \\ \hline
Retransmission Timer & 2 NPDCCH subframes \\ \hline
Inactivity Timer   	& 8 NPDCCH subframes \\ \hline
DRX Start Offset   	& 0              	 \\ \hline
drx-ULRetransmissionTimer & 0       	 \\ \hline
\multicolumn{2}{|c|}{\textbf{eDRX}} 	 \\ \hline
Idle eDRX State		& Off				 \\ \hline
\gls{PTW}			& 5120 subframes 	 \\ \hline
Idle eDRX Cycle     & 20480 subframes	 \\ \hline
\multicolumn{2}{|c|}{\textbf{PSM}}  	 \\ \hline
Power Saving Mode	& Off				 \\ \hline
T3324 (DRX period)	& 10 s    			 \\ \hline
T3412 (PSM period)	& 10 s		  		 \\ \hline
\end{tabular}
\caption{Specific parameter used to measure device idle power}
\label{tab:UXM_idle_values}
\end{table}


\subsection{Test Procedure}
\begin{enumerate}
\item Setup the \gls{DUT} as shown on \autoref{fig:IPE_test_setup}
\item Turn on power supply 
\item Put in settings as described in \autoref{tab:setup_parameters}
\item Put in settings as described in \autoref{tab:UXM_idle_values}  
\item Start to measure power output
\item Connect device to cell
\item Release device from cell
\item Wait for measurements to stop
\item Save measurements as "Idle/DRX.csv"
\item Set Idle mode eDRX state to On
\item Repeat step 5-8
\item Save measurements as "Idle/eDRX.csv"
\item Set Idle mode eDRX state to Off and PSM On
\item Repeat step 5-8
\item Save measurements as "Idle/PSM.csv"
\end{enumerate}


\subsection{Results}
\begin{minipage}{0.48\textwidth}
\begin{figure}[H]
\tikzsetnextfilename{DRX1}
\centering
\resizebox{\textwidth}{!}{
% This file was created by matlab2tikz.
%
%The latest updates can be retrieved from
%  http://www.mathworks.com/matlabcentral/fileexchange/22022-matlab2tikz-matlab2tikz
%where you can also make suggestions and rate matlab2tikz.
%
\definecolor{mycolor1}{rgb}{0.00000,0.44700,0.74100}%
%
\begin{tikzpicture}

\begin{axis}[%
width=\textwidth,
height=0.66\textwidth,
at={(2.08in,0.858in)},
scale only axis,
xmin=0,
xmax=51.1998976,
xlabel={Time [s]},
ymin=-0.1,
ymax=0.8,
ylabel={Power Consumption [W]},
axis background/.style={fill=white},
%title style={font=\bfseries},
%title={$\text{DRXDefConfQuectel}_\text{\$}\text{1\$}$}
]
\addplot [color=mycolor1,only marks,mark=*,mark options={solid},forget plot]
  table[row sep=crcr]{%
0.0001024	-1.3347342e-05\\
0.0513024	3.01396464e-05\\
0.1025024	4.6453356e-05\\
0.1537024	1.9208808e-05\\
0.2049024	1.67797332e-05\\
0.2561024	-1.01010312e-05\\
0.3073024	-1.43632296e-05\\
0.3585024	4.8192588e-05\\
0.4097024	2.50241724e-05\\
0.4609024	-2.46168108e-05\\
0.5121024	9.6894792e-07\\
0.5633024	3.00072132e-06\\
0.6145024	-4.7064384e-06\\
0.6657024	4.583394e-05\\
0.7169024	1.4475546e-06\\
0.7681024	1.6579404e-06\\
0.8193024	3.7063944e-05\\
0.8705024	0.000157729824\\
0.9217024	-2.13939684e-05\\
0.9729024	4.1361336e-05\\
1.0241024	-1.28243124e-07\\
1.0753024	-3.00700764e-05\\
1.1265024	-1.65584484e-05\\
1.1777024	2.282895e-05\\
1.2289024	-3.5255124e-05\\
1.2801024	2.71732932e-05\\
1.3313024	1.39189896e-05\\
1.3825024	2.02255308e-05\\
1.4337024	0.000240173388\\
1.4849024	1.77495192e-05\\
1.5361024	5.4253548e-05\\
1.5873024	1.98978012e-05\\
1.6385024	-4.7446596e-05\\
1.6897024	3.25720764e-06\\
1.7409024	7.7375016e-05\\
1.7921024	3.53003868e-05\\
1.8433024	3.0104442e-05\\
1.8945024	2.18717388e-05\\
1.9457024	-7.578918e-06\\
1.9969024	-7.1237808e-06\\
2.0481024	8.8395552e-06\\
2.0993024	5.0090256e-06\\
2.1505024	-1.59046596e-05\\
2.2017024	6.464628e-05\\
2.2529024	-1.65006144e-05\\
2.3041024	5.2467372e-05\\
2.3553024	-1.1801718e-06\\
2.4065024	-4.3452612e-05\\
2.4577024	-2.37870036e-05\\
2.5089024	4.6469268e-06\\
2.5601024	-6.014016e-06\\
2.6113024	1.9419192e-05\\
2.6625024	-6.0011064e-05\\
2.7137024	-7.1698788e-06\\
2.7649024	8.1789768e-05\\
2.8161024	-4.5204444e-05\\
2.8673024	1.88006076e-05\\
2.9185024	2.73836772e-05\\
2.9697024	1.80177408e-05\\
3.0209024	-6.8547204e-06\\
3.0721024	-6.7960476e-06\\
3.1233024	2.99175264e-05\\
3.1745024	1.13382e-05\\
3.2257024	2.52462924e-05\\
3.2769024	1.57755816e-05\\
3.3281024	3.09459852e-06\\
3.3793024	6.5504556e-06\\
3.4305024	4.1524776e-05\\
3.4817024	-3.9143484e-05\\
3.5329024	-1.06617816e-05\\
3.5841024	-6.2428428e-05\\
3.6353024	-2.18725776e-05\\
3.6865024	2.73132684e-05\\
3.7377024	-1.86606324e-05\\
3.7889024	-1.3803318e-05\\
3.8401024	2.94389208e-05\\
3.8913024	-3.01748508e-05\\
3.9425024	4.4256456e-06\\
3.9937024	3.86523e-05\\
4.0449024	3.50279748e-06\\
4.0961024	4.1770368e-05\\
4.1473024	2.7827082e-05\\
4.1985024	-5.78016e-06\\
4.2497024	-4.618512e-05\\
4.3009024	3.13659216e-05\\
4.3521024	3.322251e-05\\
4.4033024	2.58766128e-05\\
4.4545024	-8.1396648e-06\\
4.5057024	1.68618744e-05\\
4.5569024	2.7079416e-05\\
4.6081024	1.4444532e-05\\
4.6593024	3.58728696e-05\\
4.7105024	8.453988e-06\\
4.7617024	1.95709068e-05\\
4.8129024	1.88467092e-05\\
4.8641024	-3.06182556e-05\\
4.9153024	1.43858592e-05\\
4.9665024	4.0042044e-05\\
5.0177024	2.6503578e-06\\
5.0689024	4.8905064e-05\\
5.1201024	-1.8625428e-05\\
5.1713024	1.83328992e-05\\
5.2225024	2.27585448e-05\\
5.2737024	0.000242287308\\
5.3249024	-1.41880464e-05\\
5.3761024	3.21479532e-05\\
5.4273024	6.083586e-06\\
5.4785024	3.28947804e-05\\
5.5297024	2.82822192e-05\\
5.5809024	6.7021704e-06\\
5.6321024	1.56238668e-05\\
5.6833024	-6.6795372e-06\\
5.7345024	6.2427564e-05\\
5.7857024	3.7904652e-05\\
5.8369024	5.7332232e-06\\
5.8881024	4.1945544e-05\\
5.9393024	-3.14824284e-05\\
5.9905024	-4.952448e-05\\
6.0417024	3.9646404e-07\\
6.0929024	1.36625004e-05\\
6.1441024	3.08278008e-05\\
6.1953024	-3.9505572e-05\\
6.2465024	-2.19890844e-05\\
6.2977024	-3.9587724e-06\\
6.3489024	5.941512e-05\\
6.4001024	6.076962e-05\\
6.4513024	-4.0964868e-05\\
6.5025024	-2.33553348e-05\\
6.5537024	1.13239512e-06\\
6.6049024	-1.11403872e-05\\
6.6561024	-7.4733048e-06\\
6.7073024	-2.5153254e-05\\
6.7585024	1.12568976e-05\\
6.8097024	5.0676984e-06\\
6.8609024	-1.36516068e-05\\
6.9121024	2.32614576e-05\\
6.9633024	4.5215352e-05\\
7.0145024	-1.47186216e-06\\
7.0657024	2.66242788e-05\\
7.1169024	7.9376616e-07\\
7.1681024	-3.9003516e-05\\
7.2193024	-8.9334324e-06\\
7.2705024	1.86480576e-05\\
7.3217024	1.08947988e-05\\
7.3729024	1.83681036e-05\\
7.4241024	5.1567984e-05\\
7.4753024	1.8403308e-05\\
7.5265024	-1.5122628e-05\\
7.5777024	4.3837344e-05\\
7.6289024	0.0199351152\\
7.6801024	0.10303776\\
7.7313024	0.104827716\\
7.7825024	0.105791364\\
7.8337024	0.108096444\\
7.8849024	0.11392794\\
7.9361024	0.116557272\\
7.9873024	0.120682872\\
8.0385024	0.12594384\\
8.0897024	0.0188719452\\
8.1409024	0.154284408\\
8.1921024	0.0200538612\\
8.2433024	0.0199176012\\
8.2945024	0.208794564\\
8.3457024	0.210701088\\
8.3969024	0.15352128\\
8.4481024	0.0249259752\\
8.4993024	0.161845524\\
8.5505024	0.208368252\\
8.6017024	0.0228432528\\
8.6529024	0.150860304\\
8.7041024	0.0207715572\\
8.7553024	0.0250076124\\
8.8065024	0.150689268\\
8.8577024	0.15134994\\
8.9089024	0.1512909\\
8.9601024	0.0250060788\\
9.0113024	0.020781072\\
9.0625024	0.0207200124\\
9.1137024	0.0208070388\\
9.1649024	0.0206997696\\
9.2161024	0.0207937476\\
9.2673024	0.0207832428\\
9.3185024	0.0208215396\\
9.3697024	0.0208099908\\
9.4209024	0.0207891864\\
9.4721024	0.0208092132\\
9.5233024	0.0208234476\\
9.5745024	0.0249726024\\
9.6257024	0.0285985944\\
9.6769024	0.0250067628\\
9.7281024	0.0209289132\\
9.7793024	0.0208719684\\
9.8305024	0.0207531648\\
9.8817024	0.0208653804\\
9.9329024	0.02086911\\
9.9841024	0.020876166\\
10.0353024	0.0208785312\\
10.0865024	0.020900088\\
10.1377024	0.0208500444\\
10.1889024	0.0208981584\\
10.2401024	0.0280131732\\
10.2913024	0.0278755236\\
10.3425024	0.0250580016\\
10.3937024	0.0208112724\\
10.4449024	0.0208738296\\
10.4961024	0.0207869292\\
10.5473024	0.0207606888\\
10.5985024	0.0207415188\\
10.6497024	0.0207309708\\
10.7009024	0.020779308\\
10.7521024	0.0208486584\\
10.8033024	1.1572056e-05\\
10.8545024	-1.51804656e-05\\
10.9057024	-1.28494584e-06\\
10.9569024	1.21906404e-05\\
11.0081024	-1.94200308e-05\\
11.0593024	-5.3612316e-05\\
11.1105024	4.3669728e-06\\
11.1617024	2.75471244e-05\\
11.2129024	2.11240728e-05\\
11.2641024	-1.04865984e-05\\
11.3153024	4.2774516e-05\\
11.3665024	0.0200678292\\
11.4177024	0.020173842\\
11.4689024	0.0280076076\\
11.5201024	0.0279150768\\
11.5713024	0.0277283556\\
11.6225024	0.0249987024\\
11.6737024	0.728325\\
11.7249024	0.0283334976\\
11.7761024	0.02777445\\
11.8273024	0.72592164\\
11.8785024	0.72608184\\
11.9297024	0.195912792\\
11.9809024	0.72651672\\
12.0321024	0.027727182\\
12.0833024	0.0308310624\\
12.1345024	0.030968568\\
12.1857024	0.115462332\\
12.2369024	0.0337531356\\
12.2881024	0.256169808\\
12.3393024	0.0323021664\\
12.3905024	0.0251955972\\
12.4417024	0.23196726\\
12.4929024	0.0251941572\\
12.5441024	0.0250945812\\
12.5953024	0.0267261228\\
12.6465024	0.0250068456\\
12.6977024	0.26257392\\
12.7489024	0.199599768\\
12.8001024	0.0235381464\\
12.8513024	0.0277234164\\
12.9025024	0.0307102464\\
12.9537024	0.72585288\\
13.0049024	0.64620216\\
13.0561024	0.227542824\\
13.1073024	0.263043684\\
13.1585024	0.0251079408\\
13.2097024	0.221081652\\
13.2609024	0.024917364\\
13.3121024	0.223861932\\
13.3633024	0.19419228\\
13.4145024	0.0249408324\\
13.4657024	0.123592716\\
13.5169024	0.0248616828\\
13.5681024	0.218612412\\
13.6193024	0.0273884112\\
13.6705024	0.0252271692\\
13.7217024	0.211425156\\
13.7729024	0.025119558\\
13.8241024	0.0250067052\\
13.8753024	0.028041642\\
13.9265024	0.0251603676\\
13.9777024	0.261418536\\
14.0289024	0.0333495684\\
14.0801024	0.0234402552\\
14.1313024	0.0277242516\\
14.1825024	0.030903066\\
14.2337024	0.70493328\\
14.2849024	0.73729692\\
14.3361024	0.210575484\\
14.3873024	0.260115408\\
14.4385024	0.025053948\\
14.4897024	0.221661252\\
14.5409024	0.0251830152\\
14.5921024	0.209955564\\
14.6433024	0.194872572\\
14.6945024	0.0250024932\\
14.7457024	0.1305837\\
14.7969024	0.0250194384\\
14.8481024	0.209866716\\
14.8993024	0.027714348\\
14.9505024	0.0252617112\\
15.0017024	0.205328844\\
15.0529024	0.0249093108\\
15.1041024	0.0250121376\\
15.1553024	0.0281312136\\
15.2065024	0.0252221832\\
15.2577024	0.258469128\\
15.3089024	0.200008476\\
15.3601024	0.0234274932\\
15.4113024	0.027740934\\
15.4625024	0.0309356928\\
15.5137024	0.71017452\\
15.5649024	0.7610778\\
15.6161024	0.72104652\\
15.6673024	0.265644036\\
15.7185024	0.0301432572\\
15.7697024	0.221364036\\
15.8209024	0.02509056\\
15.8721024	0.226569636\\
15.9233024	0.195592968\\
15.9745024	0.0249082308\\
16.0257024	0.132866352\\
16.0769024	0.025156962\\
16.1281024	0.232486308\\
16.1793024	0.0275248116\\
16.2305024	0.0252820548\\
16.2817024	0.22811634\\
16.3329024	0.0253370376\\
16.3841024	0.0247916412\\
16.4353024	0.0276118776\\
16.4865024	0.025149042\\
16.5377024	0.25890354\\
16.5889024	0.165810636\\
16.6401024	0.0234741924\\
16.6913024	0.0277516008\\
16.7425024	0.0310229784\\
16.7937024	0.72624204\\
16.8449024	0.77297976\\
16.8961024	0.238438764\\
16.9473024	0.261871272\\
16.9985024	0.025118478\\
17.0497024	0.22203954\\
17.1009024	0.0251678052\\
17.1521024	0.23655438\\
17.2033024	0.194836068\\
17.2545024	0.0250567812\\
17.3057024	0.1339794\\
17.3569024	0.02519109\\
17.4081024	0.232363188\\
17.4593024	0.0277310196\\
17.5105024	0.0252927684\\
17.5617024	0.229553172\\
17.6129024	0.0249882984\\
17.6641024	0.025060068\\
17.7153024	0.0278109252\\
17.7665024	0.0249430284\\
17.8177024	0.2575989\\
17.8689024	0.101195028\\
17.9201024	0.0250827948\\
17.9713024	0.0266997312\\
18.0225024	0.024985404\\
18.0737024	0.263091384\\
18.1249024	0.0287482788\\
18.1761024	0.218758788\\
18.2273024	0.262426104\\
18.2785024	0.0249756192\\
18.3297024	0.221911812\\
18.3809024	0.0251394408\\
18.4321024	0.215327412\\
18.4833024	0.195097248\\
18.5345024	0.0250595244\\
18.5857024	0.134426412\\
18.6369024	0.0250777764\\
18.6881024	0.212219028\\
18.7393024	0.027910782\\
18.7905024	0.0252245268\\
18.8417024	0.20687868\\
18.8929024	0.0251775648\\
18.9441024	0.0250093512\\
18.9953024	0.0278805384\\
19.0465024	0.0253429992\\
19.0977024	0.257762016\\
19.1489024	0.019898802\\
19.2001024	0.0252754164\\
19.2513024	0.0270424908\\
19.3025024	0.0250199856\\
19.3537024	0.262497168\\
19.4049024	0.0287253792\\
19.4561024	0.208518336\\
19.5073024	0.261546156\\
19.5585024	0.0260910396\\
19.6097024	0.22109094\\
19.6609024	0.0250980156\\
19.7121024	0.210750552\\
19.7633024	0.195751764\\
19.8145024	0.0250226892\\
19.8657024	0.134345196\\
19.9169024	0.025092324\\
19.9681024	0.213509628\\
20.0193024	0.068726484\\
20.0705024	0.0252628632\\
20.1217024	0.211375152\\
20.1729024	0.024901092\\
20.2241024	0.0247144788\\
20.2753024	0.073123992\\
20.3265024	0.0250517376\\
20.3777024	0.254542428\\
20.4289024	0.0305754012\\
20.4801024	0.0249196608\\
20.5313024	0.028073034\\
20.5825024	0.0251284248\\
20.6337024	0.25800012\\
20.6849024	0.0287396352\\
20.7361024	0.229525344\\
20.7873024	0.260032824\\
20.8385024	0.0273174084\\
20.8897024	0.216350748\\
20.9409024	0.0249624864\\
20.9921024	0.234159912\\
21.0433024	0.196200216\\
21.0945024	0.0251046612\\
21.1457024	0.134225856\\
21.1969024	0.025105824\\
21.2481024	0.23679306\\
21.2993024	0.0283907592\\
21.3505024	0.0252085068\\
21.4017024	0.232931232\\
21.4529024	0.0249138072\\
21.5041024	0.02512287\\
21.5553024	0.070455564\\
21.6065024	0.0251048952\\
21.6577024	0.256064328\\
21.7089024	0.0206443476\\
21.7601024	0.024905034\\
21.8113024	0.0295501572\\
21.8625024	0.0250314624\\
21.9137024	0.256896684\\
21.9649024	0.0287357652\\
22.0161024	0.232188336\\
22.0673024	0.259568244\\
22.1185024	0.0267835212\\
22.1697024	0.215772732\\
22.2209024	0.0251327448\\
22.2721024	0.229847832\\
22.3233024	0.195798312\\
22.3745024	0.0251154108\\
22.4257024	0.13424472\\
22.4769024	0.0250659828\\
22.5281024	0.22423158\\
22.5793024	0.042133032\\
22.6305024	0.0251454132\\
22.6817024	0.218408868\\
22.7329024	0.02526516\\
22.7841024	0.0249169968\\
22.8353024	0.064354752\\
22.8865024	0.0251093556\\
22.9377024	0.258482052\\
22.9889024	0.0213241032\\
23.0401024	0.024943518\\
23.0913024	0.04018518\\
23.1425024	0.0249809508\\
23.1937024	0.258717096\\
23.2449024	0.71411148\\
23.2961024	0.212285412\\
23.3473024	0.262335636\\
23.3985024	0.027599922\\
23.4497024	0.217956348\\
23.5009024	0.0250252092\\
23.5521024	0.209873052\\
23.6033024	0.195941016\\
23.6545024	0.0250895124\\
23.7057024	0.149109876\\
23.7569024	0.0251678268\\
23.8081024	0.2083446\\
23.8593024	0.05939748\\
23.9105024	0.0250814232\\
23.9617024	0.20447964\\
24.0129024	0.0249591924\\
24.0641024	0.024959934\\
24.1153024	0.047088432\\
24.1665024	0.025109046\\
24.2177024	0.260258004\\
24.2689024	0.0203570316\\
24.3201024	0.023549958\\
24.3713024	0.0312678288\\
24.4225024	0.0207306864\\
24.4737024	0.0206958492\\
24.5249024	0.0250086636\\
24.5761024	0.0208189224\\
24.6273024	0.0208302624\\
24.6785024	0.0208787832\\
24.7297024	0.0208354248\\
24.7809024	0.0208027152\\
24.8321024	0.020807352\\
24.8833024	0.020761992\\
24.9345024	0.020790828\\
24.9857024	0.020681604\\
25.0369024	0.020803932\\
25.0881024	0.0208750104\\
25.1393024	0.020802978\\
25.1905024	0.0207509148\\
25.2417024	0.0206656164\\
25.2929024	0.0207867384\\
25.3441024	0.0207752328\\
25.3953024	0.020853792\\
25.4465024	0.0207490644\\
25.4977024	0.0207474192\\
25.5489024	-5.7173796e-05\\
25.6001024	4.9640976e-05\\
25.6513024	1.4690124e-05\\
25.7025024	-5.6167128e-06\\
25.7537024	4.876506e-05\\
25.8049024	0.0001505943\\
25.8561024	-4.9402944e-06\\
25.9073024	-5.877306e-05\\
25.9585024	0.019907352\\
26.0097024	0.0200097144\\
26.0609024	0.0200609928\\
26.1121024	0.204453504\\
26.1633024	4.2552396e-05\\
26.2145024	1.31361192e-05\\
26.2657024	-1.89179568e-06\\
26.3169024	2.05172208e-05\\
26.3681024	2.56083912e-05\\
26.4193024	3.44596788e-05\\
26.4705024	1.69666488e-05\\
26.5217024	4.2097284e-05\\
26.5729024	4.4187732e-05\\
26.6241024	-3.8746188e-05\\
26.6753024	5.1158952e-05\\
26.7265024	6.3518076e-06\\
26.7777024	-1.32073632e-05\\
26.8289024	-1.72474416e-05\\
26.8801024	5.8247532e-05\\
26.9313024	1.9710882e-05\\
26.9825024	-4.6008252e-06\\
27.0337024	4.6703952e-06\\
27.0849024	-3.845448e-05\\
27.1361024	3.07347624e-05\\
27.1873024	3.50556336e-05\\
27.2385024	0.0199123812\\
27.2897024	0.0200673504\\
27.3409024	0.0200144844\\
27.3921024	0.205295724\\
27.4433024	-7.124616e-07\\
27.4945024	-3.9248244e-05\\
27.5457024	7.694334e-05\\
27.5969024	-8.594802e-06\\
27.6481024	-1.3125222e-05\\
27.6993024	-2.22338376e-05\\
27.7505024	-2.54223108e-05\\
27.8017024	3.7944864e-06\\
27.8529024	6.5160072e-05\\
27.9041024	8.640234e-05\\
27.9553024	4.8846384e-05\\
28.0065024	4.8788532e-05\\
28.0577024	-3.8653164e-05\\
28.1089024	4.2035256e-06\\
28.1601024	7.00065e-05\\
28.2113024	-2.83811256e-06\\
28.2625024	5.530464e-05\\
28.3137024	-3.5009532e-05\\
28.3649024	-1.4601276e-06\\
28.4161024	3.701448e-06\\
28.4673024	-4.3557408e-05\\
28.5185024	0.0198970236\\
28.5697024	0.019943568\\
28.6209024	0.0200915424\\
28.6721024	0.204602292\\
28.7233024	-3.3628194e-06\\
28.7745024	4.53042e-06\\
28.8257024	-4.6477656e-06\\
28.8769024	2.87264592e-05\\
28.9281024	9.2368584e-06\\
28.9793024	3.25100484e-05\\
29.0305024	-3.42149292e-06\\
29.0817024	-4.2950556e-05\\
29.1329024	-1.21680072e-05\\
29.1841024	6.8255496e-05\\
29.2353024	1.47831624e-05\\
29.2865024	1.67101632e-05\\
29.3377024	4.4139096e-06\\
29.3889024	3.9061332e-05\\
29.4401024	3.06760896e-05\\
29.4913024	1.09878372e-05\\
29.5425024	2.9088558e-05\\
29.5937024	-9.5721336e-07\\
29.6449024	9.5137128e-05\\
29.6961024	-3.48016608e-06\\
29.7473024	1.40589648e-05\\
29.7985024	0.019943406\\
29.8497024	0.0201050676\\
29.9009024	0.0200327472\\
29.9521024	0.205416684\\
30.0033024	3.22409916e-05\\
30.0545024	-6.4808892e-06\\
30.1057024	5.5456344e-05\\
30.1569024	-1.48535712e-05\\
30.2081024	8.208396e-06\\
30.2593024	5.7710232e-05\\
30.3105024	4.5530496e-05\\
30.3617024	-4.0291812e-06\\
30.4129024	-1.48653036e-05\\
30.4641024	-1.63715328e-05\\
30.5153024	-5.3132868e-06\\
30.5665024	2.81891772e-05\\
30.6177024	1.55886624e-05\\
30.6689024	2.4475158e-07\\
30.7201024	5.6869524e-05\\
30.7713024	4.3171848e-05\\
30.8225024	5.7044736e-05\\
30.8737024	1.18176444e-05\\
30.9249024	-1.85793264e-05\\
30.9761024	3.34789992e-05\\
31.0273024	1.78542936e-05\\
31.0785024	0.0198766872\\
31.1297024	0.0199951524\\
31.1809024	0.0200548008\\
31.2321024	0.204279588\\
31.2833024	4.7048472e-05\\
31.3345024	2.57835708e-05\\
31.3857024	-4.6862388e-05\\
31.4369024	-5.3602272e-06\\
31.4881024	5.768424e-06\\
31.5393024	-3.662136e-05\\
31.5905024	5.6401812e-06\\
31.6417024	8.5243968e-06\\
31.6929024	2.90994516e-05\\
31.7441024	2.16144144e-05\\
31.7953024	5.4101844e-05\\
31.8465024	4.1683212e-06\\
31.8977024	-2.87381916e-05\\
31.9489024	1.28243124e-07\\
32.0001024	-1.86916428e-07\\
32.0513024	1.16650944e-05\\
32.1025024	4.6604232e-05\\
32.1537024	-1.39785012e-05\\
32.2049024	3.47404752e-05\\
32.2561024	-2.29580316e-05\\
32.3073024	-3.49391268e-05\\
32.3585024	0.0198829548\\
32.4097024	0.0201045204\\
32.4609024	0.0200977956\\
32.5121024	0.205858332\\
32.5633024	1.82515932e-05\\
32.6145024	7.0525332e-06\\
32.6657024	2.45572992e-05\\
32.7169024	-5.7360708e-05\\
32.7681024	7.7649948e-06\\
32.8193024	-1.45853496e-05\\
32.8705024	0.000196533828\\
32.9217024	1.68140988e-06\\
32.9729024	3.23231328e-05\\
33.0241024	4.898718e-05\\
33.0753024	-2.84900904e-06\\
33.1265024	-7.2448128e-05\\
33.1777024	-2.43594864e-05\\
33.2289024	1.2821796e-05\\
33.2801024	-5.3836956e-06\\
33.3313024	3.28134744e-05\\
33.3825024	-4.3669728e-06\\
33.4337024	3.255615e-05\\
33.4849024	-5.1813576e-05\\
33.5361024	2.92050648e-05\\
33.5873024	2.02716324e-05\\
33.6385024	0.0199823544\\
33.6897024	0.0200268\\
33.7409024	0.0200947392\\
33.7921024	0.20495286\\
33.8433024	-4.4958852e-05\\
33.8945024	1.04974956e-05\\
33.9457024	3.9785544e-05\\
33.9969024	1.48183668e-05\\
34.0481024	2.22103656e-05\\
34.0993024	1.46197152e-05\\
34.1505024	6.4574172e-06\\
34.2017024	3.6573588e-05\\
34.2529024	5.2655112e-06\\
34.3041024	2.75705928e-05\\
34.3553024	-5.1953544e-05\\
34.4065024	3.12720408e-05\\
34.4577024	-5.2319844e-06\\
34.5089024	4.0228956e-05\\
34.5601024	2.17552284e-05\\
34.6113024	3.12955128e-05\\
34.6625024	-3.6539244e-05\\
34.7137024	2.18834712e-05\\
34.7649024	1.22141088e-05\\
34.8161024	2.83760964e-05\\
34.8673024	4.6698084e-05\\
34.9185024	0.0199321344\\
34.9697024	0.0200574648\\
35.0209024	0.0200899728\\
35.0721024	0.204345576\\
35.1233024	-4.1023548e-05\\
35.1745024	1.48275864e-06\\
35.2257024	2.56318596e-05\\
35.2769024	-8.5478652e-06\\
35.3281024	1.04740272e-05\\
35.3793024	-6.29397e-06\\
35.4305024	6.1120008e-05\\
35.4817024	3.759282e-06\\
35.5329024	-1.02527424e-05\\
35.5841024	4.8613356e-05\\
35.6353024	-1.79951076e-05\\
35.6865024	-6.0093216e-05\\
35.7377024	-4.2856668e-06\\
35.7889024	5.3716248e-05\\
35.8401024	-1.3650768e-05\\
35.8913024	1.95826392e-05\\
35.9425024	-3.31060032e-05\\
35.9937024	1.61728812e-05\\
36.0449024	-2.338551e-07\\
36.0961024	0.000175899276\\
36.1473024	2.26655028e-05\\
36.1985024	0.0199526328\\
36.2497024	0.0200850336\\
36.3009024	0.0200841192\\
36.3521024	0.200312352\\
36.4033024	2.22916716e-05\\
36.4545024	1.45727784e-05\\
36.5057024	-2.98948968e-05\\
36.5569024	0.0002582037\\
36.6081024	3.39576048e-05\\
36.6593024	-4.5195228e-06\\
36.7105024	8.967798e-06\\
36.7617024	8.523558e-06\\
36.8129024	-1.74930336e-05\\
36.8641024	-1.98164952e-05\\
36.9153024	1.72591776e-05\\
36.9665024	-5.2437204e-06\\
37.0177024	-2.88907452e-05\\
37.0689024	2.82126492e-05\\
37.1201024	1.8694998e-05\\
37.1713024	1.12099572e-05\\
37.2225024	-3.7169532e-05\\
37.2737024	-4.4030124e-06\\
37.3249024	1.89179568e-06\\
37.3761024	4.9127184e-05\\
37.4273024	5.038866e-05\\
37.4785024	0.0199912572\\
37.5297024	0.0199704384\\
37.5809024	0.0200724624\\
37.6321024	0.19306944\\
37.6833024	-6.4691532e-06\\
37.7345024	8.5126608e-06\\
37.7857024	4.063716e-05\\
37.8369024	4.7340144e-05\\
37.8881024	0.000190987524\\
37.9393024	-3.8466216e-05\\
37.9905024	-5.6484792e-05\\
38.0417024	4.3743492e-05\\
38.0929024	1.26231444e-05\\
38.1441024	-2.61582444e-05\\
38.1953024	-1.4620554e-05\\
38.2465024	-7.122942e-06\\
38.2977024	3.7333008e-05\\
38.3489024	-5.05521e-05\\
38.4001024	1.10892564e-06\\
38.4513024	-3.6866124e-05\\
38.5025024	4.5647856e-06\\
38.5537024	-5.2327368e-05\\
38.6049024	-4.6652004e-05\\
38.6561024	6.7098816e-05\\
38.7073024	5.5515024e-05\\
38.7585024	0.0198180468\\
38.8097024	0.0200715696\\
38.8609024	0.0201181536\\
38.9121024	0.190391148\\
38.9633024	-1.42701912e-05\\
39.0145024	-2.1256506e-06\\
39.0657024	7.4421252e-05\\
39.1169024	6.3222192e-05\\
39.1681024	-8.1975024e-06\\
39.2193024	-6.3526428e-06\\
39.2705024	-2.3822208e-05\\
39.3217024	-2.84900904e-06\\
39.3729024	2.20469184e-05\\
39.4241024	3.37455432e-06\\
39.4753024	6.3245664e-05\\
39.5265024	1.22543424e-06\\
39.5777024	6.7139028e-06\\
39.6289024	-1.40363352e-05\\
39.6801024	5.902956e-05\\
39.7313024	-2.09044656e-06\\
39.7825024	2.9905794e-05\\
39.8337024	-1.7518176e-07\\
39.8849024	-4.2395652e-06\\
39.9361024	-2.97901224e-05\\
39.9873024	2.14861716e-05\\
40.0385024	0.0200774052\\
40.0897024	0.0200886048\\
40.1409024	0.0200652048\\
40.1921024	0.192271428\\
40.2433024	1.36742364e-05\\
40.2945024	-1.44688428e-05\\
40.3457024	-5.2584696e-05\\
40.3969024	-3.12980256e-06\\
40.4481024	3.1213368e-05\\
40.4993024	-7.1816148e-06\\
40.5505024	3.7029564e-05\\
40.6017024	2.35648836e-05\\
40.6529024	2.15557416e-05\\
40.7041024	1.23188832e-05\\
40.7553024	-5.6986884e-05\\
40.8065024	5.32494e-05\\
40.8577024	-5.8154472e-05\\
40.9089024	8.1723564e-07\\
40.9601024	1.54604196e-05\\
41.0113024	-2.75353884e-05\\
41.0625024	1.47102408e-06\\
41.1137024	-2.75353884e-05\\
41.1649024	-1.98282312e-05\\
41.2161024	1.8391572e-05\\
41.2673024	5.6988576e-06\\
41.3185024	0.0198292284\\
41.3697024	0.0200469168\\
41.4209024	0.0200438604\\
41.4721024	0.193774392\\
41.5233024	-2.583135e-05\\
41.5745024	3.40749504e-05\\
41.6257024	3.02326848e-05\\
41.6769024	-3.00072132e-06\\
41.7281024	2.39269824e-05\\
41.7793024	-5.71059e-06\\
41.8305024	7.9376616e-07\\
41.8817024	6.0933096e-05\\
41.9329024	1.60555356e-05\\
41.9841024	-3.6690948e-05\\
42.0353024	1.11630204e-05\\
42.0865024	-2.27476476e-05\\
42.1377024	-1.21445388e-05\\
42.1889024	4.8309948e-05\\
42.2401024	1.04773788e-07\\
42.2913024	1.54721556e-05\\
42.3425024	5.9999328e-05\\
42.3937024	3.3023862e-05\\
42.4449024	3.7566e-05\\
42.4961024	2.79905292e-05\\
42.5473024	-1.10699784e-05\\
42.5985024	0.0199399572\\
42.6497024	0.0200525976\\
42.7009024	0.0201307608\\
42.7521024	0.19522116\\
42.8033024	5.963724e-05\\
42.8545024	6.8149908e-05\\
42.9057024	2.048202e-05\\
42.9569024	1.34521164e-05\\
43.0081024	8.0105832e-06\\
43.0593024	3.47404752e-05\\
43.1105024	5.049342e-05\\
43.1617024	-2.28063204e-05\\
43.2129024	2.3810472e-05\\
43.2641024	3.53800116e-06\\
43.3153024	-3.6960012e-05\\
43.3665024	-1.89992592e-05\\
43.4177024	3.26382912e-05\\
43.4689024	-2.96845092e-05\\
43.5201024	-3.50363556e-06\\
43.5713024	1.16064216e-05\\
43.6225024	1.6324596e-05\\
43.6737024	4.6522908e-05\\
43.7249024	-5.0541192e-05\\
43.7761024	6.3969012e-05\\
43.8273024	7.1698788e-06\\
43.8785024	0.0198715788\\
43.9297024	0.0200436624\\
43.9809024	0.020019798\\
44.0321024	0.195036732\\
44.0833024	5.6857788e-05\\
44.1345024	1.83563676e-05\\
44.1857024	5.3050752e-05\\
44.2369024	8.687844e-06\\
44.2881024	1.03851792e-06\\
44.3393024	-7.8940764e-06\\
44.3905024	-3.8255832e-05\\
44.4417024	-3.10633344e-06\\
44.4929024	3.7262592e-05\\
44.5441024	1.54025856e-05\\
44.5953024	1.58753232e-06\\
44.6465024	1.03809888e-05\\
44.6977024	-7.2419652e-07\\
44.7489024	1.0076724e-05\\
44.8001024	1.25292672e-05\\
44.8513024	-3.26039256e-05\\
44.9025024	7.6828536e-06\\
44.9537024	-2.34483732e-05\\
45.0049024	3.6970884e-05\\
45.0561024	1.09526328e-05\\
45.1073024	-8.396154e-06\\
45.1585024	0.0200032056\\
45.2097024	0.020094462\\
45.2609024	0.0199887588\\
45.3121024	0.194746716\\
45.3633024	-8.1161964e-06\\
45.4145024	2.33310276e-05\\
45.4657024	1.79129664e-05\\
45.5169024	2.05172208e-05\\
45.5681024	-8.0692596e-06\\
45.6193024	2.82822192e-05\\
45.6705024	-3.22284168e-06\\
45.7217024	-5.2969428e-05\\
45.7729024	2.98898676e-06\\
45.8241024	-2.13470316e-05\\
45.8753024	6.9492672e-05\\
45.9265024	-3.23239716e-05\\
45.9777024	-1.43749656e-05\\
46.0289024	-2.27736324e-06\\
46.0801024	-2.04476544e-05\\
46.1313024	8.3374776e-06\\
46.1825024	-4.5122292e-05\\
46.2337024	-2.97892836e-05\\
46.2849024	3.8472948e-07\\
46.3361024	2.18834712e-05\\
46.3873024	-4.3078788e-05\\
46.4385024	0.0199598508\\
46.4897024	0.0201514572\\
46.5409024	0.020082114\\
46.5921024	0.195359076\\
46.6433024	5.0889888e-05\\
46.6945024	3.6001116e-05\\
46.7457024	1.05327e-05\\
46.7969024	2.4931134e-05\\
46.8481024	-2.16085464e-06\\
46.8993024	-2.11391604e-06\\
46.9505024	-2.8190016e-05\\
47.0017024	2.53167012e-05\\
47.0529024	-2.30980104e-05\\
47.1041024	1.10113056e-05\\
47.1553024	6.221718e-05\\
47.2065024	-8.1975024e-06\\
47.2577024	4.4408988e-05\\
47.3089024	1.09643688e-05\\
47.3601024	1.16759916e-06\\
47.4113024	-8.1396648e-06\\
47.4625024	-3.58032996e-05\\
47.5137024	3.58611336e-05\\
47.5649024	-8.0223192e-06\\
47.6161024	-2.22807744e-05\\
47.6673024	1.30204476e-05\\
47.7185024	0.0198962604\\
47.7697024	0.0200421864\\
47.8209024	0.0200440332\\
47.8721024	0.196013052\\
47.9233024	-1.9817334e-05\\
47.9745024	2.64608316e-05\\
48.0257024	6.6153348e-05\\
48.0769024	2.43586476e-05\\
48.1281024	-2.68639992e-06\\
48.1793024	3.03726636e-05\\
48.2305024	1.55895012e-05\\
48.2817024	1.76330088e-05\\
48.3329024	5.1649272e-05\\
48.3841024	-1.96999884e-05\\
48.4353024	-2.85169104e-05\\
48.4865024	-3.662136e-05\\
48.5377024	7.3159776e-05\\
48.5889024	-6.0961572e-06\\
48.6401024	-6.9038388e-05\\
48.6913024	1.58107824e-05\\
48.7425024	0.000199079424\\
48.7937024	4.8064356e-05\\
48.8449024	0.000210886164\\
48.8961024	1.54017468e-05\\
48.9473024	-2.80148364e-05\\
48.9985024	0.0198899712\\
49.0497024	0.020090736\\
49.1009024	0.0201293784\\
49.1521024	0.195468768\\
49.2033024	-8.5478652e-06\\
49.2545024	9.1077768e-06\\
49.3057024	2.11944816e-05\\
49.3569024	-2.81782836e-05\\
49.4081024	1.20196476e-06\\
49.4593024	5.1812748e-05\\
49.5105024	-6.3912024e-05\\
49.5617024	-4.0556664e-05\\
49.6129024	-4.2043644e-06\\
49.6641024	-2.74892904e-05\\
49.7153024	-3.20783832e-05\\
49.7665024	-1.94896008e-05\\
49.8177024	0.000180208404\\
49.8689024	3.7953252e-06\\
49.9201024	2.24551188e-05\\
49.9713024	-4.608036e-05\\
50.0225024	-2.0400714e-05\\
50.0737024	-2.45346696e-05\\
50.1249024	9.0956232e-05\\
50.1761024	6.10848e-05\\
50.2273024	-2.20242888e-05\\
50.2785024	0.0198536976\\
50.3297024	0.0200802672\\
50.3809024	0.0201572316\\
50.4321024	0.195431616\\
50.4833024	4.2330276e-05\\
50.5345024	3.6317124e-05\\
50.5857024	2.751192e-05\\
50.6369024	4.3743492e-05\\
50.6881024	5.2046604e-05\\
50.7393024	-2.26898136e-05\\
50.7905024	4.4140788e-05\\
50.8417024	2.98898676e-06\\
50.8929024	-3.6271008e-05\\
50.9441024	5.0559624e-06\\
50.9953024	2.74767156e-05\\
51.0465024	-2.53988424e-05\\
51.0977024	-2.26898136e-05\\
51.1489024	-2.6426466e-05\\
};

\addplot[area legend,solid,draw=black,fill=black,fill opacity=0.1,forget plot]
table[row sep=crcr] 
\caption{Raw measurement of \gls{DRX}. Area A is connection phase and area B is DRX phase.}
\label{fig:DRX}
\end{figure}
\vspace{0.8em}
\end{minipage}%
\hfill
\begin{minipage}{0.48\textwidth}
\begin{figure}[H]
\tikzsetnextfilename{PSM1}
\centering
\resizebox{\textwidth}{!}{
% This file was created by matlab2tikz.
%
%The latest updates can be retrieved from
%  http://www.mathworks.com/matlabcentral/fileexchange/22022-matlab2tikz-matlab2tikz
%where you can also make suggestions and rate matlab2tikz.
%
\definecolor{mycolor1}{rgb}{0.00000,0.44700,0.74100}%
%
\begin{tikzpicture}

\begin{axis}[%
width=\textwidth,
height=0.66\textwidth,
at={(2.08in,0.858in)},
scale only axis,
xmin=0,
xmax=51.1998976,
xlabel={Time [s]},
ymin=-0.1,
ymax=0.8,
ylabel={Power Consumption [W]},
axis background/.style={fill=white},
%title style={font=\bfseries},
%title={$\text{PSM}_\text{N}\text{BIoTQuectel}_\text{T}\text{AU}_\text{\$}\text{1\$}$}
]
\addplot [color=mycolor1,only marks,mark=*,mark options={solid},forget plot]
  table[row sep=crcr]{%
0.0001024	0.000100730376\\
0.0513024	6.3987444e-06\\
0.1025024	1.61611488e-05\\
0.1537024	2.28993588e-05\\
0.2049024	8.967798e-06\\
0.2561024	1.30196088e-05\\
0.3073024	1.35216864e-05\\
0.3585024	3.9259152e-05\\
0.4097024	-3.37455432e-06\\
0.4609024	-2.50484796e-05\\
0.5121024	1.41411096e-05\\
0.5633024	-4.0649724e-05\\
0.6145024	0.000265105332\\
0.6657024	-8.0003592e-05\\
0.7169024	3.08982096e-05\\
0.7681024	4.2739308e-05\\
0.8193024	5.2899012e-05\\
0.8705024	-2.6718156e-05\\
0.9217024	-4.6301616e-05\\
0.9729024	1.33699752e-05\\
1.0241024	3.02209524e-05\\
1.0753024	5.59467e-05\\
1.1265024	4.4152524e-05\\
1.1777024	4.5168408e-05\\
1.2289024	2.74884516e-05\\
1.2801024	2.34131688e-05\\
1.3313024	1.20623976e-05\\
1.3825024	4.78305e-05\\
1.4337024	1.03692528e-05\\
1.4849024	-1.0673514e-05\\
1.5361024	3.55577112e-05\\
1.5873024	5.2583868e-05\\
1.6385024	-2.26780776e-05\\
1.6897024	5.6167128e-06\\
1.7409024	-8.3609496e-06\\
1.7921024	2.77801416e-05\\
1.8433024	-2.02842072e-05\\
1.8945024	3.44496204e-06\\
1.9457024	2.62035072e-05\\
1.9969024	2.70215784e-05\\
2.0481024	9.8437068e-06\\
2.0993024	9.3768336e-06\\
2.1505024	2.202345e-05\\
2.2017024	1.48879368e-05\\
2.2529024	2.2852422e-05\\
2.3041024	2.84691348e-05\\
2.3553024	-2.55505572e-05\\
2.4065024	9.8906472e-06\\
2.4577024	4.0451076e-05\\
2.5089024	-8.2913796e-06\\
2.5601024	-2.49319692e-05\\
2.6113024	-1.34060148e-05\\
2.6625024	1.61494128e-05\\
2.7137024	1.00775628e-05\\
2.7649024	-5.94738e-05\\
2.8161024	-1.77847236e-05\\
2.8673024	-1.09534716e-05\\
2.9185024	5.5991124e-07\\
2.9697024	9.4346712e-06\\
3.0209024	2.33016912e-07\\
3.0721024	-1.21680072e-05\\
3.1233024	2.11944816e-05\\
3.1745024	-2.44081008e-06\\
3.2257024	2.34827388e-05\\
3.2769024	1.79481708e-05\\
3.3281024	-2.90424564e-05\\
3.3793024	0.103864464\\
3.4305024	0.103352112\\
3.4817024	0.104455008\\
3.5329024	0.107675928\\
3.5841024	0.110758824\\
3.6353024	0.115395984\\
3.6865024	0.117880668\\
3.7377024	0.124679448\\
3.7889024	0.13064274\\
3.8401024	0.154161108\\
3.8913024	0.0200584188\\
3.9425024	0.0199371636\\
3.9937024	0.020880702\\
4.0449024	0.199970208\\
4.0961024	0.155603916\\
4.1473024	0.155343636\\
4.1985024	0.156356208\\
4.2497024	0.0285480072\\
4.3009024	0.0207597456\\
4.3521024	0.0208391832\\
4.4033024	0.15083136\\
4.4545024	0.0250282404\\
4.5057024	0.0207703512\\
4.5569024	0.0208668384\\
4.6081024	0.0207972864\\
4.6593024	0.151222212\\
4.7105024	0.151454556\\
4.7617024	0.151535448\\
4.8129024	0.0207819216\\
4.8641024	0.0207587772\\
4.9153024	0.0207394092\\
4.9665024	0.0207600372\\
5.0177024	0.0207053748\\
5.0689024	0.224381484\\
5.1201024	0.20290194\\
5.1713024	0.02517759\\
5.2225024	0.020743056\\
5.2737024	0.02073942\\
5.3249024	0.0208322136\\
5.3761024	0.0208742256\\
5.4273024	0.0208060236\\
5.4785024	0.0208253916\\
5.5297024	0.0209075976\\
5.5809024	0.020738124\\
5.6321024	0.0208274256\\
5.6833024	0.0206970588\\
5.7345024	0.0207894384\\
5.7857024	0.0208783152\\
5.8369024	0.0207801612\\
5.8881024	0.02074509\\
5.9393024	0.0208177092\\
5.9905024	0.0207723276\\
6.0417024	0.0208009548\\
6.0929024	0.0206764776\\
6.1441024	0.0250793172\\
6.1953024	0.0207748584\\
6.2465024	0.020752686\\
6.2977024	0.0207268956\\
6.3489024	0.0207335772\\
6.4001024	0.03079089\\
6.4513024	0.0250467048\\
6.5025024	0.0249816204\\
6.5537024	0.0208406592\\
6.6049024	0.0208646496\\
6.6561024	0.020861946\\
6.7073024	0.020737908\\
6.7585024	0.0206756532\\
6.8097024	0.0207348084\\
6.8609024	0.020821464\\
6.9121024	0.0207543276\\
6.9633024	1.80060048e-05\\
7.0145024	-8.1975024e-06\\
7.0657024	-1.02996828e-05\\
7.1169024	-5.963724e-05\\
7.1681024	-1.99573128e-05\\
7.2193024	-7.2989604e-06\\
7.2705024	4.5320112e-05\\
7.3217024	-4.1887728e-05\\
7.3729024	-2.04476544e-05\\
7.4241024	-2.12766228e-05\\
7.4753024	1.99916784e-05\\
7.5265024	0.0201398832\\
7.5777024	0.0200624724\\
7.6289024	0.085475772\\
7.6801024	0.0278117172\\
7.7313024	0.51729588\\
7.7825024	0.0209310696\\
7.8337024	0.025094088\\
7.8849024	0.0207996444\\
7.9361024	0.0209242368\\
7.9873024	0.0207635184\\
8.0385024	0.0207280224\\
8.0897024	0.0207732564\\
8.1409024	0.0208429272\\
8.1921024	0.0208325628\\
8.2433024	-1.52274024e-05\\
8.2945024	5.5573704e-05\\
8.3457024	2.49780708e-05\\
8.3969024	4.2902784e-05\\
8.4481024	3.44948832e-05\\
8.4993024	-8.759088e-07\\
8.5505024	-1.05611976e-07\\
8.6017024	5.007348e-05\\
8.6529024	-1.3055652e-05\\
8.7041024	3.17045484e-05\\
8.7553024	-6.9251292e-06\\
8.8065024	0.0202142556\\
8.8577024	0.0201304512\\
8.9089024	0.08521992\\
8.9601024	0.0277353216\\
9.0113024	0.51743304\\
9.0625024	0.0208749852\\
9.1137024	0.72843948\\
9.1649024	0.71765892\\
9.2161024	0.7255098\\
9.2673024	0.72017676\\
9.3185024	0.72072612\\
9.3697024	0.71845992\\
9.4209024	0.0324441864\\
9.4721024	0.258669828\\
9.5233024	0.0322068204\\
9.5745024	0.03857724\\
9.6257024	0.22024602\\
9.6769024	0.025017408\\
9.7281024	0.251754516\\
9.7793024	0.0279427284\\
9.8305024	0.0248753304\\
9.8817024	0.261805716\\
9.9329024	0.02501847\\
9.9841024	0.0252559764\\
10.0353024	0.0277994088\\
10.0865024	0.0251101656\\
10.1377024	0.201129264\\
10.1889024	0.0283241628\\
10.2401024	0.0234372708\\
10.2913024	0.49752036\\
10.3425024	0.0311227956\\
10.3937024	0.70438392\\
10.4449024	0.710289\\
10.4961024	0.256461012\\
10.5473024	0.198715464\\
10.5985024	0.0249708168\\
10.6497024	0.200193876\\
10.7009024	0.0250201044\\
10.7521024	0.253512756\\
10.8033024	0.0277290756\\
10.8545024	0.0247244976\\
10.9057024	0.206926236\\
10.9569024	0.0250781796\\
11.0081024	0.253834668\\
11.0593024	0.0249754644\\
11.1105024	0.0248920272\\
11.1617024	0.258876612\\
11.2129024	0.025092972\\
11.2641024	0.0249311592\\
11.3153024	0.0257420664\\
11.3665024	0.0250744608\\
11.4177024	0.201938508\\
11.4689024	0.0283981176\\
11.5201024	0.02354427\\
11.5713024	0.5075226\\
11.6225024	0.0312388704\\
11.6737024	0.71754444\\
11.7249024	0.72447984\\
11.7761024	0.252228672\\
11.8273024	0.194748336\\
11.8785024	0.0250760628\\
11.9297024	0.201239208\\
11.9809024	0.0249131628\\
12.0321024	0.25212906\\
12.0833024	0.0277967448\\
12.1345024	0.0247136328\\
12.1857024	0.231321492\\
12.2369024	0.0250299252\\
12.2881024	0.252189108\\
12.3393024	0.025040322\\
12.3905024	0.0249147972\\
12.4417024	0.257808672\\
12.4929024	0.024984234\\
12.5441024	0.0248834844\\
12.5953024	0.0246883392\\
12.6465024	0.0250800012\\
12.6977024	0.201678984\\
12.7489024	0.0283571712\\
12.8001024	0.023528286\\
12.8513024	0.52008804\\
12.9025024	0.0311971536\\
12.9537024	0.72763812\\
13.0049024	0.73125468\\
13.0561024	0.7282332\\
13.1073024	0.73599228\\
13.1585024	0.0284462928\\
13.2097024	0.199963008\\
13.2609024	0.0250275636\\
13.3121024	0.252655596\\
13.3633024	0.0280109448\\
13.4145024	0.0247579488\\
13.4657024	0.220065408\\
13.5169024	0.0249773688\\
13.5681024	0.250779168\\
13.6193024	0.0283868316\\
13.6705024	0.025046838\\
13.7217024	0.258488604\\
13.7729024	0.0250181316\\
13.8241024	0.0248805432\\
13.8753024	0.0250044732\\
13.9265024	0.0250313256\\
13.9777024	0.201074256\\
14.0289024	0.0285286104\\
14.0801024	0.0235093896\\
14.1313024	0.50262444\\
14.1825024	0.0310548312\\
14.2337024	0.70834356\\
14.2849024	0.71106732\\
14.3361024	0.236550528\\
14.3873024	0.194500368\\
14.4385024	0.0249807348\\
14.4897024	0.201157092\\
14.5409024	0.024991542\\
14.5921024	0.229930344\\
14.6433024	0.0275774508\\
14.6945024	0.0248071428\\
14.7457024	0.204483492\\
14.7969024	0.0250279596\\
14.8481024	0.225947556\\
14.8993024	0.0251199468\\
14.9505024	0.02499876\\
15.0017024	0.257938812\\
15.0529024	0.024894126\\
15.1041024	0.0249275988\\
15.1553024	0.0248282316\\
15.2065024	0.0280489932\\
15.2577024	0.200861964\\
15.3089024	0.0282840264\\
15.3601024	0.0248969952\\
15.4113024	0.027909792\\
15.4625024	0.0251595432\\
15.5137024	0.200646036\\
15.5649024	0.0249630408\\
15.6161024	0.225115596\\
15.6673024	0.195838848\\
15.7185024	0.0249633864\\
15.7697024	0.202101408\\
15.8209024	0.0251929188\\
15.8721024	0.22814172\\
15.9233024	0.0287766504\\
15.9745024	0.0248600808\\
16.0257024	0.22122018\\
16.0769024	0.02514825\\
16.1281024	0.23299758\\
16.1793024	0.025184286\\
16.2305024	0.0250613892\\
16.2817024	0.257886828\\
16.3329024	0.025123014\\
16.3841024	0.0248892804\\
16.4353024	0.024742134\\
16.4865024	0.0249476472\\
16.5377024	0.201858264\\
16.5889024	0.028737756\\
16.6401024	0.0250032528\\
16.6913024	0.0265769568\\
16.7425024	0.0249917724\\
16.7937024	0.200293164\\
16.8449024	0.0249923772\\
16.8961024	0.24207606\\
16.9473024	0.195315444\\
16.9985024	0.0250188696\\
17.0497024	0.201564288\\
17.1009024	0.0249708636\\
17.1521024	0.243078264\\
17.2033024	0.047835936\\
17.2545024	0.0247897152\\
17.3057024	0.229966632\\
17.3569024	0.0250281108\\
17.4081024	0.231857496\\
17.4593024	0.0249288912\\
17.5105024	0.02505861\\
17.5617024	0.258035328\\
17.6129024	0.0249172344\\
17.6641024	0.0249796764\\
17.7153024	0.0248330088\\
17.7665024	0.0250390404\\
17.8177024	0.199898604\\
17.8689024	0.0286031232\\
17.9201024	0.0207958644\\
17.9713024	0.0248804496\\
18.0225024	0.0249631128\\
18.0737024	0.199900584\\
18.1249024	0.024957648\\
18.1761024	0.245730852\\
18.2273024	0.195166764\\
18.2785024	0.0250736076\\
18.3297024	0.201173508\\
18.3809024	0.0251282088\\
18.4321024	0.246340044\\
18.4833024	0.064742652\\
18.5345024	0.0247825692\\
18.5857024	0.208223388\\
18.6369024	0.0250547508\\
18.6881024	0.214684344\\
18.7393024	0.025280676\\
18.7905024	0.025044444\\
18.8417024	0.258745824\\
18.8929024	0.0249422508\\
18.9441024	0.024846696\\
18.9953024	0.0247275648\\
19.0465024	0.025041132\\
19.0977024	0.200000916\\
19.1489024	0.0285531876\\
19.2001024	0.020837286\\
19.2513024	0.0249944868\\
19.3025024	0.0251536392\\
19.3537024	0.200293704\\
19.4049024	0.0250300944\\
19.4561024	0.027684666\\
19.5073024	0.19604916\\
19.5585024	0.0251181324\\
19.6097024	0.201871728\\
19.6609024	0.02500056\\
19.7121024	0.245916972\\
19.7633024	0.072077292\\
19.8145024	0.0248379228\\
19.8657024	0.210613392\\
19.9169024	0.0250853436\\
19.9681024	0.206072748\\
20.0193024	0.025080264\\
20.0705024	0.0250294788\\
20.1217024	0.255735936\\
20.1729024	0.0249529788\\
20.2241024	0.024744204\\
20.2753024	0.0248444172\\
20.3265024	0.025100892\\
20.3777024	0.200587824\\
20.4289024	0.0288904248\\
20.4801024	0.0233222292\\
20.5313024	0.51061248\\
20.5825024	0.020826468\\
20.6337024	0.0207668736\\
20.6849024	0.0248379588\\
20.7361024	0.0207880092\\
20.7873024	0.0207965376\\
20.8385024	0.02080773\\
20.8897024	0.0208891656\\
20.9409024	0.0207718812\\
20.9921024	0.0208333584\\
21.0433024	0.0207813852\\
21.0945024	0.0207937368\\
21.1457024	0.0207734976\\
21.1969024	0.0207887292\\
21.2481024	0.0208559124\\
21.2993024	0.0207612072\\
21.3505024	0.0208209384\\
21.4017024	0.0208446876\\
21.4529024	0.020716668\\
21.5041024	0.02072322\\
21.5553024	0.0207621576\\
21.6065024	0.0208247724\\
21.6577024	0.0207628668\\
21.7089024	1.13616684e-05\\
21.7601024	4.0752828e-06\\
21.8113024	7.0263828e-05\\
21.8625024	-1.43665812e-06\\
21.9137024	-4.8661128e-05\\
21.9649024	4.0940568e-05\\
22.0161024	-2.80148364e-05\\
22.0673024	2.08675872e-05\\
22.1185024	0.0198705564\\
22.1697024	0.020075544\\
22.2209024	0.0200339352\\
22.2721024	0.0228009672\\
22.3233024	5.860962e-05\\
22.3745024	-4.602168e-05\\
22.4257024	-1.79716392e-05\\
22.4769024	2.26889748e-05\\
22.5281024	-1.5134364e-05\\
22.5793024	4.438638e-05\\
22.6305024	8.4774564e-06\\
22.6817024	2.7757512e-05\\
22.7329024	4.6534644e-05\\
22.7841024	-6.1575984e-05\\
22.8353024	-4.4760204e-05\\
22.8865024	4.004874e-06\\
22.9377024	-8.326584e-06\\
22.9889024	9.1225296e-05\\
23.0401024	-1.54143216e-05\\
23.0913024	3.7286064e-05\\
23.1425024	-4.9618368e-05\\
23.1937024	-1.8310266e-05\\
23.2449024	-9.2485908e-06\\
23.2961024	1.80060048e-05\\
23.3473024	4.9314096e-05\\
23.3985024	0.019914444\\
23.4497024	0.020111976\\
23.5009024	0.0200932524\\
23.5521024	0.0237098052\\
23.6033024	5.7835116e-08\\
23.6545024	-6.5462652e-07\\
23.7057024	-1.60010532e-06\\
23.7569024	1.40589648e-05\\
23.8081024	-3.8688336e-05\\
23.8593024	1.96061112e-05\\
23.9105024	6.4144188e-05\\
23.9617024	5.4650844e-05\\
24.0129024	2.5724898e-05\\
24.0641024	2.00386152e-05\\
24.1153024	1.41411096e-05\\
24.1665024	-1.76095404e-05\\
24.2177024	0.000213723432\\
24.2689024	1.0298844e-05\\
24.3201024	2.65656024e-05\\
24.3713024	1.13382e-05\\
24.4225024	2.42823744e-06\\
24.4737024	4.23873e-06\\
24.5249024	-1.18763172e-05\\
24.5761024	2.83727412e-06\\
24.6273024	2.07971784e-05\\
24.6785024	0.0199057536\\
24.7297024	0.0202195764\\
24.7809024	0.0201283704\\
24.8321024	0.0245727252\\
24.8833024	-3.3245982e-05\\
24.9345024	1.66984272e-05\\
24.9857024	-1.413189e-06\\
25.0369024	6.1738596e-05\\
25.0881024	3.22284168e-06\\
25.1393024	-6.6326004e-06\\
25.1905024	9.1786032e-05\\
25.2417024	3.15159552e-07\\
25.2929024	0.00025669746\\
25.3441024	2.50359048e-05\\
25.3953024	1.06374744e-05\\
25.4465024	4.1373072e-05\\
25.4977024	7.1279712e-05\\
25.5489024	2.54558412e-06\\
25.6001024	1.50052824e-05\\
25.6513024	-1.99916784e-05\\
25.7025024	1.29584232e-06\\
25.7537024	2.08550124e-05\\
25.8049024	3.9107448e-05\\
25.8561024	3.21010128e-05\\
25.9073024	-2.92880448e-05\\
25.9585024	0.019897308\\
26.0097024	0.020099412\\
26.0609024	0.0200840652\\
26.1121024	0.0253999296\\
26.1633024	2.0470284e-05\\
26.2145024	7.4421252e-05\\
26.2657024	-1.52550648e-07\\
26.3169024	2.34475344e-05\\
26.3681024	-7.7306292e-06\\
26.4193024	5.049342e-05\\
26.4705024	-5.32737e-05\\
26.5217024	1.43975952e-05\\
26.5729024	7.5428748e-06\\
26.6241024	-4.6862388e-05\\
26.6753024	-6.4691532e-06\\
26.7265024	-5.5354104e-06\\
26.7777024	6.0302772e-05\\
26.8289024	4.9897476e-05\\
26.8801024	4.7642724e-06\\
26.9313024	-4.1350428e-05\\
26.9825024	-9.4003056e-06\\
27.0337024	2.91354948e-05\\
27.0849024	-2.54801484e-05\\
27.1361024	5.8849344e-06\\
27.1873024	-3.27908448e-05\\
27.2385024	0.0198309384\\
27.2897024	0.0200522412\\
27.3409024	0.0201315996\\
27.3921024	0.0261339588\\
27.4433024	-4.0381488e-05\\
27.4945024	4.096404e-05\\
27.5457024	-3.17162844e-05\\
27.5969024	-3.8187936e-06\\
27.6481024	-2.28063204e-05\\
27.6993024	-3.6084096e-06\\
27.7505024	3.7052208e-05\\
27.8017024	8.5009248e-06\\
27.8529024	-4.8229488e-06\\
27.9041024	-4.2647112e-05\\
27.9553024	1.36742364e-05\\
28.0065024	-7.287228e-06\\
28.0577024	-1.03231512e-05\\
28.1089024	-2.94506532e-05\\
28.1601024	3.0185748e-05\\
28.2113024	1.9653048e-05\\
28.2625024	-2.02724712e-05\\
28.3137024	-2.67298884e-05\\
28.3649024	-2.72789028e-05\\
28.4161024	3.50086932e-05\\
28.4673024	4.0451076e-05\\
28.5185024	0.0200019852\\
28.5697024	0.0200021832\\
28.6209024	0.020016684\\
28.6721024	0.0266250456\\
28.7233024	-3.0280464e-05\\
28.7745024	4.599738e-05\\
28.8257024	3.14002872e-05\\
28.8769024	-4.2915348e-05\\
28.9281024	-1.29743496e-05\\
28.9793024	4.009986e-05\\
29.0305024	1.35216864e-05\\
29.0817024	5.0645124e-05\\
29.1329024	-2.25389376e-06\\
29.1841024	2.13461928e-05\\
29.2353024	-1.017144e-05\\
29.2865024	7.261074e-05\\
29.3377024	1.33934436e-05\\
29.3889024	-1.43514936e-05\\
29.4401024	-3.58502364e-05\\
29.4913024	-2.382975e-06\\
29.5425024	4.8017412e-05\\
29.5937024	-2.66016456e-05\\
29.6449024	-9.4631688e-07\\
29.6961024	-1.75869108e-05\\
29.7473024	1.25762076e-05\\
29.7985024	0.0199175004\\
29.8497024	0.0200014524\\
29.9009024	0.0199896948\\
29.9521024	0.0268038396\\
30.0033024	1.85432832e-05\\
30.0545024	2.14509672e-05\\
30.1057024	1.68032016e-05\\
30.1569024	2.23042428e-06\\
30.2081024	0.000183700296\\
30.2593024	-4.526226e-05\\
30.3105024	-1.67805684e-05\\
30.3617024	-2.49437052e-05\\
30.4129024	1.0590534e-05\\
30.4641024	1.09719108e-06\\
30.5153024	5.4008784e-05\\
30.5665024	3.44479464e-05\\
30.6177024	-2.20703904e-05\\
30.6689024	6.0021972e-05\\
30.7201024	3.10381884e-05\\
30.7713024	5.579496e-05\\
30.8225024	0.00020948472\\
30.8737024	-1.06383132e-05\\
30.9249024	4.8648564e-05\\
30.9761024	8.5596012e-06\\
31.0273024	-2.36595996e-05\\
31.0785024	-9.634158e-06\\
31.1297024	1.8169452e-05\\
31.1809024	-1.96304184e-05\\
31.2321024	-7.636752e-06\\
31.2833024	4.729068e-06\\
31.3345024	9.0418932e-05\\
31.3857024	3.6993528e-05\\
31.4369024	1.70714232e-05\\
31.4881024	3.02209524e-05\\
31.5393024	0.000159083496\\
31.5905024	2.641473e-05\\
31.6417024	3.42149292e-05\\
31.6929024	1.84619808e-05\\
31.7441024	8.6530572e-05\\
31.7953024	2.40200208e-05\\
31.8465024	1.17472356e-05\\
31.8977024	-3.25804572e-06\\
31.9489024	-3.1623246e-05\\
32.0001024	-4.740804e-06\\
32.0513024	-3.6119304e-05\\
32.1025024	-2.173176e-05\\
32.1537024	2.78857548e-05\\
32.2049024	3.7893744e-05\\
32.2561024	2.9134656e-05\\
32.3073024	3.0396132e-05\\
32.3585024	3.29299812e-05\\
32.4097024	3.45418236e-05\\
32.4609024	2.21868972e-06\\
32.5121024	4.6359468e-05\\
32.5633024	-6.3995832e-06\\
32.6145024	-8.664372e-06\\
32.6657024	-1.11169188e-05\\
32.7169024	-3.01329432e-06\\
32.7681024	2.86443144e-05\\
32.8193024	-3.08403756e-05\\
32.8705024	5.4312228e-05\\
32.9217024	-3.04765992e-06\\
32.9729024	-2.73250044e-06\\
33.0241024	4.4129052e-05\\
33.0753024	-2.50954164e-05\\
33.1265024	1.30665492e-05\\
33.1777024	-2.18256372e-05\\
33.2289024	1.21789044e-05\\
33.2801024	-7.9644852e-06\\
33.3313024	4.529664e-05\\
33.3825024	2.09371572e-05\\
33.4337024	-7.053372e-06\\
33.4849024	6.1902036e-05\\
33.5361024	1.99656936e-06\\
33.5873024	-1.81937592e-05\\
33.6385024	9.8671752e-06\\
33.6897024	3.11722992e-06\\
33.7409024	-3.9925512e-05\\
33.7921024	5.157972e-05\\
33.8433024	-3.59432784e-05\\
33.8945024	1.12334256e-05\\
33.9457024	1.06282548e-06\\
33.9969024	-2.96845092e-05\\
34.0481024	-1.5540048e-06\\
34.0993024	2.00729808e-05\\
34.1505024	5.2291332e-05\\
34.2017024	-4.8346812e-06\\
34.2529024	-2.49780708e-05\\
34.3041024	6.6970584e-05\\
34.3553024	-3.20197068e-05\\
34.4065024	2.19647772e-05\\
34.4577024	-9.6919956e-06\\
34.5089024	3.56398524e-05\\
34.5601024	0.000347455872\\
34.6113024	3.7870272e-05\\
34.6625024	1.4944932e-06\\
34.7137024	1.99212696e-05\\
34.7649024	-1.63371672e-05\\
34.8161024	4.4922816e-05\\
34.8673024	3.86523e-05\\
34.9185024	0.000262034208\\
34.9697024	-3.59667468e-06\\
35.0209024	-3.7613808e-05\\
35.0721024	-2.7162396e-05\\
35.1233024	1.44327996e-05\\
35.1745024	-2.5597494e-05\\
35.2257024	3.7612944e-05\\
35.2769024	1.95591708e-05\\
35.3281024	1.88123436e-05\\
35.3793024	-4.2504624e-06\\
35.4305024	-1.16776692e-05\\
35.4817024	-9.0734112e-06\\
35.5329024	2.84926032e-05\\
35.5841024	-1.15846272e-05\\
35.6353024	-1.81585548e-05\\
35.6865024	-1.33004052e-05\\
35.7377024	4.070754e-05\\
35.7889024	-1.15611588e-05\\
35.8401024	5.2981164e-05\\
35.8913024	2.04350796e-06\\
35.9425024	4.2506316e-05\\
35.9937024	6.4086372e-05\\
36.0449024	-9.8562816e-06\\
36.0961024	6.5269872e-06\\
36.1473024	3.10616568e-05\\
36.1985024	1.0076724e-05\\
36.2497024	-5.7566916e-06\\
36.3009024	-4.3149204e-05\\
36.3521024	-1.104651e-05\\
36.4033024	2.9065086e-05\\
36.4545024	-2.67298884e-05\\
36.5057024	9.303912e-08\\
36.5569024	2.8563012e-05\\
36.6081024	-2.42890776e-05\\
36.6593024	2.19061044e-05\\
36.7105024	-1.49826528e-05\\
36.7617024	0.000127787976\\
36.8129024	3.6421884e-05\\
36.8641024	2.96258364e-05\\
36.9153024	4.1256576e-05\\
36.9665024	-1.91627064e-05\\
37.0177024	2.82478536e-05\\
37.0689024	7.3977012e-05\\
37.1201024	-3.7882008e-05\\
37.1713024	-1.31143248e-05\\
37.2225024	-2.04359184e-05\\
37.2737024	-9.7280352e-06\\
37.3249024	-3.33515916e-05\\
37.3761024	-5.6049804e-06\\
37.4273024	8.5269096e-05\\
37.4785024	-6.0844248e-06\\
37.5297024	-1.09417356e-05\\
37.5809024	-1.149075e-05\\
37.6321024	4.8741588e-05\\
37.6833024	-1.9653048e-05\\
37.7345024	5.238522e-05\\
37.7857024	-4.401252e-05\\
37.8369024	3.7706004e-05\\
37.8881024	3.46843152e-06\\
37.9393024	5.5970172e-05\\
37.9905024	-9.1555524e-06\\
38.0417024	1.44914724e-05\\
38.0929024	1.3416912e-05\\
38.1441024	0.000301703256\\
38.1953024	4.1361336e-05\\
38.2465024	-3.9705084e-06\\
38.2977024	1.43749656e-05\\
38.3489024	1.4525838e-05\\
38.4001024	-1.38376836e-05\\
38.4513024	0.000108870012\\
38.5025024	-1.65123504e-05\\
38.5537024	2.52211464e-06\\
38.6049024	6.342084e-05\\
38.6561024	-2.11014396e-05\\
38.7073024	4.7772648e-05\\
38.7585024	-4.7714832e-05\\
38.8097024	-1.98399636e-05\\
38.8609024	1.631286e-05\\
38.9121024	4.0979124e-06\\
38.9633024	1.27513908e-05\\
39.0145024	4.1339556e-06\\
39.0657024	-8.3257452e-06\\
39.1169024	-3.22887672e-05\\
39.1681024	0.00011063358\\
39.2193024	8.6063688e-05\\
39.2705024	5.33592e-06\\
39.3217024	7.35093e-07\\
39.3729024	9.5050764e-06\\
39.4241024	2.40904296e-05\\
39.4753024	1.631286e-05\\
39.5265024	7.0407972e-06\\
39.5777024	1.88349732e-05\\
39.6289024	5.0703804e-05\\
39.6801024	-1.19693592e-05\\
39.7313024	-2.78748576e-05\\
39.7825024	-1.75517064e-05\\
39.8337024	3.26382912e-05\\
39.8849024	2.30510736e-05\\
39.9361024	-3.8512332e-05\\
39.9873024	-8.6417424e-07\\
40.0385024	4.0357188e-05\\
40.0897024	-1.6235748e-06\\
40.1409024	4.6674612e-05\\
40.1921024	-5.0450688e-06\\
40.2433024	7.0041708e-05\\
40.2945024	4.4538084e-05\\
40.3457024	1.32417324e-05\\
40.3969024	-1.87654032e-05\\
40.4481024	-2.53284336e-05\\
40.4993024	-5.2386048e-05\\
40.5505024	-3.52039932e-08\\
40.6017024	0.0237689316\\
40.6529024	0.0352633392\\
40.7041024	0.03475071\\
40.7553024	0.038573856\\
40.8065024	0.150915492\\
40.8577024	0.0282501036\\
40.9089024	0.0249537996\\
40.9601024	0.0308980944\\
41.0113024	0.49237056\\
41.0625024	0.0208366236\\
41.1137024	0.7087554\\
41.1649024	0.70438392\\
41.2161024	0.213219072\\
41.2673024	0.031039758\\
41.3185024	0.0249676632\\
41.3697024	0.0199263672\\
41.4209024	0.0246877056\\
41.4721024	0.259106472\\
41.5233024	0.028401246\\
41.5745024	0.211422996\\
41.6257024	0.222590556\\
41.6769024	0.0249585516\\
41.7281024	0.216969408\\
41.7793024	0.0250562808\\
41.8305024	0.214402824\\
41.8817024	0.204967368\\
41.9329024	0.0251224632\\
41.9841024	0.0250012944\\
42.0353024	0.0250479612\\
42.0865024	0.216435816\\
42.1377024	0.0252040752\\
42.1889024	0.025006572\\
42.2401024	0.0234830124\\
42.2913024	0.50628672\\
42.3425024	0.0278242308\\
42.3937024	0.7213896\\
42.4449024	0.0250898256\\
42.4961024	0.256245804\\
42.5473024	0.200483496\\
42.5985024	0.025014312\\
42.6497024	0.0205779348\\
42.7009024	0.0250394076\\
42.7521024	0.257317956\\
42.8033024	0.0280414188\\
42.8545024	0.215029116\\
42.9057024	0.240606612\\
42.9569024	0.02496078\\
43.0081024	0.2162124\\
43.0593024	0.0250720452\\
43.1105024	0.213630588\\
43.1617024	0.197622468\\
43.2129024	0.025209432\\
43.2641024	0.0207637632\\
43.3153024	0.0251222904\\
43.3665024	0.21116772\\
43.4177024	0.0223375212\\
43.4689024	0.0251562024\\
43.5201024	0.0207539568\\
43.5713024	0.0249911784\\
43.6225024	0.0249281784\\
43.6737024	0.0256452876\\
43.7249024	0.0250511688\\
43.7761024	0.258744276\\
43.8273024	0.199863324\\
43.8785024	0.0249711012\\
43.9297024	0.0335998764\\
43.9809024	0.0250047252\\
44.0321024	0.259521552\\
44.0833024	0.0283496364\\
44.1345024	0.207570348\\
44.1857024	0.245236932\\
44.2369024	0.0250508988\\
44.2881024	0.218184912\\
44.3393024	0.025380504\\
44.3905024	0.20673342\\
44.4417024	0.194773284\\
44.4929024	0.025151346\\
44.5441024	0.0250025148\\
44.5953024	0.0249987168\\
44.6465024	0.206025372\\
44.6977024	0.0265422996\\
44.7489024	0.0252769752\\
44.8001024	0.0248701032\\
44.8513024	0.024963732\\
44.9025024	0.0248968728\\
44.9537024	0.0259568352\\
45.0049024	0.025011162\\
45.0561024	0.260003628\\
45.1073024	0.199876356\\
45.1585024	0.0250652952\\
45.2097024	0.019967994\\
45.2609024	0.0250984692\\
45.3121024	0.260623692\\
45.3633024	0.0283997268\\
45.4145024	0.207298116\\
45.4657024	0.245870676\\
45.5169024	0.0251386704\\
45.5681024	0.218518056\\
45.6193024	0.0249492888\\
45.6705024	0.209956932\\
45.7217024	0.194605668\\
45.7729024	0.0249645204\\
45.8241024	0.0249896196\\
45.8753024	0.0250003908\\
45.9265024	0.213900372\\
45.9777024	0.0234063288\\
46.0289024	0.0251808948\\
46.0801024	0.0249995664\\
46.1313024	0.0253232424\\
46.1825024	0.0248536872\\
46.2337024	0.0234335628\\
46.2849024	0.0249253992\\
46.3361024	0.2610009\\
46.3873024	0.200819556\\
46.4385024	0.0251097732\\
46.4897024	0.0203538276\\
46.5409024	0.0248806188\\
46.5921024	0.261136008\\
46.6433024	0.0283485024\\
46.6945024	0.233512416\\
46.7457024	0.245194704\\
46.7969024	0.0250071048\\
46.8481024	0.219806568\\
46.8993024	0.0249655428\\
46.9505024	0.238044312\\
47.0017024	0.19396134\\
47.0529024	0.024915222\\
47.1041024	0.02454768\\
47.1553024	0.0251011404\\
47.2065024	0.239855292\\
47.2577024	0.0275245596\\
47.3089024	0.0251288748\\
47.3601024	0.0250846416\\
47.4113024	0.0250768224\\
47.4625024	0.0249492312\\
47.5137024	0.0240889896\\
47.5649024	0.0251528688\\
47.6161024	0.261371196\\
47.6673024	0.199791504\\
47.7185024	0.0248689404\\
47.7697024	0.0211933728\\
47.8209024	0.0251396712\\
47.8721024	0.262160856\\
47.9233024	0.0288355716\\
47.9745024	0.233811936\\
48.0257024	0.244437408\\
48.0769024	0.0250313544\\
48.1281024	0.219193488\\
48.1793024	0.0250138656\\
48.2305024	0.229652028\\
48.2817024	0.193446504\\
48.3329024	0.0251859276\\
48.3841024	0.0242562924\\
48.4353024	0.025034922\\
48.4865024	0.224154612\\
48.5377024	0.0276385248\\
48.5889024	0.0249833808\\
48.6401024	0.0207558396\\
48.6913024	0.0250919136\\
48.7425024	0.0249856092\\
48.7937024	0.0251777916\\
48.8449024	0.0250507368\\
48.8961024	0.262145628\\
48.9473024	0.19952352\\
48.9985024	0.0251896572\\
49.0497024	0.0216344268\\
49.1009024	0.0249697872\\
49.1521024	0.262220688\\
49.2033024	0.028406088\\
49.2545024	0.214270812\\
49.3057024	0.24061068\\
49.3569024	0.0249733512\\
49.4081024	0.221317092\\
49.4593024	0.0249037164\\
49.5105024	0.212406336\\
49.5617024	0.193782636\\
49.6129024	0.0250553592\\
49.6641024	0.0288417168\\
49.7153024	0.0250437744\\
49.7665024	0.211756356\\
49.8177024	0.0262041156\\
49.8689024	0.0251067888\\
49.9201024	0.0206385624\\
49.9713024	0.0250043724\\
50.0225024	0.0248790168\\
50.0737024	0.027664542\\
50.1249024	0.0250556652\\
50.1761024	0.2625822\\
50.2273024	0.200288016\\
50.2785024	0.024805008\\
50.3297024	0.0226017\\
50.3809024	0.0251095176\\
50.4321024	0.262350576\\
50.4833024	0.0283451364\\
50.5345024	0.222107904\\
50.5857024	0.249648876\\
50.6369024	0.0250437528\\
50.6881024	0.22104378\\
50.7393024	0.025081398\\
50.7905024	0.228362688\\
50.8417024	0.1948176\\
50.8929024	0.025153974\\
50.9441024	0.07169256\\
50.9953024	0.0250504092\\
51.0465024	0.232958016\\
51.0977024	0.0266505552\\
51.1489024	0.0250756236\\
};

\addplot[area legend,solid,draw=black,fill=black,fill opacity=0.1,forget plot]
table[row sep=crcr] {%
x	y\\
0	-0.1\\
0	0.8\\
21.685	0.8\\
21.685	-0.1\\
}--cycle;

\addplot[area legend,solid,draw=black,fill=black,fill opacity=0.1,forget plot]
table[row sep=crcr] {%
x	y\\
30.582	-0.1\\
30.582	0.8\\
40.563	0.8\\
40.563	-0.1\\
}--cycle;
\end{axis}

\draw [thick,decoration={brace,raise=-0.6cm},decorate]
   (5.3,7.9) -- (8.54,7.9) node [pos=0.5,anchor=north] {A};
\draw [thick,decoration={brace,raise=-0.6cm},decorate]
   (8.56,7.9) -- (9.86,7.9) node [pos=0.5,anchor=north] {B};
\draw [thick,decoration={brace,raise=-0.6cm},decorate]
   (9.88,7.9) -- (11.36,7.9) node [pos=0.5,anchor=north] {C};
\draw [thick,decoration={brace,raise=-0.6cm},decorate]
   (11.38,7.9) -- (12.95,7.9) node [pos=0.5,anchor=north] {D};
   
\end{tikzpicture}%}
\caption{Raw measurement of \gls{PSM}. Area A is connection phase, area B is DRX phase, area C is the PSM phase and area D is the tracking area update phase.}
\label{fig:PSM}
\end{figure}
\end{minipage}
\vspace{1em}
\begin{figure}[H]
\tikzsetnextfilename{eDRX1}
\centering
\begin{minipage}{0.48\textwidth}
\resizebox{\textwidth}{!}{
% This file was created by matlab2tikz.
%
%The latest updates can be retrieved from
%  http://www.mathworks.com/matlabcentral/fileexchange/22022-matlab2tikz-matlab2tikz
%where you can also make suggestions and rate matlab2tikz.
%
\definecolor{mycolor1}{rgb}{0.00000,0.44700,0.74100}%
%
\begin{tikzpicture}

\begin{axis}[%
width=\textwidth,
height=0.66\textwidth,
at={(2.08in,0.858in)},
scale only axis,
xmin=0,
xmax=51.1998976,
xlabel={Time [s]},
ymin=-0.1,
ymax=0.8,
ylabel={Power Consumption [W]},
axis background/.style={fill=white},
%title style={font=\bfseries},
%title={eDRXDefConfQuectel1}
]
\addplot [color=mycolor1,only marks,mark=*,mark options={solid},forget plot]
  table[row sep=crcr]{%
0.0001024	7.7029704e-07\\
0.0513024	1.24940664e-05\\
0.1025024	-2.18491056e-05\\
0.1537024	3.7333008e-05\\
0.2049024	-1.40715396e-05\\
0.2561024	7.1350128e-05\\
0.3073024	-2.9229372e-05\\
0.3585024	-2.80031004e-05\\
0.4097024	0.0199919736\\
0.4609024	0.102939084\\
0.5121024	0.103248576\\
0.5633024	0.106699824\\
0.6145024	0.108234252\\
0.6657024	0.113970996\\
0.7169024	0.11649294\\
0.7681024	0.119489112\\
0.8193024	0.125794224\\
0.8705024	0.14745726\\
0.9217024	0.154090044\\
0.9729024	0.020089962\\
1.0241024	0.019979046\\
1.0753024	0.20567358\\
1.1265024	0.25423038\\
1.1777024	0.152694432\\
1.2289024	0.200813436\\
1.2801024	0.204115356\\
1.3313024	0.203274576\\
1.3825024	0.203266476\\
1.4337024	0.0254531664\\
1.4849024	0.02071917\\
1.5361024	0.153060984\\
1.5873024	0.0208186056\\
1.6385024	0.0248926176\\
1.6897024	0.0208589364\\
1.7409024	0.0208804428\\
1.7921024	0.153088632\\
1.8433024	0.153103392\\
1.8945024	0.15318108\\
1.9457024	0.0205936272\\
1.9969024	0.0206871588\\
2.0481024	0.0206399484\\
2.0993024	0.020726964\\
2.1505024	0.0207921456\\
2.2017024	0.0208234728\\
2.2529024	0.0207169452\\
2.3041024	0.0207289044\\
2.3553024	0.0208390104\\
2.4065024	0.0207935208\\
2.4577024	0.0206852508\\
2.5089024	0.0207536436\\
2.5601024	0.0207819828\\
2.6113024	0.0246643092\\
2.6625024	0.0207312264\\
2.7137024	0.0206887248\\
2.7649024	0.0207711648\\
2.8161024	0.020741508\\
2.8673024	0.0207978516\\
2.9185024	0.0207806796\\
2.9697024	0.02071188\\
3.0209024	0.0208756332\\
3.0721024	0.0207344016\\
3.1233024	0.0207605556\\
3.1745024	0.0207926028\\
3.2257024	0.0207366912\\
3.2769024	0.0207922752\\
3.3281024	0.0207094068\\
3.3793024	0.0208351404\\
3.4305024	0.0207739512\\
3.4817024	0.0207603108\\
3.5329024	0.020797146\\
3.5841024	0.0250019316\\
3.6353024	0.0208000476\\
3.6865024	0.0207856764\\
3.7377024	0.02088918\\
3.7889024	0.0208038384\\
3.8401024	0.028506006\\
3.8913024	0.0280209384\\
3.9425024	0.0251486244\\
3.9937024	0.0208765188\\
4.0449024	0.0208602036\\
4.0961024	0.0208050084\\
4.1473024	0.02082897\\
4.1985024	0.020856006\\
4.2497024	0.0209245716\\
4.3009024	0.0208914192\\
4.3521024	0.0208030356\\
4.4033024	4.860162e-05\\
4.4545024	-4.6582416e-05\\
4.5057024	-1.30791204e-05\\
4.5569024	3.55342392e-05\\
4.6081024	8.465724e-06\\
4.6593024	-2.597049e-05\\
4.7105024	4.4514612e-05\\
4.7617024	-6.78582e-05\\
4.8129024	2.00964528e-05\\
4.8641024	2.62973844e-05\\
4.9153024	5.3343252e-05\\
4.9665024	0.020100132\\
5.0177024	0.0200740932\\
5.0689024	0.083704752\\
5.1201024	0.0277630308\\
5.1713024	0.027735606\\
5.2225024	0.0252309132\\
5.2737024	0.7101288\\
5.3249024	0.7056198\\
5.3761024	0.71106732\\
5.4273024	0.70630632\\
5.4785024	0.71202852\\
5.5297024	0.230119272\\
5.5809024	0.043383564\\
5.6321024	0.222184548\\
5.6833024	0.215447292\\
5.7345024	0.0335190564\\
5.7857024	0.0250802352\\
5.8369024	0.0251543052\\
5.8881024	0.222656976\\
5.9393024	0.0264803076\\
5.9905024	0.0252675972\\
6.0417024	0.217936224\\
6.0929024	0.0250614432\\
6.1441024	0.024842106\\
6.1953024	0.0239217588\\
6.2465024	0.0249073668\\
6.2977024	0.260980236\\
6.3489024	0.200060604\\
6.4001024	0.0234902088\\
6.4513024	0.0276438852\\
6.5025024	0.0307504872\\
6.5537024	0.7215498\\
6.6049024	0.323648244\\
6.6561024	0.237851856\\
6.7073024	0.249788916\\
6.7585024	0.0251300556\\
6.8097024	0.22033602\\
6.8609024	0.0253042632\\
6.9121024	0.24031638\\
6.9633024	0.193637052\\
7.0145024	0.0251965296\\
7.0657024	0.0244612728\\
7.1169024	0.024945732\\
7.1681024	0.239956632\\
7.2193024	0.025162686\\
7.2705024	0.0251992836\\
7.3217024	0.23252382\\
7.3729024	0.0250028748\\
7.4241024	0.0249630192\\
7.4753024	0.0244549476\\
7.5265024	0.0250219512\\
7.5777024	0.261335232\\
7.6289024	0.199260576\\
7.6801024	0.0233929548\\
7.7313024	0.027658638\\
7.7825024	0.03092229\\
7.8337024	0.72665388\\
7.8849024	0.32229342\\
7.9361024	0.230351652\\
7.9873024	0.243691632\\
8.0385024	0.025028388\\
8.0897024	0.220980024\\
8.1409024	0.0250960464\\
8.1921024	0.225455436\\
8.2433024	0.193112028\\
8.2945024	0.0250755012\\
8.3457024	0.0238707864\\
8.3969024	0.0250081704\\
8.4481024	0.22136724\\
8.4993024	0.0275557968\\
8.5505024	0.0251262468\\
8.6017024	0.212216652\\
8.6529024	0.0251337276\\
8.7041024	0.0249195852\\
8.7553024	0.0252014004\\
8.8065024	0.0251040888\\
8.8577024	0.262219284\\
8.9089024	0.199160352\\
8.9601024	0.023480262\\
9.0113024	0.0277658388\\
9.0625024	0.0306867456\\
9.1137024	0.70680996\\
9.1649024	0.323626392\\
9.2161024	0.7132644\\
9.2673024	0.243424404\\
9.3185024	0.0301940352\\
9.3697024	0.220417884\\
9.4209024	0.0248028444\\
9.4721024	0.2116323\\
9.5233024	0.193554036\\
9.5745024	0.0249675588\\
9.6257024	0.0265901724\\
9.6769024	0.0249975612\\
9.7281024	0.211396716\\
9.7793024	0.0262931724\\
9.8305024	0.0250708104\\
9.8817024	0.204676452\\
9.9329024	0.0249666804\\
9.9841024	0.024944724\\
10.0353024	0.0274927212\\
10.0865024	0.024913638\\
10.1377024	0.262284624\\
10.1889024	0.19165842\\
10.2401024	0.0235066032\\
10.2913024	0.0277248348\\
10.3425024	0.030685194\\
10.3937024	0.71024328\\
10.4449024	0.38793348\\
10.4961024	0.221931324\\
10.5473024	0.24930036\\
10.5985024	0.0249782148\\
10.6497024	0.221510304\\
10.7009024	0.0251134308\\
10.7521024	0.227558088\\
10.8033024	0.194492052\\
10.8545024	0.0251455212\\
10.9057024	0.07060302\\
10.9569024	0.0249953976\\
11.0081024	0.232209756\\
11.0593024	0.0278158212\\
11.1105024	0.0251230824\\
11.1617024	0.22743396\\
11.2129024	0.0249403572\\
11.2641024	0.0250531452\\
11.3153024	0.0263919852\\
11.3665024	0.0250739352\\
11.4177024	0.26249472\\
11.4689024	0.098069616\\
11.5201024	0.0250649064\\
11.5713024	0.0237451536\\
11.6225024	0.02496267\\
11.6737024	0.26228682\\
11.7249024	0.0284809356\\
11.7761024	0.240172812\\
11.8273024	0.25611696\\
11.8785024	0.0250774416\\
11.9297024	0.221157036\\
11.9809024	0.0249971184\\
12.0321024	0.23838282\\
12.0833024	0.19400508\\
12.1345024	0.0249748884\\
12.1857024	0.1111005\\
12.2369024	0.0249414624\\
12.2881024	0.23473764\\
12.3393024	0.0273118068\\
12.3905024	0.025295166\\
12.4417024	0.226800936\\
12.4929024	0.0249752052\\
12.5441024	0.0249785712\\
12.5953024	0.0266730372\\
12.6465024	0.0250844148\\
12.6977024	0.261697176\\
12.7489024	0.05347422\\
12.8001024	0.0248633712\\
12.8513024	0.0246768516\\
12.9025024	0.0250930368\\
12.9537024	0.262619316\\
13.0049024	0.0284961384\\
13.0561024	0.221588712\\
13.1073024	0.260311896\\
13.1585024	0.0249715584\\
13.2097024	0.221197716\\
13.2609024	0.0251860248\\
13.3121024	0.217806192\\
13.3633024	0.194090868\\
13.4145024	0.0251423604\\
13.4657024	0.126577332\\
13.5169024	0.0249991272\\
13.5681024	0.213997752\\
13.6193024	0.0275242572\\
13.6705024	0.0252109476\\
13.7217024	0.207031644\\
13.7729024	0.0248978664\\
13.8241024	0.02510514\\
13.8753024	0.0280696644\\
13.9265024	0.0250011864\\
13.9777024	0.262548036\\
14.0289024	0.0287131428\\
14.0801024	0.02485197\\
14.1313024	0.0273779316\\
14.1825024	0.0250964604\\
14.2337024	0.262822752\\
14.2849024	0.0288388008\\
14.3361024	0.209629152\\
14.3873024	0.261810756\\
14.4385024	0.025056342\\
14.4897024	0.221395392\\
14.5409024	0.0249345252\\
14.5921024	0.21062286\\
14.6433024	0.194736816\\
14.6945024	0.0250574112\\
14.7457024	0.131423724\\
14.7969024	0.0250040448\\
14.8481024	0.213145776\\
14.8993024	0.027405612\\
14.9505024	0.0252559008\\
15.0017024	0.210145392\\
15.0529024	0.0248484384\\
15.1041024	0.0248762664\\
15.1553024	0.0273844404\\
15.2065024	0.0250223328\\
15.2577024	0.258789744\\
15.3089024	0.198440316\\
15.3601024	0.025107012\\
15.4113024	0.026053686\\
15.4625024	0.0250775568\\
15.5137024	0.262652868\\
15.5649024	0.0283830228\\
15.6161024	0.229168224\\
15.6673024	0.261491148\\
15.7185024	0.025125534\\
15.7697024	0.221919948\\
15.8209024	0.0250911072\\
15.8721024	0.234169272\\
15.9233024	0.195330888\\
15.9745024	0.0250109208\\
16.0257024	0.176981832\\
16.0769024	0.0253520064\\
16.1281024	0.23712912\\
16.1793024	0.027729558\\
16.2305024	0.0251454168\\
16.2817024	0.23259654\\
16.3329024	0.024919002\\
16.3841024	0.0250163244\\
16.4353024	0.02756583\\
16.4865024	0.0250751052\\
16.5377024	0.2584926\\
16.5889024	0.0215791272\\
16.6401024	0.0235818\\
16.6913024	0.0310584312\\
16.7425024	0.0206883\\
16.7937024	0.0207709848\\
16.8449024	0.0249071724\\
16.8961024	0.020835972\\
16.9473024	0.0207620244\\
16.9985024	0.0207497916\\
17.0497024	0.0207797688\\
17.1009024	0.0206956656\\
17.1521024	0.0207864828\\
17.2033024	0.0207225432\\
17.2545024	0.0207333396\\
17.3057024	0.0207816012\\
17.3569024	0.0206705952\\
17.4081024	0.020738808\\
17.4593024	0.020756754\\
17.5105024	0.02070495\\
17.5617024	0.0207342684\\
17.6129024	0.0208696212\\
17.6641024	0.0206948556\\
17.7153024	0.0207305856\\
17.7665024	0.020771784\\
17.8177024	0.0207610128\\
17.8689024	-3.2206626e-05\\
17.9201024	-4.5834768e-05\\
17.9713024	3.9177864e-05\\
18.0225024	6.888582e-05\\
18.0737024	-2.57257368e-05\\
18.1249024	5.0090256e-06\\
18.1761024	8.0449524e-06\\
18.2273024	9.4229352e-06\\
18.2785024	1.54025856e-05\\
18.3297024	1.50538968e-06\\
18.3809024	1.82867976e-05\\
18.4321024	3.6398412e-05\\
18.4833024	3.9808188e-05\\
18.5345024	-3.47522076e-05\\
18.5857024	0.000171077148\\
18.6369024	4.529664e-05\\
18.6881024	2.99175264e-05\\
18.7393024	-3.27673728e-05\\
18.7905024	4.3451784e-05\\
18.8417024	2.53745352e-05\\
18.8929024	-5.842188e-07\\
18.9441024	2.5444944e-05\\
18.9953024	1.29265704e-05\\
19.0465024	-4.4409852e-05\\
19.0977024	-4.483062e-05\\
19.1489024	-4.5312552e-06\\
19.2001024	-1.00079928e-05\\
19.2513024	-5.3132868e-06\\
19.3025024	1.521567e-05\\
19.3537024	1.04044572e-05\\
19.4049024	-8.9921052e-06\\
19.4561024	2.05867908e-05\\
19.5073024	1.90453608e-05\\
19.5585024	-3.8641428e-05\\
19.6097024	-4.145688e-06\\
19.6609024	2.03655096e-05\\
19.7121024	4.751532e-05\\
19.7633024	4.6289052e-05\\
19.8145024	-3.45543948e-05\\
19.8657024	-5.613444e-05\\
19.9169024	-2.03303052e-05\\
19.9681024	3.8605356e-05\\
20.0193024	2.224557e-05\\
20.0705024	-1.4106744e-05\\
20.1217024	-1.11873276e-05\\
20.1729024	-1.91979108e-05\\
20.2241024	-3.9587724e-06\\
20.2753024	3.632886e-05\\
20.3265024	8.4891924e-06\\
20.3777024	-1.59046596e-05\\
20.4289024	5.0995512e-05\\
20.4801024	5.860962e-05\\
20.5313024	-2.09966688e-05\\
20.5825024	4.0182012e-05\\
20.6337024	-1.10976408e-06\\
20.6849024	-2.07510768e-05\\
20.7361024	1.22543424e-06\\
20.7873024	6.2235648e-06\\
20.8385024	2.3074542e-05\\
20.8897024	-3.20783832e-05\\
20.9409024	3.3854508e-06\\
20.9921024	-2.27945844e-05\\
21.0433024	6.1972452e-05\\
21.0945024	-7.5319776e-06\\
21.1457024	1.72591776e-05\\
21.1969024	5.2139628e-05\\
21.2481024	1.2028032e-06\\
21.2993024	3.53355876e-05\\
21.3505024	2.24668548e-05\\
21.4017024	-2.98831608e-05\\
21.4529024	2.15331084e-05\\
21.5041024	4.5063612e-05\\
21.5553024	9.6802596e-06\\
21.6065024	-1.19115216e-05\\
21.6577024	-2.58422472e-05\\
21.7089024	4.849686e-05\\
21.7601024	0.000118258596\\
21.8113024	2.89946808e-05\\
21.8625024	2.84221944e-05\\
21.9137024	3.6561852e-05\\
21.9649024	-8.4665592e-06\\
22.0161024	1.81350864e-05\\
22.0673024	2.08441152e-05\\
22.1185024	-3.4051482e-05\\
22.1697024	-1.6579404e-06\\
22.2209024	1.9956474e-05\\
22.2721024	-1.7060526e-05\\
22.3233024	7.7767308e-06\\
22.3745024	1.6324596e-05\\
22.4257024	3.3628194e-06\\
22.4769024	2.33016912e-07\\
22.5281024	-1.40145408e-06\\
22.5793024	4.4304228e-05\\
22.6305024	3.58024608e-05\\
22.6817024	-1.67923044e-05\\
22.7329024	-3.04263072e-07\\
22.7841024	3.15989352e-05\\
22.8353024	2.6034192e-06\\
22.8865024	4.9886568e-05\\
22.9377024	-4.5787824e-05\\
22.9889024	4.5273168e-05\\
23.0401024	-1.02301128e-05\\
23.0913024	-4.5437472e-05\\
23.1425024	-3.990204e-05\\
23.1937024	0.000168332076\\
23.2449024	3.8605356e-05\\
23.2961024	-8.214264e-08\\
23.3473024	-3.8862684e-05\\
23.3985024	2.30393376e-05\\
23.4497024	1.1572056e-05\\
23.5009024	-1.88123436e-05\\
23.5521024	4.2319404e-05\\
23.6033024	1.27396548e-05\\
23.6545024	-1.40128668e-05\\
23.7057024	1.7060526e-05\\
23.7569024	6.1283448e-05\\
23.8081024	9.2988828e-05\\
23.8593024	-4.1724288e-05\\
23.9105024	1.09526328e-05\\
23.9617024	3.50673696e-05\\
24.0129024	2.92285332e-05\\
24.0641024	5.8982616e-05\\
24.1153024	-2.16269856e-05\\
24.1665024	-5.9318712e-06\\
24.2177024	-5.3727984e-07\\
24.2689024	-2.13470316e-05\\
24.3201024	-1.89640548e-05\\
24.3713024	2.58187752e-05\\
24.4225024	-3.10977e-05\\
24.4737024	2.099583e-05\\
24.5249024	-8.4782952e-06\\
24.5761024	3.1377654e-05\\
24.6273024	6.525396e-05\\
24.6785024	3.8472948e-07\\
24.7297024	-3.55116096e-05\\
24.7809024	4.660506e-05\\
24.8321024	-8.4078864e-06\\
24.8833024	2.08089144e-05\\
24.9345024	0.000175210272\\
24.9857024	-7.789302e-06\\
25.0369024	-4.5425736e-05\\
25.0881024	0.00011821248\\
25.1393024	-1.98516996e-05\\
25.1905024	9.1195092e-06\\
25.2417024	4.9395384e-05\\
25.2929024	4.8519468e-05\\
25.3441024	3.1213368e-05\\
25.3953024	5.8027932e-06\\
25.4465024	1.92205404e-05\\
25.4977024	-6.9133932e-06\\
25.5489024	-1.94896008e-05\\
25.6001024	2.82822192e-05\\
25.6513024	4.7678796e-05\\
25.7025024	7.0290648e-06\\
25.7537024	1.95709068e-05\\
25.8049024	1.73756844e-05\\
25.8561024	2.27585448e-05\\
25.9073024	4.4666316e-05\\
25.9585024	0.0199496628\\
26.0097024	0.020042712\\
26.0609024	0.0273470688\\
26.1121024	0.20430252\\
26.1633024	4.9386168e-06\\
26.2145024	4.6100484e-08\\
26.2657024	5.2782516e-05\\
26.3169024	5.08203e-05\\
26.3681024	1.40128668e-05\\
26.4193024	-3.25108872e-05\\
26.4705024	1.98399636e-05\\
26.5217024	-1.15846272e-05\\
26.5729024	-3.963384e-05\\
26.6241024	1.96178436e-05\\
26.6753024	-3.48016608e-06\\
26.7265024	3.6386676e-05\\
26.7777024	0.000170026056\\
26.8289024	1.79708004e-05\\
26.8801024	2.94204816e-06\\
26.9313024	2.4639444e-05\\
26.9825024	3.46239648e-05\\
27.0337024	-4.8346812e-06\\
27.0849024	-6.1132572e-05\\
27.1361024	-3.9505572e-05\\
27.1873024	3.15159552e-07\\
27.2385024	0.019976724\\
27.2897024	0.0200798568\\
27.3409024	0.0200967876\\
27.3921024	0.205462908\\
27.4433024	3.8664036e-05\\
27.4945024	5.8338036e-07\\
27.5457024	-1.06282548e-06\\
27.5969024	3.8883636e-06\\
27.6481024	-2.64381984e-05\\
27.6993024	1.8671526e-05\\
27.7505024	1.88006076e-05\\
27.8017024	-7.648488e-06\\
27.8529024	-1.12694688e-05\\
27.9041024	-1.36164024e-05\\
27.9553024	3.2252724e-05\\
28.0065024	1.30665492e-05\\
28.0577024	4.3113168e-05\\
28.1089024	1.12216932e-05\\
28.1601024	-3.58619724e-05\\
28.2113024	-4.5986472e-05\\
28.2625024	6.1446888e-05\\
28.3137024	-5.6333088e-05\\
28.3649024	-7.6375908e-06\\
28.4161024	4.3521372e-05\\
28.4673024	1.30078764e-05\\
28.5185024	0.019892718\\
28.5697024	0.0200542176\\
28.6209024	0.0200350224\\
28.6721024	0.205180488\\
28.7233024	2.07267696e-05\\
28.7745024	7.986276e-06\\
28.8257024	-5.5784088e-05\\
28.8769024	4.493538e-05\\
28.9281024	-9.3458232e-07\\
28.9793024	-1.07456004e-06\\
29.0305024	-3.03969708e-05\\
29.0817024	-3.44839896e-05\\
29.1329024	2.66360112e-05\\
29.1841024	0.000189970812\\
29.2353024	-1.81007208e-05\\
29.2865024	1.4970078e-05\\
29.3377024	-2.72076588e-06\\
29.3889024	-2.11710096e-05\\
29.4401024	-4.2249816e-05\\
29.4913024	6.4319364e-05\\
29.5425024	-1.16307288e-05\\
29.5937024	-3.9505572e-05\\
29.6449024	2.82361176e-05\\
29.6961024	-1.2845268e-05\\
29.7473024	5.5654992e-05\\
29.7985024	0.0199372536\\
29.8497024	0.0201271572\\
29.9009024	0.0200915352\\
29.9521024	0.205534368\\
30.0033024	3.768336e-05\\
30.0545024	-9.9023796e-06\\
30.1057024	-9.2024928e-06\\
30.1569024	0.00021262626\\
30.2081024	-9.5059152e-06\\
30.2593024	-4.6360296e-06\\
30.3105024	-3.43322748e-05\\
30.3617024	-2.8481706e-05\\
30.4129024	2.53519056e-05\\
30.4641024	-2.76292692e-05\\
30.5153024	6.2649684e-05\\
30.5665024	4.5869112e-05\\
30.6177024	6.0793092e-05\\
30.6689024	-2.37878424e-05\\
30.7201024	-1.48359672e-06\\
30.7713024	-3.09459852e-05\\
30.8225024	2.6764254e-05\\
30.8737024	-1.25887788e-05\\
30.9249024	-2.10897072e-05\\
30.9761024	1.24127604e-05\\
31.0273024	3.7566e-05\\
31.0785024	-1.22627232e-06\\
31.1297024	-7.0651044e-06\\
31.1809024	9.3174912e-05\\
31.2321024	3.9749508e-05\\
31.2833024	-2.06697744e-06\\
31.3345024	-5.0096952e-05\\
31.3857024	2.22916716e-05\\
31.4369024	-1.156032e-05\\
31.4881024	-1.59281316e-05\\
31.5393024	4.5890928e-06\\
31.5905024	5.5281168e-05\\
31.6417024	-2.49898068e-05\\
31.6929024	4.1683212e-06\\
31.7441024	4.1676516e-05\\
31.7953024	1.68853428e-05\\
31.8465024	-1.95046884e-06\\
31.8977024	1.77847236e-05\\
31.9489024	5.5385928e-05\\
32.0001024	-1.47957372e-05\\
32.0513024	-2.74188816e-05\\
32.1025024	-7.6836924e-06\\
32.1537024	3.9563424e-05\\
32.2049024	3.41101548e-05\\
32.2561024	-1.65584484e-05\\
32.3073024	-3.5231652e-05\\
32.3585024	2.61691416e-05\\
32.4097024	-1.79833716e-05\\
32.4609024	3.05939448e-06\\
32.5121024	6.5499552e-05\\
32.5633024	-2.04350796e-06\\
32.6145024	-2.7911736e-06\\
32.6657024	1.99212696e-05\\
32.7169024	7.6602204e-06\\
32.7681024	3.15159552e-07\\
32.8193024	4.2284196e-05\\
32.8705024	3.23927028e-05\\
32.9217024	-1.73765232e-05\\
32.9729024	2.53049652e-05\\
33.0241024	-1.5122628e-05\\
33.0753024	-3.15763056e-05\\
33.1265024	6.2470332e-06\\
33.1777024	-1.84393476e-05\\
33.2289024	-1.0451394e-05\\
33.2801024	-9.3877308e-08\\
33.3313024	3.53355876e-05\\
33.3825024	-3.713184e-06\\
33.4337024	1.28561616e-05\\
33.4849024	-5.0809428e-05\\
33.5361024	1.76908464e-05\\
33.5873024	3.9344652e-06\\
33.6385024	-3.83841e-05\\
33.6897024	3.06643536e-05\\
33.7409024	2.9660202e-05\\
33.7921024	3.24631116e-06\\
33.8433024	-3.643614e-06\\
33.8945024	-3.14706924e-05\\
33.9457024	2.88077616e-05\\
33.9969024	-1.70261604e-05\\
34.0481024	-4.2039432e-05\\
34.0993024	1.53204444e-05\\
34.1505024	3.5685954e-05\\
34.2017024	1.12099572e-05\\
34.2529024	1.50748524e-05\\
34.3041024	6.2860068e-05\\
34.3553024	3.5685954e-05\\
34.4065024	-3.19032e-05\\
34.4577024	-1.413189e-06\\
34.5089024	4.4013384e-06\\
34.5601024	6.374772e-05\\
34.6113024	-1.29500388e-05\\
34.6625024	5.7395088e-05\\
34.7137024	4.9594032e-05\\
34.7649024	-4.7925216e-05\\
34.8161024	2.79092232e-05\\
34.8673024	9.2946924e-06\\
34.9185024	2.1813066e-05\\
34.9697024	3.9819924e-05\\
35.0209024	-1.66523256e-05\\
35.0721024	-3.27673728e-05\\
35.1233024	-1.45040436e-05\\
35.1745024	2.3062806e-05\\
35.2257024	1.34873208e-05\\
35.2769024	5.5991124e-07\\
35.3281024	-4.8065184e-05\\
35.3793024	0.000108531396\\
35.4305024	2.9963628e-05\\
35.4817024	5.6379204e-05\\
35.5329024	-6.014016e-06\\
35.5841024	-1.70144244e-05\\
35.6353024	-3.18562596e-05\\
35.6865024	-2.62512828e-05\\
35.7377024	2.18834712e-05\\
35.7889024	-3.8559276e-05\\
35.8401024	-1.8042048e-05\\
35.8913024	1.8917118e-05\\
35.9425024	-4.2249816e-05\\
35.9937024	-4.2974028e-06\\
36.0449024	3.45996576e-05\\
36.0961024	3.8033712e-05\\
36.1473024	4.924368e-05\\
36.1985024	-3.183363e-05\\
36.2497024	0.000188032896\\
36.3009024	-1.63019628e-05\\
36.3521024	6.2352972e-06\\
36.4033024	6.8594148e-05\\
36.4545024	-9.9149544e-06\\
36.5057024	-2.37291696e-05\\
36.5569024	3.0933414e-05\\
36.6081024	5.091336e-05\\
36.6593024	2.5947018e-05\\
36.7105024	0.000116203356\\
36.7617024	2.24090172e-05\\
36.8129024	-6.4530612e-05\\
36.8641024	2.41952004e-05\\
36.9153024	-9.4915836e-05\\
36.9665024	3.9270888e-05\\
37.0177024	5.5012932e-05\\
37.0689024	2.1952206e-06\\
37.1201024	-5.6146176e-05\\
37.1713024	3.44412396e-06\\
37.2225024	3.951648e-05\\
37.2737024	4.3451784e-05\\
37.3249024	1.73991564e-05\\
37.3761024	2.09136852e-05\\
37.4273024	-3.15880416e-05\\
37.4785024	6.678702e-06\\
37.5297024	2.9134656e-05\\
37.5809024	2.31675804e-05\\
37.6321024	-2.4838092e-05\\
37.6833024	-4.2160968e-06\\
37.7345024	5.4790812e-05\\
37.7857024	5.5596312e-05\\
37.8369024	2.28758904e-05\\
37.8881024	-1.28570004e-05\\
37.9393024	3.9213072e-05\\
37.9905024	-1.95046884e-06\\
38.0417024	2.47442148e-05\\
38.0929024	-2.74306176e-05\\
38.1441024	1.6838406e-05\\
38.1953024	5.0995512e-05\\
38.2465024	-6.3888552e-05\\
38.2977024	-3.7368216e-05\\
38.3489024	-2.28180564e-05\\
38.4001024	1.33004052e-05\\
38.4513024	-1.45501452e-05\\
38.5025024	1.52969724e-06\\
38.5537024	-2.2548996e-05\\
38.6049024	6.4222128e-06\\
38.6561024	3.8792304e-05\\
38.7073024	8.7243048e-05\\
38.7585024	2.64960324e-05\\
38.8097024	1.02519072e-05\\
38.8609024	-5.8621356e-05\\
38.9121024	-2.1725892e-06\\
38.9633024	1.24479648e-05\\
39.0145024	3.9890304e-05\\
39.0657024	-5.0412132e-05\\
39.1169024	3.1155534e-05\\
39.1681024	-3.7730304e-05\\
39.2193024	1.53087084e-05\\
39.2705024	2.63669544e-05\\
39.3217024	-1.71074628e-05\\
39.3729024	7.077762e-05\\
39.4241024	1.26466164e-05\\
39.4753024	-2.9202552e-06\\
39.5265024	-1.96304184e-05\\
39.5777024	5.1733116e-06\\
39.6289024	1.73924496e-06\\
39.6801024	-2.35657224e-05\\
39.7313024	-1.17480744e-05\\
39.7825024	5.8948236e-05\\
39.8337024	-6.8209416e-05\\
39.8849024	1.87184664e-05\\
39.9361024	1.09291644e-05\\
39.9873024	-2.05289568e-05\\
40.0385024	2.62739124e-05\\
40.0897024	-2.2701546e-05\\
40.1409024	1.93957236e-05\\
40.1921024	0.000114988824\\
40.2433024	-1.697922e-05\\
40.2945024	4.849686e-05\\
40.3457024	9.4003056e-06\\
40.3969024	9.924174e-07\\
40.4481024	-3.8883636e-06\\
40.4993024	1.54143216e-06\\
40.5505024	1.90336248e-05\\
40.6017024	-2.46168108e-05\\
40.6529024	8.0214804e-06\\
40.7041024	-6.4787076e-05\\
40.7553024	-2.25263664e-05\\
40.8065024	5.1905772e-05\\
40.8577024	-2.09731968e-05\\
40.9089024	1.09643688e-05\\
40.9601024	2.69629056e-05\\
41.0113024	0.000252447012\\
41.0625024	-4.8392064e-05\\
41.1137024	-9.5285484e-06\\
41.1649024	-1.52626068e-05\\
41.2161024	1.9442664e-05\\
41.2673024	4.5331848e-05\\
41.3185024	0.000114276348\\
41.3697024	3.0104442e-05\\
41.4209024	4.94658e-05\\
41.4721024	9.5285484e-06\\
41.5233024	0.0002468529\\
41.5745024	4.4689788e-05\\
41.6257024	4.401252e-05\\
41.6769024	1.92616128e-06\\
41.7281024	9.1312452e-06\\
41.7793024	-3.7216476e-05\\
41.8305024	-1.521567e-05\\
41.8817024	-1.39080924e-05\\
41.9329024	-6.7928616e-05\\
41.9841024	1.70714232e-05\\
42.0353024	7.7632344e-05\\
42.0865024	5.8983444e-05\\
42.1377024	1.75156632e-05\\
42.1889024	3.54529368e-05\\
42.2401024	3.57094224e-05\\
42.2913024	4.5471816e-05\\
42.3425024	1.59859656e-05\\
42.3937024	-4.0635468e-06\\
42.4449024	5.8924764e-05\\
42.4961024	-1.499355e-05\\
42.5473024	-1.6032906e-05\\
42.5985024	2.7055944e-05\\
42.6497024	-1.35107892e-05\\
42.7009024	1.73924496e-06\\
42.7521024	4.4782848e-05\\
42.8033024	-5.3612316e-05\\
42.8545024	-1.8264168e-05\\
42.9057024	-4.5612648e-05\\
42.9569024	8.5243968e-06\\
43.0081024	3.6468828e-05\\
43.0593024	5.719644e-05\\
43.1105024	-1.60454772e-05\\
43.1617024	-1.8018576e-05\\
43.2129024	-2.8049202e-05\\
43.2641024	3.9983364e-05\\
43.3153024	6.0513156e-05\\
43.3665024	-9.8914824e-06\\
43.4177024	-3.7022868e-06\\
43.4689024	-2.90424564e-05\\
43.5201024	7.8354036e-06\\
43.5713024	-1.64184732e-05\\
43.6225024	4.2284196e-05\\
43.6737024	1.53556452e-05\\
43.7249024	-4.3326072e-06\\
43.7761024	3.19023612e-05\\
43.8273024	-3.40053804e-05\\
43.8785024	2.47794192e-05\\
43.9297024	-3.6201456e-06\\
43.9809024	3.13298784e-05\\
44.0321024	2.39504508e-05\\
44.0833024	-1.36281348e-05\\
44.1345024	3.2790006e-05\\
44.1857024	3.6071532e-05\\
44.2369024	-1.22493132e-05\\
44.2881024	-6.186852e-05\\
44.3393024	0.000166581108\\
44.3905024	4.5109728e-05\\
44.4417024	-1.9081404e-05\\
44.4929024	1.49583456e-05\\
44.5441024	1.66967496e-06\\
44.5953024	1.46197152e-05\\
44.6465024	-2.36470248e-05\\
44.6977024	1.19106828e-06\\
44.7489024	1.65115116e-05\\
44.8001024	3.03726636e-05\\
44.8513024	-1.11873276e-05\\
44.9025024	0.000231590304\\
44.9537024	7.2981216e-06\\
45.0049024	1.72357092e-05\\
45.0561024	8.120556e-05\\
45.1073024	1.91627064e-05\\
45.1585024	-6.44652e-06\\
45.2097024	3.6526644e-05\\
45.2609024	-3.07238652e-05\\
45.3121024	2.12414184e-05\\
45.3633024	6.4903608e-05\\
45.4145024	-1.00540944e-05\\
45.4657024	3.45300876e-05\\
45.5169024	4.3965612e-05\\
45.5681024	-3.02217912e-05\\
45.6193024	-1.02879468e-05\\
45.6705024	5.0794344e-06\\
45.7217024	3.929436e-05\\
45.7729024	1.78190892e-05\\
45.8241024	-1.30841496e-06\\
45.8753024	2.9310678e-05\\
45.9265024	-1.60446384e-05\\
45.9777024	1.76908464e-05\\
46.0289024	-4.8801132e-05\\
46.0801024	0.000159410376\\
46.1313024	4.3253136e-05\\
46.1825024	-7.5437136e-06\\
46.2337024	0.000128616948\\
46.2849024	3.08982096e-05\\
46.3361024	3.832542e-05\\
46.3873024	4.9034124e-07\\
46.4385024	0.0209229516\\
46.4897024	0.0201117816\\
46.5409024	0.0295496748\\
46.5921024	0.194821128\\
46.6433024	1.46431836e-05\\
46.6945024	4.0427604e-05\\
46.7457024	7.206174e-05\\
46.7969024	-4.2285024e-05\\
46.8481024	4.6651176e-05\\
46.8993024	2.5444944e-05\\
46.9505024	2.12405796e-05\\
47.0017024	-5.8966704e-06\\
47.0529024	2.574837e-05\\
47.1041024	-3.6025416e-05\\
47.1553024	7.3442232e-06\\
47.2065024	5.205834e-05\\
47.2577024	1.8694998e-05\\
47.3089024	2.14828164e-06\\
47.3601024	4.0964868e-05\\
47.4113024	4.1800536e-06\\
47.4625024	1.90336248e-05\\
47.5137024	3.52308168e-05\\
47.5649024	6.9477588e-06\\
47.6161024	-3.28143132e-05\\
47.6673024	3.04765992e-06\\
47.7185024	0.0199704168\\
47.7697024	0.0200516256\\
47.8209024	0.0200953368\\
47.8721024	0.195674472\\
47.9233024	-4.5752616e-05\\
47.9745024	-1.36859724e-05\\
48.0257024	1.49583456e-05\\
48.0769024	-5.3635788e-05\\
48.1281024	3.8079828e-05\\
48.1793024	-2.64616668e-05\\
48.2305024	3.1002984e-05\\
48.2817024	-4.7181744e-06\\
48.3329024	2.34014364e-05\\
48.3841024	-1.88827524e-05\\
48.4353024	2.37283308e-05\\
48.4865024	-2.02255308e-05\\
48.5377024	1.12920984e-05\\
48.5889024	1.660455e-05\\
48.6401024	1.50622812e-06\\
48.6913024	4.718844e-05\\
48.7425024	-4.2039432e-05\\
48.7937024	-2.79209592e-05\\
48.8449024	1.87536708e-05\\
48.8961024	3.6363204e-05\\
48.9473024	7.0760016e-06\\
48.9985024	0.01993203\\
49.0497024	0.0199989108\\
49.1009024	0.0199553652\\
49.1521024	0.1953045\\
49.2033024	-4.846248e-05\\
49.2545024	-2.77943904e-06\\
49.3057024	2.99409948e-05\\
49.3569024	2.87381916e-05\\
49.4081024	4.5424872e-05\\
49.4593024	4.5180144e-05\\
49.5105024	8.6526396e-06\\
49.5617024	-6.2244e-06\\
49.6129024	2.94037164e-05\\
49.6641024	0.00016837902\\
49.7153024	-7.7775696e-06\\
49.7665024	-4.8120516e-06\\
49.8177024	4.7118852e-05\\
49.8689024	4.6312524e-05\\
49.9201024	0.000158301468\\
49.9713024	-2.20360248e-05\\
50.0225024	-6.5982348e-06\\
50.0737024	1.4164578e-05\\
50.1249024	1.04974956e-05\\
50.1761024	1.45928952e-06\\
50.2273024	3.6445356e-05\\
50.2785024	0.0198738972\\
50.3297024	0.0201281616\\
50.3809024	0.0200363364\\
50.4321024	0.195443928\\
50.4833024	3.42610308e-05\\
50.5345024	1.1280366e-05\\
50.5857024	9.4698756e-06\\
50.6369024	-1.34294868e-05\\
50.6881024	-8.289954e-05\\
50.7393024	2.49864516e-06\\
50.7905024	3.42727632e-05\\
50.8417024	-7.3802664e-06\\
50.8929024	2.06337312e-05\\
50.9441024	5.62761e-06\\
50.9953024	3.15754668e-05\\
51.0465024	6.7905144e-05\\
51.0977024	1.11630204e-05\\
51.1489024	4.574088e-05\\
};

\addplot[area legend,solid,draw=black,fill=black,fill opacity=0.1,forget plot]
table[row sep=crcr] {%
x	y\\
0	-0.1\\
0	0.8\\
17.87	0.8\\
17.87	-0.1\\
}--cycle;

\addplot[area legend,solid,draw=black,fill=black,fill opacity=0.1,forget plot]
table[row sep=crcr] {%
x	y\\
25.9366	-0.1\\
25.9366	0.8\\
29.992	0.8\\
29.992	-0.1\\
}--cycle;

\addplot[area legend,solid,draw=black,fill=black,fill opacity=0.1,forget plot]
table[row sep=crcr] {%
x	y\\
46.415	-0.1\\
46.415	0.8\\
50.4706	0.8\\
50.4706	-0.1\\
}--cycle;
\end{axis}

\draw [thick,decoration={brace,raise=-0.6cm},decorate]
   (5.3,7.9) -- (7.95,7.9) node [pos=0.5,anchor=north] {A};
\draw [thick,decoration={brace,raise=-0.6cm},decorate]
   (7.97,7.9) -- (9.16,7.9) node [pos=0.5,anchor=north] {B};
\draw [thick,decoration={brace,raise=-0.6cm},decorate]
   (9.18,7.9) -- (9.77,7.9) node [pos=0.5,anchor=north] {C};
\draw [thick,decoration={brace,raise=-0.6cm},decorate]
   (9.79,7.9) -- (12.25,7.9) node [pos=0.5,anchor=north] {D};
\draw [thick,decoration={brace,raise=-0.6cm},decorate]
   (12.26,7.9) -- (12.85,7.9) node [pos=0.5,anchor=north] {E};

\end{tikzpicture}%}
\caption{Raw measurement of \gls{eDRX}. Area A is connection phase, area B is the sleep phase of the first eDRX cycle, area C is the \gls{PTW} of the first eDRX cycle, area D is the sleep phase of the second eDRX cycle and area E is the \gls{PTW} of the second eDRX cycle.}
\label{fig:eDRX1}
\end{minipage}
\end{figure}

As can be seen in the raw measurements each mode has several phases. Phase A of each measurement can be omitted from the analysis since this is the attach phase, which has already been covered in \autoref{sec:performance_attach}. The average power consumption is found for each of the other areas following \autoref{eq:mean_over_phase}, for each area. The results can be seen in \autoref{tab:idle_results}.

\begin{table}[H]
\centering
\begin{tabular}{|c|c|c|c|c|} 
\multicolumn{1}{c}{ }	& \multicolumn{4}{c}{Area} \\ \hline
\textbf{Mode}	& \textbf{B}	& \textbf{C} 	& \textbf{D} 	& \textbf{E} \\ \hline
DRX				& 1.898 mW		& - 			& - 			& - \\ \hline
eDRX			& 3.653 $\mu$W	& 1.901 mW		& 5.738 $\mu$W	& 1.903 mW \\ \hline
PSM 			& 1.520 mW		& 3.919 $\mu$W	& 31.62 mW		& - \\ \hline 
\end{tabular}
\caption{Average power consumption of the individual areas for the the three idle measurements.}
\label{tab:idle_results}
\end{table}

It can be seen that the average power consumption of the DRX period is almost equal to the average power consumption of the PTW of the eDRX cycle, this is also expected. It can further be seen that the in the off periods of the eDRX cycle the Quectel device shuts down to the same state as in PSM mode. This could  mean one of two things, either the device has very bad hardware meaning the PSM power consumption is very high, or as in this case the hardware is very good and it can keep the timing sufficiently with only the timer needed to run PSM mode. This means that the power consumption during the eDRX idle periods can be counted as PSM period. This means that only two values is needed to describe the power consumption during the idle phase $P_{DRX}$ and $P_{PSM}$:
\begin{align}
P_{DRX} &= 1.901 [mW] \\
P_{PSM} &= 4.437 [\mu W]
\end{align}


\section{Battery Life Time Estimations}

Now that all the power values are known the battery lifetime can be estimated. To do this however the influence of some cell specific parameters as well as some assumptions of the transmit time is needed. As stated in \appref{app:bat_model} the model used to estimated the battery life time is:

\begin{equation}
L(t_i) = \frac{C_{bat}\cdot SF_{bat}}{8760\cdot (P_m(t_i) + P_{device})}
\end{equation}
\begin{where}
\va{$L(t_i)$}{is the expected lifetime of the battery}{years}
\va{$t_i$}{is the transmission time interval}{s}
\va{$C_{bat}$}{is the capacity of the battery}{Wh}
\va{$SF_{bat}$}{is the safety factor of the battery}{1}
\va{$P_m(t_i)$}{is the power consumption of the modem}{W}
\va{$P_{device}$}{is the power consumption of the IoT device}{W}
\end{where}

This equation is used under the assumption that $P_{device}$ is negligible, that there is no mobile terminated data occurring and setting $SF_{bat}$ to 0.5 and $C_{bat}$ to 5 Wh, giving:
\begin{equation}\label{eq:bat_est}
\begin{gathered}
	L(t_i) = \frac{1.0274}{P_m(t_i)} \\
	P_m(t_i) = \frac{E_{active} + E_{idle}}{t_i} \\
	E_{active} = E_{sync} + P_{attach}\cdot T_{attach} + P_{transmit}\cdot T_{transmit} + E_{release} \\
	E_{idle} = \begin{cases} P_{DRX}\cdot T_{DRX} + P_{PSM}\cdot (t_i - T_{attach} - T_{transmit} - T_{DRX}) \quad \text{for PSM}\\
							P_{DRX}\cdot T_{PTW} + P_{PSM} \cdot (T_{PSM} - T_{PTW}) \qquad\qquad\qquad\quad\;\;\, \text{for eDRX}
				\end{cases}
\end{gathered}
\end{equation}

From \autoref{eq:bat_est} a couple of problems become visible. First is that the time to sync and release of the devise from the system has not been taken into account. This however will only influence the estimation slightly as the time is essentially added to the idle time and only amounts to a few seconds. Second is that $E_{active}$ depends on two parameters one of which has not been measured or in any other way classified, namely $P_{TX}$ and $T_{transmit}$\todo{Refresh sentence, so Mads understands}. Because $T_{transmit}$ depends not only on the amount of data the devise wants to transmit, but also on some cell specific parameters such as repetition etc. will it be considered as a separate parameter for the battery lifetime estimation. As $P_{TX}$ depends on the channel condition for the device it is also as a separate parameter. The third problem is that $E_{idle}$ is depending on several things first is which idle mode is used second is what timer values are associated with the idle mode. Since this potentially has severe consequences for the power consumption the impact of this is also elaborated upon. The forth problem is also with the model used to estimate $E_{idle}$ for PSM mode it is assumed that T3412 is always bigger than $t_i$ meaning that only when the device has data will it wake up from PSM and for eDRX the energy is calculated based on a complete cycle which might be interrupted by the data transmission, this issue become less relevant with larger $t_i$ and smaller $T_{PSM}$. 

A common factor for all these problems is the model will underestimate the battery lifetime except in the case of T3412 being larger than $t_i$, however as the assumptions made here helps simplify the model it is considered to be reasonable. Another important step in this regard is that for all the measured parameters a statistical analysis was made, based on this analysis values are chosen such that there is a 95\% confidence that the device will life for the estimated time or in other words such that at least 95\% of the values are below the chosen value. To start the analysis of the battery lifetime four cases are chosen for idle mode eDRX and PSM respectively, the values for these cases can be seen in \autoref{tab:case_description}. This is done to visualize the impact of the chosen mode and timer values, a case where only DRX is used is not considered as this should only be relevant for very low $t_i$'s and then it is covered by eDRX and PSM already.

\begin{table}[H]
\centering
\begin{tabular}{|c|c|c|c|c|c|} \hline
		& $T_{DRX}$	& $T_{PSM}$	& $T_{PTW}$	& $T_{Transmit}$	& $P_{TX}$ 	\\ \hline
Case 1	& 2 s 		& 20.48 s	& 5.12 s	& 1 s				& 20 dBm	\\ \hline
Case 2	& 10 s		& 163.84 s	& 10.24 s	& 1 s				& 20 dBm	\\ \hline
Case 3	& 50 s		& 1310.72 s	& 20.48 s	& 1 s				& 20 dBm	\\ \hline
Case 4	& 250 s		& 10485.76 s& 40.96 s	& 1 s				& 20 dBm	\\ \hline
\end{tabular}
\caption{Timer settings for the four cases.}
\label{tab:case_description}
\end{table}

The expected battery lifetime for these four cases is found for different $t_i$'s as can be seen in \autoref{fig:eDRXvsPSM}.

\begin{figure}[H]
\centering
\tikzsetnextfilename{eDRXvsPSM}
\resizebox{0.7\textwidth}{!}{
% This file was created by matlab2tikz.
%
%The latest updates can be retrieved from
%  http://www.mathworks.com/matlabcentral/fileexchange/22022-matlab2tikz-matlab2tikz
%where you can also make suggestions and rate matlab2tikz.
%
\definecolor{mycolor1}{rgb}{0.00000,0.44700,0.74100}%
\definecolor{mycolor2}{rgb}{0.85000,0.32500,0.09800}%
\definecolor{mycolor3}{rgb}{0.92900,0.69400,0.12500}%
\definecolor{mycolor4}{rgb}{0.49400,0.18400,0.55600}%
%
\begin{tikzpicture}

\begin{axis}[%
width=\textwidth,
height=0.66\textwidth,
at={(0.758in,0.481in)},
scale only axis,
xmin=0,
xmax=25,
xlabel={$t_i$ [hours]},
xmajorgrids,
ymin=0,
ymax=12,
ylabel={Battery lifetime [years]},
ymajorgrids,
axis background/.style={fill=white},
legend style={at={(0.03,0.97)},anchor=north west,legend cell align=left,align=left,draw=white!15!black}
]
\addplot [color=mycolor1,solid]
  table[row sep=crcr]{%
0.000277777777777778	0.000152920330453197\\
0.250275	0.137485508387185\\
0.500272222222222	0.274232894892845\\
0.750269444444444	0.410398812385694\\
1.00026666666667	0.545986961728788\\
1.25026388888889	0.68100101244527\\
1.50026111111111	0.815444603049412\\
1.75025833333333	0.949321341373472\\
2.00025555555556	1.08263480489042\\
2.25025277777778	1.21538854103262\\
2.50025	1.34758606750645\\
2.75024722222222	1.47923087260307\\
3.00024444444444	1.61032641550518\\
3.25024166666667	1.74087612659009\\
3.50023888888889	1.87088340772884\\
3.75023611111111	2.00035163258178\\
4.00023333333333	2.12928414689036\\
4.25023055555556	2.2576842687653\\
4.50022777777778	2.3855552889713\\
4.750225	2.51290047120814\\
5.00022222222222	2.63972305238837\\
5.25021944444444	2.76602624291154\\
5.50021666666667	2.89181322693515\\
5.75021388888889	3.01708716264218\\
6.00021111111111	3.14185118250545\\
6.25020833333333	3.26610839354868\\
6.50020555555556	3.38986187760446\\
6.75020277777778	3.51311469156898\\
7.0002	3.63586986765377\\
7.25019722222222	3.75813041363436\\
7.50019444444444	3.87989931309593\\
7.75019166666667	4.00117952567603\\
8.00018888888889	4.12197398730434\\
8.25018611111111	4.24228561043961\\
8.50018333333333	4.36211728430372\\
8.75018055555556	4.48147187511297\\
9.00017777777778	4.6003522263066\\
9.250175	4.71876115877259\\
9.50017222222222	4.83670147107081\\
9.75016944444445	4.95417593965344\\
10.0001666666667	5.07118731908293\\
10.2501638888889	5.18773834224723\\
10.5001611111111	5.30383172057259\\
10.7501583333333	5.4194701442338\\
11.0001555555556	5.53465628236203\\
11.2501527777778	5.64939278325013\\
11.50015	5.76368227455566\\
11.7501472222222	5.87752736350144\\
12.0001444444444	5.99093063707382\\
12.2501416666667	6.10389466221866\\
12.5001388888889	6.21642198603496\\
12.7501361111111	6.32851513596638\\
13.0001333333333	6.44017661999038\\
13.2501305555556	6.55140892680535\\
13.5001277777778	6.66221452601545\\
13.750125	6.77259586831342\\
14.0001222222222	6.88255538566122\\
14.2501194444444	6.99209549146865\\
14.5001166666667	7.10121858076996\\
14.7501138888889	7.20992703039834\\
15.0001111111111	7.31822319915853\\
15.2501083333333	7.42610942799742\\
15.5001055555556	7.53358804017275\\
15.7501027777778	7.64066134141983\\
16.0001	7.74733162011646\\
16.2500972222222	7.85360114744597\\
16.5000944444444	7.95947217755837\\
16.7500916666667	8.06494694772976\\
17.0000888888889	8.17002767851992\\
17.2500861111111	8.27471657392816\\
17.5000833333333	8.37901582154739\\
17.7500805555556	8.4829275927165\\
18.0000777777778	8.58645404267105\\
18.250075	8.68959731069228\\
18.5000722222222	8.79235952025446\\
18.7500694444444	8.89474277917062\\
19.0000666666667	8.99674917973665\\
19.2500638888889	9.09838079887385\\
19.5000611111111	9.19963969826991\\
19.7500583333333	9.30052792451825\\
20.0000555555556	9.40104750925597\\
20.2500527777778	9.50120046930023\\
20.50005	9.60098880678307\\
20.7500472222222	9.7004145092849\\
21.0000444444444	9.79947954996641\\
21.2500416666667	9.89818588769913\\
21.5000388888889	9.99653546719454\\
21.7500361111111	10.0945302191318\\
22.0000333333333	10.192172060284\\
22.2500305555556	10.2894628936432\\
22.5000277777778	10.3864046085443\\
22.750025	10.4829990807868\\
23.0000222222222	10.5792481727562\\
23.2500194444444	10.6751537335437\\
23.5000166666667	10.7707175990643\\
23.7500138888889	10.8659415921743\\
24.0000111111111	10.9608275227871\\
24.2500083333333	11.0553771879877\\
24.5000055555556	11.1495923721463\\
24.7500027777778	11.2434748470306\\
25	11.3370263719168\\
};
\addlegendentry{Case 1};

\addplot [color=mycolor2,solid]
  table[row sep=crcr]{%
0.000277777777777778	0.000151689060355743\\
0.250275	0.136380859600234\\
0.500272222222222	0.272034184728727\\
0.750269444444444	0.407115307922963\\
1.00026666666667	0.541627841987517\\
1.25026388888889	0.675575369376499\\
1.50026111111111	0.808961442511606\\
1.75025833333333	0.941789584096164\\
2.00025555555556	1.07406328742525\\
2.25025277777778	1.20578601669194\\
2.50025	1.33696120728973\\
2.75024722222222	1.4675922661112\\
3.00024444444444	1.59768257184294\\
3.25024166666667	1.72723547525689\\
3.50023888888889	1.85625429949802\\
3.75023611111111	1.98474234036846\\
4.00023333333333	2.11270286660813\\
4.25023055555556	2.24013912017199\\
4.50022777777778	2.36705431650377\\
4.750225	2.49345164480646\\
5.00022222222222	2.61933426830942\\
5.25021944444444	2.74470532453227\\
5.50021666666667	2.86956792554558\\
5.75021388888889	2.99392515822834\\
6.00021111111111	3.11778008452237\\
6.25020833333333	3.24113574168363\\
6.50020555555556	3.36399514253048\\
6.75020277777778	3.486361275689\\
7.0002	3.60823710583528\\
7.25019722222222	3.72962557393487\\
7.50019444444444	3.85052959747938\\
7.75019166666667	3.97095207072015\\
8.00018888888889	4.09089586489925\\
8.25018611111111	4.21036382847768\\
8.50018333333333	4.32935878736081\\
8.75018055555556	4.44788354512128\\
9.00017777777778	4.56594088321909\\
9.250175	4.68353356121926\\
9.50017222222222	4.8006643170068\\
9.75016944444445	4.91733586699921\\
10.0001666666667	5.03355090635649\\
10.2501638888889	5.14931210918865\\
10.5001611111111	5.26462212876084\\
10.7501583333333	5.37948359769602\\
11.0001555555556	5.4938991281754\\
11.2501527777778	5.60787131213637\\
11.50015	5.72140272146828\\
11.7501472222222	5.83449590820588\\
12.0001444444444	5.94715340472055\\
12.2501416666667	6.05937772390933\\
12.5001388888889	6.17117135938173\\
12.7501361111111	6.28253678564447\\
13.0001333333333	6.39347645828401\\
13.2501305555556	6.50399281414706\\
13.5001277777778	6.61408827151902\\
13.750125	6.72376523030035\\
14.0001222222222	6.83302607218093\\
14.2501194444444	6.94187316081251\\
14.5001166666667	7.05030884197913\\
14.7501138888889	7.15833544376564\\
15.0001111111111	7.26595527672429\\
15.2501083333333	7.37317063403954\\
15.5001055555556	7.47998379169082\\
15.7501027777778	7.58639700861372\\
16.0001	7.69241252685912\\
16.2500972222222	7.79803257175071\\
16.5000944444444	7.90325935204069\\
16.7500916666667	8.00809506006374\\
17.0000888888889	8.11254187188926\\
17.2500861111111	8.21660194747193\\
17.5000833333333	8.32027743080065\\
17.7500805555556	8.42357045004572\\
18.0000777777778	8.52648311770452\\
18.250075	8.62901753074548\\
18.5000722222222	8.73117577075053\\
18.7500694444444	8.83295990405598\\
19.0000666666667	8.9343719818918\\
19.2500638888889	9.03541404051946\\
19.5000611111111	9.13608810136817\\
19.7500583333333	9.23639617116973\\
20.0000555555556	9.33634024209183\\
20.2500527777778	9.43592229186998\\
20.50005	9.53514428393789\\
20.7500472222222	9.63400816755662\\
21.0000444444444	9.73251587794211\\
21.2500416666667	9.83066933639156\\
21.5000388888889	9.92847045040827\\
21.7500361111111	10.0259211138252\\
22.0000333333333	10.1230232069274\\
22.2500305555556	10.2197785965725\\
22.5000277777778	10.3161891363109\\
22.750025	10.4122566665038\\
23.0000222222222	10.5079830144402\\
23.2500194444444	10.6033699944533\\
23.5000166666667	10.6984194080345\\
23.7500138888889	10.793133043947\\
24.0000111111111	10.8875126783383\\
24.2500083333333	10.9815600748509\\
24.5000055555556	11.0752769847322\\
24.7500027777778	11.1686651469434\\
25	11.261726288267\\
};
\addlegendentry{Case 2};

\addplot [color=mycolor3,solid]
  table[row sep=crcr]{%
0.000277777777777778	0.000145818614636649\\
0.250275	0.131113596279814\\
0.500272222222222	0.261549106364857\\
0.750269444444444	0.391455587081296\\
1.00026666666667	0.520836250426298\\
1.25026388888889	0.649694282447398\\
1.50026111111111	0.778032843504026\\
1.75025833333333	0.905855068525886\\
2.00025555555556	1.03316406726821\\
2.25025277777778	1.15996292456395\\
2.50025	1.28625470057296\\
2.75024722222222	1.41204243102816\\
3.00024444444444	1.53732912747879\\
3.25024166666667	1.66211777753072\\
3.50023888888889	1.786411345084\\
3.75023611111111	1.91021277056744\\
4.00023333333333	2.03352497117054\\
4.25023055555556	2.1563508410726\\
4.50022777777778	2.27869325166915\\
4.750225	2.40055505179573\\
5.00022222222222	2.52193906794898\\
5.25021944444444	2.64284810450521\\
5.50021666666667	2.76328494393635\\
5.75021388888889	2.88325234702341\\
6.00021111111111	3.00275305306739\\
6.25020833333333	3.12178978009787\\
6.50020555555556	3.24036522507899\\
6.75020277777778	3.35848206411318\\
7.0002	3.47614295264248\\
7.25019722222222	3.59335052564752\\
7.50019444444444	3.71010739784425\\
7.75019166666667	3.82641616387835\\
8.00018888888889	3.94227939851746\\
8.25018611111111	4.05769965684113\\
8.50018333333333	4.17267947442871\\
8.75018055555556	4.28722136754497\\
9.00017777777778	4.40132783332368\\
9.250175	4.51500134994908\\
9.50017222222222	4.62824437683531\\
9.75016944444445	4.74105935480373\\
10.0001666666667	4.85344870625833\\
10.2501638888889	4.96541483535912\\
10.5001611111111	5.0769601281935\\
10.7501583333333	5.18808695294586\\
11.0001555555556	5.29879766006507\\
11.2501527777778	5.40909458243026\\
11.50015	5.5189800355147\\
11.7501472222222	5.62845631754775\\
12.0001444444444	5.73752570967517\\
12.2501416666667	5.84619047611753\\
12.5001388888889	5.95445286432687\\
12.7501361111111	6.06231510514168\\
13.0001333333333	6.16977941294011\\
13.2501305555556	6.27684798579153\\
13.5001277777778	6.38352300560634\\
13.750125	6.48980663828427\\
14.0001222222222	6.59570103386089\\
14.2501194444444	6.70120832665264\\
14.5001166666667	6.8063306354002\\
14.7501138888889	6.91107006341032\\
15.0001111111111	7.01542869869611\\
15.2501083333333	7.1194086141158\\
15.5001055555556	7.22301186750993\\
15.7501027777778	7.32624050183718\\
16.0001	7.42909654530862\\
16.2500972222222	7.53158201152053\\
16.5000944444444	7.63369889958583\\
16.7500916666667	7.73544919426405\\
17.0000888888889	7.83683486608991\\
17.2500861111111	7.93785787150058\\
17.5000833333333	8.03852015296144\\
17.7500805555556	8.13882363909065\\
18.0000777777778	8.23877024478229\\
18.250075	8.33836187132825\\
18.5000722222222	8.43760040653875\\
18.7500694444444	8.53648772486165\\
19.0000666666667	8.63502568750045\\
19.2500638888889	8.73321614253109\\
19.5000611111111	8.83106092501747\\
19.7500583333333	8.92856185712575\\
20.0000555555556	9.02572074823749\\
20.2500527777778	9.12253939506158\\
20.50005	9.21901958174499\\
20.7500472222222	9.31516307998232\\
21.0000444444444	9.41097164912434\\
21.2500416666667	9.5064470362852\\
21.5000388888889	9.60159097644872\\
21.7500361111111	9.69640519257345\\
22.0000333333333	9.79089139569663\\
22.2500305555556	9.88505128503722\\
22.5000277777778	9.97888654809765\\
22.750025	10.0723988607647\\
23.0000222222222	10.1655898874093\\
23.2500194444444	10.2584612809851\\
23.5000166666667	10.3510146831266\\
23.7500138888889	10.4432517242452\\
24.0000111111111	10.5351740236257\\
24.2500083333333	10.6267831895206\\
24.5000055555556	10.718080819244\\
24.7500027777778	10.8090684992643\\
25	10.8997478052964\\
};
\addlegendentry{Case 3};

\addplot [color=mycolor4,solid]
  table[row sep=crcr]{%
0.000277777777777778	0.00012217703908731\\
0.250275	0.109892424286448\\
0.500272222222222	0.219288635809301\\
0.750269444444444	0.328312720113509\\
1.00026666666667	0.436966572742852\\
1.25026388888889	0.545252076388894\\
1.50026111111111	0.653171100999718\\
1.75025833333333	0.760725503887568\\
2.00025555555556	0.867917129835402\\
2.25025277777778	0.974747811202362\\
2.50025	1.08121936802819\\
2.75024722222222	1.18733360813658\\
3.00024444444444	1.2930923272375\\
3.25024166666667	1.39849730902849\\
3.50023888888889	1.50355032529495\\
3.75023611111111	1.60825313600938\\
4.00023333333333	1.71260748942972\\
4.25023055555556	1.8166151221966\\
4.50022777777778	1.92027775942971\\
4.750225	2.02359711482318\\
5.00022222222222	2.12657489074\\
5.25021944444444	2.22921277830552\\
5.50021666666667	2.33151245750003\\
5.75021388888889	2.43347559725042\\
6.00021111111111	2.53510385552096\\
6.25020833333333	2.6363988794031\\
6.50020555555556	2.73736230520454\\
6.75020277777778	2.83799575853726\\
7.0002	2.93830085440482\\
7.25019722222222	3.03827919728873\\
7.50019444444444	3.13793238123403\\
7.75019166666667	3.23726198993396\\
8.00018888888889	3.3362695968139\\
8.25018611111111	3.43495676511443\\
8.50018333333333	3.5333250479736\\
8.75018055555556	3.63137598850843\\
9.00017777777778	3.72911111989557\\
9.250175	3.82653196545124\\
9.50017222222222	3.92364003871038\\
9.75016944444445	4.02043684350497\\
10.0001666666667	4.11692387404176\\
10.2501638888889	4.21310261497905\\
10.5001611111111	4.30897454150292\\
10.7501583333333	4.40454111940261\\
11.0001555555556	4.49980380514523\\
11.2501527777778	4.59476404594978\\
11.50015	4.68942327986037\\
11.7501472222222	4.78378293581889\\
12.0001444444444	4.87784443373685\\
12.2501416666667	4.97160918456666\\
12.5001388888889	5.06507859037212\\
12.7501361111111	5.15825404439835\\
13.0001333333333	5.25113693114099\\
13.2501305555556	5.34372862641479\\
13.5001277777778	5.43603049742155\\
13.750125	5.52804390281736\\
14.0001222222222	5.61977019277934\\
14.2501194444444	5.71121070907163\\
14.5001166666667	5.80236678511084\\
14.7501138888889	5.89323974603083\\
15.0001111111111	5.98383090874696\\
15.2501083333333	6.07414158201971\\
15.5001055555556	6.16417306651766\\
15.7501027777778	6.25392665487996\\
16.0001	6.34340363177822\\
16.2500972222222	6.43260527397775\\
16.5000944444444	6.52153285039833\\
16.7500916666667	6.61018762217437\\
17.0000888888889	6.69857084271452\\
17.2500861111111	6.78668375776073\\
17.5000833333333	6.8745276054468\\
17.7500805555556	6.96210361635638\\
18.0000777777778	7.0494130135804\\
18.250075	7.13645701277403\\
18.5000722222222	7.22323682221312\\
18.7500694444444	7.30975364285005\\
19.0000666666667	7.3960086683692\\
19.2500638888889	7.48200308524182\\
19.5000611111111	7.56773807278041\\
19.7500583333333	7.65321480319265\\
20.0000555555556	7.73843444163484\\
20.2500527777778	7.82339814626481\\
20.50005	7.90810706829441\\
20.7500472222222	7.99256235204149\\
21.0000444444444	8.07676513498144\\
21.2500416666667	8.16071654779824\\
21.5000388888889	8.2444177144351\\
21.7500361111111	8.32786975214459\\
22.0000333333333	8.41107377153836\\
22.2500305555556	8.49403087663641\\
22.5000277777778	8.57674216491592\\
22.750025	8.65920872735966\\
23.0000222222222	8.74143164850392\\
23.2500194444444	8.82341200648611\\
23.5000166666667	8.90515087309183\\
23.7500138888889	8.98664931380161\\
24.0000111111111	9.0679083878372\\
24.2500083333333	9.14892914820744\\
24.5000055555556	9.22971264175379\\
24.7500027777778	9.31025990919535\\
25	9.39057198517364\\
};
\addlegendentry{Case 4};

\addplot [color=mycolor1,dashed,forget plot]
  table[row sep=crcr]{%
0.000277777777777778	0.000153233017096898\\
0.250275	0.137275381729568\\
0.500272222222222	0.272844996089213\\
0.750269444444444	0.406888294993121\\
1.00026666666667	0.539430910270325\\
1.25026388888889	0.670497903021654\\
1.50026111111111	0.800113779416866\\
1.75025833333333	0.928302505969974\\
2.00025555555556	1.05508752431276\\
2.25025277777778	1.18049176548559\\
2.50025	1.30453766376391\\
2.75024722222222	1.42724717003777\\
3.00024444444444	1.54864176476133\\
3.25024166666667	1.66874247048823\\
3.50023888888889	1.7875698640083\\
3.75023611111111	1.90514408810028\\
4.00023333333333	2.02148486291461\\
4.25023055555556	2.13661149699994\\
4.50022777777778	2.25054289798607\\
4.750225	2.36329758293603\\
5.00022222222222	2.47489368837894\\
5.25021944444444	2.58534898003525\\
5.50021666666667	2.69468086224519\\
5.75021388888889	2.80290638711103\\
6.00021111111111	2.91004226336319\\
6.25020833333333	3.01610486495989\\
6.50020555555556	3.1211102394296\\
6.75020277777778	3.22507411596535\\
7.0002	3.32801191327928\\
7.25019722222222	3.42993874722578\\
7.50019444444444	3.53086943820123\\
7.75019166666667	3.63081851832768\\
8.00018888888889	3.72980023842808\\
8.25018611111111	3.8278285747999\\
8.50018333333333	3.92491723579392\\
8.75018055555556	4.02107966820484\\
9.00017777777778	4.11632906347974\\
9.250175	4.21067836375062\\
9.50017222222222	4.3041402676967\\
9.75016944444445	4.39672723624204\\
10.0001666666667	4.48845149809397\\
10.2501638888889	4.57932505512734\\
10.5001611111111	4.66935968761969\\
10.7501583333333	4.75856695934214\\
11.0001555555556	4.84695822251046\\
11.2501527777778	4.93454462260102\\
11.50015	5.02133710303571\\
11.7501472222222	5.10734640973999\\
12.0001444444444	5.19258309557826\\
12.2501416666667	5.27705752467005\\
12.5001388888889	5.36077987659107\\
12.7501361111111	5.44376015046247\\
13.0001333333333	5.52600816893188\\
13.2501305555556	5.60753358204953\\
13.5001277777778	5.68834587104259\\
13.750125	5.768454351991\\
14.0001222222222	5.84786817940759\\
14.2501194444444	5.92659634972547\\
14.5001166666667	6.00464770469556\\
14.7501138888889	6.08203093469678\\
15.0001111111111	6.15875458196163\\
15.2501083333333	6.23482704371966\\
15.5001055555556	6.31025657526121\\
15.7501027777778	6.38505129292379\\
16.0001	6.45921917700342\\
16.2500972222222	6.53276807459304\\
16.5000944444444	6.60570570235014\\
16.7500916666667	6.67803964919572\\
17.0000888888889	6.74977737894645\\
17.2500861111111	6.82092623288207\\
17.5000833333333	6.89149343224977\\
17.7500805555556	6.96148608070744\\
18.0000777777778	7.03091116670744\\
18.250075	7.09977556582266\\
18.5000722222222	7.16808604301643\\
18.7500694444444	7.23584925485789\\
19.0000666666667	7.30307175168426\\
19.2500638888889	7.36975997971167\\
19.5000611111111	7.43592028309574\\
19.7500583333333	7.50155890594353\\
20.0000555555556	7.56668199427799\\
20.2500527777778	7.63129559795632\\
20.50005	7.69540567254358\\
20.7500472222222	7.7590180811425\\
21.0000444444444	7.82213859618098\\
21.2500416666667	7.88477290115823\\
21.5000388888889	7.94692659235071\\
21.7500361111111	8.00860518047897\\
22.0000333333333	8.06981409233636\\
22.2500305555556	8.1305586723808\\
22.5000277777778	8.19084418429025\\
22.750025	8.25067581248327\\
23.0000222222222	8.31005866360526\\
23.2500194444444	8.36899776798131\\
23.5000166666667	8.42749808103675\\
23.7500138888889	8.48556448468603\\
24.0000111111111	8.54320178869072\\
24.2500083333333	8.60041473198766\\
24.5000055555556	8.6572079839878\\
24.7500027777778	8.71358614584657\\
25	8.76955375170649\\
};
\addplot [color=mycolor2,dashed,forget plot]
  table[row sep=crcr]{%
0.000277777777777778	0.000153237801541707\\
0.250275	0.135828347965958\\
0.500272222222222	0.267176303433843\\
0.750269444444444	0.394400867039551\\
1.00026666666667	0.517693205916842\\
1.25026388888889	0.637232849948952\\
1.50026111111111	0.753188564012513\\
1.75025833333333	0.865719142927649\\
2.00025555555556	0.974974136988905\\
2.25025277777778	1.08109451504793\\
2.50025	1.18421327133014\\
2.75024722222222	1.28445598147792\\
3.00024444444444	1.38194131270855\\
3.25024166666667	1.47678149244441\\
3.50023888888889	1.56908273930674\\
3.75023611111111	1.65894565995251\\
4.00023333333333	1.74646561487159\\
4.25023055555556	1.83173305594016\\
4.50022777777778	1.9148338382423\\
4.750225	1.99584950841943\\
5.00022222222222	2.07485757158351\\
5.25021944444444	2.15193173863034\\
5.50021666666667	2.22714215561183\\
5.75021388888889	2.30055561666752\\
6.00021111111111	2.37223576187393\\
6.25020833333333	2.4422432612436\\
6.50020555555556	2.51063598599188\\
6.75020277777778	2.57746916808764\\
7.0002	2.64279554901246\\
7.25019722222222	2.70666551857031\\
7.50019444444444	2.76912724451561\\
7.75019166666667	2.83022679370033\\
8.00018888888889	2.89000824538067\\
8.25018611111111	2.94851379726893\\
8.50018333333333	3.00578386486666\\
8.75018055555556	3.06185717457073\\
9.00017777777778	3.1167708510028\\
9.250175	3.17056049897619\\
9.50017222222222	3.22326028048046\\
9.75016944444445	3.27490298703328\\
10.0001666666667	3.32552010772189\\
10.2501638888889	3.37514189323055\\
10.5001611111111	3.42379741612769\\
10.7501583333333	3.47151462766523\\
11.0001555555556	3.51832041132311\\
11.2501527777778	3.56424063331472\\
11.50015	3.60930019025246\\
11.7501472222222	3.65352305415812\\
12.0001444444444	3.69693231498891\\
12.2501416666667	3.73955022083778\\
12.5001388888889	3.78139821595502\\
12.7501361111111	3.82249697672765\\
13.0001333333333	3.86286644574353\\
13.2501305555556	3.90252586405795\\
13.5001277777778	3.9414938017725\\
13.750125	3.97978818702832\\
14.0001222222222	4.01742633350877\\
14.2501194444444	4.05442496654024\\
14.5001166666667	4.09080024787364\\
14.7501138888889	4.1265677992239\\
15.0001111111111	4.16174272463934\\
15.2501083333333	4.19633963176831\\
15.5001055555556	4.23037265208595\\
15.7501027777778	4.26385546013991\\
16.0001	4.29680129187017\\
16.2500972222222	4.32922296205445\\
16.5000944444444	4.36113288092748\\
16.7500916666667	4.39254307001954\\
17.0000888888889	4.42346517725663\\
17.2500861111111	4.4539104913621\\
17.5000833333333	4.48388995559724\\
17.7500805555556	4.51341418087596\\
18.0000777777778	4.54249345828638\\
18.250075	4.57113777105073\\
18.5000722222222	4.59935680595237\\
18.7500694444444	4.62715996425772\\
19.0000666666667	4.65455637215876\\
19.2500638888889	4.68155489076059\\
19.5000611111111	4.70816412563699\\
19.7500583333333	4.73439243597549\\
20.0000555555556	4.76024794333261\\
20.2500527777778	4.78573854001837\\
20.50005	4.81087189712822\\
20.7500472222222	4.83565547223967\\
21.0000444444444	4.86009651678985\\
21.2500416666667	4.88420208314917\\
21.5000388888889	4.90797903140576\\
21.7500361111111	4.93143403587433\\
22.0000333333333	4.95457359134235\\
22.2500305555556	4.97740401906596\\
22.5000277777778	4.99993147252713\\
22.750025	5.02216194296318\\
23.0000222222222	5.04410126467894\\
23.2500194444444	5.0657551201517\\
23.5000166666667	5.08712904493802\\
23.7500138888889	5.1082284323916\\
24.0000111111111	5.12905853820039\\
24.2500083333333	5.14962448475122\\
24.5000055555556	5.16993126532928\\
24.7500027777778	5.18998374815995\\
25	5.20978668029968\\
};
\addplot [color=mycolor3,dashed,forget plot]
  table[row sep=crcr]{%
0.000277777777777778	0.000153256942309041\\
0.250275	0.130332936977038\\
0.500272222222222	0.246676282527326\\
0.750269444444444	0.351277901449182\\
1.00026666666667	0.445829922818649\\
1.25026388888889	0.531714236326446\\
1.50026111111111	0.610070476313885\\
1.75025833333333	0.681846879264268\\
2.00025555555556	0.747838845528813\\
2.25025277777778	0.808718539140944\\
2.50025	0.865057865026214\\
2.75024722222222	0.917346490146441\\
3.00024444444444	0.966006112463573\\
3.25024166666667	1.01140185860314\\
3.50023888888889	1.05385146242395\\
3.75023611111111	1.09363271271169\\
4.00023333333333	1.13098953919839\\
4.25023055555556	1.16613701876899\\
4.50022777777778	1.19926551895217\\
4.750225	1.23054414730243\\
5.00022222222222	1.26012363864376\\
5.25021944444444	1.28813878422769\\
5.50021666666667	1.31471048541344\\
5.75021388888889	1.33994749788099\\
6.00021111111111	1.36394791944942\\
6.25020833333333	1.38680046442006\\
6.50020555555556	1.40858555934467\\
6.75020277777778	1.42937628874684\\
7.0002	1.44923921423258\\
7.25019722222222	1.46823508633381\\
7.50019444444444	1.48641946512318\\
7.75019166666667	1.50384326295522\\
8.00018888888889	1.52055322050043\\
8.25018611111111	1.53659232544545\\
8.50018333333333	1.55200018175712\\
8.75018055555556	1.56681333618876\\
9.00017777777778	1.58106556769537\\
9.250175	1.59478814458238\\
9.50017222222222	1.60801005350813\\
9.75016944444445	1.62075820387001\\
10.0001666666667	1.63305761060695\\
10.2501638888889	1.64493155803155\\
10.5001611111111	1.65640174694943\\
10.7501583333333	1.66748842702167\\
11.0001555555556	1.67821051606888\\
11.2501527777778	1.68858570779549\\
11.50015	1.69863056922477\\
11.7501472222222	1.70836062897304\\
12.0001444444444	1.71779045735262\\
12.2501416666667	1.72693373917235\\
12.5001388888889	1.73580334000087\\
12.7501361111111	1.74441136656734\\
13.0001333333333	1.75276922189596\\
13.2501305555556	1.76088765570235\\
13.5001277777778	1.76877681051999\\
13.750125	1.77644626397309\\
14.0001222222222	1.78390506756614\\
14.2501194444444	1.79116178232071\\
14.5001166666667	1.79822451155419\\
14.7501138888889	1.80510093106461\\
15.0001111111111	1.8117983169577\\
15.2501083333333	1.81832357132847\\
15.5001055555556	1.82468324598766\\
15.7501027777778	1.83088356440472\\
16.0001	1.83693044202164\\
16.2500972222222	1.84282950507713\\
16.5000944444444	1.84858610806695\\
16.7500916666667	1.85420534995442\\
17.0000888888889	1.85969208923416\\
17.2500861111111	1.86505095794258\\
17.5000833333333	1.87028637470021\\
17.7500805555556	1.87540255686282\\
18.0000777777778	1.88040353185169\\
18.250075	1.88529314772711\\
18.5000722222222	1.89007508306311\\
18.7500694444444	1.89475285617712\\
19.0000666666667	1.89932983376284\\
19.2500638888889	1.90380923897108\\
19.5000611111111	1.90819415897921\\
19.7500583333333	1.91248755208665\\
20.0000555555556	1.9166922543706\\
20.2500527777778	1.92081098593352\\
20.50005	1.92484635677131\\
20.7500472222222	1.92880087228871\\
21.0000444444444	1.93267693848646\\
21.2500416666667	1.93647686684284\\
21.5000388888889	1.94020287891026\\
21.7500361111111	1.94385711064627\\
22.0000333333333	1.94744161649659\\
22.2500305555556	1.95095837324667\\
22.5000277777778	1.95440928365691\\
22.750025	1.9577961798955\\
23.0000222222222	1.96112082678211\\
23.2500194444444	1.96438492485423\\
23.5000166666667	1.96759011326758\\
23.7500138888889	1.97073797254082\\
24.0000111111111	1.97383002715434\\
24.2500083333333	1.97686774801204\\
24.5000055555556	1.97985255477439\\
24.7500027777778	1.98278581807078\\
25	1.98566886159806\\
};
\addplot [color=mycolor4,dashed,forget plot]
  table[row sep=crcr]{%
0.000277777777777778	0.000153333553219328\\
0.250275	0.11217862252938\\
0.500272222222222	0.188747176936526\\
0.750269444444444	0.244392460368606\\
1.00026666666667	0.286659841889377\\
1.25026388888889	0.31985598206762\\
1.50026111111111	0.346618129016932\\
1.75025833333333	0.368651557502802\\
2.00025555555556	0.387107836039876\\
2.25025277777778	0.402792690421381\\
2.50025	0.416286767418033\\
2.75024722222222	0.428019098855727\\
3.00024444444444	0.438313555161441\\
3.25024166666667	0.447419205513731\\
3.50023888888889	0.455530734161642\\
3.75023611111111	0.462802515938813\\
4.00023333333333	0.469358535414257\\
4.25023055555556	0.475299514404404\\
4.50022777777778	0.480708123400159\\
4.750225	0.485652852088157\\
5.00022222222222	0.49019092493384\\
5.25021944444444	0.494370525840372\\
5.50021666666667	0.498232515646906\\
5.75021388888889	0.50181177241703\\
6.00021111111111	0.505138247757618\\
6.25020833333333	0.508237806966725\\
6.50020555555556	0.51113290291973\\
6.75020277777778	0.513843120854344\\
7.0002	0.516385622016601\\
7.25019722222222	0.518775507416193\\
7.50019444444444	0.521026117986322\\
7.75019166666667	0.523149283752313\\
8.00018888888889	0.52515553183686\\
8.25018611111111	0.527054261023003\\
8.50018333333333	0.528853888983972\\
8.75018055555556	0.530561977046048\\
9.00017777777778	0.532185336385056\\
9.250175	0.533730118801882\\
9.50017222222222	0.535201894627805\\
9.75016944444445	0.536605719839372\\
10.0001666666667	0.537946194087105\\
10.2501638888889	0.539227511041516\\
10.5001611111111	0.54045350221747\\
10.7501583333333	0.541627675241661\\
11.0001555555556	0.542753247368238\\
11.2501527777778	0.543833174917014\\
11.50015	0.544870179201486\\
11.7501472222222	0.545866769425463\\
12.0001444444444	0.54682526295388\\
12.2501416666667	0.547747803302529\\
12.5001388888889	0.548636376140677\\
12.7501361111111	0.549492823558003\\
13.0001333333333	0.550318856811593\\
13.2501305555556	0.551116067738622\\
13.5001277777778	0.551885938994856\\
13.750125	0.552629853257567\\
14.0001222222222	0.553349101513006\\
14.2501194444444	0.554044890532971\\
14.5001166666667	0.554718349631526\\
14.7501138888889	0.555370536781476\\
15.0001111111111	0.556002444160257\\
15.2501083333333	0.556615003186407\\
15.5001055555556	0.557209089100367\\
15.7501027777778	0.557785525137017\\
16.0001	0.558345086331756\\
16.2500972222222	0.558888502997148\\
16.5000944444444	0.559416463902919\\
16.7500916666667	0.559929619188421\\
17.0000888888889	0.560428583033468\\
17.2500861111111	0.560913936110601\\
17.5000833333333	0.561386227839375\\
17.7500805555556	0.561845978461069\\
18.0000777777778	0.562293680950282\\
18.250075	0.562729802778199\\
18.5000722222222	0.563154787540759\\
18.7500694444444	0.563569056463677\\
19.0000666666667	0.563973009795024\\
19.2500638888889	0.564367028095067\\
19.5000611111111	0.564751473432093\\
19.7500583333333	0.565126690492119\\
20.0000555555556	0.565493007609634\\
20.2500527777778	0.565850737725862\\
20.50005	0.566200179280407\\
20.7500472222222	0.566541617041641\\
21.0000444444444	0.566875322880665\\
21.2500416666667	0.567201556493296\\
21.5000388888889	0.567520566074075\\
21.7500361111111	0.567832588946004\\
22.0000333333333	0.568137852149343\\
22.2500305555556	0.568436572992562\\
22.5000277777778	0.568728959568234\\
22.750025	0.569015211236462\\
23.0000222222222	0.569295519078188\\
23.2500194444444	0.569570066320552\\
23.5000166666667	0.569839028736302\\
23.7500138888889	0.570102575019066\\
24.0000111111111	0.570360867136181\\
24.2500083333333	0.570614060660631\\
24.5000055555556	0.570862305083521\\
24.7500027777778	0.571105744108398\\
25	0.571344515928651\\
};
\end{axis}
\end{tikzpicture}%}
\caption{Battery life time estimation for the four cases for different $t_i$'s. Full line using PSM, dashed line using eDRX.}
\label{fig:eDRXvsPSM}
\end{figure}

As can be seen in \autoref{fig:eDRXvsPSM} the longer the $t_i$ the more impactful the these parameters become which makes sense as more time is spent in idle mode. What is also interesting to note is that each case is going asymptotic towards certain values and that eDRX is more influenced by the timer values than PSM is. To give a more clear picture of the influence of the timer values the $t_i$ is set to 24 hours and the timer values are changed are varied to a further extent this can be seen in \autoref{fig:TimerPlot_plots}.


\begin{figure}[H]
\centering
\begin{minipage}{0.48\textwidth}
\tikzsetnextfilename{eDRX_each_day_TimerPlot}
\resizebox{\textwidth}{!}{
\input{figures/QuectelMeas/eDRX_each_day_TimerPlot.tex}}
\end{minipage}
\hfill
\begin{minipage}{0.48\textwidth}
\tikzsetnextfilename{PSM_each_day_TimerPlot}
\resizebox{\textwidth}{!}{
% This file was created by matlab2tikz.
%
%The latest updates can be retrieved from
%  http://www.mathworks.com/matlabcentral/fileexchange/22022-matlab2tikz-matlab2tikz
%where you can also make suggestions and rate matlab2tikz.
%
\definecolor{mycolor1}{rgb}{0.00000,0.44700,0.74100}%
%
\begin{tikzpicture}

\begin{axis}[%
width=\textwidth,
height=0.66\textwidth,
at={(0.758in,0.481in)},
scale only axis,
xmin=0,
xmax=100,
xlabel={$T_{DRX}$ [s]},
ymin=10,
ymax=11,
ylabel={Battery Life Time [years]},
axis background/.style={fill=white},
title style={font=\bfseries},
title={$t_i$ 24 hours (PSM)}
%title={Scenario 4}
]
\addplot [color=mycolor1,solid,forget plot]
  table[row sep=crcr]{%
1	10.9700571495878\\
2	10.9608233130734\\
3	10.9516050082971\\
4	10.9424021961044\\
5	10.9332148374719\\
6	10.9240428935075\\
7	10.9148863254493\\
8	10.9057450946651\\
9	10.8966191626524\\
10	10.8875084910371\\
11	10.8784130415736\\
12	10.8693327761439\\
13	10.860267656757\\
14	10.8512176455488\\
15	10.8421827047811\\
16	10.8331627968415\\
17	10.8241578842424\\
18	10.8151679296209\\
19	10.8061928957381\\
20	10.7972327454786\\
21	10.78828744185\\
22	10.7793569479823\\
23	10.7704412271278\\
24	10.7615402426598\\
25	10.7526539580729\\
26	10.7437823369822\\
27	10.7349253431225\\
28	10.7260829403484\\
29	10.7172550926333\\
30	10.7084417640692\\
31	10.6996429188661\\
32	10.6908585213515\\
33	10.6820885359699\\
34	10.6733329272826\\
35	10.6645916599666\\
36	10.655864698815\\
37	10.6471520087358\\
38	10.6384535547515\\
39	10.6297693019991\\
40	10.6210992157292\\
41	10.6124432613058\\
42	10.6038014042055\\
43	10.5951736100176\\
44	10.5865598444429\\
45	10.5779600732939\\
46	10.5693742624941\\
47	10.5608023780774\\
48	10.552244386188\\
49	10.5437002530795\\
50	10.535169945115\\
51	10.526653428766\\
52	10.5181506706128\\
53	10.5096616373431\\
54	10.5011862957525\\
55	10.4927246127432\\
56	10.4842765553244\\
57	10.4758420906111\\
58	10.4674211858244\\
59	10.4590138082903\\
60	10.45061992544\\
61	10.4422395048091\\
62	10.4338725140372\\
63	10.4255189208675\\
64	10.4171786931466\\
65	10.4088517988236\\
66	10.4005382059504\\
67	10.3922378826806\\
68	10.3839507972693\\
69	10.3756769180731\\
70	10.3674162135492\\
71	10.359168652255\\
72	10.3509342028482\\
73	10.3427128340858\\
74	10.3345045148243\\
75	10.3263092140186\\
76	10.3181269007222\\
77	10.3099575440867\\
78	10.3018011133612\\
79	10.293657577892\\
80	10.2855269071222\\
81	10.2774090705917\\
82	10.2693040379361\\
83	10.2612117788869\\
84	10.2531322632709\\
85	10.2450654610098\\
86	10.2370113421199\\
87	10.2289698767119\\
88	10.2209410349901\\
89	10.2129247872524\\
90	10.2049211038897\\
91	10.1969299553858\\
92	10.1889513123167\\
93	10.1809851453506\\
94	10.1730314252473\\
95	10.1650901228577\\
96	10.157161209124\\
97	10.1492446550788\\
98	10.1413404318448\\
99	10.133448510635\\
100	10.1255688627515\\
};
\end{axis}
\end{tikzpicture}%}
\end{minipage}
\caption{Influence of timer settings for the estimated battery lifetime.}
\label{fig:TimerPlot_plots}
\end{figure}

From \autoref{fig:TimerPlot_plots} it can be seen that the values for the eDRX timers are very influential as the expected lifetime is varying  from 0.075 years to 10.78 years, where for PSM the lifetime only varies from 10.97 to 10.13 years. The next step is to look at the influence of $T_{transmit}$ and $P_{TX}$. To do this four scenarios are picked such that the differences for high and low $t_i$'s in each idle mode can be compared, the values chosen to showcase this can be seen in \autoref{tab:scenario_description}.


\begin{table}[H]
\centering
\begin{tabular}{|c|c|c|c|c|c|} \hline
			& $t_i$ 		& Idle mode	& $T_{DRX}$	& $T_{PSM}$	& $T_{PTW}$	\\ \hline
Scenario 1	& 1 hour 	& eDRX 		& -			& 5242.88 s	& 10.24 s	\\ \hline
Scenario 2	& 24 hours 	& eDRX 		& -			& 5242.88 s	& 10.24 s	\\ \hline
Scenario 3	& 1 hour 	& PSM		& 10 s		& -			& -			\\ \hline
Scenario 4	& 24 hours 	& PSM		& 10 s		& -			& -			\\ \hline
\end{tabular}
\caption{Parameter values chosen for each scenario.}
\label{tab:scenario_description}
\end{table}

For each of the four scenarios the transmit power and transmit time is varied independent of each other and the battery lifetime is estimated for each combination as seen in \autoref{fig:lifetime_plots}.


\begin{figure}[H]
\centering
\begin{minipage}{0.48\textwidth}
\tikzsetnextfilename{eDRX_each_hour_lifetime}
\resizebox{\textwidth}{!}{
\input{figures/QuectelMeas/eDRX_each_hour_lifetime.tex}}
\end{minipage}
\hfill
\begin{minipage}{0.48\textwidth}
\tikzsetnextfilename{eDRX_each_day_lifetime}
\resizebox{\textwidth}{!}{
\input{figures/QuectelMeas/eDRX_each_day_lifetime.tex}}
\end{minipage}
\begin{minipage}{0.48\textwidth}
\tikzsetnextfilename{PSM_each_hour_lifetime}
\resizebox{\textwidth}{!}{
\input{figures/QuectelMeas/PSM_each_hour_lifetime.tex}}
\end{minipage}
\hfill
\begin{minipage}{0.48\textwidth}
\tikzsetnextfilename{PSM_each_day_lifetime}
\resizebox{\textwidth}{!}{
\input{figures/QuectelMeas/PSM_each_day_lifetime.tex}}
\end{minipage}
\caption{The estimated battery lifetime for the mentioned scenarios with respect to transmit power and transmit duration.}
\label{fig:lifetime_plots}
\end{figure}

Here it can be seen again that the values of these parameters have a huge effect. Especially $T_{transmit}$ which alone can bring the lifetime down to less than a year no matter the other parameters. A realistic value for $T_{transmit}$ is not really known as it depends heavily on the usages as well as a multitude of cell parameters. To clarify the utilization of the energy spent four points has been picked out from each plot in \autoref{fig:lifetime_plots}. The values of each point can be seen in \autoref{tab:point_description}.

\begin{table}[H]
\centering
\begin{tabular}{|c|c|c|} \hline
& $T_{Transmit}$ & $P_{TX}$ \\ \hline
Point 1 & 100 ms& -20 dBm  	\\ \hline
Point 2 & 100 s	& -20 dBm  	\\ \hline
Point 3 & 100 ms& 23 dBm 	\\ \hline
Point 4 & 100 s	& 23 dBm 	\\ \hline
\end{tabular}
\caption{Points chosen to investigate.}
\label{tab:point_description}
\end{table} 

For each of the points in each of plots the energy usages has been divided up into five parts corresponding to the five parts in the model: 	sync, attach, transmit, release and idle. The result of which can be seen in \autoref{fig:barplot_plots}.
 
 
\begin{figure}[H]
\centering
\begin{minipage}{0.48\textwidth}
\tikzsetnextfilename{eDRX_each_hour_barplot}
\resizebox{\textwidth}{!}{
% This file was created by matlab2tikz.
%
%The latest updates can be retrieved from
%  http://www.mathworks.com/matlabcentral/fileexchange/22022-matlab2tikz-matlab2tikz
%where you can also make suggestions and rate matlab2tikz.
%
\definecolor{mycolor1}{rgb}{0.00000,0.44700,0.74100}%
\definecolor{mycolor2}{rgb}{0.85000,0.32500,0.09800}%
\definecolor{mycolor3}{rgb}{0.92900,0.69400,0.12500}%
\definecolor{mycolor4}{rgb}{0.49400,0.18400,0.55600}%
\definecolor{mycolor5}{rgb}{0.46600,0.67400,0.18800}%
%
\begin{tikzpicture}

\begin{axis}[%
width=\textwidth,
height=0.66\textwidth,
at={(0.758in,0.481in)},
scale only axis,
bar width=0.8,
xmin=0.5,
xmax=4.5,
xtick={1,2,3,4},
xticklabels={Point 1, Point 2, Point 3, Point 4},
ymin=0,
ymax=100,
ylabel={Energy spenditure [\%]},
axis background/.style={fill=white},
title style={font=\bfseries},
title={Scenario 1}
]
\addplot[ybar stacked,draw=black,fill=mycolor1,area legend] plot table[row sep=crcr] {%
1	0.000416948476652553\\
2	3.30796564229127e-05\\
3	0.000204920892803428\\
4	2.78447749929594e-07\\
};
\addplot[ybar stacked,draw=black,fill=mycolor2,area legend] plot table[row sep=crcr] {%
1	80.4314444549567\\
2	6.38123100851546\\
3	83.0893922694818\\
4	0.112902369318916\\
};
\addplot[ybar stacked,draw=black,fill=mycolor3,area legend] plot table[row sep=crcr] {%
1	0.116074671790491\\
2	92.0907612628466\\
3	7.35014649609497\\
4	99.8742356375432\\
};
\addplot[ybar stacked,draw=black,fill=mycolor4,area legend] plot table[row sep=crcr] {%
1	12.51264260014\\
2	0.992721982050193\\
3	6.14968524057609\\
4	0.00835622954100753\\
};
\addplot[ybar stacked,draw=black,fill=mycolor5,area legend] plot table[row sep=crcr] ;
\node[align=center, text=black]
at (axis cs:1,40.216) {80.43\%};
\node[align=center, text=black]
at (axis cs:1,80.49) {0.12\%};
\node[align=center, text=black]
at (axis cs:1,86.804) {12.51\%};
\node[below, align=center, text=black]
at (axis cs:1,100) {6.94\%};
\node[above, align=center, text=black]
at (axis cs:2,0) {0.00\%};
\node[align=center, text=black]
at (axis cs:2,13) {6.38\%};
\node[align=center, text=black]
at (axis cs:2,52.427) {92.09\%};
\node[align=center, text=black]
at (axis cs:2,87) {0.99\%};
\node[below, align=center, text=black]
at (axis cs:2,100) {0.54\%};
\node[above, align=center, text=black]
at (axis cs:3,0) {0.00\%};
\node[align=center, text=black]
at (axis cs:3,41.545) {83.09\%};
\node[above, align=center, text=black]
at (axis cs:3,74) {7.35\%};
\node[align=center, text=black]
at (axis cs:3,87) {6.15\%};
\node[below, align=center, text=black]
at (axis cs:3,100) {3.41\%};
\node[above, align=center, text=black]
at (axis cs:4,0) {0.00\%};
\node[align=center, text=black]
at (axis cs:4,13) {0.11\%};
\node[align=center, text=black]
at (axis cs:4,50.05) {99.87\%};
\node[align=center, text=black]
at (axis cs:4,87) {0.01\%};
\node[below, align=center, text=black]
at (axis cs:4,100) {0.00\%};
\end{axis}
\end{tikzpicture}%}
\end{minipage}
\hfill
\begin{minipage}{0.48\textwidth}
\tikzsetnextfilename{eDRX_each_day_barplot}
\resizebox{\textwidth}{!}{
% This file was created by matlab2tikz.
%
%The latest updates can be retrieved from
%  http://www.mathworks.com/matlabcentral/fileexchange/22022-matlab2tikz-matlab2tikz
%where you can also make suggestions and rate matlab2tikz.
%
%\definecolor{mycolor1}{rgb}{0.00000,0.44700,0.74100}%
\definecolor{mycolor1}{rgb}{0.85000,0.32500,0.09800}%
\definecolor{mycolor2}{rgb}{0.49400,0.18400,0.55600}%
\definecolor{mycolor3}{rgb}{0.46600,0.67400,0.18800}%
\definecolor{mycolor4}{rgb}{0.46600,0.67400,0.18800}%
%
\begin{tikzpicture}

\begin{axis}[%
width=\textwidth,
height=0.66\textwidth,
at={(0.758in,0.481in)},
scale only axis,
bar width=0.8,
xmin=0.5,
xmax=4.5,
xtick={1,2,3,4},
xticklabels={Point 1, Point 2, Point 3, Point 4},
ymin=0,
ymax=100,
ylabel={Energy spenditure [\%]},
axis background/.style={fill=white},
title style={font=\bfseries},
title={Scenario 2}
]
\addplot[ybar stacked,draw=black,fill=mycolor2,area legend] plot table[row sep=crcr] {%
1	21.6458916810439\\
2	6.71046797067417\\
3	35.8731879704054\\
4	0.140888017705163\\
};
\addplot[ybar stacked,draw=black,fill=mycolor3,area legend] plot table[row sep=crcr] {%
1	0.0222653408733227\\
2	69.025041328013\\
3	2.53647846936899\\
4	99.6174144868408\\
};
\addplot[ybar stacked,draw=black,fill=mycolor4,area legend] plot table[row sep=crcr] {%
1	24.6652203883008\\
2	7.64649356303764\\
3	19.3936347377525\\
4	0.076166376864918\\
};
\addplot[ybar stacked,draw=black,fill=mycolor5,area legend] plot table[row sep=crcr] 
\end{minipage}
\begin{minipage}{0.48\textwidth}
\tikzsetnextfilename{PSM_each_hour_barplot}
\resizebox{\textwidth}{!}{
% This file was created by matlab2tikz.
%
%The latest updates can be retrieved from
%  http://www.mathworks.com/matlabcentral/fileexchange/22022-matlab2tikz-matlab2tikz
%where you can also make suggestions and rate matlab2tikz.
%
\definecolor{mycolor1}{rgb}{0.00000,0.44700,0.74100}%
\definecolor{mycolor2}{rgb}{0.85000,0.32500,0.09800}%
\definecolor{mycolor3}{rgb}{0.92900,0.69400,0.12500}%
\definecolor{mycolor4}{rgb}{0.49400,0.18400,0.55600}%
\definecolor{mycolor5}{rgb}{0.46600,0.67400,0.18800}%
%
\begin{tikzpicture}

\begin{axis}[%
width=\textwidth,
height=0.66\textwidth,
at={(0.758in,0.481in)},
scale only axis,
bar width=0.8,
xmin=0.5,
xmax=4.5,
xtick={1,2,3,4},
xticklabels={Point 1, Point 2, Point 3, Point 4},
ymin=0,
ymax=100,
ylabel={Energy spenditure [\%]},
axis background/.style={fill=white},
title style={font=\bfseries},
title={Scenario 3}
]
\addplot[ybar stacked,draw=black,fill=mycolor1,area legend] plot table[row sep=crcr] {%
1	0.000411484740862782\\
2	3.30425462359728e-05\\
3	0.000203592271867335\\
4	2.78445117589701e-07\\
};
\addplot[ybar stacked,draw=black,fill=mycolor2,area legend] plot table[row sep=crcr] {%
1	79.3774625212183\\
2	6.37407226803143\\
3	82.5506755743404\\
4	0.112901301982545\\
};
\addplot[ybar stacked,draw=black,fill=mycolor3,area legend] plot table[row sep=crcr] {%
1	0.11455361733398\\
2	91.9874498704244\\
3	7.30249123564561\\
4	99.8732914642296\\
};
\addplot[ybar stacked,draw=black,fill=mycolor4,area legend] plot table[row sep=crcr] {%
1	12.3486756425248\\
2	0.99160830366544\\
3	6.10981326632674\\
4	0.00835615054436856\\
};
\addplot[ybar stacked,draw=black,fill=mycolor5,area legend] plot table[row sep=crcr] ;
\node[align=center, text=black]
at (axis cs:1,39.689) {79.38\%};
\node[align=center, text=black]
at (axis cs:1,79.435) {0.11\%};
\node[align=center, text=black]
at (axis cs:1,85.667) {12.35\%};
\node[below, align=center, text=black]
at (axis cs:1,100) {8.16\%};
\node[above, align=center, text=black]
at (axis cs:2,0) {0.00\%};
\node[align=center, text=black]
at (axis cs:2,13) {6.37\%};
\node[align=center, text=black]
at (axis cs:2,52.368) {91.99\%};
\node[align=center, text=black]
at (axis cs:2,87) {0.99\%};
\node[below, align=center, text=black]
at (axis cs:2,100) {0.65\%};
\node[above, align=center, text=black]
at (axis cs:3,0) {0.00\%};
\node[align=center, text=black]
at (axis cs:3,41.276) {82.55\%};
\node[above, align=center, text=black]
at (axis cs:3,74) {7.30\%};
\node[align=center, text=black]
at (axis cs:3,87) {6.11\%};
\node[below, align=center, text=black]
at (axis cs:3,100) {4.04\%};
\node[above, align=center, text=black]
at (axis cs:4,0) {0.00\%};
\node[align=center, text=black]
at (axis cs:4,13) {0.11\%};
\node[align=center, text=black]
at (axis cs:4,50.05) {99.87\%};
\node[align=center, text=black]
at (axis cs:4,87) {0.01\%};
\node[below, align=center, text=black]
at (axis cs:4,100) {0.01\%};
\end{axis}
\end{tikzpicture}%}
\end{minipage}
\hfill
\begin{minipage}{0.48\textwidth}
\tikzsetnextfilename{PSM_each_day_barplot}
\resizebox{\textwidth}{!}{
% This file was created by matlab2tikz.
%
%The latest updates can be retrieved from
%  http://www.mathworks.com/matlabcentral/fileexchange/22022-matlab2tikz-matlab2tikz
%where you can also make suggestions and rate matlab2tikz.
%
\definecolor{mycolor1}{rgb}{0.00000,0.44700,0.74100}%
\definecolor{mycolor2}{rgb}{0.85000,0.32500,0.09800}%
\definecolor{mycolor3}{rgb}{0.92900,0.69400,0.12500}%
\definecolor{mycolor4}{rgb}{0.49400,0.18400,0.55600}%
\definecolor{mycolor5}{rgb}{0.46600,0.67400,0.18800}%
%
\begin{tikzpicture}

\begin{axis}[%
width=\textwidth,
height=0.66\textwidth,
at={(0.758in,0.481in)},
scale only axis,
bar width=0.8,
xmin=0.5,
xmax=4.5,
xtick={1,2,3,4},
xticklabels={Point 1,Point 2,Point 3,Point 4},
ymin=0,
ymax=100,
ylabel={Energy spenditure [\%]},
axis background/.style={fill=white},
title style={font=\bfseries},
title={Scenario 4}
]
\addplot[ybar stacked,draw=black,fill=mycolor1,area legend] plot table[row sep=crcr] {%
1	0.0002213549734827\\
2	3.09105458526233e-05\\
3	0.000142873785362242\\
4	2.7828337143635e-07\\
};
\addplot[ybar stacked,draw=black,fill=mycolor2,area legend] plot table[row sep=crcr] {%
1	42.700480398537\\
2	5.96279874141242\\
3	57.931115927631\\
4	0.112835718676714\\
};
\addplot[ybar stacked,draw=black,fill=mycolor3,area legend] plot table[row sep=crcr] {%
1	0.0616232156607875\\
2	86.0521542976644\\
3	5.12462755016133\\
4	99.8152760073356\\
};
\addplot[ybar stacked,draw=black,fill=mycolor4,area legend] plot table[row sep=crcr] {%
1	6.64287274338821\\
2	0.927626875949508\\
3	4.28764874624202\\
4	0.0083512965350073\\
};
\addplot[ybar stacked,draw=black,fill=mycolor5,area legend] plot table[row sep=crcr] ;
\node[align=center, text=black]
at (axis cs:1,21.35) {42.70\%};
\node[below, align=center, text=black]
at (axis cs:1,42.732) {0.06\%};
\node[align=center, text=black]
at (axis cs:1,46.084) {6.64\%};
\node[below, align=center, text=black]
at (axis cs:1,100) {50.59\%};
\node[above, align=center, text=black]
at (axis cs:2,0) {0.00\%};
\node[align=center, text=black]
at (axis cs:2,13) {5.96\%};
\node[align=center, text=black]
at (axis cs:2,48.989) {86.05\%};
\node[align=center, text=black]
at (axis cs:2,87) {0.93\%};
\node[below, align=center, text=black]
at (axis cs:2,100) {7.06\%};
\node[above, align=center, text=black]
at (axis cs:3,0) {0.00\%};
\node[align=center, text=black]
at (axis cs:3,28.966) {57.93\%};
\node[align=center, text=black]
at (axis cs:3,58.494) {5.12\%};
\node[align=center, text=black]
at (axis cs:3,67.2) {4.29\%};
\node[below, align=center, text=black]
at (axis cs:3,100) {32.66\%};
\node[above, align=center, text=black]
at (axis cs:4,0) {0.00\%};
\node[align=center, text=black]
at (axis cs:4,13) {0.11\%};
\node[align=center, text=black]
at (axis cs:4,50.02) {99.82\%};
\node[align=center, text=black]
at (axis cs:4,87) {0.01\%};
\node[below, align=center, text=black]
at (axis cs:4,100) {0.06\%};
\end{axis}
\end{tikzpicture}%}
\end{minipage}
\caption{The distribution of power for each phase of a cycle. Blue is the sync phase, red is the attach phase, yellow is the transmit phase, purple is the release phase and green is the idle phase.}
\label{fig:barplot_plots}
\end{figure}

It can be seen that depending on the point and scenario the energy is used quite differently, however some commonalities can be seen. For instance if each transmission is 100 s more than 80\% of the energy is spent during the transmit phase if $P_{TX}$ is also high then more than 99\% of the energy is spent here. Likewise can it be seen that the energy used to synchronise the devise is less negligible. It can also be seen that only when the $t_i$ is high does $E_idle$ take op more than 9\%. Also the energy spent during the attach phase is quite high only really topped by the energy spent during a long transmission and comparable to the energy used during the idle phase for scenario 2 and 4. 

With this analysis it can be concluded that a device will find it hard to achieve a battery lifetime of 10 years in extreme conditions based on the fact that it will increase not only the needed $P_{TX}$ but also the repetition which increases $T_{transmit}$. When these parameters are high it has been shown that a battery lifetime might not even exceed one year independently on all other parameters. To combat this the only way is really to focus in decreasing $T_{transmit}$ as much as possible. 
