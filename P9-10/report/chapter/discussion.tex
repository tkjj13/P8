\chapter{Discussion}
The discussion will follow up on the results found in \autoref{ch:mass_test} and \autoref{ch:BatTest} and suggestions are made to explain some of the findings, as well as some of the potential problems that might influence the results. 

\section{Domain: Massiveness}


\section{Domain: Energy Consumption}
It was found in \autoref{ch:BatTest} that it is possible to achieve a battery lifetime of more than 10 years on a 5 Wh battery, however it was also discovered that this was not a guarantee. It was found that to achieve this the device is only allowed to transmit for a maximum of \todo{find value}{} per 24 hours. 

Taking a step back to evaluate the method of finding this value, a couple of issues can be seen. The way the model is designed limits the precision a bit, this comes in two parts first is that TX and RX can not be told apart. This is needed for a couple of different reasons to better the results, first is this could be used to approximate the energy spent during the attach and release procedures more accurately, second is it would give way better options for designing a use case including e.g. software updates. The model handles this issue by always assuming a worst case scenario energy consumption, this however leads to the second issue. During the transmit phase RX is included, this lowers the power average power used here, as the model does not assume any relation between $T_{transmit}$ and the throughput this is not a major issue, but is does not accurately describe most use cases. The third issue is a bit more hidden however, the parameter repetition was set to 1, this means that any transmission is rather short when this parameter is increase factors like the transmission gap comes into play this is not accounted for in the model. The forth issue is that the model uses a full attach procedure for each attach, this should not be the case in a real scenario, however as this issue makes the model underestimate the lifetime the design objective can still be evaluated.

Fixing these flaws in the model could indeed bring the accuracy of the model up, however it would also complicate the model further. As the model is intended to be as simple as possible while still providing accurate estimates this is not desired. To evaluate if the model actually predicts the battery life time accurately would require a longer test run, setting all the parameters and including idle mode and data transmission interval and evaluating the energy consumption over multiple days. This has however not been possible as the UXM used in the project is a shared resource. 

Another concern is that only four parameters was chosen out of a multitude of parameters, this of course could influence the estimate quite a bit. This has been accounted for indirectly as more or less all of these parameters would affect the time a transmission takes, but not the average power consumption. A concern however is that repetition not only affects the data transfer but also the attach, release and paging messages. A few other parameters might also affect these procedures, in an uncertain way.

When all of these issues and concerns are taking into account the model still predicts that the main influence the battery lifetime is the transmit time and in the end this parameter is very much use case dependent. Other parameters is also highly influential such as idle mode timer values, however the scenarios for which they are a problem is considered quite unrealistic. 

