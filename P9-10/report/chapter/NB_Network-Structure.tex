 \section{Network Structure}

The network structure of \gls{NB-IoT} is very similar to that of legacy \gls{LTE}, as can be seen on \autoref{fig:network_structure}. The system is divided into a control plane \gls{CIoT} \gls{EPS} optimization and a user plane \gls{CIoT} \gls{EPS} optimization.

\tikzsetnextfilename{NB-IoT_Network_Architechture}
\begin{figure}[H]
\centering
%\includegraphics[width=\textwidth]{figures/NB-Network.png}
\definecolor{purple}{HTML}{7030A0}
\usetikzlibrary{arrows}
\definecolor{red}{HTML}{FF0000}
\definecolor{blue}{HTML}{00B0F0}

\resizebox{\textwidth}{!}{
\begin{tikzpicture}[scale=0.5]


\draw  (-14,5) rectangle (-10,3);
\node at (-12,4) {\acrshort{UE}};
\draw  (-4,14) rectangle (2,-8);
\draw  (6,9) rectangle (10,7);
\draw  (6,1) rectangle (10,-1);
\node at (-1,-7) {\acrshort{CIoT} \acrshort{RAN}};
\node at (8,8) {\acrshort{MME}};
\node at (8,0) {\acrshort{SGW}};
 




\draw  (14,1) rectangle (18,-1);
\draw  (14,13) rectangle (18,11);
\draw  (25,6) ellipse (3 and 2);
\node at (16,12) {\acrshort{SCEF}};

\node at (16,0) {\acrshort{PGW}};
\node at (25,6) {\acrshort{CIoT} Services};

\draw  (-3,13) rectangle (1,11);
\draw  (-3,5) rectangle (1,3);
\draw  (-3,-3) rectangle (1,-5);
\node at (-1,12) {\acrshort{eNB}};
\node at (-1,4) {\acrshort{eNB}};
\node at (-1,-4) {\acrshort{eNB}};

\draw (6,18) -- (6,15); 
\draw (10,15) -- (10,18);
\draw  (8,18) ellipse (2 and 0.5);
\node at (8,15.4) {\acrshort{HSS}};
\draw (6,15) arc (-120:-60:4);

\draw (14,8.6) -- (14,5.6);
\draw (18,5.6) -- (18,8.6);
\draw  (16,8.6) ellipse (2 and 0.5);
\node at (16,6) {\acrshort{PCRF}};
\draw (14,5.6) arc (-120:-60:4);
\node (v27) at (16,5) {};


\node (v2) at (-7,4) {\acrshort{CIoT}-Uu};
\node (v1) at (-10,4) {};
\node (v3) at (-4,4) {};
\draw [draw=purple, arrows={triangle 45-},fill=purple] (v1) edge (v2);
\draw [draw=purple, arrows={-triangle 45},fill=purple] (v2) edge (v3);

\node (v5) at (-1,8) {X2};
\node (v8) at (-1,0) {X2};
\node (v4) at (-1,11) {};
\node (v6) at (-1,5) {};
\node (v7) at (-1,3) {};
\node (v9) at (-1,-3) {};
\draw [draw=red, arrows={triangle 45-},fill=red] (v4) edge (v5);
\draw [draw=red, arrows={-triangle 45},fill=red] (v5) edge (v6);
\draw [draw=red, arrows={triangle 45-},fill=red] (v7) edge (v8);
\draw [draw=red, arrows={-triangle 45},fill=red] (v8) edge (v9);
\node (v10) at (2,4) {};
\node (v14) at (6,0) {};
\node (v12) at (6,8) {};
\node (v18) at (8,7) {};
\node (v20) at (8,1) {};
\node (v17) at (8,9) {};
\node (v15) at (8,14.4) {};
\node (v21) at (10,8) {};
\node (v23) at (14,12) {};

\node (v24) at (10,0) {};
\node (v26) at (14,0) {};

\node (v33) at (18,0) {};
\node (v32) at (22,6) {};
\node (v30) at (18,12) {};
\node (v29) at (16,1) {};

\node (v11) at (4,6) {S1-\acrshort{MME}};
\node (v13) at (4,2) {S1-U};
\node (v19) at (8,4) {S11};
\node (v16) at (8,12) {S6a};
\node (v22) at (12,10) {T6a};
\node (v25) at (12,0) {S5};
\node (v28) at (16,3) {S7};
%\node (v31) at (20,9) {};
\node (v34) at (20,3) {SGi};

\draw [draw=red, arrows={triangle 45-},fill=red]  (v10) edge (v11);
\draw [draw=red, arrows={-triangle 45},fill=red]  (v11) edge (v12);
\draw [draw=blue, arrows={triangle 45-},fill=blue]  (v10) edge (v13);
\draw [draw=blue, arrows={-triangle 45},fill=blue]  (v13) edge (v14);


\draw [draw=red, arrows={triangle 45-},fill=red]   (v15) edge (v16);
\draw [draw=red, arrows={-triangle 45},fill=red]  (v16) edge (v17);
\draw [draw=red, arrows={triangle 45-},fill=red]  (v18) edge (v19);
\draw [draw=red, arrows={-triangle 45},fill=red]  (v19) edge (v20);
\draw [draw=red, arrows={triangle 45-},fill=red]  (v21) edge (v22);
\draw [draw=red, arrows={-triangle 45},fill=red]  (v22) edge (v23);
\draw [draw=purple, arrows={triangle 45-},fill=purple]  (v24) edge (v25);
\draw [draw=purple, arrows={-triangle 45},fill=purple]  (v25) edge (v26);
\draw [draw=red, arrows={triangle 45-},fill=red]  (v27) edge (v28);
\draw [draw=red, arrows={-triangle 45},fill=red]  (v28) edge (v29);
\draw [draw=red, arrows={triangle 45-triangle 45},fill=red]  (v30) edge (v32);
%\draw [draw=red, arrows={-triangle 45},fill=red]  (v31) edge (v32);
\draw [draw=purple, arrows={triangle 45-},fill=purple]  (v33) edge (v34);
\draw [draw=purple, arrows={-triangle 45},fill=purple]  (v34) edge (v32);
\end{tikzpicture}
}
\caption{Overview over the network blocks and interfaces between blocks in \gls{NB-IoT}. Blue lines are user plane \gls{CIoT} \gls{EPS} optimization, the red lines are control plane \gls{CIoT} \gls{EPS} optimization plane and the purple lines are both \citep{NB_slide}.}
\label{fig:network_structure}
\end{figure}


\subsection{Device}
The device is the smart meters or other products as mentioned before, they do not need to transmit a lot of data and the latency of the payload is not critical. They do however require a long battery lifetime. The problem comes in terms of placement because many of these devices might be placed in a basement like environment, with a high path loss. \citep{REL-13,book_LTE_for_UMTS}

\subsection{\gls{CIoT} \gls{RAN}}
The \gls{CIoT} \gls{RAN} is the network of base stations, where the most typically used is the \gls{eNB} base station. Any device that wish to use an external service, communicates with the \gls{eNB} \citep{book_LTE_for_UMTS}. The \gls{eNB} communicates with both the \gls{MME} and the \gls{SGW}. The \gls{eNB} is in charge of \gls{RRM}, i.e. allocating radio resources in the user plane to the individual device based on \gls{QoS} measures. 

\subsection{\gls{MME}}
The \gls{MME} takes care of mobility issues and keep track of where in the network different devices are connected \citepalias{3GPP_MME_spec}. Another very important function of the \gls{MME} is to handle authentication of devices and setting up security for the data bearers. The \gls{MME} might be connected to multiple devices, however, a device may only be connected to a single \gls{MME} \citep[ch. 3]{book_LTE_for_UMTS}. The \gls{MME} might serve a particular geographic area with several \gls{eNB}s. In \gls{NB-IoT} handovers are omitted and the only way to change cell is by releasing the existing connection and connect to the new cell\citep{REL-13}. The \gls{MME} also handles paging procedures \citep{NB-IoT_Book}.

\subsection{\gls{HSS}}
The \gls{HSS} stores the identity of the devices, which the \gls{MME} uses for authentication purposes. It records the location of the device in level of visited network control nodes, such as \gls{MME}. It also keeps track of which networks the device is allowed to roam to \citep[ch. 3]{book_LTE_for_UMTS}.

\subsection{\gls{SCEF}}
The \gls{SCEF} is in charge of multiple tasks, including device trigger delivery, sponsored data, device reachability, \gls{3GPP} network issues, \gls{QoS} for a device session etc. Many of these functionalities are meant for normal \gls{LTE} use. Uses meant for \gls{NB-IoT} include device reachability, which enables the application layer to be informed when a device reconnects to the network i.e. after a \gls{eDRX} or  \gls{PSM} period. Another functionality it handles is \gls{NIDD}, which enables devices with small data volumes to send its data with less overhead and thereby have a longer battery lifetime \citepalias{3GPP_SCEF_primer}.

\subsection{\gls{SGW}}
The \gls{SGW} is primarily a routing unit. It interfaces with the \gls{eNB}, the \gls{MME} and the \gls{PGW}. When the device transmits data, it is sent to the \gls{eNB} and then routed via the \gls{SGW} to the \gls{PGW}, before reaching the providers. The \gls{SGW} typically serve a particular geographic area with several \gls{eNB}s. In \gls{LTE} this was the last node in the network that could change during a connected state, meaning that all \gls{SGW}s needs to be connected to all \gls{PGW}s \citep[ch. 3]{book_LTE_for_UMTS}. This is however not equally important in \gls{NB-IoT}, as no handovers are possible \citep{REL-13}. During connected state the \gls{SGW} works as a relay, however in idle mode the resources are released in the \gls{eNB} and the data path terminates at the \gls{SGW}. It then stores the data from the \gls{PGW} and requests the \gls{MME} to initiate paging of the device \citep[ch. 3]{book_LTE_for_UMTS}.

\subsection{\gls{PGW}}
The \gls{PGW} is the edge of the \gls{EPS}. It function as the point of attachment for the devices \gls{IP} traffic. The \gls{IP}-address of the device is allocated during the connection procedure, when the device request a \gls{PDN} connection and during any subsequent \gls{PDN} connection request. It is the \gls{PGW} that performs the \gls{DHCP} functionality \citep[ch. 3]{book_LTE_for_UMTS}.

\subsection{\gls{PCRF}}
The \gls{PCRF} is a server that makes a decision on how to handle services provided for the device, based on \gls{QoS} measures. It informs the \gls{PGW} about appropriate bearer policy that can be set up. A default bearer is set up during connection request and either the device or the service domain can request additional bearers, which is handled by the \gls{PCRF} \citep[ch. 3]{book_LTE_for_UMTS}.

\subsection{\gls{CIoT} Services}
The \gls{CIoT} services are typically storage functionalities but could be control algorithms or other services needed for specific products. 

%connections across the network