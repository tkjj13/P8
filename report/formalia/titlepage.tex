%pagestyle{fancy} %enable headers and footers again

\begin{comment}
\pdfbookmark[0]{Danish title page}{label:titlepage_en}
\aautitlepage{%
  \englishprojectinfo{
    Project Title %title
  }{%
    Analoge kredsløb og systemer %theme
  }{%
    P3: 2. September 2014 - 17. December 2014 %project period
  }{%
    14gr313 % project group
  }{%
    %list of group members
    Amalie Vistoft Petersen\\
    Mikkel Krogh Simonsen\\
    Rasmus Gundorff Sæderup\\
    Simon Bjerre Krogh\\
    Thomas Kær Juel Jørgensen\\
    Thomas 'Godlike' Rasmussen
  }{%
    %list of supervisors
    Tom S. Pedersen

  }{%
    9 % number of printed copies
  }{%
    \today % date of completion
  }%
}{%department and address
  \textbf{Institut for Elektroniske Systemer}\\
  Fredrik Bajers Vej 7\\
  DK-9220 Aalborg Ø\\
  }{% the abstract
  Here is the abstract
}

\cleardoublepage

\end{comment}

\selectlanguage{english}
\pdfbookmark[0]{Titelblad}{label:titelblad}
\aautitlepage{%
  \danishprojectinfo{
    Practical investigation of Rayleigh fading model in URC conditions%title
  }{%
    AAU MSc Project (WCS8)  %theme
  }{%
    1. February - 29. May %project period
  }{%
    17gr850 % project group
  }{%
    %list of group members
    Andreas Vembe Jäger\\
    Mads Røgeskov Gotthardsen\\
    Markos Vourvachis\\
    Thomas Kær Juel Jørgensen
  }{%
    %list of supervisors
    Patrick Eggers 
    }{
    Petar Popovski
    }{%
    6 % number of printed copies
  }
  {%
    \today % date of completion
  }%
}{%department and address
  \textrm{\textbf{Institute of Electronic Systems  }\\
  Fredrik Bajers Vej 7\\
  DK-9220 Aalborg Ø\\}
 }{
\setlength{\parindent}{3ex} As technology progresses, more and more electronics around us become wireless. It is very important to have high reliability and coverage to maintain these wireless communication links. The upcoming 5G technology promises a feature of ultra-reliable communication with outage probability less than 1E-5. Today's existing statistical models of the wireless channel are extrapolated to untested low fades which might not hold true.\par
The purpose of this paper is to design a system that can measure the fading at these low probabilities and a statistical method to interpret the data to verify if the extrapolation of existing models are still valid.\par
It was found that 4.04E+6 independent samples were needed to test a fading probability level of 1E-5. To achieve the high number of samples the use of both the temporal and spatial domain was examined. The analysis found that the assumptions of a stationary channel were slightly dubious with regards to the spatial domain due to the relatively small measurement area. In the end 2.09E+6 uncorrelated samples were collected, verifying the measurements down to a fading probability level of 1E-5 deviate from the Rayleigh fading model by only ($1.59\pm 1.37$) dB. 

%The first step to this investigation was to determine a sufficient number of samples so the rare phenomena can be observed. After that it was needed to analyze the different ways to obtain these samples due to limited time and environment to measure. Alongside the research for the different domains, it was important to find a dynamic range that would let us detect the deep fades of the signal.

}