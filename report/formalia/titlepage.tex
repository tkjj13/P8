%pagestyle{fancy} %enable headers and footers again

\begin{comment}
\pdfbookmark[0]{Danish title page}{label:titlepage_en}
\aautitlepage{%
  \englishprojectinfo{
    Project Title %title
  }{%
    Analoge kredsløb og systemer %theme
  }{%
    P3: 2. September 2014 - 17. December 2014 %project period
  }{%
    14gr313 % project group
  }{%
    %list of group members
    Amalie Vistoft Petersen\\
    Mikkel Krogh Simonsen\\
    Rasmus Gundorff Sæderup\\
    Simon Bjerre Krogh\\
    Thomas Kær Juel Jørgensen\\
    Thomas 'Godlike' Rasmussen
  }{%
    %list of supervisors
    Tom S. Pedersen

  }{%
    9 % number of printed copies
  }{%
    \today % date of completion
  }%
}{%department and address
  \textbf{Institut for Elektroniske Systemer}\\
  Fredrik Bajers Vej 7\\
  DK-9220 Aalborg Ø\\
  }{% the abstract
  Here is the abstract
}

\cleardoublepage

\end{comment}

\selectlanguage{english}
\pdfbookmark[0]{Titelblad}{label:titelblad}
\aautitlepage{%
  \danishprojectinfo{
    Assessment of practical channel impairment and antenna processing\\- in realistic URC conditions%title
  }{%
    AAU MSc Project (WCS8)  %theme
  }{%
    1. February - 29. May %project period
  }{%
    17gr850 % project group
  }{%
    %list of group members
    Andreas Vembe Jäger\\
    Mads Røgeskov Gotthardsen\\
    Markos Vourvachis\\
    Thomas Kær Juel Jørgensen
  }{%
    %list of supervisors
    Patrick Eggers 
    }{
    Petar Popovski
    }{%
    6 % number of printed copies
  }
  {%
    \today % date of completion
  }%
}{%department and address
  \textrm{\textbf{Institute of Electronic Systems  }\\
  Fredrik Bajers Vej 7\\
  DK-9220 Aalborg Ø\\}
 }{
\setlength{\parindent}{3ex} As technology progresses, more and more electronics around us become wireless. It’s very important to have high reliability and coverage to maintain these wireless communication links. The upcoming 5g technology promises a feature of ultra-reliable communication with outage probability less than $10^{-6}$. The problem is that there are no models to confirm that low probability and the way that are assumed is statistical by extrapolating other known cellular models.\par
The purpose of this paper is to design a system that can measure the fading gain in these low outage probabilities and a statistical method to interpret the data in a way to confirm if the existing models are correct.\par
The first step to this investigation was to determine a sufficient number of samples so the rare phenomena can be observed. After that it was needed to analyze the different ways to obtain these samples due to limited time and environment to measure. Alongside the research for the different domains, it was important to find a dynamic range that would let us detect the deep fades of the signal.

}