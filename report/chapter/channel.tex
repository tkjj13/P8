\section{Not report ready (Thomas)}
\begin{equation}
\lambda = \frac{c}{f}
\end{equation}

\begin{equation}
f_D=\frac{v}{\lambda}
\end{equation}

\begin{equation}
BW = 2\cdot f_D
\end{equation}

\begin{equation}
w = k \cdot T \cdot BW
\end{equation}


\begin{equation}
w \propto BW \propto f \propto \frac{1}{v}
\end{equation}


Rayleigh fading
\begin{equation}
CDF: p(xs < x | \theta) = 1-exp\left(-\frac{x^2}{2\cdot \theta^2}\right)
\end{equation}
\begin{equation}
CDF: p(SNRs < SNR | E\left(SNR\right)) = 1-exp\left(-\frac{SNRs}{E\left(SNR\right)}\right)
\end{equation}

\begin{equation}
\mu(x) = \theta\cdot\sqrt{\frac{\pi}{2}}
\end{equation}

\begin{equation}
var(x) = \frac{4-\pi}{2}\cdot \theta^2
\end{equation}

Interval:
\begin{equation}
\bar{x} \pm z_{\frac{\alpha}{2}} \cdot \sqrt{\frac{var(x)}{N}}
\end{equation}
Threshold, A

\begin{equation}
N \leq \left(z_{\frac{\alpha}{2}} \cdot \frac{\sqrt{var(x)}}{A} \right)^2
\end{equation}

Strict limitations
\begin{equation}
SNRs > 1
\end{equation}

Other limitations
\begin{equation}
E\left(SNR\right) \;>\, \frac{1}{raylinv(p,\theta)} 
\end{equation}


\chapter{The wireless channel}
% \chapter{Channel Dynamics} possible other headline




\section{The idea of Multipath}
% (basic intro / symbol introduction)
%A general way to interpret communication is by assuming two persons Alice and Bob. When Alice successfully deliver her message to Bob the communication has been a success, unfortunately Charlie likes to tease Alice and Bob by interrupting the message while its under way, making the communication a failure. The interesting part here is what happens to the message under way that determines whether it results in a success or failure. 

%In wireless communication the path travelled is what is interesting, this is especially true when more than one path is possible.
%Communication happens when a message is transmitted from a to b

\section{Rayleigh Fading}
% (confidence level vs number of samples)

%\section{Statistics}
%\subsection{Confidence Interval}
%\subsection{Distributions?}

In a wireless communication system the signal can reach the receiver from many different reflective paths. This is a phenomenon called multipath propagation. It has many effects on the signal, usually on its amplitude, phase and angle of arrival.\citep{Fading} A more specific term called multipath fading is used to describe these effects. Depending on those effects the receiver may experience constructive or distractive interference. Basically it means that the signal it might get amplified or attenuated depending on the interference. Strong distractive interference is usually called deep fade and can lead to failure of the communication system for a short period of time. Cause of the multipath fading the channel can be affected in certain ways.
	\begin{enumerate}
	\item Flat Fading : All the frequencies experience changes in amplitude over a period of time. That has as a result some reduction on the SNR.
	\item Selective Fading : Different frequencies across the channel experience different changes in phase and amplitude. Sometimes deep fades may observed in the channel, causing problem in the communication.\citep{FlatSelective}
	\end{enumerate}
Rayleigh fading is a propagation model that is used to describe situations where there are no line of sight connections and the signal may vary randomly due to scatterings. Usually this situation fits on an urban environment scenario with many buildings and obstacles.


\section{Correlation/coherence function}
% (Relation of number of samples to physical dimensions)

\section{Doppler}
% (relation to bandwidth and time estimate)

\section{Introduction of noisy signal}