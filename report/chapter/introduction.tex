\chapter{Introduction}
\section{Motivation}


\chapter{Key Concepts}
\section{URC}
\section{Diversity}
Diversity is used to combat fading in a wireless system. By using clever techniques you can effectively deliver several copies or replicas of a signal to the receiver. There is a lower chase that all of these copies of signals are going to have a deep fade at the same time \citep[p. 4-6]{diversityFuture}. The overall goal is to provide a gain in signal quality by having several signals that independently fade from each other without the cost of more power consumption, reduced bit rate and complexity or other resources. There are several ways to achieve diversity in a wireless system. 
Time diversity: By retransmitting the same information you get some diversity gain on the cost of decreased bit/rate. Usually a forward error correction code is added instead of just re transmitting the signals several times.
Antenna diversity / Spatial diversity
By having several antennas at both/or the receiver and transmitter a antenna diversity is achieved. With several antennas the signal can be combined together to process the signals into one. To get independently fading signals the signals need to be uncorrelated by separating the antennas least half a wavelength (micro diversity) [2]. Antenna diversity comes at the cost of having to power more than one antenna.


\section{Dynamic range}
\section{Rayleigh Fading}
\section{Statistics}
\subsection{Confidence Interval}
\subsection{Distributions?}

\chapter{Measurement Campaign}
\section{Setup}
\section{Results}

\chapter{Discussion}
\section{Uncertainties}
\section{Comparison of Measurements and Theory}

\chapter{Conclusion}




