\chapter{Introduction}
\label{chap:introduction}
%Channel limit
%
%URC
%Deep fade
%drop out
%5G





For all types of communication an important aspect is the channel through which the message is transmitted. For wireless communication, this is even more the case, as the channel changes constantly. The transmission of the message can be done in different environments, which have different impacts on the received signal. Common for nearly all the environments is they introduce fading, which if not accounted for can have devastating effects for the transmission. 

Fading occurs when multipath propagation is present, that will introduce points in space where the waves adds either constructively or destructively, as can be seen on \autoref{intro_fading}. The constructive spots is not of much interest as it is only a couple of dB's difference, however the destructive interference can create spots with losses that borders minus infinite dB's. These spots are called deep fades and is is where the communication might suffer an outage. An outage occurs if the transmitted signal drops below the receivers sensitivity level so the signal is lost. There are different tools to heighten the chance for the receiver to receive signals that might experience deep fades and therefore increase the reliability of the communication link


\begin{figure}[H]
\centering
\includegraphics[width=\textwidth]{figures/intro_fading.png}
\caption{Fading gain from two wave fronts meeting, the deep fade spots have been elevated to allow for a better color resolution.}
\label{intro_fading}
\end{figure}

\section{Motivation}

The problem about fading is addressed already in cellular networks like \gls{LTE} or \gls{LTE-A} and many others, however with the development of the 5G network, new concepts is introduced. The 5G network is proposed to consist of three parts: \Gls{URLLC}, \gls{mMTC} and \gls{eMBB} \citep{5G}. Both the \gls{mMTC} and \gls{eMBB} will be sort of extensions to the existing networks, focusing on either the number of users or the data rate respectively, the \gls{URLLC} will introduce some new problems mostly in the reliability aspect where earlier networks have run with an outage probability of 0.1\%, the URLLC part shall have an outage probability under 0.001-0.0001\% \citep{LTE,Petar5G}, but also by lowering the latency from (>30 to <1) ms \citep{LTE,5G_Latency}. By introducing this new feature, critical system with special needs might be able to use the cellular network instead of a wired channel, which today provide a higher reliability, but comes with a greater cost in both price and installation. It is predicted that some of the applications needing a URLLC channel in the future could be self driving cars, emergency services and sensitive machinery like a remote surgery system \citep{Petar5G}. Common for these examples is that they all require both low latency and low outage probability.
%Some of the applications needing a URLLC channel could be self driving cars, in emergency cases, is it not only necessary to provide low latency, the certainty of the message to arrive is also desired. %where short message about position and speed can be send to other cars in the area, with low chance of needing re-transmission, which is importing in case of high alert situations. 

Other system that can gain advantages with URLLC, is systems where the communication window is very small and therefore do not have time for re-transmissions or a lot of data processing for error coding.


The problem is that in the URLLC scenarios, an in depth description of the channel is needed. A way to get higher reliability is to transmit with a higher power level, so the \gls{SNR} gets stronger on the communication link. But even if the overall signal is stronger, there will still be problem with fading in some points in space as seen in \autoref{intro_fading}. Today descriptions of such channels have been verified to a probability level down to $10^{-3}$ - $10^{-4}$. Existing research are assuming that an extrapolation of the models extend to probability levels that are multiple magnitudes lower as seen in \autoref{fading_gain}. This might have been all right so far, but to achieve URLLC conditions the channel needs to be tested for very low fades down to $-50$dB. Therefore this report wants to investigate untested assumptions made in multipath fading channels. The problem when testing such channels is that current measuring systems are not designed for the needed dynamic range or the large amount of samples needed to measure these very deep fades. 

\textbf{General problem statement:}
Design a measurement system that can handle very deep fades and a statistical method to process the data and compare to existing models used in URLLC channels.

%This can be further specified to a more technical problem statement: \todo{Where do these numbers come from and why do we have two statements just after each other?!?}

%Ascertain with 90\% confidence with a 1 dB (25\%) margin that in a multipath environment, the cumulative distribution function of the fading gain can be assumed to increase in a log log linear fashion between -50dB and -20dB.

To this it is important to investigate the following points: 
\begin{itemize}
	\item Find connections between channel statistics and sample size.
	\item Find the reasonable confidence interval based on the minimum required sample size.
	\begin{itemize}
	\item Look into if there is a way to reduce the amount of samples needed by using statistical techniques.
 	\end{itemize}
	\item Figure out how to obtain uncorrelated samples in space,time and frequency.
	\item Balance the sample acquisition for available resources. 
	\item Design  a measurement setup that can acquire the samples needed.
	\item Specify a method to process the obtained channel measurements.
	\item Analyse the measurement and link the results to the current assumptions.
\end{itemize}

\begin{figure}[H]
\centering
\includegraphics[width=0.65\textwidth]{figures/fading_gain.pdf}
\caption{\Gls{CDF} of fading gain in different channel environments.}
\label{fading_gain}
\end{figure}



\section{Project outline}

As stated the focus will be to measure and compare channel characteristics at URLLC conditions. The project will not address the issue of low latency, as this is a completely different issue. This leaves a main aspects that needs investigation: The wireless channel and its practical limitations e.g. with a focus on fading. To investigate the channel, a test setup is needed, that can measure the varying signal power and therefore the limitations in the channel. This gives two main aspects for this project as seen on \autoref{ChannelAndEquip}.

\begin{figure}[H]
\centering
\includegraphics[width=0.85\textwidth]{figures/ProOutline.png}
\caption{Two of the main aspects of the project.}
\label{ChannelAndEquip}
\end{figure}



The rest of the report is structured as follows. In part I the general and theoretical problems are solved. Chapter 2 of the report will explain some basic statistics in relation to brute force estimation. When a number of samples required is established, statistical techniques is explored to see if it's possible to reduce the amount of samples needed. In Chapter 3 the wireless channel is discussed in general together with  the ideas of multipath and fading channels. This will give a method to obtain uncorrelated samples in wireless channels. Closing part I chapter 4 goes through basic receiver structure to examine potential problems and requirements of the measurements. By establishing these principals of the wireless channels and multipath fading the measurement setup is designed in part II chapter 5 describe the available resources for the measurements followed by chapter 6 that goes through the general setup parameters used for the measurements. Chapter 7 and 8 discusses the pilot test and the actual measurement campaign and the results of this. Next is discussion of uncertainties in the measurement campaign in chapter 9, and finally chapter 10 is the conclusion.

%First some values from previous scientific research and theory is used to estimate  parameters needed in the measurement. This gives a insight in our thought process when designing the measurement setup how do address  our limitations. The final settings and values used in the measurement is given together with the raw results. The following part is the analysis and presentation of the findings from the measurement. After that, the next chapter includes the  discussion of the uncertainties and difficulties that occurred. The final chapter contains the conclusions of the report.




 




