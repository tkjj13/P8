\chapter{Conclusion}

When designing a measurement campaign prerequisites are important. A large part of the project was understanding how channel behaviour and how this can be represented in a stochastic manner. These models needs to be justified and connected to real world multipath environment. The deep fades that needs to be addressed in \gls{URLLC} are not empirically tested for but assumed. (assumed not necessarily known)

A statistical method of estimating the number of samples needed to validate the measurement to a 90\% $\pm 1dB$ was found. To further decrease the required amount of samples, several statistical techniques where investigated. But no such technique can be used because it changes the overall distribution. However the bootstrap is a useful method that can provide the confidence level and margin with the given samples.

To investigate the deep fades that needs to be dealt with to deliver an connection that fulfils the \gls{URLLC} conditions, requires an extensive amount of samples. Therefore the spatial and temporal domains has to be taken advantage of. To achieve this optimal frequency and space separation needed to be estimated

To validate the measurement of these predicted deep fades a large dynamic range is required. Together with the large dynamic range a large amount of samples are needed. These two part are linked has to be balanced and adjusted to make a measurement possible to do. An SNR margin is also needed to confirm that it's the signal and not the noise being measured

The results presented shows that the wanted confidence level of 90\% with a $\pm 1dB$ margin was not achieved and it had to be increased to $\pm 1.33dB$.\todo{update numbers if needed}


%\textbf{General problem statement:}
%Design a measurement system that can handle very deep fades and a statistical method to process the data and compare to existing models used in URLLC channels.

%This can be further specified to a more technical problem statement: \todo{Where do these numbers come from and why do we have two statements just after each other?!?}

%Ascertain with 90\% confidence with a 1 dB (25\%) margin that in a multipath environment, the cumulative distribution function of the fading gain can be assumed to increase in a log log linear fashion between -50dB and -20dB.

%To this it is important to investigate the following points: 
%\begin{itemize}
%	\item Find connections between channel statistics and sample size.
%	\item Find the reasonable confidence interval based on the minimum required sample size.
%	\begin{itemize}
%	\item Look into if there is a way to reduce the amount of samples needed by using statistical techniques.
% 	\end{itemize}
%	\item Figure out how to obtain uncorrelated samples in space,time and frequency.
%	\item Balance the sample acquisition for available resources. 
%	\item Design  a measurement setup that can acquire the samples needed.
%	\item Specify a method to process the obtained channel measurements.
%	\item Analyse the measurement and link the results to the current assumptions.
%\end{itemize}