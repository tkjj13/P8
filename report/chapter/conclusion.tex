\chapter{Conclusion}

The project found that designing a measurement campaign prerequisites are important. A large part of the project was understanding channel behaviour and how this can be represented in a stochastic manner. This was used to get an understanding of how to measure a real world multipath environment. The statistics of the very deep fades that needs to be addressed in \gls{URLLC} had not been empirically tested for but assumed Rayleigh based on extensions from other measurements of fading.

A statistical method of estimating the number of samples needed to validate the measurement was used and it was found that for a confidence of $\pm 1dB$ at the 90\% level $4.04E+6$. Several statistical techniques to decrease the required amount of samples were investigated, without luck as they require to changes the overall distribution, which is not feasible for real measurements. The bootstrap method was found to be a useful method that can provide the confidence level and interval based on the measurements.

To obtain the required amount of samples it was found that use of both spatial and temporal domain was necessary. To reduce the correlation between samples the needed separation was estimated based on a Rayleigh channel. This was found to be 0.4$\lambda$ in space and 25 MHz for the room that was measured. 

%To investigate the deep fades that needs to be dealt with to deliver an connection that fulfils the \gls{URLLC} conditions, requires an extensive amount of samples. Therefore the spatial and temporal domains has to be taken advantage of. To achieve this optimal frequency and space separation needed to be estimated

To validate the measurement of these predicted deep fades it was found that an average SNR of 64 dB was required, this includes not only the level of expected fading but a margin ensuring that the measurement is not distorted by noise. This together with the needed amount of samples means the measurement is quite time consuming. 

In the end a total of 4.18$E+6$ samples was collected. However analysis of the samples showed that the assumptions of a WSSUS channel could only be scarcely justified. Further more was it found that half of the samples was to correlated ending with an effective sample pool of 2.09$E+6$ samples. The result being that the Rayleigh fading model can be extended to some degree but verification to the needed levels for \gls{URLLC} can not be confirmed, what can be confirmed is the model is valid down to a probability of $10^{-5}$ with an error of (1.26$\pm$1.23) dB with 90\% confidence.

Further work in this regard would include introducing a PA or similar that allows for the TX and RX antennas to be space further apart, simultaneously increasing the SNR giving a larger region to measure the fading. Also investigation of using diversity gain to reduce the needed SNR might prove useful.

%\textbf{General problem statement:}
%Design a measurement system that can handle very deep fades and a statistical method to process the data and compare to existing models used in URLLC channels.

%This can be further specified to a more technical problem statement: \todo{Where do these numbers come from and why do we have two statements just after each other?!?}

%Ascertain with 90\% confidence with a 1 dB (25\%) margin that in a multipath environment, the cumulative distribution function of the fading gain can be assumed to increase in a log log linear fashion between -50dB and -20dB.

%To this it is important to investigate the following points: 
%\begin{itemize}
%	\item Find connections between channel statistics and sample size.
%	\item Find the reasonable confidence interval based on the minimum required sample size.
%	\begin{itemize}
%	\item Look into if there is a way to reduce the amount of samples needed by using statistical techniques.
% 	\end{itemize}
%	\item Figure out how to obtain uncorrelated samples in space,time and frequency.
%	\item Balance the sample acquisition for available resources. 
%	\item Design  a measurement setup that can acquire the samples needed.
%	\item Specify a method to process the obtained channel measurements.
%	\item Analyse the measurement and link the results to the current assumptions.
%\end{itemize}