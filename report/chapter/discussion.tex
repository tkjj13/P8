\chapter{Discussion}
\section{Uncertainties}
One of the biggest unsertiency in this prosject is when the freuqnecy span is realtivily big 1Ghz do you acually measure the same channel;Stationariness principle
\todo {Errors when measureing the PDP and the autocorr of the area, becase we moved around and there is not unefomly power}

When designing a measurement campaign prerequisites are important. A large part of the project was understanding how channel behaviour and how this can be represented in a stochastic manner. 

Gathering the number of samples needed for to validate the measured data has proven to be difficult. A big point is that the theory assumes uncorrelated data, but in reality completely uncorrelated samples are hard to obtain. For this reason a much larger sample pool then estimated is needed. Since the equipment was available for a limited time period the measurement was done over a whole day in one room. A more extensive measurements over several days in different rooms was desired, but analysing and validating the method was prioritized.

Stationery of a large frequency sweeps.
To achieve a large number of samples in frequency a large $f_{span}$ was used. When doing such a large span one has to be careful to consider the stationery of the channel.

Since the spatial part of the measurements was done by a human holding the antenna and moving human errors is present in the measurement. The way the antenna array has held or the consistency of the speed of movement. This is a very hard thing to accomplish without a elaborate setup with motorized rail or arm. Because of the limited time where the equipment was available all the measurements was limited to one day.

The limited time with the equipment.


Averaging over the spatial  area also has the effect of averaging over eventual fades, but a completely homogeneous path-loss is hard to obtain in small sized measurement.

\section{Comparison of Measurements and Theory}