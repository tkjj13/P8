\chapter{Discussion of Uncertainties}
%
%When designing a measurement campaign prerequisites are important. A large part of the project was understanding channel behaviour and how this can be represented in a stochastic manner.
%
%A more extensive measurement over several days in different rooms was considered but analysing and validating the method was prioritized.

%\section{Uncertainties}
%\chapter{Uncertainties}

Gathering the number of samples needed to validate the measured data has proven to be difficult. A big point is that the theory assumes uncorrelated data but in reality completely uncorrelated samples are hard to obtain. A problem in the performed measurement campaign was the movement during the measurement campaign might have been to slow to obtain uncorrelated samples, analysis found that removing every second sample, which corresponds to moving with double speed, reduces the correlation to a more acceptable level. However correlation was still present and therefore a larger sample pool is needed compared to the estimation. 

Since the measurements was done by holding and moving the antenna, its fair to say that human errors are present. This is due to the difficulty on maintaining constant speed and stable movement. Also the human presence might obscure some scatterers, which could be a contributing factor to why the analysis found it hard to validate the measurements as having \gls{US}. To remove the human factor from the measurement a more elaborate setup with a motorized rail or arm is needed and even then it might still affects the scatterers. Because of the limited time where the equipment was available all the measurements were done in one day, and with a simple setup.

While analysing the measured data an antenna pattern became apparent, to account for this a moving average was used. Averaging over the spatial area also has the effect of averaging over eventual fades, however to reduce any such implication an average was done over 5043 samples or 41 space samples. Due to the relatively small environment a completely homogeneous path-loss across all space samples can not be achieved, likewise to remove disturbances from e.g. the antenna pattern a much greater distance needs to be put between transmitter and receiver which again complicates the setup. To increase the distance between transmitter and receiver would also result in a greater \gls{PL} which would influence the SNR, requiring a greater transmit power.

To achieve a large number of samples in frequency, a wide $f_{span}$ was used. In a case like this it is important to consider the stationarity of the channel. A wide $f_{span}$ will give more uncorrelated samples in frequency but increases the risk of measuring a completely different channel. An argument could be made that deep fades are so dependant on the environment that a measurement would only be valid for that specific room or situation being measured. The analysis found that the span was not a problem in the measurement campaign, however only one environment has been measured so it might still be the case it is only valid for this environment.

%\section{Comparison of Measurements and Theory}