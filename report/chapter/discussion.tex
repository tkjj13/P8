\chapter{Discussion}
\section{Uncertainties}
One of the biggest unsertiency in this prosject is when the freuqnecy span is realtivily big 1Ghz do you acually measure the same channel;Stationariness principle
\todo {Errors when measureing the PDP and the autocorr of the area, becase we moved around and there is not unefomly power}

When designing a measurement campaign prerequisites are important. A large part of the project was understanding how channel behaviour and how this can be represented in a stochastic manner. 

Gathering the number of samples needed for to validate the measured data has proven to be difficult. A big point is that the theory assumes uncorrelated data, but in reality completely uncorrelated samples are hard to obtain. For this reason a much larger sample pool then estimated is needed and will take a long time, where measurements needs to be over several days in different places.

Since the spatial part of the measurements was done by a human holding and moving with the  antenna it would be fair to say that human errors would be present in the measurement. This is due to the different heights needed to hold the antenna and the low speed that the subject was supposed to move. This is a very hard thing to accomplish without an elaborate setup with motorized rail or arm. Because of the limited time where the equipment was available all the measurements was limited to one day.

The limited time with the equipment.


Averaging over the spatial area also has the effect of averaging over eventual fades, but a completely homogeneous path-loss is hard to obtain in small sized measurement.

\section{Comparison of Measurements and Theory}